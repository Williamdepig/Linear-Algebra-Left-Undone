\makeatletter
   \def\input@path{{..}}
\makeatother

\documentclass[
    % colors = false,
    geometry = 16k,
]{LALUbook}

\usepackage{booktabs} % Excel 导出的大表格
\usepackage{rotating}
\usepackage{extarrows}

\usepackage{float}
\usepackage{diagbox}
\usepackage{caption}

\usepackage{pgfplots}
\usetikzlibrary{cd, arrows, arrows.meta, calc, intersections, decorations.pathreplacing, patterns, decorations.markings}
\pgfplotsset{compat=newest}

\usepackage[xindy, splitindex]{imakeidx}
\makeindex[
    columns=1,
    program=truexindy,
    intoc=true,
    options=-M texindy -I xelatex -C utf8,
    title={名词索引}
] % 名词索引
\makeindex[
    columns=3,
    program=truexindy,
    intoc=true,
    options=-M numeric-sort -M latex -M latex-loc-fmts -M makeindex -I xelatex -C utf8,
    name=sym,
    title={符号索引}
] % 符号索引

% 嵌套 enumerate 环境的 label
\setlist[enumerate,2]{label=(\arabic*)}
\setlist[enumerate,3]{label=\roman*.}

\usepackage{xparse}
\NewDocumentCommand{\term}{m}{{\sffamily\heiti\bfseries{#1}}}

% 标题格式
% \ResetChapterNumberingStyle 重置章节编号为正常样式
% \SetLUChapterNumberingStyle 设置章节编号为未竟专题样式
% 参数为其上一章节的编号
\newcounter{LUchapter}
\NewDocumentCommand{\ResetChapterNumberingStyle}{m}{%
\setcounter{chapter}{#1}
\ctexset{
    chapter = {format={\centering\Huge\bfseries},name={第,讲},number=\arabic{chapter}},
    section = {format={\raggedright\Large\bfseries},name={,},number={\arabic{chapter}.\arabic{section}}},
    subsection = {format={\raggedright\large\bfseries},name={,},number={\arabic{chapter}.\arabic{section}.\arabic{subsection}}},
    subsubsection = {format={\raggedright\normalsize\bfseries},name={,},number={\arabic{chapter}.\arabic{section}.\arabic{subsection}.\arabic{subsubsection}}},
}
\renewcommand{\thechapter}{\arabic{chapter}}
}
\NewDocumentCommand{\SetLUChapterNumberingStyle}{m}{%
\setcounter{chapter}{#1}
\refstepcounter{LUchapter}
\ctexset{
    chapter = {format={\centering\Huge\bfseries},name={未竟专题,},number={\zhnumber{\arabic{LUchapter}}}},
    section = {format={\raggedright\Large\bfseries},name={,},number={\texorpdfstring{\arabic{chapter}$\boldsymbol{\varepsilon}$}{\arabic{chapter}ε}.\arabic{section}}},
    subsection = {format={\raggedright\large\bfseries},name={,},number={\texorpdfstring{\arabic{chapter}$\boldsymbol{\varepsilon}$}{\arabic{chapter}ε}.\arabic{section}.\arabic{subsection}}},
    subsubsection = {format={\raggedright\normalsize\bfseries},name={,},number={\texorpdfstring{\arabic{chapter}$\boldsymbol{\varepsilon}$}{\arabic{chapter}ε}.\arabic{section}.\arabic{subsection}.\arabic{subsubsection}}},
}
\renewcommand{\thechapter}{\arabic{chapter}$\boldsymbol{\varepsilon}$}
}

\ResetChapterNumberingStyle{0}

\title{\heiti 浙江大学 2023--2024 学年 \\ 线性代数荣誉课辅学讲义}
\author{2023--2024 学年线性代数 I/II(H)辅学授课 \\ 吴一航 \quad \verb|yhwu_is@zju.edu.cn|}

\AtEndPreamble{\hypersetup{
    hypertexnames = true,
    pdfauthor = {吴一航},
    pdftitle = {线性代数荣誉课辅学讲义},
}}

\begin{document}
\frontmatter

% 封面,代替 \maketitle
\includepdf[
    pages={1},
    noautoscale=true,
    trim=0 0 0 10mm,
    clip,
]{./figs/cover-16k.pdf}

\songti

{% 插入空页
\null
\thispagestyle{empty}
\clearpage
\thispagestyle{empty}
\input{./其它/扉页.tex}
}
\clearpage
\setcounter{page}{1}
\input{./其它/序.tex}
\input{./其它/致读者.tex}

\clearpage
\pdfbookmark[0]{目录}{contents}
\tableofcontents

\addtolength{\parskip}{.5em}

\mainmatter
\setcounter{page}{1} % 将页码计数设置为 1
\chapter{预备知识}

线性代数作为大学的第一门数学课,预修要求并不高. 我们默认读者具有基本的高中数学知识,因此关于集合、映射以及向量的基本知识我们不在此赘述. 这一讲我们将从基本代数结构开始,以便后续线性空间的引入,然后我们将介绍本书中常见的概念——等价类和最常用的算法之一——高斯消元法.

\section{基本代数结构}

我们选择从基本代数结构谈起,因为在以往的实践中我们深切地体会到直接引入线性空间的跳跃. 因此我们希望从更具象的例子开始,首先引入``代数结构''这一基本概念,然后在下一节中自然地引出线性空间的定义.

我们首先考察一个简单的例子:实数集$\mathbf{R}$,它是一个集合. 在初中我们便知道,在$\mathbf{R}$上我们可以定义加法和乘法两种运算. 本质而言,运算是一种映射(或者更通俗而言,函数):

\begin{center}
    \begin{tabular}{rrcl}
        $+\enspace\colon$      & $\mathbf{R}\times\mathbf{R}$ & $\to$     & $\mathbf{R}$ \\
                               & $(a,b)$                      & $\mapsto$ & $a+b$        \\
        $\times\enspace\colon$ & $\mathbf{R}\times\mathbf{R}$ & $\to$     & $\mathbf{R}$ \\
                               & $(a,b)$                      & $\mapsto$ & $a\times b$
    \end{tabular}
\end{center}

上面的定义中出现了一个新的记号,即两个集合之间出现了乘号,这实际上是集合的笛卡尔积运算,定义如下:

\begin{definition}[笛卡尔积] \index{dikaerji@笛卡尔积 (Cartesian product)}
    设$A$和$B$是两个非空集合,我们把集合
    \[A\times B=\{(a,b) \mid a\in A, b\in B\}\]
    称为集合$A$和$B$的\term{笛卡尔积}.
\end{definition}

因此我们很容易理解$\mathbf{R}\times\mathbf{R}$作为集合长什么样,它的元素是形如$(a,b)$的有序对,其中$a,b\in\mathbf{R}$. 事实上,我们可以将$\mathbf{R}\times\mathbf{R}$看作平面上的点集,其中的元素$(a,b)$对应于平面上的一个点,这一点的横坐标为$a$,纵坐标为$b$.

我们回到运算的映射表示,我们发现$+$和$\times$都以实数的有序对作为函数的自变量,函数值也是一个实数. 或许读者看到这里还是对运算的定义有些许迷茫,但如果我们回忆映射的基本定义$f:A\to B$表示给$A$中的任意元素$a$指派一个$B$中的元素$f(a)$,并将加法乘法写成$+(2,3)=5$,$\times(2,3)=6$,想必就会恍然大悟:$+$和$\times$实际上就是函数名,函数做的事情就是输入两个自变量然后进行加法/乘法运算得到结果,并把这个结果指派给自变量作为函数值.

在上述讨论中,我们所做的事情很简单,就是给定一个集合,然后在这一集合的元素之间定义运算. 实际上这就是代数系统的定义:
\begin{definition}[代数系统] \index{daishuxitong@代数系统 (algebraic system)}
    一般地,我们把一个非空集合$X$和在$X$上定义的若干代数运算$f_1,\ldots,f_k$组成的系统称为\term{代数系统}(简称代数系),记作$\langle X : f_1,\ldots,f_k\rangle$.
\end{definition}

特别注意的是,代数系统上定义的运算必须保证封闭性,也就是运算后的结果必须仍然在集合$X$中. 这事实上早已由映射的方式对运算的定义保证了.

不难理解,代数系统其中蕴含的性质与其中定义的运算具有的性质是关联很大的. 我们仍然以实数域为例,介绍在代数学中关心的几个运算性质. 我们首先讨论实数域上的加法运算,以下性质对于任意$a,b,c\in\mathbf{R}$都成立:

\begin{enumerate}
    \item 结合律:$(a+b)+c=a+(b+c)$;

    \item 单位元:存在一个元素$0$,使得$a+0=0+a=a$;

    \item 逆元:对于任意$a$,存在一个元素$-a$,使得$a+(-a)=(-a)+a=0$(0为单位元);

    \item 交换律:$a+b=b+a$.
\end{enumerate}

对于乘法运算(可记为$\cdot$或$\times$),单位元一般记为$1$(更一般的可以记为$e$),逆元记为$a^{-1}$. 事实上,我们可以给出更多的例子:
\begin{example}\label{ex:1:Abel 群}
    \begin{enumerate}
        \item 代数系统$\langle \mathbf{R}\backslash\{0\}:\circ\rangle$定义的一般乘法运算

        \item 代数系统$\langle \mathbf{R}^2:+\rangle$定义的平面向量的加法
    \end{enumerate}
    均满足上述四条运算性质.
\end{example}

事实上,我们可以对上面的定义做进一步的抽象. 我们可以忽略集合中元素的意义差异(元素可以表示实数,也可以在上述例子中表示平面向量等几何对象),同时也可以忽略运算定义的差异,只关心运算作用于集合元素的性质. 对于一般的代数系统$\langle G:\circ\rangle$,我们有如下定义:
\begin{definition}[群] \label{def:1:群} \index{qun@群 (group)}
    若运算$\circ$满足结合律,则称代数系统$\langle G:\circ\rangle$为\term{半群}\index{qun!banqun@半群 (semigroup)};若在半群基础上存在单位元,则称之为\term{含幺半群}\index{qun!hanyaobanqun@含幺半群 (monoid)};若在含幺半群基础上每个元素存在逆元,则称之为\term{群};若在群的基础上运算还满足交换律,则称之为\term{Abel 群},也称\term{交换群}\index{qun!abel@Abel 群 (Abelian group), 交换群 (commutative group)}.
\end{definition}

\autoref{def:1:群} 给出了我们本节第一个要讨论的代数结构——群的定义. 简而言之,代数结构就是在集合上定义具有某些特定性质的运算后得到的一类代数系统. 事实上,教材中42--44页给出了大量抽象的例子有助于同学们理解上述一系列群的定义,并且我们在后续学习矩阵的时候也会遇到一些群结构,相信这些实例能使读者体会到``在集合上定义运算''的方式的多样与抽象.

为方便书写,对于\autoref{def:1:群} 定义的群$\langle G:\circ\rangle$,在不引起混淆的情况下我们可以简写为群$G$. 除此之外,我们还需要指出以下两点:
\begin{theorem}\label{thm:1:群的单位元逆元唯一}
    \begin{enumerate}
        \item 群的单位元唯一;

        \item 群中每个元素的逆元唯一.
    \end{enumerate}
\end{theorem}

\begin{proof}
    \begin{enumerate}
        \item 设$e_1$和$e_2$都是群$G$的单位元,则
              \[e_1=e_1\circ e_2=e_2.\]

        \item 设$b$和$c$都是$a$的逆元,则
              \[b=b\circ e=b\circ(a\circ c)=(b\circ a)\circ c=e\circ c=c.\]
    \end{enumerate}
\end{proof}

其中第一点的证明直接使用了单位元的性质,第二点的证明则使用了结合律和逆元的性质. 这里关于唯一性的证明是非常重要的:我们只需假设要证明唯一的东西有两个,然后说明这两个必然相等即可. 这一思想在之后证明矩阵的逆唯一等问题时也会用到,因此此处特别给出证明强调.

事实上,在很多集合上我们不仅可以定义一种运算,也可以定义两种甚至更多运算,在代数结构中我们仅讨论最多两种运算的情况. 事实上,我们最开始的实数集合定义加法和乘法的例子便可以引入一个新的代数结构——域:
\begin{definition}[域] \index{yu@域 (field)}
    我们称代数系统$\langle F:+,\circ\rangle$为一个\term{域},如果
    \begin{enumerate}
        \item $\langle F:+\rangle$是交换群,其单位元记作0;

        \item $\langle F\backslash\{0\}:\circ\rangle$是交换群;

        \item 运算$\circ$对$+$满足左、右分配律,即
              \begin{gather*}
                  a\circ(b+c)=a\circ b+a\circ c \\
                  (b+c)\circ a=b\circ a+c\circ a
              \end{gather*}
    \end{enumerate}
\end{definition}

显然,实数域$\mathbf{R}$上定义一般的实数加法和乘法后构成一个域. 实际上我们熟悉的例如有理数、实数等集合关于一般的加法和乘法运算都构成域,因此我们会经常使用``有理数域''、``实数域''等说法. 我们称数集对数的加法和乘法构成的域为数域,注意此处运算的定义必须是数学分析中定义的数的加法和乘法,不能是自定义的运算.
\begin{theorem}
    关于数域,我们有如下两个结论:
    \begin{enumerate}
        \item 数集$F$对数的加法和乘法构成数域的充要条件为:$F$包含0,1且对数的加、减、乘、除(除数不为0)运算封闭;

        \item 任何数域都包含有理数域$\mathbf{Q}$,即$\mathbf{Q}$是最小的数域.
    \end{enumerate}
\end{theorem}

上述定理的证明可见教材46页. 事实上,如果加法和乘法的定义不是数的加法和乘法,我们可以定义除了数域之外的域,我们将在本讲介绍完等价类的概念后给出这样的例子.

当然,还有一种代数结构对于$\circ$运算的要求有所降低,但也有广泛的应用,这就是环:
\begin{definition}[环] \index{huan@环 (ring)}
    我们称代数系统$\langle R:+,\circ\rangle$为一个\term{环},如果
    \begin{enumerate}
        \item $\langle R:+\rangle$是交换群,其单位元记作0;

        \item $\langle R:\circ\rangle$是幺半群;

        \item 运算$\circ$对$+$满足左、右分配律,即
              \begin{gather*}
                  a\circ(b+c)=a\circ b+a\circ c \\
                  (b+c)\circ a=b\circ a+c\circ a
              \end{gather*}
    \end{enumerate}

    若进一步每个非$0$($+$运算单位元)元素关于$\circ$都有逆元,则称之为\term{除环}\index{huan!chu@除环 (division ring)}. 另外,若上述定义中$\circ$运算满足交换律,则称为\term{交换环}\index{huan!jiaohuan@交换环 (commutative ring)},结合上述除环和交换环两个定义,我们可以发现,交换除环即为域.
\end{definition}

\begin{example}
    利用定义验证下述关于代数系统的结论:
    \begin{enumerate}
        \item 整数集$\mathbf{Z}$对整数的加法和乘法构成一个交换环,但不是域;

        \item 设$C[a,b]$是闭区间$[a,b]$上的连续函数的集合;它对函数的加法和乘法构成一个环;

        \item 设$Q(\sqrt{2})=\{a+b\sqrt{2} \mid a,b\in\mathbf{Q}\}$,则$Q(\sqrt{2})$是一个数域.
    \end{enumerate}
\end{example}

我想大部分读者都会对抽象出代数结构的原因表示不解,如果这个问题无法解答,我想在下一章直接引入抽象的线性空间更会引发同学们对于``学了这个有什么用''的怀疑. 我们可以举一些不那么贴切但具象的例子来说明这其中的意义. 读者高中阶段想必大都经受过解析几何的摧残,大家在拿到题目时总会首先观察到题目属于``定点''、``定值''或是``极值''等问题,大家将自动与自己做题的经验或技巧匹配用于解答这几类问题. 同理,在研究一个特定的代数系统(例如定义了加法和乘法的实数域)的性质时,我们可以首先将其归类为群、环或是域等,然后我们只需要利用群环域各自的性质来研究这个代数系统的性质,而不需要再去研究这个代数系统的具体定义. 在这一过程中我们找到了一个模型,即将一个孤立的问题转化为了对一个更广泛的问题的研究,正如将解决上千道解析几何问题转化为研究几种作为模型的题目的解法. 这一``寻找模型''的思想在将来的学习生活中我们将经常遇见,在实际中例如投资股票时我们可以将投资转化为提高投资组合的期望收益而尽力降低方差(风险)的求取极值的问题,在理论中,例如在计算理论的学习中我们会将各种各样不同的计算机架构抽象成图灵机模型,这在可计算性的研究中是最基础的模型. 对于这类抽象问题感兴趣的同学不妨可以选择数学科学学院的抽象代数等课程,或是阅读本讲义的``后继''教程\href{https://frightenedfoxcn.github.io/notes/series/alg-for-cs/}{《写给计算机系学生的代数》}作进一步的了解. 事实上,对于对理论感兴趣的同学,抽象代数将是必不可少的基础课程,它将是密码学、量子计算、计算理论以及编程语言理论等诸多领域的必要基础.

当然,这段描述因为涉及的知识容量较大,大概无法说服每一个读者. 但我们会在学习线性空间、线性映射的过程中不断重复这些思想,直到读者具备的知识容量足够时,一定能领会其中的奥妙.

\section{复数域的引入}

本书前半段讨论的框架是实数域、复数域都适用的,当然为了简化,我们的例子大都来源于实数. 从多项式一讲开始,我们便会开始强调实数域和复数域结论的不同,因此我们有必要在此引入复数域.

直观来看,实数位于数轴上,复数则分布在二维平面上,因此我们可以先考虑平面点集$\mathbf{R}^2$,并在其上定义加法和乘法运算使其成为一个域. 我们回顾高中学习的平面向量知识,我们记$\vec{e}_1=(1,0)$,$\vec{e}_2=(0,1)$,则$\mathbf{R}^2$上的任一向量$\vec{u}=(x,y)$可写为$x\vec{e}_1+y\vec{e}_2$. 此外,我们仍沿袭高中对向量长度的定义,即$\lvert\vec{u}\rvert=\sqrt{x^2+y^2}$.

在\autoref{ex:1:Abel 群} 中我们已经验证了$\mathbf{R}^2$上的向量加法满足Abel群的条件,因此我们只需要定义$\mathbf{R}^2$上的乘法使得代数系统$\langle\mathbf{R}^2\backslash\{(0,0)\}:\circ\rangle$也为Abel群. 这一乘法的构造需要满足一些自然的条件,同时也能实现构成Abel群的要求. 事实上,我们有如下定理:
\begin{theorem}\label{thm:1:复数乘法构造}
    平面点集$\mathbf{R}^2$上存在唯一的乘法$\circ$,满足
    \begin{enumerate}
        \item (单位元) $\vec{u}\circ\vec{e}_1=\vec{e}_1\circ\vec{u}=\vec{u},\enspace\forall\vec{u}\in\mathbf{R}^2$;

        \item (长度可乘性) $\lvert\vec{u}\circ\vec{v}\rvert=\lvert\vec{u}\rvert\lvert\vec{v}\rvert$.
    \end{enumerate}
    此乘法满足交换律,且使得$\langle\mathbf{R}^2:+,\circ\rangle$成为域.
\end{theorem}

上述定理中第一个条件是非常自然的,因为在二维平面上,$\{(x,0) \mid x\in\mathbf{R}\}$实际上就是实数轴,因此$\vec{e}_1=(1,0)$相当于实数1,因此作为乘法单位元是非常自然的. 第二条长度可乘则看起来没那么自然,但在接下来的证明中我们将会了解到其意义.

\begin{proof}
    对任意向量$\vec{u}=(a,b)=a\vec{e}_1+b\vec{e}_2,\enspace \vec{v}=(c,d)=c\vec{e}_1+d\vec{e}_2$,我们利用乘法的第一条性质有
    \[\vec{u}\circ\vec{v}=ac\vec{e}_1+(ad+bc)\vec{e}_2+bd\vec{e}_2\circ\vec{e}_2.\]
    由此可见$\vec{u}\circ\vec{v}=\vec{v}\circ\vec{u}$,因此乘法满足交换律. 同时可知,要定义乘法,关键是定义$\vec{e}_2\circ\vec{e}_2$的值.

    记$\vec{e}_2\circ\vec{e}_2=(x,y)$,由长度可乘性知$x^2+y^2=1$,另一方面
    \[(\vec{e}_1+\vec{e}_2)\circ(\vec{e}_1-\vec{e}_2)=\vec{e}_1-\vec{e}_2\circ\vec{e}_2=(1-x,y).\]
    由$|\vec{e}_1+\vec{e}_2|=|\vec{e}_1-\vec{e}_2|=\sqrt{2}$以及长度可乘性可得
    \[4=|(\vec{e}_1+\vec{e}_2)\circ(\vec{e}_1-\vec{e}_2)|^2=(1-x)^2+y^2.\]
    由此求出$x=-1,\enspace y=0$. 这说明
    \[\vec{e}_2\circ\vec{e}_2=-\vec{e}_1.\]
    由此得乘法的定义$\vec{u}\circ\vec{v}=(ac-bd)\vec{e}_1+(ad+bc)\vec{e}_2$,即
    \[(a,b)\circ(c,d)=(ac-bd,ad+bc).\]
    可验证,此乘法以$\vec{e}_1$为单位元,等式$(ac-bd)^2+(ad+bc)^2=(a^2+b^2)(c^2+d^2)$表明乘法满足长度可乘性. 上述证明亦表明乘法唯一(只能这么构造$\vec{e}_2\circ\vec{e}_2$).

    接下来我们很容易验证$\langle\mathbf{R}^2:+,\circ\rangle$满足域的定义,我们留作习题供读者自行验证.
\end{proof}

在\autoref{thm:1:复数乘法构造} 赋予的乘法下,$\langle\mathbf{R}^2:+,\circ\rangle$称为复数域$\mathbf{C}$. 我们自然地将$\vec{e}_1$合理简记为1,同时$\vec{e}_2$简记为$\i$,因为此时$(a,b)$即为$a+b\i$,并且利用$\vec{e}_2^2=-\vec{e}_1$可知$\i^2=-1$,这与我们熟知的虚数单位的定义是统一的. 这一代数表示引入的相关概念,如实部、虚部、纯虚数,以及复数四则运算法则在高中阶段大家都已熟知,在此不再赘述.

非零复数$z=x+y\i$也可写为极坐标的形式,即$z=|z|(\cos\theta+\i\sin\theta)$,其中$|z|=\sqrt{x^2+y^2}$为复数的平面表示的模长,$\theta\in\mathbf{R}$为连接原点与$z$的有向线段与$x$轴正方向的夹角(在相差$2\pi$整数倍的意义下唯一). 我们称$\theta$为复数$z$的辐角. 关于复数的模长我们有经典的三角不等式:
\begin{theorem}
    设$z,w\in\mathbf{C}$,则有$|z+w|\leqslant|z|+|w|$.
\end{theorem}

这一定理的几何意义是非常显然的,我们将$z$和$w$放在平面直角坐标系中观察就可以明白这就是经典三角不等式的复数版本. 等号成立的条件也显而易见,即$z$和$w$要么至少一个为0,要么都非零且$z$和$w$位于从原点出发的同一条射线上. 按照我们给出的定义,这个式子是不言自明的,我们需要证明的反而是一些大家在高中时就已经熟知的结果,例如:

\begin{theorem}
    设$z = (a, b) \in \mathbf{C}$,称 $\overline{z} = (a, -b)$ 为 $z$ 的共轭复数,则 $\overline{z} z = |z|^2$
\end{theorem}

严格的证明如下:

\begin{proof}
    首先验证 $z$ 和 $\overline{z}$ 具备相同的长度,然后由长度可乘性得到。
\end{proof}

这样定义的复数域的所有性质都和高中所学是完全一致的,此后我们默认读者具有足够的基础知识,不在此赘述.

\section{等价关系}

我们时常需要讨论集合中元素之间的关系. 例如直线间的平行、垂直、相交,或是数之间的大于、等于、小于关系.``关系''在我们的讲义中将会多次出现,因此我们很有必要在此形式化定义这一概念,并强调其中一类特定的关系——等价关系.

我们首先从(二元)关系这一概念入手. 实际上,这里的二元关系和日常生活中的关系是紧密相连的,例如将全人类作为谈论的背景集合,那么$(\text{小头爸爸}, \text{大头儿子})$这一有序二元组是符合这一关系的,但$(\text{章鱼哥}, \text{海绵宝宝})$显然不符合. 因此我们可以将父子关系看作笛卡尔积集合$\text{人类}\times\text{人类}$的子集. 更一般化的,集合$A$中的关系可以由$A\times A$的子集
\[\{(a,b) \mid a,b\in A, \enspace a\,R\,b\}\]
来刻画,其中$R$是这个关系本身(实质上是两个元素之间的某种性质),例如之前讨论的父子关系,或是数学中的大于、小于或同余等. 事实上,反过来,由$A\times A$的子集可以确定一个关系,例如我把全世界所有的父子组合放在这个集合中,那么这个集合就定义了人类中的父子关系.
\begin{example}
    以下是一些关系的例子:
    \begin{enumerate}
        \item 设$A=\mathbf{R}$,则$A\times A$的子集
              \[\{(a,b)\in A\times A \mid a^2+b^2=1\}\]
              定义了一个关系$R$,即
              \[a\,R\,b \iff a^2+b^2=1.\]

        \item 设$A=\{1,2,3\}$,则$A\times A$的子集
              \[\{(1,1),(1,2),(1,3),(2,2),(2,3),(3,3)\}\]
              定义了一个关系$R$,即
              \[a\,R\,b \iff a\leqslant b.\]

        \item 设$A$为任意数集,定义在$A$上的函数$f$也是一种关系,集合$A\times A$的子集
              \[B=\{(a,b)\in A\times A \mid b=f(a)\}\]
              刻画了这一关系. 换言之,函数是一种特殊的关系,它要求$\forall a\in A$有且仅有一个元素$b\in A$使得$(a,b)\in B$,其中$B$为上述定义的$A\times A$的子集.

        \item 设$A=\mathbf{Z}$,关系$R$满足$a\,R\,b\iff a\equiv b \pmod n$,即模$n$同余,则$A\times A$的子集
              \[\{(a,b)\in A\times A \mid a\equiv b \pmod n\}\]
              可以刻画这一关系.
    \end{enumerate}
\end{example}

接下来我们要讨论一种特别的关系,即等价关系. 它对关系$R$有一定的规定:
\begin{definition}\label{def:1:等价关系}
    集合$A$中关系若满足以下条件:
    \begin{itemize}
        \item (自反性) $\forall a\in A, \enspace a\,R\,a$;

        \item (对称性) 若$a\,R\,b$,则$b\,R\,a$;

        \item (传递性) 若$a\,R\,b$,$b\,R\,c$,则$a\,R\,c$,
    \end{itemize}
    则称$R$为$A$的一个等价关系. 进一步地,若$R$是集合$A$的一个等价关系且$a,b\in A$,若$a\,R\,b$,则称$a$,$b$关于$R$是等价的,并把$A$中所有与$a$等价的元素集合
    \[\overline{a}=\{b\in A \mid b\,R\,a\}\]
    称为$a$所在的等价类,$a$称为这个等价类的代表元素,并记$\{\overline{a}\}$为所有等价类为元素构成的集族.
\end{definition}

我们可能需要一个例子来理解这些概念. 我们不难证明,初等数论中的同余关系是一种等价关系,以模3同余为例,我们取整体集合为正整数集合,对于3,它的等价类就是所有和3模3同余的元素集合,即所有3的倍数. 同理,对于1,它所在的等价类就是模3余1的全体正整数,2所在的等价类是全体模3余2的正整数. 除此之外,我们还发现一个特点,即这三个等价类将原集合分成了三个无交集的子集
\begin{gather*}
    \overline{0}=\{3k\mid k\in\mathbf{Z}\} \\
    \overline{1}=\{3k+1\mid k\in\mathbf{Z}\} \\
    \overline{2}=\{3k+2\mid k\in\mathbf{Z}\}
\end{gather*}
且它们的并集就是原集合,即这三个等价类构成了原集合的一个\term{分划}\index{fenhua@分划 (partition)}(即分为并为原集合且互不相交的子集). 这一结论对所有等价类都成立,是很直观的结论:
\begin{theorem}\label{thm:1:等价类的性质}
    设$R$是集合$A$的等价关系,则由所有不同的等价类构成的子集族$\{\overline{a}\}$是$A$的分划. 反之,我们也可以基于分划在$A$中定义等价关系.
\end{theorem}

证明这一定理需要一个引理:
\begin{lemma}
    设$R$是集合$A$的等价关系,$a,b\in A$,则$\overline{a}=\overline{b}\iff a\,R\,b$.
\end{lemma}
这一引理说明$a$和$b$等价当且仅当它们等价类相同,或者说在同一个等价类中,相信根据等价类的定义这是很显然的结论.

这一引理还有一个重要的推论:
\begin{corollary}\label{cor:1:等价类的性质}
    设$R$是集合$A$的等价关系,$a,b\in A$,则下面二者必成立其一:
    \begin{enumerate}
        \item $\overline{a}\cap\overline{b}=\varnothing$;

        \item $\overline{a}=\overline{b}$.
    \end{enumerate}
\end{corollary}
即等价类要么相等要么不相交,这一结论也是非常自然的,且由这一结论我们很容易证明\autoref{thm:1:等价类的性质}. 如果对这些定理的证明细节感兴趣的读者可以参看教材第5页的定理1.1和1.2.

进一步此我们可以定义商集的概念:
\begin{definition}[商集] \index{shangji@商集 (quotient set)}
    设$R$是集合$A$的等价关系,以关于$R$的等价类为元素的集合(实际上是集合构成的集合,又称集族)$\{\overline{a}\}$称为$A$对$R$的\term{商集},记为$A/R$. 由
    \[\pi(a) = \overline{a}, \enspace \forall a\in A\]
    定义的$A$到$A/R$上的映射$\pi$称为$A$到$A/R$上的自然映射.
\end{definition}
我们可以看到,自然映射$\pi$将$A$中的元素$a$映到自己所在的等价类$\overline{a}$. 基于上述定义,我们可以完成在基本代数结构一节中遗留的一个问题:我们能否定义非数域的域?答案是肯定的,如果同学们对密码学感兴趣的应当听闻过有限域这一概念,接下来我们将通过简单的例子来说明这一概念.

\begin{example}\label{ex:1:有限域}
    设$Z_n$是$\mathbf{Z}$关于模$n$同余关系$R$的商集,即
    \[Z_n=\mathbf{Z}/R=\{\overline{0},\overline{1},\ldots,\overline{n-1}\}.\]
    即$Z_n$中的元素是$n$个集合,其中第$i$个集合是全体模$n$余$i-1\enspace(i=1,2,\ldots,n)$的整数构成的集合.

    在$Z_n$上定义加法$\oplus$为$\overline{a}\oplus\overline{b}=\overline{a+b}$. 这里$a$和$b$并不一定要在$0$到$n-1$之间,因为事实上$\overline{a}=\overline{kn+a}\enspace(k\in\mathbf{Z})$. 我们只需对$a$,$b$以及$a+b$对$n$取模就可以将它们控制在$0$到$n-1$之间且表示的是同一个运算表达式(因为本质上只是我们选取了同一个等价类的不同代表元素进行计算,例如$n=3$时,$\overline{1}+\overline{2}=\overline{4}+\overline{8}=\overline{0}$),这被称为是等价关系与加法的相容性. 任何时候如果我们要在商集上定义某个由原来的集合上的东西继承下来的东西,对相容性的检验都是不可或缺的.

    接下来我们需要定义乘法$\circ$,同样是一个自然的定义,即$\overline{a}\circ\overline{b}=\overline{ab}$. 读者不妨自行验证它与等价关系的相容性. 我们很容易验证$\forall n\in\mathbf{Z}$且$n\geqslant 2$,$\langle Z_n:\oplus,\circ\rangle$构成一个含幺交换环. 教材43页例8和45页例3中有详细的证明,因为较为显然此处从略. 我们要讨论的是何时$\langle Z_n:\oplus,\circ\rangle$构成域,由此我们便构造了一个非数域的域,并且元素个数是有限的.

    我们这里可以给出结论:$\langle Z_n:\oplus,\circ\rangle$是域当且仅当$n$是素数. 这一结论的证明需要一些数论的知识,我们放在习题中供感兴趣的同学证明.
\end{example}

在此,我们需要再次强调一下相容性的重要性:因为相容性保证了这个二元运算在等价类上是良定义的. 所谓良定义,即这个东西就其自身不产生矛盾,这在形而上学家的论述中最常出现的例子是``圆的方''这个语汇. 而在这里,我们遇到的问题是,当我们称它是一个运算时,它究竟是不是运算?因为如果它将两个相同的东西映到不同的东西,那它当然不是一个映射,这和它的定义矛盾,我们就称这种东西是不良定义的. 从此往后,我们希望读者在每一处构造定义之后,都注意检查其是否是良定义的. 这个问题在商集中特别典型,因为这是这个系列中最容易出现从某个集合``诱导''(即自然地投射到商结构)定义出新东西的情形,它也有相当大的可能性(尽管本讲义中显然避免了这种情况)是不良定义的.

\section{高斯消元法}

高斯消元法是线性代数中最常用的算法之一,是之后解决大量问题所需要掌握的基本方法,同时也是考试中一定会考察的内容,无论是单独一个大题考察,还是嵌入在其它问题中. 教材中相关概念和算法的介绍已经非常详细,这里只作总结.

注意考试中单独考察解方程时,时间充足时建议将过程写完整,标明初等行变换的具体步骤,并且至少写出阶梯矩阵和行简化阶梯矩阵. 除此之外,需要保证计算中尽量减少错误,时间充足可以解完方程后将答案代入进行检查.

需要强调的是,不要认为本节内容很简单就放过了,实际上如果长期不计算高斯消元法很容易陷入眼高手低的窘境,因此希望各位同学熟悉高斯消元法的基本步骤并熟练应用.

一般的,对于一个由$m$个方程组成的$n$元(即变量数为$n$)线性方程组
\[ \begin{cases} \begin{aligned}
            a_{11}x_1+a_{12}x_2+\cdots+a_{1n}x_n & = b_1           \\
            a_{21}x_1+a_{22}x_2+\cdots+a_{2n}x_n & = b_2           \\
                                                 & \vdotswithin{=} \\
            a_{m1}x_1+a_{m2}x_2+\cdots+a_{mn}x_n & = b_m
        \end{aligned} \end{cases} \]
将其系数排列成矩阵
\[\begin{pmatrix}
        a_{11} & a_{12} & \cdots & a_{1n} \\
        a_{21} & a_{22} & \cdots & a_{2n} \\
        \vdots & \vdots & \ddots & \vdots \\
        a_{m1} & a_{m2} & \cdots & a_{mn}
    \end{pmatrix}\]
且记$\vec{b}=(b_1,b_2,\ldots,b_m)^\mathrm{T}$,若$\vec{b}=\vec{0}$则称此方程为齐次线性方程组,否则为非齐次线性方程组. 再将$n$个未知量记为$n$元列向量$X=(x_1,x_2,\ldots,x_n)^\mathrm{T}$,我们便可以把方程组简记为$AX=\vec{b}$.

令$\vec{\beta}_i=(a_{1i},a_{2i},\ldots,a_{mi})^\mathrm{T}$,即方程组系数矩阵的某一列,则方程组还可以记为$x_1\vec{\beta}_1+x_2\vec{\beta}_2+\cdots+x_n\vec{\beta}_n=\vec{b}$,这一形式将在之后多次见到.

在以上的记号下,我们可以将解线性方程组的过程转化为矩阵的初等行变换. 高斯消元法的一般步骤如下:
\begin{center}
    线性方程组$\overset{1}{\longrightarrow}$增广矩阵$\overset{2}{\longrightarrow}$阶梯矩阵$\overset{3}{\longrightarrow}$(行)简化阶梯矩阵$\overset{4}{\longrightarrow}$解
\end{center}

\begin{enumerate}[label=步骤\arabic*~]
    \item 只需要将线性方程组转化为$(A, \vec{b})$的形式,得到左$n$列为系数矩阵,最右列为列向量$\vec{b}$的$n+1$列的增广矩阵;

    \item 通过初等行变换后,得到教材P34(1--13)的形式的矩阵——阶梯矩阵. 阶梯矩阵系数全零行在最下方,并且非零行中,在下方的行的第一个非零元素一定在上方行的右侧(每行第一个非零元素称主元素);

    \item 将主元素化1后将主元素所在列的其他元素均通过初等行变换化为0即可;

    \item \label{item:1:解方程组}
          我们分三种情况讨论:
          \begin{enumerate}
              \item 有唯一解:没有全零行,最后一个主元素的行号与系数矩阵的列数相等,且行简化阶梯矩阵对角线上全为1,其余元素均为0,此时可以直接写出解;

              \item 无解:出现矛盾方程,即系数为0的行的行末元素不为0,此时直接写无解即可;

              \item 有无穷解:非上述情况. 此时设出自由未知量将其令为$k_1,k_2,\ldots$,然后代入增广矩阵对应的方程组即可. 注意选取自由未知量时,选取没有主元素出现的列对应的未知量会与标准答案更贴近(如教材P33选取$x_2,x_5$),当然选择其他作为自由未知量也可以.
          \end{enumerate}
\end{enumerate}

从高斯消元法开始,我们正式进入线性代数的学习. 实际上,上述 \ref*{item:1:解方程组} 中关于方程组解的情况的讨论我们是浮于表面,是基于算法最后得到的矩阵的形式进行的讨论,但事实上,这背后蕴含着更深刻的意义. 我们将会在接下来的十余个章节中讲述线性代数中的核心概念,并在\hyperref[chap:朝花夕拾]{朝花夕拾}中回过头来重新审视线性方程组解的问题. 相信在那时,经历十余章各式抽象概念和运算技巧的洗礼后再来回味这一问题的你,定有``守得云开见月明''之感,对线性代数的理解也会更深一层.

\vspace{2ex}
\centerline{\heiti \Large 内容总结}

本讲为了后续章节讲述方便引入了一些基本概念和算法. 尽管这是一门面向理工科应用的数学课,但我们仍然希望以最自然的方式引入概念,而非填鸭式地轰炸,因此我们首先从大家最熟悉的实数集合开始,讨论在集合上定义运算的方法:我们逐步加强条件,引入了三种基本的代数结构——群、环和域,并且给出了一些例子,并简单讨论了定义代数系统的意义. 事实上,下一讲开始要介绍的线性空间也是一种特殊的代数结构,因此首先引入代数结构对于我们自然展开接下来的讨论有很大的帮助,不至于让读者觉得非常突兀.

接下来我们也从域的定义入手,构造了$\mathbf{R}^2$上的乘法运算使其构成了一个域,并且我们发现这里的定义与高中学习的复数乘法是完全一致的. 之后我们引入了等价关系的概念,这一概念在后续的讲义中将会多次出现,其重要意义就是将一个集合划分成了几个等价的区域. 最后我们讨论了高斯消元法的一般步骤,这是我们接下来解决线性空间中各类问题绕不开的算法.

\vspace{2ex}
\centerline{\heiti \Large 习题}

\vspace{2ex}
{\kaishu 我这门课很简单,只有简单的加减乘除四则运算,甚至除法都不太需要.}
\begin{flushright}
    \kaishu
    ——浙江大学数学科学学院教授吴志祥
\end{flushright}

\centerline{\heiti A组}
\begin{enumerate}
    \item 完善\autoref{thm:1:复数乘法构造} 中的证明,即证明$\mathbf{R}^2$在平面向量加法和如\autoref*{thm:1:复数乘法构造} 定义的乘法下构成一个域.

    \item 完成教材48页第13题.

    \item 求齐次线性方程组$\begin{cases}
                  x_1+x_2+x_3+4x_4-3x_5=0   \\
                  2x_1+x_2+3x_3+5x_4-5x_5=0 \\
                  x_1-x_2+3x_3-2x_4-x_5=0   \\
                  3x_1+x_2+5x_3+6x_4-7x_5=0
              \end{cases}$的通解.

    \item 求非齐次线性方程组$\begin{cases}
                  x_1-x_2+2x_3-2x_4+3x_5=1     \\
                  2x_1-x_2+5x_3-9x_4+8x_5=-1   \\
                  3x_1-2x_2+7x_3-11x_4+11x_5=0 \\
                  x_1-x_2+x_3-x_4+3x_5=3
              \end{cases}$的通解.

    \item 求解线性方程组$\begin{cases}
                  x_1+x_2+x_3=1   \\
                  x_1+2x_2-5x_3=2 \\
                  2x_1+3x_2-4x_3=5
              \end{cases}$.
\end{enumerate}

\centerline{\heiti B组}
\begin{enumerate}
    \item 设$A$是一个Abel群,$A$的运算是加法. 在$A$中定义乘法运算为$ab=0,\enspace\forall a,b\in A$. 证明:$A$为一个环,而且它的加法单位元与乘法单位元相同(我们称这种环为\term{零环}\index{huan!ling@零环 (zero ring)}).

    \item 证明:若集合$A$上的二元关系$R$满足
          \begin{enumerate}
              \item $a\,R\,a,\enspace\forall a\in A$;

              \item $\forall a,b,c\in A$,若$a\,R\,b$且$a\,R\,c$,则$b\,R\,c$.
          \end{enumerate}
          则$R$为$A$上的等价关系.
\end{enumerate}

\centerline{\heiti C组}
\begin{enumerate}
    \item 证明:\autoref{ex:1:有限域} 中定义的$\langle Z_n:\oplus,\circ\rangle$是域当且仅当$n$是素数.
          (提示:无论$n$是否为素数,$n\in\mathbf{Z}$且$n\geqslant 2$时$\langle Z_n:\oplus,\circ\rangle$为交换环,因此是否为素数将决定这一结构中每个元素是否有逆元. 在初等数论中,我们熟知的裴蜀定理可以解决这一问题.)

    \item 本讲我们构造了$\mathbf{R}^2$上的乘法,从而定义了复数域的乘法运算. 本题希望探讨的是:$\mathbf{R}^3$无法构造出乘法使其成为一个域. 在高中的学习中我们知道,$\mathbf{R}^3$空间向量的一组基底为$\{\vec{e}_1=(1,0,0),\vec{e}_2=(0,1,0),\vec{e}_3=(0,0,1)\}$. 证明:$\mathbf{R}^3$没有乘法同时满足以下性质:
          \begin{enumerate}
              \item (单位元) $\forall \vec{u}\in\mathbf{R}^3,\enspace\vec{e}_1\cdot \vec{u}=\vec{u}\cdot \vec{e}_1$;

              \item (交换性) $\forall \vec{u},\vec{v}\in\mathbf{R}^3,\enspace\vec{u}\cdot \vec{v}=\vec{v}\cdot \vec{u}$;

              \item (长度可乘性) $\forall \vec{u},\vec{v}\in\mathbf{R}^3,\enspace|\vec{u}\cdot\vec{v}|=|\vec{u}||\vec{v}|$.
          \end{enumerate}
          按照如下思路给出详细证明过程:采用反证法. 假设乘法存在,则
          \begin{enumerate}
              \item 通过计算$(\vec{e}_1+\vec{e}_2)\cdot(\vec{e}_1-\vec{e}_2)$,$(\vec{e}_1+\vec{e}_3)\cdot(\vec{e}_1-\vec{e}_3)$,证明\[\vec{e}_2\cdot\vec{e}_2=\vec{e}_3\cdot\vec{e}_3=-\vec{e}_1.\]

              \item 证明$(\vec{e}_2+\vec{e}_3)\cdot(\vec{e}_2-\vec{e}_3)=0$得出矛盾.
          \end{enumerate}

    \item 尝试构造一个 $4$ 元的域 $\mathbf{F}_4$.
          (提示:我们在上面的第一题中已经给出了 $2$ 元的域 $\mathbf{Z}_2$,而且我们也知道 $\mathbf{Z}_4$ 不构成一个域,因此,我们考虑怎么把两个二元的域拼起来,可以尝试笛卡尔积的方式,并在其中定义二元运算来满足所需的性质.进一步的两个事实是,对于所有的素数 $p$,都可以用类似的方法构造 $p^n$ 元的域;以及不是任意两个域的笛卡尔积都是域,这个例子已经被上面的第二题所说明.)
\end{enumerate}

\begingroup
\SetLUChapterNumberingStyle{0}
\def\theHchapter{\arabic{chapter}ε}

\chapter{预备思想}

在第一讲中我们介绍了一些预备知识,但是在正式开始我们的学习旅程前,我希望在这里先讨论一些和知识本身关系不大的话题,也就是一些学习这门课的一些数学思想的准备,目的主要是给刚刚进入大学的同学一个思维上升的台阶,以便更好地接受接下来抽象的内容.

\section{数学证明初探}

我想我们很有必要在入门课程的开头简要介绍一些关于数学证明的问题. 因为我们高中阶段很多时候的证明不过是计算性的验证——例如证明圆锥曲线的一些结论,或是导数题的证明,很多时候都只是通过计算验证这一结论是否正确,而非从定义出发对命题进行``证明''. 我们举一个简单的例子:

\begin{example}
    \begin{enumerate}
        \item 已知椭圆$C:\dfrac{x^2}{4}+\dfrac{y^2}{3}=1$,若直线$l:y=kx+m$与椭圆$C$相交于$A,B$两点($A,B$不是左右顶点),且以$AB$为直径的圆过椭圆$C$的右顶点. 求证:直线$l$过定点.

        \item 设$A_i\enspace(i=1,2,\ldots,n)$是$X$的子集. 证明:$\overline{\displaystyle\bigcup_{i=1}^nA_i}=\displaystyle\bigcap_{i=1}^n\overline{A_i}$(其中集合上一横代表对全集$X$的补集).
    \end{enumerate}
\end{example}

第一题一定是各位再熟悉不过的经典高中圆锥曲线习题了. 这一证明过程我们只需要联立方程,结合已知条件不断计算即可得出结论,事实上虽是证明题确更偏向于计算性验证,与第二题的风格相去甚远. 我们来分析并书写一下第二题的证明:

\begin{proof}
    要证明两个集合相等,我们必须回忆两个集合相等的定义:即我们需要证明两个集合互相包含. 而要证明集合的包含,例如$A\subseteq B$,我们应当利用包含的定义,证明$\forall x\in A$,都有$x\in B$.

    接下来我们开始证明. 根据上面的分析,我们需要证明等号两边互相包含,因此分成如下两个部分证明:
    \begin{enumerate}
        \item 根据补集的定义,$\forall x\in \overline{\displaystyle\bigcup_{i=1}^nA_i},\enspace x\notin \displaystyle\bigcup_{i=1}^nA_i$,因此根据并集的定义(属于并集表明至少属于参与并集的某一个,因此根据逆否命题不属于并集表明一定不属于任何一个参与并集的集合),即$x\notin A_i,\enspace i=1,2,\ldots,n$. 所以,根据补集的定义,$x\in \overline{A_i},\enspace i=1,2,\ldots,n$. 根据交集的定义(属于交集表示属于参与交集的每一个集合),$x\in \displaystyle\bigcap_{i=1}^n\overline{A_i}$,因此根据集合包含的定义,$\overline{\displaystyle\bigcup_{i=1}^nA_i}\subseteq \displaystyle\bigcap_{i=1}^n\overline{A_i}$.

        \item 另一方面,$\forall x\in \displaystyle\bigcap_{i=1}^n\overline{A_i}$,根据交集的定义,$x\in \overline{A_i},\enspace i=1,2,\ldots,n$. 因此根据补集的定义,$x\notin A_i,\enspace i=1,2,\ldots,n$. 因此根据并集的定义,$x\notin \displaystyle\bigcup_{i=1}^nA_i$,因此根据补集的定义,$x\in \overline{\displaystyle\bigcup_{i=1}^nA_i}$,因此根据集合包含的定义,$\displaystyle\bigcap_{i=1}^n\overline{A_i}\subseteq \overline{\displaystyle\bigcup_{i=1}^nA_i}$.

              综上,根据集合相等的定义有$\overline{\displaystyle\bigcup_{i=1}^nA_i}=\displaystyle\bigcap_{i=1}^n\overline{A_i}$.
    \end{enumerate}
\end{proof}

我们可以看到,在整个证明过程中,我们的证明出发点就是集合相等、包含的定义,后面的证明中也在不断重复使用集合的交、并和补等运算的定义,这与之前的计算性验证风格完全不同. 事实上,我们未来的很多证明都要求我们从定义出发,而非计算性验证,因此我们希望在本节相对较为完整地展示我们经常会遇到的一些真正的证明问题以及证明策略,因此我们首先需要介绍一些简单逻辑. 我们默认读者应当在高中阶段学习过基本命题的概念以及基本逻辑运算:与、或、非以及任意、存在以及真值表的概念等,这些基本符号我们不在此赘述,如果读者不熟悉可以参考教材1.6节开头.

我们从高中没有介绍的两个逻辑运算符号开始. 其一为$\to$,读作``蕴含''. $p\to q$的真值表如下. 这表明当$p$为假命题时,无论$q$如何,$p\to q$都为真命题,而当$p$为真命题时,$p\to q$的真值与$q$相同,即必须$q$为真时才为真.

\begin{center}
    \begin{minipage}[c]{0.45\textwidth}
        \centering
        \begin{tabular}{|c|c|c|}
            \hline
            \diagbox{$p$}{$q$} & 0 & 1 \\
            \hline
            0                  & 1 & 1 \\ \hline
            1                  & 0 & 1 \\
            \hline
        \end{tabular}
        \captionof{table}{$p\to q$}
    \end{minipage}
    \begin{minipage}[c]{0.45\textwidth}
        \centering
        \begin{tabular}{|c|c|c|}
            \hline
            \diagbox{$p$}{$q$} & 0 & 1 \\
            \hline
            0                  & 1 & 0 \\ \hline
            1                  & 0 & 1 \\
            \hline
        \end{tabular}
        \captionof{table}{$p\leftrightarrow q$}
    \end{minipage}
\end{center}

事实上这很容引起人们的困惑,因为我们直觉上会将$p\to q$认为是$p$可以推导出$q$,那么为什么$p$是错误的时候$p\to q$一定是正确的呢?我们可以作如下论断为例:如果猪会飞,那么你的线性代数会考59分. 这一论断我们会认为是正确的,因为事实上猪并不会飞,因此我们不必再关心你线性代数考不考59分——从谬误出发,你想怎么干就怎么干,因此我们有$p\to q$当$p$为假命题时,$p\to q$一定为真命题. 而$p$为真命题时,我们就会关心$q$的真值,比如我说``如果明天太阳东升西落,那么你的线性代数会考59分'',你一定会瞪大你的眼睛猛摇你的头说``不可能,不是这样的''.

下面一个逻辑运算符是$\leftrightarrow$,读作``等价'',$p\leftrightarrow q$的真值表如上所示,我们不难看出,$p\leftrightarrow q$为真要求$p$与$q$的真值相同,这也符合``等价''二字的直觉. 当然,$p\leftrightarrow q$事实上就是$(p\rightarrow q)\land(q\rightarrow p)$,即$p$蕴含$q$且$q$蕴含$p$,这也是符合直觉的,因为一般的等价也需要两边能互相推导出.

当然我们必须强调逻辑上的蕴含和等价与数学问题中的推出、等价之间的区别. 对于推出$\Rightarrow$,我们需要依靠公理、定理等进行证明,而蕴含$\to$不一定需要两边有什么实际的数学联系;对于等价,逻辑上$p$和$q$的等价$\leftrightarrow$只要求两边真值相同,而数学上的等价$\Leftrightarrow$则要求我们能通过数学的公理、定理从$p$推出$q$,从$q$推出$p$. 前者被称为语义层面上的推理(真值表符合),而后者是语法层面上的推理(能够通过一套形式规则给出). 比方说,以下规则在真值表意义上成立:$((p \rightarrow q) \land p) \rightarrow q$,这可以通过检查真值表的方式表明,而所谓的肯定前件(modus ponens)规则就是``如果 $p \rightarrow q$ 成立且 $p$ 成立,则 $q$ 成立'',这是我们的一条推理规则,它是语法层面的推理,因为它检查两个命题是否具备某个结构,然后通过它们的形式结构给出另一个成立的命题.

我们对推理系统(即这套语法层面检查的形式规则)的要求往往是它和语义上的结果一致,这是一个推理系统可靠性和完备性的问题,不在本书的讨论范围内,但可参见图灵系列丛书的有关离散数学的书目\href{https://github.com/FrightenedFoxCN/Discrete-Mathematics-Made-Concrete}{《Discrete\ Mathematics\ Made\ Concrete》}(下简称 DMMC).

因为我们通常采用的推理系统中,这二者是基本上等价的,我们就可以将它作为一些证明方法的基本依据. 接下来我们将考察几类最常见的证明方法与问题,并逐一进行分析. 可能有些内容比较枯燥,读者也可以先留个印象,或记住一些经典的例子,将来遇到此类问题再回过头来看.

\begin{enumerate}
    \item 逆否命题、反证与否证:高中阶段我们都学习过逆否命题与原命题的等价性,即$p\to q$和$\lnot q\to\lnot p$等价. 事实上严谨而言,我们也可以用真值表验证$(p\to q)\leftrightarrow(\lnot q\to\lnot p)$一定是真命题,因此很多时候我们证明$p\to q$,正向证明有困难时,可以考虑$\lnot q\to\lnot p$,即导出了与条件的矛盾,这也是反证法的理论基础. 当然这只是反证法的一个情况,因为反证法不一定需要证明逆否命题,我只要假设$\lnot q$能够与任意的数学定理、公理等矛盾都可以,不一定需要得出$\lnot p$,如下面这一经典的例子:

          \begin{example}
              证明:$\sqrt{2}$不是有理数.
          \end{example}
          \begin{proof}
              假设$\sqrt{2}$是有理数,即$\sqrt{2}=\dfrac{a}{b}$,其中$a,b$互素,$b\neq 0$,则平方后得到$a^2=2b^2$,因此$a^2$是偶数,因此$a$是偶数,设$a=2c$,则$(2c)^2=2b^2$,即$2c^2=b^2$,因此$b^2$是偶数,因此$b$是偶数,因此$a,b$不互素,矛盾,因此假设不成立.
          \end{proof}

          我们发现,这一结论甚至没有前提$p$,因此我们使用反证法并非是证明了逆否命题.

          最后我们讨论一个题外话,即事实上上面的证明严谨而言不应该称作反证法,而应称作否证法. 事实上如果在中学,我们一定会倾向于将上面这一证明思想称为反证法——事实上所有假设条件成立然后推出矛盾的方法我们曾经都称之为反证法. 读者可能会产生疑问,仿佛我们只是在做一些文字游戏,但实际上,反证和否证是有一定区别的(注意区别不在于是否与逆否命题相关,而是更深层次的),简单而言反证法需要使用到排中律(即$p$和$\lnot p$必有一个为真),但排中律是否一定在逻辑系统中必要未有定论——这与直觉不符,但目前我们只要接受这一点,或者忘记它们,将这种风格的证明都成为反证法. 如果读者希望深入理解反证和否证的区别,可以参考 DMMC,其中会较为仔细地讨论二者的区别,而在本书中我们不再区分反证和否证,统称反证.

    \item 数学归纳法:我们将在本专题后半部分讲解公理化时,跟随皮亚诺公理系统的引入一起介绍数学归纳法.

    \item 任意性证明:有时候我们会遇到这样的证明问题,要求我们证明对于任意的满足某一条件的元素,都有某一命题成立. 这一问题十分平凡,我们只需``任取''一个满足条件的元素,然后证明它满足命题即可,其关键就在于我们选取元素是任意的. 例如我们证明整数集合关于加法运算构成群,我们验证运算封闭性、逆元存在性都是任意性证明.

    \item 存在性证明:存在性证明相对于任意性较为复杂,因为我们通常有两种策略,我们先看第一种最直观的例子:
          \begin{example}
              证明:素数有无穷多个.
          \end{example}
          \begin{proof}
              反证法,我们假设素数只有有限个,由小到大记为$p_1,\cdots,p_n$(则$p_n$是最大素数). 事实上我们只需证明存在比$p_n$大的素数即可,这就将原问题转化为了``存在性''的问题.

              事实上,我们构造$A=p_1\cdots p_n+1$,则$p_1,\cdots,p_n$都不可能是$A$的因数,否则若$p_i$能整除$A$,所得的商为$p_1\cdots p_{i-1}p_{i+1}\cdots p_n+\dfrac{1}{p_i}$不是整数. 因此$A$一定是素数,故存在比$p_n$大的素数,与我们假设的$p_n$是最大素数矛盾. 故素数有无穷多个.
          \end{proof}

          在上面的证明中,我们将问题转化为了证明``存在比$p_n$大的素数''这一问题,我们的证明方法就是构造了出一个更大的素数$A$,非常的直接,因为只要构造出了那么一定就是``存在''的. 但事实上我们还有另一种证明存在性的方法,不一定要构造出对应的元素,如下例:
          \begin{example*}
              证明:存在无理数 $x, y$ 使得 $x^y$ 为有理数.
          \end{example*}
          \begin{proof}
              假定 $\sqrt{2}^{\sqrt{2}}$ 为有理数,则原命题得证;假定其为无理数,则 $\left(\sqrt{2}^{\sqrt{2}}\right)^{\sqrt{2}}$ 为有理数,原命题仍得证.
          \end{proof}

          我们发现,这样的证明并没有实际构造出 $x, y$ 到底是什么. 这样的证明是非构造性的. 题外话是,这样的证明也依赖于排中律,因为我们的前提是``$\sqrt{2}^{\sqrt{2}}$ 要么是有理数,要么是无理数''这个命题一定是真的. 当然读者学习过数学分析或者微积分之后,应当熟悉单调有界数列有极限这一定理,事实上它经常被用来证明一个数列极限存在,但极限值是多少我们并不需要求出,因此也是非构造性证明的一个手段. 比方说下面的例子:

          \begin{example*}
              证明:欧拉常数存在,或者说数列
              \[a_n=1+\dfrac{1}{2}+\cdots+\dfrac{1}{n}-\ln n\]
              有极限(极限值记为$\gamma$,称之为欧拉常数).
          \end{example*}
          \begin{proof}
              高中熟知对数不等式
              \[\frac{x}{1+x}\leqslant\ln(1+x)\leqslant x,\enspace x>-1,\]
              这一不等式使用求导的方法即可证明,等号成立当且仅当$x=0$,因此我们有
              令$x=\dfrac{1}{n}$,则
              \[\frac{1}{n+1}<\ln\left(1+\frac{1}{n}\right)<\frac{1}{n},\]
              基于此,我们有
              \[a_{n+1}-a_n=\frac{1}{n+1}-\ln\left(1+\frac{1}{n}\right)<0,\]
              因此数列$\{a_n\}$单调递减,又因为
              \begin{align*}
                  x_n&=\sum\limits_{k=1}^n\frac{1}{k}-\ln(\frac{n}{n-1}\cdot \frac{n-1}{n-2}\cdots\frac{2}{1})\\
                  &=\sum\limits_{k=1}^n\frac{1}{k}-\sum\limits_{k=1}^{n-1}\ln(1+\frac{1}{k})\\
                  &=\sum\limits_{k=1}^n\left(\frac{1}{k}-\ln(1+\frac{1}{k})\right)+\frac{1}{n} \\
                  &>\frac{1}{n}>0,
              \end{align*}
              因此数列$\{a_n\}$有下界,因此数列$\{a_n\}$收敛.
          \end{proof}

    \item 存在与任意:或许这一问题应该交给数学分析或是微积分老师来讲解,因为最简单的例子就是数列极限的定义:
          \begin{definition}
              设数列$\{a_n\}$,若存在常数$A$,对于任意给定的正数$\varepsilon$,总存在正整数$N$,使得当$n>N$时,有$|a_n-A|<\varepsilon$,则称数列$\{a_n\}$收敛于$A$,否则称数列$\{a_n\}$发散.
          \end{definition}

          相信你的老师或者教材一定介绍过这一命题的逆否命题,这里我们不再赘述. 这一问题的关键在于厘清含有存在、任意的命题如何正确地取否,事实上中学阶段我们就应该拥有这样的基础.

    \item 合取式证明:即证明若$p$成立,则$q$和$r$都成立. 事实上这一类问题没有什么需要特殊说明的,就是将$q$和$r$都单独证明即可.

    \item 析取式证明:即证明若$p$成立,则$q$或$r$二者之一成立. 析取的证明比合取原理复杂,事实上一般而言我们的证明思路就是:假设$q$不成立,证明$r$(即证明$\lnot q\to r$),或者假设$r$不成立,证明$q$(即证明$\lnot r\to q$),这样我们就证明了$q$或$r$二者必有其一成立,因为它们不会同时不成立. 最典型的例子在第一讲也见到了,我们这里重述并给出证明:
          \begin{theorem}
              设$R$是集合$A$上的等价关系,$a,b\in A$,则下面二者必成立其一:
              \begin{enumerate}
                  \item $\overline{a}\cap\overline{b}=\varnothing$;

                  \item $\overline{a}=\overline{b}$.
              \end{enumerate}
          \end{theorem}
          \begin{proof}
              假设$\overline{a}\cap\overline{b}\neq\varnothing$,则存在$x\in\overline{a}\cap\overline{b}$,即$x\,R\,a$且$x\,R\,b$,因此由对称性有$a\,R\,x$,因此由传递性有$aRb$,因此根据等价类的定义可知,二者等价类必然相等,即$\overline{a}=\overline{b}$.
          \end{proof}

          事实上,上面用到的证明思想也涉及了排中律(即$p\lor\lnot p$为真). 因为证明若$p$成立,则$q$和$r$都成立. 那么我们的思想是,要么$q$成立,得证,要么$\lnot q$成立,然后得出$\lnot q\to r$成立,根据蕴含的真值表知此时$r$必然成立,得证. 显然这一过程需要基于要么$q$成立,要么$\lnot q$成立这一排中律.

    \item 唯一性证明:在第一讲\autoref{thm:1:群的单位元逆元唯一} 的证明里我们就已经强调了如何证明唯一性:我们只需假设要证明唯一的东西有两个,然后说明这两个必然相等即可,此处不再赘述.

    \item 等价性证明:熟知要证明两个命题$p$和$q$(在数学上)等价(很多时候也表达为当且仅当),即$p\Leftrightarrow q$,我们需要证明两边,即$p\Rightarrow q$和$q\Rightarrow p$. 但如果看到类似于\autoref{thm:11:可逆等价条件} 的多个命题等价情况该如何证明呢?

          假设我们要证明$n$个命题$p_1,p_2,\ldots,p_n$(在数学上)等价,实际上我们只需找出一条``推出的链条'',即我们可以证明$p_1\Rightarrow p_2,\enspace p_2\Rightarrow p_3,\enspace \cdots,\enspace p_{n-1}\Rightarrow p_n,\enspace p_n\Rightarrow p_1$,这样我们发现任意两个命题之间都可以互相推导. 例如任取$p_i, p_j\enspace (i<j)$,上面的链条告诉我们$p_i\Rightarrow p_{i+1}\Rightarrow\cdots\Rightarrow p_j$,因此$p_i\Rightarrow p_j$,而$p_j\Rightarrow p_{j+1}\Rightarrow p_1\cdots\Rightarrow p_i$,因此$p_j\Rightarrow p_i$,因此$p_i\Leftrightarrow p_j$,这样我们就证明了任意两个命题之间都是等价的,即$p_1\Leftrightarrow p_2\Leftrightarrow\cdots\Leftrightarrow p_n$.

          当然有时候我们也不必如此刻板地从$p_1$推导到$p_n$再推回$p_1$,这些命题的顺序显然都可以打乱,比如四个命题我们证明了$p_2\Rightarrow p_4\Rightarrow p_3\Rightarrow p_1 \Rightarrow p_2$也是完全可以的,我们可以根据哪些证明更加简单来决定证明的顺序. 甚至假如上面四个命题中$p_1$和$p_2$等价性显然,我们也可以直接证明$p_1\Rightarrow p_3\Rightarrow p_4\Rightarrow p_2$,这样我们就证明了$p_1\Leftrightarrow p_2\Leftrightarrow p_3\Leftrightarrow p_4$.

    \item 等号证明:如果我们要证明两个数$a,b$相等,当然可以直接说明,但有时候我们需要``曲线救国'',通过证明$a\geqslant b$和$b\geqslant a$同时成立来说明$a=b$,这种证明方法我们在将来证明秩不等式的时候非常常用.

          对于集合相等也是同理,要证明两个集合$A$和$B$相等,我们经常会通过证明$A\subseteq B$和$B\subseteq A$来说明$A=B$.

    \item 最大最小性证明:我们可以参考\autoref{thm:2:线性扩张构造子空间},事实上证明此类问题可以转化为之前所说的任意性证明:假设我们要证明某个元素或者集合是最大的,那么实际上等价于证明任意其他元素或者其他集合都不大于它或被它包含,最小性同理,此处不再赘述. 另一种证明方法则是反证法:如果存在一个更大的元素,则推出矛盾.
\end{enumerate}

\section{代数结构的引入}

在第一讲中,我们将引入代数结构的起源归于人们希望为一系列相似的结构找到一种统一的描述方式,然后就可以通过研究这种统一结构的特点来研究各个具体的结构. 本节我们希望给出一些实例展示这一思想,也简要介绍一下``抽象代数''这一学科的一些故事.

\subsection{群来源于对称}

本节我们主要介绍与群相关的几何直观,事实上它们与群的抽象定义有很大的关联. 我们回忆群的定义:在集合上定义了一种运算(运算本身满足封闭性),满足结合律、有单位元和逆元. 这一定义看起来非常抽象,但实际上它在描述一种很美丽的性质,我们来看几个例子.

\begin{example}
    平面上的平移群、反射群、旋转群. 在平面直角坐标系中,$(x,y)$为任意点的坐标.
    \begin{enumerate}
        \item 平移群:由平移$\beta_{ab}\enspace(a,b\in\mathbf{R})$构成,定义如下:
              \[\beta_{ab}(x,y)=(x+a,y+b).\]
              不难看出$\beta_{ab}\beta_{cd}=\beta_{a+c,b+d}$,因此平移群满足结合律,单位元为$\beta_{00}$,逆元为$\beta_{-a,-b}$,因此平移群构成一个群.

        \item 反射群:取平面上一直线$l$,对此直线的全体镜像映射构成群,这就是反射群. 为方便讨论,不妨假定这条直线是$y$轴,于是镜面映射$\gamma$的作用如下:
              \[\gamma(x,y)=(-x,y).\]
              不难看出$\gamma^2=e$($e$为幺映射,即$e(x,y)=(x,y)$),因此事实上反射群中只有两个元素,我们也很容易验证它真的构成群.

        \item 旋转群:取平面上一点,对此点的全体旋转映射构成群,这就是旋转群. 为方便讨论,不妨假定这个点是原点,令旋转角为$\theta\enspace(0\leqslant\theta<360^\circ)$的旋转为$\rho_\theta$,则不难看出
              \[\rho_{\theta_1}\rho_{\theta_2}=\rho_{[\theta_1+\theta_2]},\]
              其中$[\theta_1+\theta_2]$表示$\theta_1+\theta_2$的模$360^\circ$的余数. 在此群中,单位元为$\rho_0$,逆元为$\rho_{-\theta}$,因此旋转群构成一个群.
    \end{enumerate}
\end{example}

事实上,上面的例子中在集合上都具有一种所谓的``对称''美感,很巧的是这些对称都能用群描述. 因此实际上群的诞生实际上就是为了公理化(这一名词我们将在下一节中进一步阐释)描述这种对称性.

如果我们将群和对称性与物理学结合,或许会看到更多美丽的结果. 事实上,描述空间和时间变换的连续对称性需要用到Sophus Lie(1842--1899)引入的李群(Lie groups). 而诺特定理告诉我们:任何可微对称性都对应于守恒律. 例如,时间平移对称性蕴含能量守恒,空间平移对称性蕴含动量守恒,空间旋转对称性蕴含角动量守恒. 相信阅读本讲义的读者很多都具有物理学背景或对物理学感兴趣,那么在理论力学中大家将会进一步学习到这些内容,体会到群、对称性在物理学中更深层的应用.

\subsection{五次方程没有求根公式?}
事实上,群的三条公理的直接来历可能不及上一小节中的简单. 实际上,它来源于伽罗瓦(Evariste Galois,1811--1832)对五次方程没有根式解的思考. 或许我们初次见到很难想象五次方程没有求根公式会和群这三条公理产生什么联系,事实上历代很多数学家的尝试也与此无关,他们都在思考根与五次方程系数之间的联系. 然而我们可以回忆单位根在复平面上的分布,我们发现,一元高次方程的根在一定程度上具有较强的对称性——这其中的奥秘似乎可以用群来描述. 伽罗瓦基于这些观察建立了群的概念,证明了多项式可解性等价于多项式根的置换群(伽罗瓦群)具有某种结构(可解性),从而解决了5次方程不可解性这一难题.

也许这里插入一些小历史故事是合适的——毕竟伽罗瓦的生平的确令人惋惜. 细心的读者可能已经在上一段计算过伽罗瓦的去世年龄——年仅21岁. 我们可能无法想象如此困难的问题竟然能有一个天才在21岁之前解决. 事实上,1829年,年仅18岁的伽罗瓦就将他在代数方程解的结果呈交给法国科学院,由著名的大数学家奥古斯丁·路易·柯西(Augustin Louis Cauchy,不用说,柯西-施瓦茨不等式、柯西收敛准则、柯西中值定理都有他的影子)负责审阅,但柯西却将文章连同摘要都弄丢了——事实上19世纪的两个短命数学天才尼尔斯·亨利克·阿贝尔(Niels Henrik Abel,1802-1829,正是阿贝尔群的那位)与伽罗瓦都不约而同地``栽''在柯西手中,而阿贝尔事实上早于伽罗瓦用另一种方式给出了证明(于1824年),因此一元四次以上方程没有求根公式这一定理由阿贝尔命名. 阿贝尔将论文发出后,科学院秘书傅里叶(Jean Baptiste Joseph Fourier, 1768-1830,傅里叶级数就是他的成果)读了论文的引言,然后委托勒让德和柯西负责审查. 柯西把稿件带回家中,究竟放在什么地方,竟记不起来了. 直到两年以后阿贝尔已经去世,失踪的论文原稿才重新找到,而论文的正式发表,则迁延了12年之久.

让我们回到伽罗瓦的生平. 1827年,16岁的伽罗瓦自信满满地投考他理想中的(学术的与政治的)大学:综合工科学校,却因为颟顸无能的主考官而名落孙山,而当伽罗瓦第二次要报考综合工科大学时,他的父亲却因为被人在选举时恶意中伤而自杀. 正直父亲的冤死,影响他考试失败,也导致他的政治观与人生观更趋向极端.

1829年,伽罗瓦进入高等师范学院就读,次年他再次将方程式论的结果写成三篇论文,争取当年科学院的数学大奖,但是文章在送到让·巴普蒂斯·约瑟夫·傅里叶手中后,却因傅里叶过世又遭蒙尘,伽罗瓦只能眼睁睁看着大奖落入阿贝尔与卡尔·雅可比(Carl Jacobi,雅可比行列式的发明人)的手里. 1830年法国七月革命发生,保皇势力出亡,高等师范学院校长将学生锁在高墙内,引起伽罗瓦强烈不满. 12月伽罗瓦在校报上抨击校长的作法,因此被学校退学. 由于强烈支持共和主义,从1831年5月后,伽罗瓦两度因政治原因下狱,他也曾企图自杀. 在监狱中,伽罗瓦仍然顽强地进行数学研究,一面修改他关于方程论的论文及其他数学工作,一面为将要出版的著作撰写序言.

据说1832年3月他在狱中结识了一个医生的女儿并陷入狂恋. 因为这段感情,他陷入一场决斗. 自知必死的伽罗瓦在决斗前夜将他的所有数学成果狂笔疾书记录下来,希望有朝一日自己的研究成果能大白于天下,并时不时在一旁写下``我没有时间''. 第二天他果然在决斗中身亡,时间是1832年5月31日. 这个传说富浪漫主义色彩,为后世史家所质疑.

他的朋友奥古斯特·舍瓦烈(Auguste Chevalier)遵照伽罗瓦的遗愿,将他的数学论文寄给卡尔·弗里德里希·高斯(Carl Friedrich Gauss)与卡尔·雅可比,但是都石沉大海. 要一直到1843年,才由刘维尔(Joseph Liouville)肯定伽罗瓦结果之正确、独创与深邃,并在1846年将它发表.

事实上,伽罗瓦带给我们的财富也不止于此,群与对称性的关联也不止于此. 事实上,高斯为什么能尺规作图作出正十七边形而非其它形状,也正是因为他漂亮地证明高斯的论断:若用尺规作图能作出正$p$边形,$p$为质数的充要条件为$p=2^{2^k}+1$(所以正十七边形可尺规作图). 除此之外,他的理论也直接解决了古代三大尺规作图问题中的两个:三等分任意角不可能、倍立方(求作一正方体的边,使其体积为给定正方体的两倍)不可能(第三个问题是化圆为方:求作一正方形,使其与给定的圆面积相等,这一问题的不可能性由林得曼(Linderman)在1882年证明$\pi$的超越性,即$\pi$不为任何整数系数多次式的根得以确立. 有趣的是,这个结论的证明也间接地与伽罗瓦理论有关).

总而言之,我们发现,代数结构的引入实际上也源于对一些常见概念的抽象,例如群的引入源于对于对称性的抽象. 实际上这种从直觉转化为抽象的规则定义的思想可以称为``公理化'',接下来我们便用完整的一节来介绍这一思想,让读者逐步由易到难接受这一与中学学习相去可能甚远的思想.

\section{公理化思想与布尔巴基学派}

\subsection{公理化思想}

事实上,在上一讲中我们介绍了群、环和域的定义,它们的定义都有一个共同的特点:我们在定义运算的时候,都是给出一些很基本的规则,而接下来的所有性质都只能依靠这些基本规则展开. 这种思想在数学中无疑是至关重要且在将来经常遇见的,实际上从下一讲开始,我们将直面线性空间的八条公理——我想这对于初次遇见的同学而言,大概率是很难直接接受并且理解其中的目的的. 所以我们在这里给出一些基本的例子做一个台阶,以便更好地理解公理化的思想.

首先我们介绍我们熟知的``距离''这一概念的公理化描述:
\begin{definition}
    设$X$是一个非空集合,$d:X\times X\to \mathbf{R}$是一个函数,如果它满足
    \begin{enumerate}
        \item $d(x,y)\geqslant 0$,且$d(x,y)=0$当且仅当$x=y$;

        \item $d(x,y)=d(y,x)$;

        \item $d(x,y)\leqslant d(x,z)+d(z,y)$.
    \end{enumerate}
    则称$d$是$X$上的一个距离.
\end{definition}

这一定义首先表明,距离是一个双变量的函数,它能将集合$X$中两个元素映射到一个实数,这个实数代表了这两个元素的距离. 这一定义并没有给出距离的具体形式,而是给出了距离应该满足的一些性质. 我们审视这三条性质:
\begin{enumerate}
    \item 第一条性质表明两个元素之间距离非负,并且距离为0当且仅当这两个元素相等;

    \item 第二条性质表明距离是对称的,也就是说,$x$到$y$的距离和$y$到$x$的距离是相等的;

    \item 第三条性质表明距离满足三角不等式,也就是说,$x$到$y$的距离不会超过$x$到$z$的距离和$z$到$y$的距离之和.
\end{enumerate}

或许读者会觉得这些性质过于显然. 的确,它们都来源于我们对于现实世界中对于``距离''这一名词的基本认识,但是它所定义的集合$X$不一定是平面或者空间中的两个点——它可以是任意的集合,因此这一定义具有了普遍意义,我们来看一些实际例子:

\begin{example}
    定义全体$n$元实向量$(x_1,x_2,\ldots,x_n),\enspace x_i\in \mathbf{R}$构成的集合上的距离为
    \[d((x_1,x_2,\ldots,x_n),(y_1,y_2,\ldots,y_n))=\sqrt[p]{\sum_{i=1}^n|x_i-y_i|^p},\]
    其中$p \geqslant 1$,则这一定义满足距离的三条公理,其中前两条显然成立,第三条读者可以参考``闵可夫斯基不等式'',具体证明较为繁杂,此处不再赘述,我们放在习题中. 这一距离定义非常经典,我们称其为$\ell_p$范数.
\end{example}

事实上,在上面这一例子中,取定$p=2$实际上就是我们高中学习过的向量距离的定义,因此还是非常直观的. 下面我们给出的例子将涉及更为广泛的情况.

\begin{example}
    将区间$[a,b]$上的所有连续函数构成的集合记为$C[a,b]$,定义$C[a,b]$上的距离为
    \[d(f,g)=\max_{x\in [a,b]}|f(x)-g(x)|.\]
    即我们这里定义了两个连续函数之间的距离为它们在区间$[a,b]$上的最大差值. 读者可以自行验证这一定义满足距离的三条公理.
\end{example}

这一距离不是定义在平面两点之间,而是定义在两个函数之间——这就体现出公理化定义的一个重要意义,它用更抽象的语言描述使得我们可以在更广泛的背景下讨论一个概念. 例如,在数学分析(或微积分)中我们刻画了数列极限和收敛的概念,即若实数轴上点列$\{x_n\}$趋于$x_0$,实际上就是当$n\to\infty$时$|x_n-x_0|\to 0$,即二者距离趋于0,此时称$\{x_n\}$收敛于$x_0$. 而在定义了函数的距离后我们可以讨论$C[a,b]$上的极限和收敛性,即在上面例子中给出的距离定义下,一列函数$\{f_n(x)\}$收敛于$f_0(x)$当且仅当$n\to\infty$时$f_n(x)$与$f_0(x)$在$[a,b]$上的最大差值趋于0. 这实际上就是函数列一致收敛的定义,学习数学分析的同学在将来会了解到这一十分核心的概念.

由此我们在距离的公理化中能体会到如下几点:
\begin{enumerate}
    \item 公理化来源于直觉,或者说一条公理经常对应``常识''中某一条自然的性质;

    \item 公理化使得我们在讨论一个概念时可以不局限于某一特定的背景,而是可以在更广泛的背景下讨论,但前提是我们需要构造出符合公理的背景.
\end{enumerate}

接下来我们将通过其它例子体现公理化的作用:它有助于构建完备的理论体系,让数学中的基础概念有依据可循. 一个很合适的例子是皮亚诺公理(接下来的内容参考《陶哲轩实分析》),它是一个定义自然数的标准方法(当然不是唯一的,也可以用集合的基数定义). 基于皮亚诺公理,我们将看到为什么$1+1=2$是合理的,为什么结合律、分配律、交换律总是正确的. 你会发现,即使一个命题是``显然的'',但在公理体系下它可能不易证明. 除此之外,我希望读者在这里忘记以前所学的一切运算规律,甚至忘记$1,2,3,\ldots$这些数字——因为它们都还没有被定义,我们只从接下来给出的简单公理出发,推出或许很显然的很多自然数的基本性质——这就是公理化的特点,我们只有最简单的抽象公理,但我们可以以此为基础利用自己的证明和抽象思考能力还原显然的世界,但这时整个世界都是严谨、有理可循的而非杂乱无章的.

接下来我们来看皮亚诺是如何基于一些从直觉而来的原理,转化为少数几条公理使其能够构成一个完备的自然数理论体系的. 我们先从一个直觉性的不正式的定义出发,然后看皮亚诺如何将其公理化.

\begin{definition}%[非正式的自然数定义] %FIXME
    自然数是指集合
    \[\mathbf{N}=\{0,1,2,3,4,\ldots\}\]
    中的元素,此集合是由从0开始无休止地往前数所得到的一切数的集合. 我们将$\mathbf{N}$称为自然数集.
\end{definition}

实际上,这个定义在一定意义上解决了自然数是什么的问题,但这并不是完全可接受的,因为它遗留下许多没有回答的问题. 例如:怎么知道我们可以无休止地数下去而不会循环回到0?如何定义自然数的运算,如加法、乘法或指数运算?我们可以首先回答最后一个问题:可以通过简单的运算来定义复杂的运算. 指数运算只不过是重复的乘法运算:$5^3$是3个5乘在一起;乘法只不过是加法的重复:$5\times 3$是3个5加在一起;而加法只不过是每次增长一个的行为(称为增长运算,实际上与C语言代码中的$++$运算符非常类似)的重复运作:如果你把5加上3,你所做的只不过是让5增长3次. 另一方面,增长似乎是一个基本的运算,它没法再归结为更简单的运算. 于是,为了定义自然数,我们将使用两个基础性的概念:数零$0$以及增长运算,我们称之为后继. 我们用$suc(n)$代表$n$的后继,例如$suc(3)=4$,$suc(suc(3))=5$等等. 于是,似乎我们要说自然数集是由$0$和每个可由$0$经增长而得者所组成,因此我们可以自然地想到如下两条公理:

\begin{axiom}
    \begin{enumerate}
        \item $0$是自然数;

        \item 如果$n$是自然数,那么$suc(n)$也是自然数.
    \end{enumerate}
\end{axiom}

现在我们的自然数集合只有0这个元素以及$suc$运算(一定要忘记其他数字和加法,它们现在还不存在呢!),因此自然数集合可以写成
\[\mathbf{N}=\{0,suc(0),suc(suc(0)),suc(suc(suc(0))),\cdots\}.\]
这样的写法太复杂了,我们不妨记$1$是$suc(0)$,2是$suc(suc(0))$,3是$suc(suc(suc(0)))$,等等. 那么$1,2,3$实际上只是一个记号,表征它们不是0,并且互不相同. 但我们现在的公理并没有说明$suc(3)$一定不是0,事实上$suc(3)=0$并没有违反上面两条公理的任何一条,如果$suc(3)=0$,那事实上自然数集合只能有$0,1,2,3$四个数字,这与自然数集有无穷个元素不符,因此我们需要一个公理来排除这种情况:

\begin{axiom}%[无穷公理]
    如果$n$是任意自然数,那么$suc(n)\neq 0$. 即0不是任何自然数的后继.
\end{axiom}

基于此,我们可以给每个自然数集的元素一个记号,刚刚我们编到了$3$,事实上现在也可以引入$4=suc(3)$等. 但是要注意的是我们现在也没有所谓十进制的说法,记住这些数字只是记号,我完全可以定义$suc(3)=a$而不是$4$,只是记作$4$更符合我们的习惯.

然而即使我们加入了新的公理也不能阻止一些病态的情况,例如考虑由$0,1,2,3,4$组成的集合,这个集合中增长运算在4处``碰了顶'',即$suc(0)=1$,$suc(1)=2$,$suc(2)=3$,$suc(3)=4$,但$suc(4)=4$,那么接下来所有递增后的元素都将逃不开4,这也和``无穷集合''矛盾,但没有违反之前定义的三条公理的任何一条,因此我们需要补充下面这一条公理:

\begin{axiom}
    不同的自然数必有不同的后继者,即若$n,m$是自然数,且$n\neq m$,则$suc(n)\neq suc(m)$. 等价而言,如果$suc(n)=suc(m)$,则$n=m$.
\end{axiom}

到目前为止,我们大概可以保证全体自然数彼此两两不同,但我们并没有排除在0和1之间插入$0.5$这样的不该出现的元素出现在自然数集合中的可能性,于是我们需要引入接下来的这条公理,我们称之为数学归纳原理,或称第一数学归纳法:

\begin{axiom}%[数学归纳原理]
    \label{thm:1e:数学归纳原理} \index{shuxueguinayuanli@数学归纳原理 (principle of mathematical induction)}
    设$P(n)$是关于自然数的一个性质,假设$P(0)$是真的,而且$P(n) \rightarrow P(suc(n))$也是真的,那么$P(n)$对于所有自然数$n$都是真的.
\end{axiom}

因此,假如我们有定义$suc(0)=1$,$suc(1)=2$,$suc(2)=3$等构成自然数集合,但这个自然数集在$0$和$1$之间出现了数$0.5$,它不是任何数的后继,那么一定违反了上面的公理——因为假设$P(0)$是真的,我们只能保证$P(1),P(2),P(3)$是真的,$0.5$虽然也是自然数(因为它在我们定义的自然数集中),但我们没有规则保证$P(0.5)$成立,因此与``$P(n)$对于所有自然数$n$都是真的''矛盾. 所以这一公理合理防止了不应该出现的数出现在自然数集合中的可能性.

数学归纳原理事实上带来了更重要的结果,即引入了所谓的数学归纳法这一重要的证明方法. 相信在中学阶段各位也已熟知这一方法,并且这一方法实际上就是与这一公理完美对应的,因此我们不再赘述. 我们在这里只给出利用数学归纳法证明的一个框架:一般地,证明一个与自然数$n$有关的数学命题,可按如下两个步骤进行:
\begin{enumerate}
    \item(归纳奠基)证明当$n=n_0$时命题成立;

    \item(归纳递推)假设当$n=k$时命题成立,证明当$n=k+1$时命题也成立.
\end{enumerate}
由此就可以断定命题对于从$n_0$开始的所有自然数$n$都成立.

接下来我们需要解决一个更复杂的问题:我们还没有定义自然数之间的运算,例如加法、乘法以及指数运算等. 事实上,从增长到加法、从加法到乘法以及乘法到指数运算,我们都可以通过类似的方法定义,因此这里我们只严格证明加法的相关性质,而其它的运算的性质及详细证明读者可以自行证明或参考《陶哲轩实分析》.

事实上,我们在正式讨论皮亚诺公理之前就已经简单讨论过加法的思想,例如$5+3$实际上就是5增长3次,即$suc(suc(suc(5)))$. 因此我们可以如下定义加法为:

\begin{definition}
    设$m$是自然数,定义$0+m=m$. 对于任意自然数$n$,定义$suc(n)+m=suc(n+m)$.
\end{definition}

于是$0+m=m$,$1+m=suc(0+m)=suc(m)$,$2+m=suc(1+m)=suc(suc(m))$,以此类推,我们也可以基于此得到$2+3=suc(suc(3))=suc(4)=5$. 但我们目前不能仅仅依靠这些直觉推导,我们希望证明一些对于任意正整数都成立的运算规律,例如交换律. 将一个性质推演至无穷我们自然想到数学归纳法,接下来我们就来尝试使用数学归纳法证明这样定义的自然数满足加法交换律.

\begin{lemma}
    对于任意自然数$n$,有$n+0=n$.
\end{lemma}
\begin{proof}
    用归纳法. 由加法定义中$0+n=n$可知$0+0=0$,归纳基础成立. 现假设$n+0=n$,则$suc(n)+0=suc(n+0)=suc(n)$,归纳步骤成立. 由数学归纳原理可知,对于任意自然数$n$,有$n+0=n$.
\end{proof}

\begin{lemma}
    对于任意自然数$n,m$,有$n+suc(m)=suc(n+m)$.
\end{lemma}
\begin{proof}
    此处归纳法要注意准确选取对谁使用归纳法. 事实上选取$n$进行归纳更为简单,因为在原式中没有出现过独立的$suc(n)$,所以它可以在一定程度上避免套娃. $n=0$时,$0+suc(m)=suc(m)=suc(0+m)$,归纳基础成立. 现假设$n+suc(m)=suc(n+m)$,则$suc(n)+suc(m)=suc(n+suc(m))=suc(suc(n+m))$,$suc(suc(n)+m)=suc(suc(n+m))=suc(n)+suc(m)$,归纳步骤成立. 由数学归纳原理可知,对于任意自然数$n,m$,有$n+suc(m)=suc(n+m)$.
\end{proof}

有了前面两个引理的铺垫,我们可以证明自然数的加法交换律:

\begin{theorem}%[加法交换律]
    \index{jiafajiaohuanlv@加法交换律 (commutative law of addition)}
    对于任意自然数$n,m$,有$n+m=m+n$.
\end{theorem}
这里的证明仍然基于对$n$作归纳法,具体证明我们放在习题中供读者练习. 事实上我们还可以基于归纳法证明加法结合律、消去律(即由$a+c=b+c$推出$a=b$)等,以及乘法、指数运算(包括加法乘法分配律)等,但这已经超出我们的讨论范畴——我们的希望是基于皮亚诺公理给读者一个公理化构建完备体系的体验,事实上从最开始逐条加入公理排除不合理的情况,到定义加法之后不断证明一些很显然但不易证明的结论,都很好地体现了利用公理化构建完备数学体系的思想. 在讨论的最后我们为自然数引入最后一个结构——序结构,它定义了自然数之间的大小关系,这也是需要公理化的:

\begin{definition}%[序结构]
    设$m,n$是自然数,我们说$n$大于等于$m$,记作$n\geqslant m$,或$m\leqslant n$,如果存在一个自然数$k$使得$n=m+k$. 我们说$n$严格大于$m$,记作$n>m$,或$m<n$,当且仅当$n\geqslant m$且$n\neq m$.
\end{definition}

事实上,基于这一定义以及之前介绍的加法定义和导出的性质,我们可以得到以下定理:

\begin{theorem}
    设$a$和$b$是自然数,那么下面三个命题中恰有一个是真的:
    \begin{enumerate}
        \item $a=b$;

        \item $a<b$;

        \item $a>b$.
    \end{enumerate}
\end{theorem}

这事实上就表明上面的定义是合理的——因为任意两个自然数之间都可以比较,并且比较的结果是三种之中确定的一种. 定理证明只需对$a$作归纳,具体证明我们放在习题中供读者练习. 事实上,我们引入序结构主要目的是介绍下面的强归纳法原理,或称第二数学归纳法:

\begin{theorem}%[强归纳法原理]
    \label{thm:1e:强归纳法原理} \index{qiangguinafayuanli@强归纳法原理 (principle of strong induction)}
    设$m_0$是一个自然数,$P(m)$是一个依赖于任意自然数$m$的性质. 设对于每个$m\geqslant m_0$都有下述蕴含关系:如果$P(m')$对于一切满足$m_0\leqslant m'<m$的自然数$m'$都成立,那么$P(m)$也成立,那么我们可以断定$P(m)$对于所有自然数$m\geqslant m_0$都成立.
\end{theorem}

这里定理的证明我们也留作习题,当然我们更多地是直接使用它. 或许上面的描述有些抽象,我们直接给出第二数学归纳法的框架,对照理解更为直观:
\begin{enumerate}
    \item (归纳奠基) 证明当$m=m_0$时命题成立;

    \item (归纳递推) 假设当$m_0\leqslant k\enspace(k\in\mathbf{N},\enspace k>m_0)$时命题成立,证明当$m=k+1$时命题也成立.
\end{enumerate}
由此就可以断定命题对于从$m_0$开始的所有自然数$m$都成立. 如果取框架中$k+1=m'$就很容易看出这与强归纳法原理一致了. 我们不难发现能用第一数学归纳法证明的命题都可以用第二数学归纳法证明,但反之不一定成立,因此第二数学归纳法原理更强,因为它假设所有小于等于$k$的自然数都满足命题推出$k+1$时成立,而第一数学归纳法只需要假设等于$k$满足命题即可证明. 但是,需要注意的是,作为两条公理,它们是等价的,有兴趣的读者可自行完成证明.

至此我们结束对皮亚诺公理的讨论——事实上我们已经讨论了非常多,而且显得有些枯燥,因此接下来最后一个例子我们会更轻松一些. 我们希望通过罗素悖论和公理化集合论的例子和故事进一步说明公理化如何助于构建完备公理体系. 事实上,我们可以穿插着谈一些简单的历史. 十九世纪下半叶,德国数学家康托(Georg Cantor)创立了著名的集合论,在集合论刚产生时,曾遭到许多人的猛烈攻击,但不久这一开创性成果就为广大数学家所接受了,并且获得广泛而高度的赞誉. 数学家们发现,从自然数与康托尔集合论出发可建立起整个数学大厦,因而集合论成为现代数学的基石. 这一发现使数学家们为之陶醉,数学界甚至整个科学界笼罩在一片喜悦祥和的气氛之中. 数学家,尤其是弗雷格(Gottlob Frege)等逻辑主义者普遍认为,数学的系统性和严密性已经达到,数学大厦已经基本建成. 然而,1903年,包括英国数学家罗素以及创始人康托在内的几位数学家先后提出了几条悖论,它们使集合论产生了危机,在弗雷格的著作《算数的基本规律》第二卷中,就留下了那个至今仍让人胆寒的段落:

\begin{quote}
    \kaishu
    在工作完美收官之际,却突然发现整个基础都必须要放弃,对一个科学家来说没有什么能比这个更加不幸的了。是罗素的一封信件让我认识到这一点,我不得不在本书即将出版之际加以说明。
\end{quote}

罗素的版本非常浅显易懂,而且所涉及的只是集合论中最基本的东西. 所以,罗素悖论一提出就在当时的数学界与逻辑学界内引起了极大震动. 接下来我们便开始介绍这一引发第三次数学危机的悖论. 实际上,在康托的集合论中,有一条所谓的概括公理:

\begin{axiom}%[概括公理]
    设对于每个对象$x$,我们都有一个依赖于$x$的性质$P(x)$(从而对于每个$x$,$P(x)$要么是真命题,要么是假命题),则存在一个集合$A$,使得对于每个对象$x$,$x\in A$当且仅当$P(x)$为真.
\end{axiom}

事实上这一公理看起来非常符合直觉,因为它表明总是存在一个集合,它由满足某一性质的所有对象组成. 但是这一公理却导致了著名的罗素悖论:设$P(x)$是这样的命题
\[P(x)\iff x\text{~是一个集合,且~}x\notin x,\]
也就是说,$P(x)$为真当且仅当$x$是一个集合,且$x$不是自身的元素. 例如,$P(\{2,3,4\})$成立,因为集合$\{2,3,4\}$不是$\{2,3,4\}$的三个元素$2,3,4$中的任何一个.

接下来我们利用概括公理构造集合:
\[\Omega=\{x\mid P(x)\text{~成立~}\}=\{x\mid x\text{~是一个集合,且~}x\notin x\},\]
即$\Omega$是一切不以自己为元素的集合的集合,现在我们有这样一个问题,$\Omega$含有它自己为元素吗?即是否有$\Omega\in\Omega$:
\begin{enumerate}
    \item 如果$\Omega\in \Omega$,则由$\Omega$的定义知$\Omega\notin \Omega$,矛盾!

    \item 如果$\Omega\notin \Omega$,则由$\Omega$的定义知$\Omega\in \Omega$,矛盾!
\end{enumerate}

事实上这一悖论的构造是很简单的:我们只是构造了一个命题$P(x)$,然后利用概括公理基于这一命题构造出了一个集合$\Omega$,然后就发现$\Omega\in\Omega$这个问题无法回答,得到矛盾. 事实上我们有一个很经典的理发师的故事可以帮助我们理解罗素悖论:在某个城市中有一位理发师,他的广告词是这样写的:``本人的理发技艺十分高超,誉满全城. 我将为本城所有不给自己刮脸的人刮脸,我也只给这些人刮脸. 我对各位表示热诚欢迎!''来找他刮脸的人络绎不绝,自然都是那些不给自己刮脸的人. 可是,有一天,这位理发师从镜子里看见自己的胡子长了,他本能地抓起了剃刀,你们看他能不能给他自己刮脸呢?如果他不给自己刮脸,他就属于``不给自己刮脸的人'',他就要给自己刮脸,而如果他给自己刮脸呢?他又属于``给自己刮脸的人''他就不该给自己刮脸. 理发师悖论是罗素悖论的一种通俗表达:如果把每个人看成一个集合$x$,这个集合的元素被定义成这个人刮脸的对象,即$P(x)$在$x$不属于$x$时成立,即一个人不是自己的刮脸对象,即自己不给自己刮脸时成立. 那么,理发师宣称,他的元素是城里不属于自身的那些集合(即自己不给自己刮脸的人),那么理发师集合实际上就是上面的$\Omega$,那么理发师是否属于他自己这一问题就对应于罗素悖论最后导出矛盾的问题,由此我们可以看出理发师的故事和罗素悖论的对应关系.

事实上我们仔细思考理发师的故事,问题的关键就在于理发师是否考虑他自己,因为他只要不关心他自己是否给自己刮脸,他就不会陷入矛盾,因此罗素悖论的关键点也在于这一``自包含''逻辑的问题. 将来如果是学习计算机专业的同学一定会了解``停机问题'',事实上也是利用这一``自包含''思想构造的矛盾.

于是自然地,数学家们开始思考如何解决这一漏洞. 1908年,策梅罗(Ernst Zermelo)提出第一个公理化集合论体系,这一公理化集合系统很大程度上弥补了康托尔朴素集合论的缺陷,并且在通过弗兰克尔(Abraham Fraenkel)的改进后得到了著名的被称为ZF公理系统(如果有选择公理则称ZFC公理系统). 除ZF系统外,集合论的公理系统还有多种,如冯·诺伊曼(von Neumann)等人提出的NBG系统等. 在该公理系统中通过引入类(class)的概念以及相应的公理也避免了罗素悖论. 事实上,弗雷格本人也做出过一些尝试,关于他的休谟原则的讨论可以参见DMMC中相应的讨论.

总结而言,本小节我们介绍了三个公理化的例子:距离的公理化、皮亚诺公理和公理化集合论. 通过这三个例子我们可以体会到公理化很多都源于``常识'',公理通常与直觉对应,但它的表达力比常识或一些特例更强(例如公理化距离比平面两点距离公式表达力强),并且它有助于构建完备的数学体系. 因此公理化并不是``妖魔化'',虽然它让很多熟悉的概念变得陌生或是复杂,但它们只是数学家为了构建完备的数学体系而做出的努力,而这些努力很多时候仍然是基于正常人的直觉——以至于像康托那样的大数学家也可能会犯错误,即使是一些天才的设计,它们很多时候也是希望达到某个目标而进一步抽象做出的.

事实上公理化有几个重要的评价指标. 其一是一致性,即我们不能从一个公理体系导出矛盾. 其二是是否足够精简,一方面足够精简的公理系统会更加简洁美观,另一方面更多的公理使得互相之间可以推导也是没必要的. 这种观念被一些数学家奉为圭臬,其巅峰就是 Harvey Friedmann 开创的对公理的逆向工程——逆向数学(reverse mathematics),他们力求研究清楚某个结果所需的极小公理系统. 其二是表达力,事实上我们很容易提出这样的问题:为什么距离公理化只有这三条要求?群的定义为什么只有这几条?公理化集合论为什么是这些内容而不是其他,明明集合满足的基本要求我们可以有很多种表达?事实上这与表达力是分不开的. 对于这些东西的详尽讨论亦可见 DMMC. 上面介绍的这些公理化定义都能将我们最熟悉的其它可能导出的常识性性质推导出,因此表达力是足够的,例如皮亚诺公理完全可以体现出整数的特点. 而公理化集合论则更不必纠结,一方面整个数学的大厦都建立在其上,它们首先都在公理化设计中经历了反复检验和调整,然后也经历了上百年的检验和运用(除了选择公理存在一定争议),另一方面我们也不只有一套公理化集合论的方法,事实上它们在很大程度上是等价的(至少在一般的数学分析、线性代数学习中我们不关心它们的区别). 只是我们在学习这些公理的时候是被动的接受者,倘若我们站在设计者的角度思考,我们会发现这些公理的设计是自然而精妙的,是螺旋式上升的过程,并非妖魔化的. 因此相信经过这里的训练后,下一讲开始的线性空间 8 条运算公理不再能让读者产生很大的畏难心理.

\subsection{布尔巴基学派}

我们已经通过一些例子体会了公理化的思想,事实上我们要谈论公理化,不能避开的一个主题就是布尔巴基学派. 尼古拉·布尔巴基(法语:Nicolas Bourbaki)是20世纪一群法国数学家的笔名. 布尔巴基是个虚构的人物,布尔巴基团体的正式称呼是``尼古拉·布尔巴基合作者协会'',在巴黎的高等师范学校设有办公室,他们由1935年开始撰写一系列述说对现代高等数学探研所得的书籍. 以把整个数学建基于集合论为目的,在过程中,布尔巴基致力于做到最极端的严谨和泛化,建立了些新术语和概念.

布尔巴基在集合论的基础上用公理方法重新构造整个现代数学. 布尔巴基认为:数学,至少纯粹数学,是研究抽象结构的理论. 结构,就是以初始概念和公理出发的演绎系统. 有三种基本的抽象结构:代数结构——也就是我们之前一直在强调的集合上定义运算的想法;序结构——可以参考教材1.4节;拓扑结构——我们将在下面马上给出定义. 他们把全部数学看作按不同结构进行演绎的体系.

\begin{definition}
    设$X$是一个集合,$\mathscr{T}$是$X$的一个子集族,如果$\mathscr{T}$满足
    \begin{enumerate}
        \item $\varnothing,X\in \mathscr{T}$;

        \item 若$U,V\in \mathscr{T}$,则$U\cup V\in \mathscr{T}$;

        \item 若$\{U_\alpha\}_{\alpha\in I}\subset \mathscr{T}$,则$\bigcap_{\alpha\in I}U_\alpha\in \mathscr{T}$,其中$I$是一个指标集合.
    \end{enumerate}
    则称$(X,\mathscr{T})$是一个拓扑空间. 如果$U$是族$\mathscr{T}$中的元素,则称$U$是一个开集.
\end{definition}

我们无需明白上面的定义在表达什么,但至少它与我们日常科普中见到的``拓扑''一词从直观上看相去甚远,但事实上基于此我们可以得到更为本质的``连续性''概念——开集的原像是开集,这与数学分析中学习的一元函数连续性是一致的,我们也可以得到数学家笑话``拓扑学家分不清咖啡杯和甜甜圈'',得到连通性、紧致性、同伦等概念,这些概念在数学中有着重要的地位. 但是我们在这里不打算深入讨论拓扑学的内容,而是想通过这个例子说明,布尔巴基公理化的思想是如何在数学中发挥作用的.

布尔巴基著有九卷本,超过七千多页的《数学原本》,这是有史以来最大的数学巨著. 彻底追求严格性和一般性的叙述方法被称为``布尔巴基风格''. 最后的第9卷谱理论执笔始于1983年,出版工程至此告终. 只是在20世纪末,增补了交换代数的簇理论. 布尔巴基对严谨性的强调在当时产生了很大的影响. 这与当时昂利·庞加莱(Henry Poincar\'e)所强调的数学要依靠自由想像的数学直观的说法分庭抗礼. 布尔巴基的影响力随时间而减弱,一个原因是由于布尔巴基的抽象并不显得比发明者原初的想法更为有用,另一个原因是因为没有包含像范畴论等重要的现代数学理论. 尽管范畴论是由布尔巴基的成员艾伦堡(Samuel Eilenberg)所创立,格罗滕迪克(Alexander Grothendieck)所推广的,但是如果要容纳范畴论,就不得不对已经出版的著作进行根本性的改写.

布尔巴基在数学史上还承担了类似于``大一统''的工作,他们引入的记号有:$\varnothing$,代表空集;黑板粗体字母表示数集(例如:$\mathbf{N}$表示自然数集,$\mathbf{Q}$表示有理数集,$\mathbf{R}$表示实数集,$\mathbf{Z}$表示整数集);还发明了术语``单射''、``满射''和``双射''——你可能无法想象如此基本的数学名词曾经还有多种不统一的叫法. 事实上,现在我们用到的``紧集''(学习数学分析的同学应该知道,现在是用实数完备性中的有限覆盖定理表达的)在布尔巴基学派之前有数十种定义.

我们无法否认布尔巴基学派这一曾经诞生了如此众多顶级数学家的学派,同时,也作为一个曾经实践了这样一个雄心勃勃的数学统一计划的学派对数学本身的贡献. 当然他们曾经在20世纪50--60年代推行的所谓``新数学''运动,把抽象数学,特别是抽象代数的内容引入中学甚至小学的教科书当中. 这种突然的变革不但使学生无法接受新教材,就连教员都无法理解,造成了整个数学教育的混乱. 在高等数学教育方面,就连布尔巴基的奠基者们后来编的教科书也破除了布尔巴基的形式体系而采用比较自然、具体、循序渐进的体系. 所以我想对于一本教材而言,自然、具体、循序渐进是重要的,学习需要螺旋式上升的过程,而不是一蹴而就,这一点相信读者在未来学习中一定能体会到.

\vspace{2ex}
\centerline{\heiti \Large 内容总结}

\vspace{2ex}
\centerline{\heiti \Large 习题}

\vspace{2ex}
{\kaishu 教育不是灌输,而是点燃火焰.}
\begin{flushright}
    \kaishu
    ——苏格拉底
\end{flushright}

\centerline{\heiti B组}
\begin{enumerate}
    \item 依照皮亚诺公理的定义证明:对于任意自然数$n,m$,有$n+m=m+n$.
\end{enumerate}

\centerline{\heiti C组}
\begin{enumerate}
    \item 证明闵可夫斯基不等式和$\ell_p$范数的三角不等式.

    \item 利用皮亚诺公理,证明如下命题:设$a$和$b$是自然数,那么下面三个命题中恰有一个是真的:
          \begin{enumerate}
              \item $a=b$;

              \item $a<b$;

              \item $a>b$.
          \end{enumerate}

    \item 利用皮亚诺公理,证明\hyperref[thm:1e:强归纳法原理]{强归纳法原理}.
    \item 利用\hyperref[thm:1e:强归纳法原理]{强归纳法原理}反推\hyperref[thm:1e:数学归纳原理]{数学归纳原理}.
\end{enumerate}

\ResetChapterNumberingStyle{1}
\endgroup

\chapter{线性空间}

本讲我们将开始回答第 1 讲最后留下的问题,即线性方程组有唯一解、无穷解或无解的本质原因. 这段旅程或许有些漫长,中间会有很多的铺垫,我们将从其中最为基础的概念——线性空间出发进行探讨.

回忆高斯消元法,方程组中每一行或一列都可以视为向量. 我们可以先看下面这个例子:
\begin{example}\label{ex:2:线性空间引入}
    考虑如下两个方程组
    \begin{multicols}{2}
        \item $\begin{cases}
                x_1+x_2+x_3=0   \\
                x_1+2x_2+3x_3=0 \\
                2x_1+3x_2+4x_3=0
            \end{cases}$

        \item $\begin{cases}
                x_1+x_2+x_3=0   \\
                x_1+2x_2+3x_3=0 \\
                x_1+3x_2+4x_3=0
            \end{cases}$
    \end{multicols}
    不难解得,第一个方程组有无穷解,第二个方程组有唯一解. 从高斯消元法的过程来看,第一个方程组的简化阶梯矩阵出现了全零行,其原因是显而易见的:因为方程组第一行和第二行相加正好是第三行,因此可以直接消去第三行,即三行的系数矩阵的三个行向量
    \[\alpha_1=(1,1,1),\enspace\alpha_2=(1,2,3),\enspace\alpha_3=(2,3,4)\]
    满足$\alpha_1+\alpha_2=\alpha_3$. 而第二个方程组系数矩阵行向量间没有类似的可互相消去的关系.
\end{example}

从上面这一例子中我们可以看出,方程组的解与系数矩阵的行向量之间的关系密切相关. 因此我们会有一个很自然的想法,即我们需要研究向量之间的关联. 受第1讲基本代数结构的启发,我们应当自然地想到我们需要引入一个代数结构,从而使得我们可以统一地研究向量间的关联,这一代数结构便是线性空间.

\section{线性空间的定义}

\term{线性空间}\index{xianxingkongjian@线性空间 (linear space)}是我们接触的第一个核心概念,作为一种代数结构,它需要在非空集合$V$上定义运算. 其中第一个运算是我们熟知的加法$+$. 在线性空间的定义中,我们要求$\langle V:+\rangle$构成Abel群,即其中元素满足如下运算律:
\begin{enumerate}
    \item (结合律) $\alpha+(\beta+\gamma)=(\alpha+\beta)+\gamma,\enspace\forall \alpha,\beta,\gamma \in V$;

    \item (加法单位元) $\exists 0 \in V$使得$\forall\alpha\in V$ 有 $\alpha+0=0+\alpha$;

    \item (逆元) $\forall\alpha\in V,\enspace \exists \beta \in V$,有$\alpha+\beta=\beta+\alpha=0$,记$\beta=-\alpha$;

    \item (交换律) $\forall\alpha, \beta\in V,\enspace \alpha+\beta=\beta+\alpha$.
\end{enumerate}

第二种运算和之前学习的其他代数结构不同,我们需要首先引入一个数域$\mathbf{F}$,接下来在$\mathbf{F}\times V$上定义取值于$V$的数乘运算,即$\mathbf{F}\times V$中的每个元素$(\lambda,\alpha)\mapsto \lambda\alpha\in V$,并且数乘运算满足以下性质:$\forall \alpha,\beta \in V,\enspace\forall \lambda,\mu\in\mathbf{F}$以及$\mathbf{F}$上的乘法单位元1,有
\begin{enumerate}
    \item $1\cdot \alpha=\alpha$;

    \item $\lambda(\mu\alpha)=(\lambda\mu)\alpha$;

    \item $(\lambda+\mu)\alpha=\lambda\alpha+\mu\alpha$;

    \item $\lambda(\alpha+\beta)=\lambda\alpha+\lambda\beta$.
\end{enumerate}

基于此,我们完整定义了一个线性空间,我们一般称集合$V$关于上述两种运算在域$\mathbf{F}$上构成一个线性空间,简称为$V$在域$\mathbf{F}$上的线性空间,记作$V(\mathbf{F})$. 如果$\mathbf{F}$是实(复)数域,则称$V$为实(复)数域上的线性空间. 关于线性空间的定义,我们还有如下说明:
\begin{enumerate}
    \item 线性空间还有一个重要的概念是运算封闭,即线性空间中的元素进行加法或数乘运算后,得到的元素仍然是属于线性空间的. 这一点是定义要求的,加法封闭是 Abel 群的要求,因为 Abel 群要求加法运算定义为映射 $V\times V\to V$,因此$V$中两个元素相加后必须仍在$V$中(事实上这是代数系统的共性),数乘注意前述定义中数乘运算``取值于$V$''的要求,即它是 $F\times V\to V$ 的映射;

    \item 特别注意线性空间定义在非空集合上,事实上根据加法构成Abel群的要求,最小的线性空间也必须至少包含加法单位元(可以记为$V=\{0\}$).

    \item 结合我们上一讲对公理化的研究,事实上我们到目前为止也只定义了上面的加法、数乘运算和几条规则,我们需要忘记其他任何规则,由此出发进行推导出一些看似显然但公理没有直接给出的重要运算性质:
    \begin{enumerate}
        \item 由于加法运算构成Abel群,因此加法零元和逆元是唯一的,并且我们可以定义减法运算为加上一个元素的逆,即$\alpha-\beta=\alpha+(-\beta)$;

        \item 事实上,根据公理中的性质,我们可以逐步得到$\lambda(\alpha-\beta)+\lambda\beta=\lambda((\alpha-\beta)+\beta)=\lambda((\alpha+(-\beta))+\beta)=\lambda(\alpha+((-\beta)+\beta))=\lambda(\alpha+\vec{0})=\lambda\alpha$,两边分别加$-(\lambda\beta)$即可以得到
        \begin{equation}\label{eq:2:线性空间运算性质1}
            \lambda(\alpha-\beta)=\lambda\alpha-\lambda\beta.
        \end{equation}
        上面推导过程中第一个等号来源于数乘分配律,第二个等号来源于减法的定义(加上逆元),第三个等号来源于加法结合律,第四个等号来源于逆元的定义(加起来等于向量加法零元$\vec{0}$),最后一个等号来源于加法单位元的定义. 事实上这一过程是非常清晰的. 需要注意的一点是,接下来为了区分$V$中的零元和数域中的数0,我们将$V$中零元加粗,请读者务必仔细区分.

        除此之外,$(\lambda-\mu)\alpha+\mu\alpha=(\lambda-\mu+\mu)\alpha=\lambda\alpha$,两边分别加$-(\mu\alpha)$即可以得到
        \begin{equation}\label{eq:2:线性空间运算性质2}
            (\lambda-\mu)\alpha=\lambda\alpha-\mu\alpha.
        \end{equation}
        事实上,\autoref{eq:2:线性空间运算性质1} 和\autoref{eq:2:线性空间运算性质2} 可以视为数乘运算对减法也满足分配律(但我们必须时刻牢记在心,数的减法是常规的,向量的减法是加上向量的逆元).

        \item 在\autoref{eq:2:线性空间运算性质1} 中分别令$\alpha=\beta$和$\alpha=\vec{0}$,在\autoref{eq:2:线性空间运算性质2} 分别令$\lambda=\mu$和$\lambda=0$有如下四条性质:
        \begin{enumerate}
            \item $\lambda\cdot \vec{0}=\vec{0}$;

            \item $\lambda(-\beta)=-(\lambda\beta)$;

            \item $0\cdot \alpha=\vec{0}$;

            \item $(-\mu)\alpha=-(\mu\alpha)$.
        \end{enumerate}
        我们详细证明前两条如何根据公理一步步推导得到,后两条请读者依照此自行证明.
        \begin{proof}
            \begin{enumerate}
                \item 在\autoref{eq:2:线性空间运算性质1} 中令$\alpha=\beta$,则$\lambda(\alpha-\alpha)=\lambda\alpha-\lambda\alpha$,根据减法定义有$\alpha-\alpha=\alpha+(-\alpha)=\vec{0}$,且$\lambda\alpha-\lambda\alpha=\lambda\alpha+(-(\lambda\alpha))=\vec{0}$,因此$\lambda\cdot \vec{0}=\vec{0}$.

                \item 在\autoref{eq:2:线性空间运算性质1} 中令$\alpha=\vec{0}$有$\lambda(\vec{0}-\beta)=\lambda\vec{0}-\lambda\beta$,根据减法定义有$\vec{0}-\beta=\vec{0}+(-\beta)=-\beta$(第二个等号来源于加法单位元性质),且$\lambda\vec{0}-\lambda\beta=\vec{0}-\lambda\beta=\vec{0}+(-(\lambda\beta))=-(\lambda\beta)$(第一个等号来源于刚刚证明的$\lambda\cdot \vec{0}=\vec{0}$,第二个等号来源于减法的定义,第三个等号来源于加法单位元性质),因此$\lambda(-\beta)=-(\lambda\beta)$.
            \end{enumerate}
        \end{proof}
        特别地,当$\mu=1$时有$(-1)\alpha=-\alpha$. 即$-1$数乘一个元素可以得到该元素的逆元(虽然代入一般平面向量这一点非常显然,但是我们只能基于公理一步步推导得到这一显然的性质).

        \item 若$\lambda\alpha=\vec{0}$,则$\lambda=0$或$\alpha=\vec{0}$,这一点也是显然的,因为如果$\lambda\neq 0$,则$\lambda^{-1}$存在,从而$\alpha=1\alpha=(\lambda^{-1}\lambda)\alpha=\lambda^{-1}(\lambda\alpha)=\lambda^{-1}\vec{0}=\vec{0}$(这里的每一个等号都是能找到对应的,请读者自行判断).

        最后,综合上述性质我们有方程$\lambda\beta+\lambda_1\alpha_1+\lambda_2\alpha_2+\cdots+\lambda_r\alpha_r=\vec{0}$在$\lambda\neq 0$时的解为$\beta=-\lambda^{-1}\lambda_1\alpha_1-\lambda^{-1}\lambda_2\alpha_2-\cdots-\lambda^{-1}\lambda_r\alpha_r$. 我们放在习题中供读者练习.
    \end{enumerate}
\end{enumerate}

或许同学们会疑惑为什么线性空间会要求上述这8条性质(加法、数乘各4条). 事实上,这里的加法交换律是可以被其他7条推出的,感兴趣的同学可以自行尝试证明. 其余的7条公理彼此独立,每一条均不可取消. 感兴趣的同学可以试试举出反例来说明其余7条中每一条均不能由其余各条推出.

我们发现线性空间中定义的运算规则与我们高中学习的平面向量的加法和数乘是非常类似的,我们回顾未竟专题一关于公理化的讨论,实际上这就可以视为从简单的向量加法和数乘抽象出来的一些规则. 而公理的诞生应当是要尽可能简洁,而且有足够的表达力——这一点我们将来基于这一定义不断推出线性空间的性质时就会发现非常足够(事实上你现在就能通过我们上面证明的运算性质初步感知到这一点,因为7条公理中任何一条的缺失都会使得上面某条显然而合理的性质不再满足,而我们未来需要的性质都可以由此导出),因此皮亚诺在1888年正式给出这一定义并沿用至今. 但我们需要知道他的工作也是基于前人(如格拉斯曼)的工作不断修正而来的,只是我们被动接受这一概念使得这一自然的过程变得很突兀. 当然这门课只要求你记忆这8条性质,并请务必牢记于心,考试可能要求你验证线性空间. 记忆难度也并不大,Abel 群4条性质都有名称标注,数乘运算也是易于记忆的结合律和分配律加单位元性质.

除此之外,公理化定义还有一个很重要的作用就是使得我们可以不仅仅在向量集合的背景下定义线性空间,这使得我们可以将对于很多结构的研究都转化为对于线性空间的研究. 接下来我们给出一些与向量无关的线性空间的例子:

\begin{example}
    几种非常常见的线性空间,希望读者能熟知其性质:
    \begin{enumerate}
        \item (多项式)$\mathbf{F}[x]_{n+1}=\{a_0+a_1x+\cdots+a_nx^n \mid a_i\in\mathbf{F}\}$关于多项式的加法和数乘构成线性空间,但
              \[\mathbf{F}[x]'_{n+1}=\{a_0+a_1x+\cdots+a_nx^n \mid a_i\in\mathbf{F}, a_n\neq 0\}\]
              不构成线性空间.

              注:书上常将多项式记为$\mathbf{F}[x]_{n+1}$,表示次数不超过$n$的多项式的集合,而《线性代数应该这样学》中使用 $\mathcal{P}_n(\mathbf{F})$ 表示相同的集合.

              注意常见记号:$(k_1p_1+k_2p_2)(x)=k_1p_1(x)+k_2p_2(x)$.

        \item (复数与实数)可以验证:复数集$\mathbf{C}$是数域$\mathbf{C}$或数域$\mathbf{R}$上的线性空间. 此处一定注意复数集$\mathbf{C}$在此处同时出现在集合和数域中.

              注意:这一例子表明,同一集合可以在不同数域上构成不同的线性空间,在下一讲接触维数的定义后,我们也将知道二者的维数是不一样的(见\autoref{ex:3:不同数域的维数}).

              当然,不同的集合也可以在同一个数域上构成不同的线性空间,例如$\mathbf{C(R)}$和$\mathbf{R(R)}$.

        \item 对$n$维实非零系数向量空间$V$定义如下加法运算
        \begin{gather*}
            \alpha = (a_1, a_2, \ldots, a_n), \beta = (b_1, b_2, \ldots, b_n) \in V = \mathbb{R}_{+}^n, \\ \alpha \oplus \beta = (a_1b_1, a_2b_2, \ldots, a_nb_n).
        \end{gather*}
        定义如下数乘运算
        \[\forall \lambda \in \mathbf{R}, \lambda \circ \alpha = (a_1^\lambda, a_2^\lambda, \ldots, a_n^\lambda).\]
        则$V$构成线性空间.

        \item $V=\{f \mid x \in \mathbf{R}, f(x) \in \mathbf{C}$(即$f$是实变量复值函数),且$f(-x)=\overline{f(x)}$(后者为$f(x)$的共轭复数)$\}$,定义如下的加法和数乘运算:
        \begin{gather*}
            (f \oplus g)(x) = f(x) + g(x) \\
            (\lambda \circ f)(x) = \lambda f(x).
        \end{gather*}
        则$V$构成线性空间.
    \end{enumerate}
\end{example}

\begin{solution}
    \begin{enumerate}
        \item 我们对八条性质进行逐条验证即可.
                \begin{enumerate}
                    \item $\forall p_1(x), p_2(x), p_3(x) \in \mathbf{F}[x]_{n+1}=\{a_0+a_1x+\cdots+a_nx^n \mid a_i\in\mathbf{F}\}$,有
                    \begin{align*}
                        & (p_1(x) + p_2(x)) + p_3(x) \\ ={} & ((a_{10} + a_{11}x + \cdots  + a_{1n}x^n) + (a_{20} + a_{21}x + \cdots  + a_{2n}x^n)) \\ +{} & (a_{30} + a_{31}x + \cdots  + a_{3n}x^n) \\ ={} & ((a_{10} + a_{20}) + (a_{11} + a_{21}) x + \cdots  + (a_{1n} + a_{2n}) x^n) \\ +{} & (a_{30} + a_{31}x + \cdots  + a_{3n}x^n) \\ ={} & (((a_{10} + a_{20}) + a_{30}) + ((a_{11} + a_{21}) + a_{31})x + \cdots + ((a_{1n} + a_{2n}) + a_{3n})x^n) \\ ={} & ((a_{10} + (a_{20} + a_{30})) + (a_{11} + (a_{21} + a_{31}))x + \cdots + (a_{1n} + (a_{2n} + a_{3n}))x^n) \\ ={} & (a_{10} + a_{11}x + \cdots  + a_{1n}x^n) \\ +{} & ((a_{20} + a_{21}x + \cdots  + a_{2n}x^n) + (a_{30} + a_{31}x + \cdots  + a_{3n}x^n)) \\ ={} & p_1(x) + (p_2(x) + p_3(x))
                    \end{align*}
                    注意,在证明过程中,我们用了形式的加法定义(逐次数将系数相加),并诉诸域 $\mathbf{F}$ 上的结合律,这种诉诸基域性质的方式在以后的证明中会经常碰上.

                    \item 取定 $p_0(x) = 0 \in V$ 则有 $\forall p(x) \in \mathbf{F}[x]_{n+1}, p(x) + p_0(x) = p_0(x) + p(x)$.

                    \item $\forall p(x) = a_0 + a_1x + \cdots + a_nx^n \in \mathbf{F}[x]_{n+1}, \exists p^*(x) = -a_0 - a_1x - \cdots - a_nx^n \in \mathbf{F}[x]_{n+1}, p(x) + p^*(x) = p^*(x) + p(x) = p_0(x) = 0$.

                    \item $\forall p_1(x), p_2(x) \in \mathbf{F}[x]_{n+1}$有
                    \begin{align*}
                        p_1(x) + p_2(x) &= (a_{10} + a_{11}x + \cdots + a_{1n}x^n) + (a_{20} + a_{21}x + \cdots + a_{2n}x^n) \\ &= (a_{10} + a_{20}) + (a_{11} + a_{21})x + \cdots + (a_{1n} + a_{2n})x^n \\ &= (a_{20} + a_{10}) + (a_{21} + a_{11})x + \cdots + (a_{2n} + a_{1n})x^n \\ &= (a_{20} + a_{21}x + \cdots + a_{2n}x^n) + (a_{10} + a_{11}x + \cdots + a_{1n}x^n) \\ &= p_2(x) + p_1(x).
                    \end{align*}

                    \item 取定 $\lambda = 1 \in \mathbf{F},\forall p(x) \in \mathbf{F}[x]_{n+1}, \lambda \cdot p(x) = p(x)$.

                    \item $\forall \lambda, \mu \in \mathbf{F}, p(x) \in \mathbf{F}[x]_{n+1}$有
                    \begin{align*}
                        \lambda(\mu p(x)) &= \lambda(\mu(a_0 + a_1x + \cdots + a_nx^n)) = \lambda(\mu a_0 + \mu a_1x + \cdots + \mu a_nx^n) \\ &= \lambda \mu a_0 + \lambda \mu a_1x + \cdots + \lambda \mu a_nx^n = (\lambda \mu) (a_0 + a_1x + \cdots + a_nx^n) \\ &= (\lambda \mu)p(x).
                    \end{align*}

                    \item $\forall \lambda, \mu \in \mathbf{F}, p(x) \in \mathbf{F}[x]_{n+1}$有
                    \begin{align*}
                        (\lambda + \mu) p(x) &= (\lambda + \mu)(a_0 + a_1x + \cdots + a_nx^n) \\ &= (\lambda + \mu)a_0 + (\lambda + \mu)a_1x + \cdots + (\lambda + \mu)a_nx^n \\ &= \lambda a_0 + \mu a_0 + \lambda a_1x + \mu a_1x+ \cdots + \lambda a_nx^n + \mu a_nx^n \\ &= \lambda(a_0 + a_1x + \cdots + a_nx^n) + \mu(a_0 + a_1x + \cdots + a_nx^n) \\ &= \lambda p(x) + \mu p(x).
                    \end{align*}
                    这里的第二行到第三行并没有诉诸对单项式的分配律,而是利用了性质 6 和域 $\mathbf{F}$ 上的分配律.

                    \item $\forall p_1(x), p_2(x) \in \mathbf{F}[x]_{n+1}, \lambda \in \mathbf{F}$有
                    \begin{align*}
                        & \lambda(p_1(x) + p_2(x)) \\ ={} & \lambda((a_{10} + a_{11}x + \cdots + a_{1n}x^n) + (a_{20} + a_{21}x + \cdots + a_{2n}x^n)) \\ ={} & \lambda((a_{10} + a_{20}) + (a_{11} + a_{21})x + \cdots + (a_{1n} + a_{2n})x^n) \\ ={} & \lambda(a_{10} + a_{20}) + \lambda(a_{11} + a_{21})x + \cdots + \lambda(a_{1n} + a_{2n})x^n \\ ={} & \lambda(a_{10} + a_{11}x + \cdots + a_{1n}x^n) + \lambda(a_{20} + a_{21}x + \cdots + a_{2n}x^n) \\ ={} & \lambda p_1(x) + \lambda p_2(x).
                    \end{align*}
                \end{enumerate}
            但是对\[\mathbf{F}[x]'_{n+1}=\{a_0+a_1x+\cdots+a_nx^n \mid a_i\in\mathbf{F}, a_n\neq 0\}\]不构成线性空间,其原因在于我们无法找到一个零元$p_0(x)$满足$p(x) + p_0(x) = p_0(x) + p(x) = p(x)$.

        \item 同理我们应当对八条性质逐条验证,但我们在第一讲以及说明了全体复数构成一个域,因此$\mathbf{C}(\mathbf{C})$自动满足线性空间的所有条件,此处不再赘述. 除此之外,$\mathbf{C}(\mathbf{R})$的加法运算与实数无关(回顾线性空间定义,实数只用来参与数乘运算),因此加法Abel群事实上与$\mathbf{C}(\mathbf{C})$一致,都是群$\langle \mathbf{C}:+\rangle$,此处也不再验证. 因此这里只验证$\mathbf{C}(\mathbf{R})$数乘运算是否满足线性空间定义的要求:
                \begin{enumerate}
                    \item 取定 $1 \in \mathbf{R}, \forall \alpha = a+b\i \in \mathbf{C}, a, b \in \mathbf{R}, 1 \cdot \alpha = 1 \cdot (a+b\i) = a+b\i = \alpha.$

                    \item $\forall \lambda, \mu \in \mathbf{R}, \alpha = a+b\i \in \mathbf{C}, a, b \in \mathbf{R},$
                    \begin{align*}
                        \lambda(\mu \alpha) = \lambda(\mu (a+b\i)) = \lambda(\mu a+\mu b\i) = \lambda \mu a + \lambda \mu b\i = (\lambda \mu)(a+b\i) = (\lambda \mu)\alpha.
                    \end{align*}

                    \item $\forall \lambda, \mu \in \mathbf{R}, \alpha = a+b\i \in \mathbf{C}, a, b \in \mathbf{R},$
                    \begin{align*}
                        (\lambda + \mu) \alpha &= (\lambda a + \lambda b\i) + (\mu a + \mu b\i) \\ &= \lambda(a+b\i)+\mu(a+b\i) = \lambda \alpha + \mu \alpha.
                    \end{align*}

                    \item $\forall \lambda \in \mathbf{R}, \alpha_1 = a_1+b_1\i, \alpha_2 = a_2+b_2\i \in \mathbf{C}, a_i, b_i \in \mathbf{R}, i = 1, 2,$
                    \begin{align*}
                        \lambda(\alpha_1+\alpha_2) &= \lambda((a_1+b_1\i)+(a_2+b_2\i)) = \lambda((a_1+a_2)+(b_1+b_2)\i) \\ &= \lambda(a_1+a_2)+\lambda(b_1+b_2)\i = (\lambda a_1+\lambda b_1\i)+(\lambda a_2+\lambda b_2\i) \\ &= \lambda(a_1+b_1\i)+\lambda(a_2+b_2\i) = \lambda \alpha_1 + \lambda \alpha_2.
                    \end{align*}
                \end{enumerate}
                所以$\mathbf{C}(\mathbf{C})$和$\mathbf{C}(\mathbf{R})$均构成线性空间.

                \item 这里定义的``加法''和``数乘''与一般的不同,不过也只需要验证八条性质就行.
                \begin{enumerate}
                    \item $\forall \alpha = (a_1, a_2, \ldots, a_n), \beta = (b_1, b_2, \ldots, b_n), \gamma = (c_1, c_2, \ldots, c_n) \in V, $
                    \begin{align*}
                        (\alpha \oplus \beta) \oplus \gamma &= ((a_1, a_2, \ldots, a_n)\oplus (b_1, b_2, \ldots, b_n)) \oplus (c_1, c_2, \ldots, c_n) \\ &= (a_1b_1, a_2b_2, \ldots, a_nb_n) \oplus (c_1, c_2, \ldots, c_n) \\ &= (a_1b_1c_1, a_2b_2c_2, \ldots, a_nb_nc_n) \\ &= (a_1, a_2, \ldots, a_n)\oplus (b_1c_1, b_2c_2, \ldots, b_nc_n) \\ &= (a_1, a_2, \ldots, a_n) \oplus ((b_1, b_2, \ldots, b_n) \oplus (c_1, c_2, \ldots, c_n)) \\ &= \alpha \oplus (\beta \oplus \gamma)
                    \end{align*}

                    \item 取定 $e = (1, 1, \ldots , 1) \in V, \forall \alpha = (a_1, a_2, \ldots, a_n) \in V,$
                    \begin{align*}
                        e \oplus \alpha &=(1, 1, \ldots , 1) \oplus (a_1, a_2, \ldots, a_n) =(a_1, a_2, \ldots, a_n) = \alpha \\ &=(a_1, a_2, \ldots, a_n) \oplus (1, 1, \ldots , 1) =\alpha \oplus e.
                    \end{align*}

                    \item $\forall \alpha = (a_1, a_2, \ldots, a_n) \in V, \exists \beta = (\dfrac{1}{a_1}, \dfrac{1}{a_2}, \ldots, \dfrac{1}{a_n}), \alpha \oplus \beta = \beta \oplus \alpha = e.$

                    \item $\forall \alpha = (a_1, a_2, \ldots, a_n), \beta = (b_1, b_2, \ldots, b_n) \in V,$
                    \begin{align*}
                        \alpha \oplus \beta &= (a_1, a_2, \ldots, a_n) \oplus (b_1, b_2, \ldots, b_n) = (a_1b_1, a_2b_2, \ldots, a_nb_n) \\ &= (b_1a_1, b_2a_2, \ldots, b_na_n) = (b_1, b_2, \ldots, b_n) \oplus (a_1, a_2, \ldots, a_n) = \beta \oplus \alpha.
                    \end{align*}

                    \item 取定 $\lambda = 1 \in \mathbf{R}, \forall \alpha = (a_1, a_2, \ldots, a_n) \in V,$
                    \[\lambda \circ \alpha = (a_1^\lambda, a_2^\lambda, \ldots, a_n^\lambda) = (a_1, a_2, \ldots, a_n) = \alpha.\]

                    \item $\forall \lambda, \mu \in \mathbf{R}, \forall \alpha \in = (a_1, a_2, \ldots, a_n) \in V,$
                    \begin{align*}
                        \lambda \circ(\mu \circ \alpha) &= \lambda \circ(\mu \circ (a_1, a_2, \ldots, a_n)) = \lambda \circ (a_1^\mu, a_2^\mu, \ldots, a_n^\mu) \\ &= (a_1^{\lambda\mu}, a_2^{\lambda\mu}, \ldots, a_n^{\lambda\mu}) = (\lambda \mu)\circ \alpha.
                    \end{align*}

                    \item $\forall \lambda, \mu \in \mathbf{R}, \forall \alpha \in = (a_1, a_2, \ldots, a_n) \in V,$
                    \begin{align*}
                        (\lambda + \mu) \circ \alpha &= (\lambda + \mu) \circ (a_1, a_2, \ldots, a_n) = (a_1^{\lambda + \mu}, a_2^{\lambda + \mu}, \ldots, a_n^{\lambda + \mu}) \\ &= (a_1^\lambda a_1^\mu, a_2^\lambda a_2^\mu, \ldots, a_n^\lambda a_n^\mu) = (a_1^\lambda, a_2^\lambda, \ldots, a_n^\lambda) \oplus (a_1^\mu, a_2^\mu, \ldots, a_n^\mu) \\ &= (\lambda \circ (a_1, a_2, \ldots, a_n)) \oplus (\mu \circ (a_1, a_2, \ldots, a_n)) \\ &= (\lambda \circ \alpha) \oplus (\mu \circ \alpha).
                    \end{align*}

                    \item $\forall \lambda \in \mathbf{R}, \alpha = (a_1, a_2, \ldots, a_n), \beta = (b_1, b_2, \ldots, b_n) \in V,$
                    \begin{align*}
                        \lambda \circ (\alpha \oplus \beta) &= \lambda \circ ((a_1, a_2, \ldots, a_n) \oplus (b_1, b_2, \ldots, b_n)) \\ &= \lambda \circ (a_1b_1, a_2b_2, \ldots , a_nb_n) = ((a_1b_1)^\lambda, (a_2b_2)^\lambda, \ldots , (a_nb_n)^\lambda) \\ &= (a_1^\lambda b_1^\lambda, a_2^\lambda b_2^\lambda, \ldots , a_n^\lambda b_n^\lambda) = (a_1^\lambda, a_2^\lambda, \ldots, a_n^\lambda) \oplus (b_1^\lambda, b_2^\lambda, \ldots, b_n^\lambda) \\ &= (\lambda \circ (a_1, a_2, \ldots, a_n)) \oplus (\lambda \circ (b_1, b_2, \ldots, b_n)) \\ &= (\lambda \circ \alpha) \oplus (\lambda \circ \beta).
                    \end{align*}
                \end{enumerate}
                所以$V$构成在此``加法''和``数乘''下的线性空间.

                \item 这题主要注意需要验证封闭的性质是什么就可以了.
                \begin{enumerate}
                    \item $\forall f, g, h \in V,$
                    \begin{align*}
                        ((f \oplus g) \oplus h)(x) &= (f \oplus g)(x)+h(x) \\ &= (f(x)+g(x))+h(x) = f(x)+(g(x)+h(x)) \\ &= f(x)+ (g \oplus h)(x) = (f \oplus (g \oplus h))(x).
                    \end{align*}

                    \item 取定 $e(x)=0, \forall x \in \mathbf{R}, e(-x)=0=\overline{e(x)}, \forall f \in V,$
                    \begin{align*}
                        (f \oplus e)(x) = f(x) + e(x) = f(x) = e(x) + f(x) = (e \oplus f)(x).
                    \end{align*}

                    \item $\forall f \in V, \exists g \in V, g(x) := -f(x), \forall x \in \mathbf{R},$
                    \begin{gather*}
                        g(-x) = -f(-x) = -\overline{f(x)} = \overline{g(x)} \\
                        (f \oplus g)(x) = f(x)+g(x) = 0 = e(x) = g(x) + f(x) = (g \oplus f)(x).
                    \end{gather*}

                    \item $\forall f, g \in V, (f \oplus g)(x) = f(x)+g(x) = g(x)+f(x) = (g \oplus f)(x).$

                    \item 取定 $\lambda = 1 \in \mathbf{R}, \forall f \in V, (\lambda \circ f)(x) = \lambda f(x) = f(x)$.

                    \item $\forall \lambda, \mu \in \mathbf{R}, f \in V,$
                    \[(\lambda \circ (\mu \circ f))(x) = \lambda((\mu \circ f)(x)) = \lambda (\mu f(x)) = (\lambda \mu) f(x) = ((\lambda \mu) \circ f)(x).\]

                    \item $\forall \lambda, \mu \in \mathbf{R}, f \in V,$
                    \begin{align*}
                        & ((\lambda + \mu) \circ f)(x) = (\lambda + \mu)f(x) = \lambda f(x) + \mu f(x) \\ ={} & (\lambda \circ f)(x) + (\mu \circ f)(x) = ((\lambda \circ f) \oplus (\mu \circ f))(x).
                    \end{align*}

                    \item $\forall \lambda \in \mathbf{R}, f, g \in V,$
                    \begin{align*}
                        & (\lambda \circ (f \oplus g))(x) = \lambda((f \oplus g)(x)) = \lambda (f(x)+g(x)) = \lambda f(x) + \lambda g(x) \\ ={} & (\lambda \circ f)(x) + (\lambda \circ g)(x) = ((\lambda \circ f) \oplus (\lambda \circ g))(x).
                    \end{align*}
                \end{enumerate}
                所以$V$构成在此``加法''和``数乘''下的线性空间.
    \end{enumerate}
\end{solution}

在上例以及习题中我们可以看到很多特殊的线性空间,它们集合中的元素不一定是数或向量,运算也不一定是熟知的数的运算和向量的数乘,对这些空间我们需要学会熟练判断,从而加深对``在集合上定义运算''的理解.

\section{线性子空间}

我们首先介绍线性子空间的定义:
\begin{definition}[线性子空间] \index{xianxingkongjian!zi@线性子空间 (linear subspace), 子空间 (subspace)}
    设$W$是线性空间$V(\mathbf{F})$的非空子集,如果$W$对$V$中的运算也构成域$\mathbf{F}$上的线性空间,则称$W$是$V$的\term{线性子空间}(简称\term{子空间}).
\end{definition}

请一定注意定义中的非空子集,建议验证子空间时先验证非空. 接下来自然的问题便是,什么时候$V$的子集$W$对$V$中的运算也构成域$\mathbf{F}$上的线性空间?事实上这一条件是惊人地简单与美观的:
\begin{theorem}\label{thm:2:子空间判别}
    线性空间$V(\mathbf{F})$的非空子集$W$为$V$的子空间的充分必要条件是$W$对于$V(\mathbf{F})$的线性运算封闭.
\end{theorem}

这表明只要子空间非空且其中的元素满足对原空间的加法和数乘运算封闭即可构成原空间的子空间. 这一定理的证明也非常简单,必要性显然(构成线性空间必须满足运算封闭),充分性我们只需要作如下思考:
\begin{enumerate}
    \item 结合律、分配律运算律是一定不变的,例如我们回顾加法结合律的定义$a+(b+c)=(a+b)+c,\enspace\forall a,b,c\in V$,由于这一性质对于任意$V$中元素成立,则若$a,b,c\in W\subseteq V$也必有这一性质成立(更通俗而言就是子集$W$中的元素也是$V$中的,因此必然受$V$中运算性质的限制);

    \item 我们根据上面的原则对8条性质一一验证,发现加法单位元和逆元仍不能保证存在,因为这不仅与运算法则相关,更与集合中元素的存在相关——取子集可能使得加法单位元和逆元被拿掉. 但在定理要求的数乘封闭性下这是不可能的:由于$\mathbf{F}$是数域,因此所有有理数都是其子集,因此$0,-1\in\mathbf{F}$. $\forall \alpha\in V$,我们由于数乘封闭性可知,$0\cdot\alpha=0\in W$,$(-1)\cdot\alpha=-\alpha\in W$,因此$W$中也有加法单位元和逆元.
\end{enumerate}

证明具体书写见教材62--63页. 下面我们来看两个常见的例子体会子空间的判别方法:
\begin{example}\label{ex:2:常见子空间}
    回答下列关于子空间的判定问题:
    \begin{enumerate}
        \item \label{item:2:常见子空间:1}
              说明$\mathbf{R}[x]_2$是$\mathbf{R}[x]_3$的子空间;

        \item \label{item:2:常见子空间:2}
              判断$W_1=\left\{(x,y,z) \,\middle|\, \dfrac{x}{3}=\dfrac{y}{2}=z\right\},\enspace W_2=\{(x,y,z) \mid x+y+z=1,\enspace x-y+z=1\}$是否为$\mathbf{R}^3$的子空间;

        \item \label{item:2:常见子空间:3}
              (线性方程组的解)试说明齐次线性方程组$AX=0$的解集是线性空间$\mathbf{F}^n$的一个子空间,但非齐次线性方程组的解不再构成线性空间(因为加法运算不封闭,具体见教材P62的2.2节开头的例子以及P86习题3(3).
    \end{enumerate}
\end{example}

\begin{solution}
    \begin{enumerate}
        \item 只需证明$\mathbf{R}[x]_2 \subseteq \mathbf{R}[x]_3$,以及$\mathbf{R}[x]_2$对$\mathbf{R}[x]_3$中的加法和数乘封闭即可.

              $\forall v \in \mathbf{R}[x]_2$,可被写作$v=a+bx,a,b \in \mathbf{R}$. 又有$\mathbf{R}[x]_3=\{a+bx+cx^2,a,b,c \in \mathbf{R}\}$,取$c=0$,有$v=a+bx \in \mathbf{R}[x]_3$,因此$\mathbf{R}[x]_2 \subseteq \mathbf{R}[x]_3$.

              对于$\mathbf{R}[x]_3$中的加法和数乘:
              \[mv_1+nv_2=m(a_1+b_1x)+n(a_2+b_2x)=(ma_1+na_2)+(mb_1+nb_2)x \in \mathbf{R}[x]_3\]
              所以$\mathbf{R}[x]_2$是$\mathbf{R}[x]_3$的子空间.

        \item 对 $W_1$: 引入参数$t$,
              \[W_1=\left\{(3t,2t,t) \,\middle|\, \frac{x}{3} = \frac{y}{2} = z = t\right\}\]
              对于$\forall v_1, v_2 \in W_1, v_1 = (3t_1, 2t_1, t_1), v_2 = (3t_2, 2t_2, t_2)$,有
              \begin{align*}
                  av_1 + bv_2 & = (3at_1 + 3bt_2, 2at_1 + 2bt_2, at_1 + bt_2)           \\
                              & = (3(at_1 + bt_2), 2(at_1 + bt_2), at_1 + bt_2) \in W_1
              \end{align*}
              故$W_1$封闭,是 $\mathbf{R}^3$ 的子空间.

              对 $W_2$: 有反例. 取 $u_1 = (1, 0, 0), u_2 = (0, 0, 1) \in W_2$,但 $W_1 + W_2 = (1, 0, 1)$ 不满足 $x + y + z = 1$,故 $W_2$ 不封闭,不是 $\mathbf{R}^3$ 的子空间.

        \item 设齐次线性方程组 $AX=0$ 的解构成的集合是 $W_1$,$\forall X_1, X_2 \in W_1$,有 $AX_1 = AX_2 = 0$,所以 $\forall a, b \in \mathbf{F}$,
              \[A(a X_1 + b X_2) = A(a X_1) + A(b X_2) = a AX_1 + b AX_2 = 0\]
              故 $W_1$ 封闭,是 $\mathbf{F}^n$ 的子空间.

              设非齐次线性方程组 $AX = \beta,\enspace \beta \in \mathbf{F}^m,\enspace \beta \neq 0$ 的解构成的集合是 $W_2$,$\forall X_1, X_2 \in W_2$,有 $AX_1 = AX_2 = \beta$,所以 $A(X_1 + X_2) = AX_1 + AX_2 = 2\beta \neq \beta$. 故 $W_2$ 不封闭,不是 $\mathbf{F}^n$ 的子空间.
    \end{enumerate}
\end{solution}

上例中 \ref*{item:2:常见子空间:2} 表明过原点的直线/平面构成三维空间的子空间,不过原点的无法保持线性性. 事实上 \ref*{item:2:常见子空间:2} 和 \ref*{item:2:常见子空间:3} 在表述同一个问题,\ref*{item:2:常见子空间:2} 从几何角度描述了 \ref*{item:2:常见子空间:3} 中齐次/非齐次线性方程组的解集. 事实上,在定义了子空间后, 如果一个线性空间的子集也构成线性空间,我们就可以对其进行同样的研究. 这一想法在我们后续的内容中十分重要, 现在需要大家先熟知子空间的定义和判别.

最后我们需要注意一个名词的定义. 线性空间有两个子空间称为平凡子空间,即仅含零元的子集$\{0\}$和其自身$V$. 而其它子空间称为非平凡子空间.

\section{线性表示 \quad 线性扩张}

在高中平面向量的学习中我们知道,两个单位向量$(0,1)$和$(1,0)$可以表示出整个平面的所有向量,高中我们也称这样的向量为平面向量的基底. 接下来我们将二维平面扩展至任意线性空间,同样讨论有关于``表示''、``基底''的问题.

我们首先来看线性组合和线性表示的概念:
\begin{definition}
    设$V(\mathbf{F})$是一个线性空间,$\alpha_i\in V,\enspace\lambda_i\in \mathbf{F}\enspace(i=1,2,\ldots,m)$,则向量$\alpha=\lambda_1\alpha_1+\lambda_2\alpha_2+\cdots+\lambda_m\alpha_m$称为向量组$\alpha_1,\alpha_2,\ldots,\alpha_m$在域$\mathbf{F}$的线性组合,或说$\alpha$在域$\mathbf{F}$上可用向量组$\alpha_1,\alpha_2,\ldots,\alpha_m$线性表示.
\end{definition}
这和我们高中所学的用向量的基底表示其他向量是完全一致的. 基于此,我们给出线性扩张的定义:
\begin{definition}[线性扩张] \index{xianxingkuozhang@线性扩张 (linear span)}
    设$S$是线性空间$V(\mathbf{F})$的非空子集,我们称
    \[ \spa(S)=\{\lambda_1\alpha_1+\cdots+\lambda_k\alpha_k \mid \lambda_1,\ldots,\lambda_k\in\mathbf{F},\enspace\alpha_1,\ldots,\alpha_k\in S,\enspace k\in\mathbf{N}_+\} \]
    为$S$的\term{线性扩张},即$S$中所有有限子集在域$\mathbf{F}$上的一切线性组合组成的$V(\mathbf{F})$的子集.
\end{definition}
注意,$\spa$参考的是《线性代数应该这样学》的记号,《大学数学——代数与几何》中使用$L$表示线性扩张. 考虑到本讲义记号统一性,我们采用更加常用并且不会与之后其它定义的记号冲突的$\spa$.

下面的定理告诉我们可以通过线性扩张构造子空间:
\begin{theorem}\label{thm:2:线性扩张构造子空间}
    线性空间$V(\mathbf{F})$的非空子集$S$的线性扩张$\spa(S)$是$V$中包含$S$的最小子空间.
\end{theorem}
仍然利用平面向量进行直观的理解,平面(也显然在平面向量加法和数乘下构成线性空间)$\mathbf{R}^2$可以由向量$(1,0)$和$(0,1)$扩张而成. 由这一定理的结果我们可以将一个向量组的线性扩张称为向量组的张成空间. 这一定理的证明思想非常重要,因此在此给出:

\begin{proof}
    \begin{enumerate}
        \item 首先我们证明$\spa(S)$是$V$的子空间.
              \begin{enumerate}
                  \item $\spa(S)$非空:由于$S$非空,且$S\subseteq\spa(S)$显然成立:取$\lambda=1,\enspace\forall s\in S,\enspace \lambda s=s\in\spa(S)$. 因此$\spa(S)$非空;

                  \item 设$\alpha,\beta\in\spa(S)$,则存在$\lambda_1,\ldots,\lambda_k\in\mathbf{F},\enspace \alpha_1,\ldots,\alpha_k\in S,\enspace\mu_1,\ldots,\mu_l\in\mathbf{F},\enspace\beta_1,\ldots,\beta_l\in S$,使得
                        \begin{gather*}
                            \alpha=\lambda_1\alpha_1+\cdots+\lambda_k\alpha_k \\
                            \beta=\mu_1\beta_1+\cdots+\mu_l\beta_l
                        \end{gather*}
                        因此我们可以得到$\spa(S)$
                        \begin{enumerate}
                            \item 关于加法封闭:$\alpha+\beta=\lambda_1\alpha_1+\cdots+\lambda_k\alpha_k+\mu_1\beta_1+\cdots+\mu_l\beta_l\in\spa(S)$;

                            \item 关于数乘封闭:$\lambda\alpha=\lambda\lambda_1\alpha_1+\cdots+\lambda\lambda_k\alpha_k\in\spa(S)$(数域关于乘法运算封闭,故$\lambda\lambda_i\in\mathbf{F},\enspace i=1,\ldots,k$).
                        \end{enumerate}
              \end{enumerate}
              综上,$\spa(S)$是$V$的子空间;

        \item 接下来我们证明$\spa(S)$是包含$S$的最小子空间. 设$W$是$V$的任一子空间,我们只需证明$\spa(S)\subseteq W$.

              事实上,类似于前面$S\subseteq\spa(S)$的证明我们有$S\subseteq W$,故$S$中元素都在$W$中. 且由\autoref{thm:2:子空间判别} 可知子空间中元素一定关于加法、数乘封闭,因此$\forall {\alpha}=\lambda_1\alpha_1+\cdots+\lambda_k\alpha_k\in\spa(S)$. 由于$\alpha_1,\ldots,\alpha_k\in S\subseteq W,\enspace\lambda_1,\ldots,\lambda_k\in\mathbf{F}$,因此$\alpha\in W$,从而$\spa(S)\subseteq W$,由此得证.
    \end{enumerate}
\end{proof}

上述证明的重要性在于,我们在这一个证明中练习了子集的证明方法、子空间的充要条件以及对于``最小''问题证明的一般方法. 希望读者能掌握其中的每一个思想与技巧. 此外,这一定理有很强的直观性,因为线性扩张实际上就是将子集中的元素进行无穷次重复的线性组合,将所有可能经过线性运算获得的向量都生成了,因此线性扩张的结果一定保障了线性运算的封闭.

最后我们再说明有限维线性空间和无限维线性空间的定义,本课程研究的内容都在有限维线性空间,如果少数时间拓展至无限维空间我们会给出说明:
\begin{definition}
    $V(\mathbf{F})$称为有限维线性空间,如果$V$中存在一个有限子集$S$使得$\spa(S)=V$,反之称为无限维线性空间.
\end{definition}

\begin{example}
    证明:$\mathbf{R}[x]_3$是有限维线性空间,$\mathbf{R}[x]$是无限维线性空间.
\end{example}

\begin{proof}
    \begin{enumerate}
        \item 显然$\mathbf{R}[x]_3$的有限子集$S=\{1,x,x^2\}$可以张成$\mathbf{R}[x]_3$,因此$\mathbf{R}[x]_3$是有限维线性空间;

        \item 对于$\mathbf{R}[x]$,我们只需证明其任意有限子集都无法张成其本身. 我们取其任意有限子集,则其中多项式元素的次数一定有最大值,我们记为$m$,那么$z^{m+1}$以及更高次数的无法被表示,因此$\mathbf{R}[x]$是无限维线性空间.
    \end{enumerate}
\end{proof}

\vspace{2ex}
\centerline{\heiti \Large 内容总结}

本讲我们追随着第一讲最末尾关于线性方程组为什么无解、有唯一解或无穷解的问题,展开我们对线性方程组一般理论的讨论. 我们首先通过一个例子引入我们为什么要研究线性空间——因为我们需要了解向量之间的关联,从直觉上这与线性方程组解的情况是有关系的. 我们给出了线性空间的定义——其核心仍然是在集合上定义满足一定条件的运算,事实上就是对我们高中就熟知的向量加法数乘规则的抽象,然后我们讨论了基于这一公理化的定义我们可以得到的性质. 我们介绍了线性空间的子空间的定义与判别方法,引入了线性表示、线性扩张的概念并说明了我们如何通过线性扩张得到子空间——这一定理蕴含着所谓``闭包''的思想,我们将在未来讨论仿射子集时再次见到,实际上是非常符合几何直观的.

事实上,这一讲的内容是比较抽象的,因为线性空间的定义实际上就是将我们熟知的向量加法数乘运算抽象出来,从而适用于所有有类似结构的集合,因此读者在学习时可能会自动带入一些高中平面向量的直观,然后发现显然的问题不用证,复杂的问题摸不着头脑,但读者应当在未竟专题一中训练了基于定义和公理的数学证明思想,我们也尽力给出大量经典的例子,将推导过程写得非常详细,所以整体而言思路应当是清晰易懂的.

\vspace{2ex}
\centerline{\heiti \Large 习题}

\vspace{2ex}
{\kaishu 1520年以来,全世界只有85个机构存活至今,其中50家是大学. 大学依靠梦想、希望生存下去——这就是大学的历史.}
\begin{flushright}
    \kaishu
    ——美国哥伦比亚大学校长L·C·柏林格
\end{flushright}

\centerline{\heiti A组}
\begin{enumerate}
    \item 检验下列集合对指定的加法和数乘运算是否构成实数域上的线性空间.
          \begin{enumerate}
              \item 有理数集$\mathbf{Q}$对普通的数的加法和乘法;

              \item 集合$\mathbf{R}^2$对通常的向量加法和如下定义的数量乘法:$\lambda\cdot(x,y)=(\lambda x,y)$;

              \item $\mathbf{R}_+^n$(即$n$元正实数向量)对如下定义的加法和数乘运算:
                    \begin{gather*}
                        (a_1,\ldots,a_n)+(b_1,\ldots,b_n)=(a_1b_1,\ldots,a_nb_n) \\
                        \lambda\cdot(a_1,\ldots,a_n)=(a_1^\lambda,\ldots,a_n^\lambda)
                    \end{gather*}

              \item 请继续完成教材P86第二章习题第1题第(9)--(11)问关于函数的加法数乘定义线性空间的问题.
          \end{enumerate}

    \item 请完成教材P86--87第二章习题第3题. 第(5)问平常问题较多,实际上就是要判断满足一定条件的多项式是否构成子空间.
\end{enumerate}

\centerline{\heiti B组}
\begin{enumerate}
    \item 证明:已知线性空间$V(\mathbf{F})$,$\lambda,\lambda_1,\cdots,\lambda_r\in\mathbf{F}$,$\beta,\alpha_1,\cdots,\alpha_r\in V$,有$\lambda\beta+\lambda_1\alpha_1+\lambda_2\alpha_2+\cdots+\lambda_r\alpha_r=\vec{0}$在$\lambda\neq 0$时的解为$\beta=-\lambda^{-1}\lambda_1\alpha_1-\lambda^{-1}\lambda_2\alpha_2-\cdots-\lambda^{-1}\lambda_r\alpha_r$.

    \item 设$V$是一个线性空间,$W$是$V$的子集,证明:$W$是$V$的子空间$\iff \spa(W)=W$.
\end{enumerate}

\centerline{\heiti C组}
\begin{enumerate}
    \item 设$\mathbf{E}$是域$\mathbf{F}$的一个子域.
        \begin{enumerate}
            \item 证明:$\mathbf{F}$关于自身的加法和乘法构成一个$\mathbf{E}$上的向量空间,并举一例;

            \item 举例说明:$\mathbf{E}\enspace(\mathbf{E}\neq \mathbf{F})$不是$\mathbf{F}$上的线性空间;

            \item 证明:若$V$是$\mathbf{F}$上的一个线性空间,则$V$也是$\mathbf{E}$上的一个线性空间.
        \end{enumerate}
    \item 考虑在第一章定义的有限域 $\mathbf{F}_4$ 和 $\mathbf{Z}_2$. 证明:$\mathbf{Z}_2$ 可以看作 $\mathbf{F}_4$ 的一个子域. 并给出 $\mathbf{F}_4$ 在 $\mathbf{Z}_2$ 上的线性空间结构. 验证 $\mathbf{Z}_p$ 一定是 $\mathbf{F}_{p^n}$ 的一个子域.
\end{enumerate}

\chapter{有限维线性空间}

在第二讲开头的\autoref{ex:2:线性空间引入} 中,我们讨论了齐次线性方程组解的个数与方程组系数矩阵行向量间没有可互相消去的关系之间的联系. 本节我们将这种``可互相消去的关系''进行形式化定义. 另一方面,在第二讲最后探讨线性扩张的概念时,一个很自然的问题便是:一个有限维线性空间最少可以由多少个向量线性扩张而来?循此路径,我们将在本讲探寻线性空间的最基本的结构属性.

\section{线性相关性}

\subsection{线性相关性的定义}

本节我们将形式化定义在引言中我们提到的``可相互消去的关系''——线性相关性,同时这一定义也可以解决引言中提到的关于有限维线性空间至少需要多少个向量张成的问题.
\begin{definition}[线性相关性] \index{xianxingxiangguan@线性相关 (linearly dependent)} \index{xianxingwuguan@线性无关 (linearly independent)}
    设$V(\mathbf{F})$是一个线性空间,$\alpha_1,\alpha_2,\ldots,\alpha_m\in V$,若存在不全为0的$\lambda_1,\lambda_2,\ldots,\lambda_m\in\mathbf{F}$,使得
    \[\lambda_1\alpha_1+\lambda_2\alpha_2+\cdots+\lambda_m\alpha_m=0\]
    成立,则称$\alpha_1,\alpha_2,\ldots,\alpha_m$\term{线性相关},否则称\term{线性无关}(即系数只能为0).
\end{definition}

很显然,\autoref{ex:2:线性空间引入} 中的方程组1系数矩阵的三个行向量$\alpha_1,\alpha_2,\alpha_3$满足$\alpha_1+\alpha_2-\alpha_3=0$,因此满足线性相关的定义,方程组2的系数矩阵三个行向量$\beta_1,\beta_2,\beta_3$的线性组合则只有$0\cdot\beta_1+0\cdot\beta_2+0\cdot\beta_3$等于0,因此符合线性无关的定义.

事实上,直接由定义我们还可以导出以下关于零向量的结论:
\begin{enumerate}
    \item 线性空间中单个向量$\alpha$线性相关的充要条件是$\alpha$为零向量;

    \item 任何含零向量的向量组都线性相关.
\end{enumerate}

需要注意的是,很多时候线性相关和线性无关的证明就是基于定义,请务必牢牢掌握. 我们先来看几个基本的例子:
\begin{example}
    \begin{enumerate}[label=(\arabic*)]
        \item 判断$\mathbf{R}^3$中向量$(1,1,0),(0,1,1),(1,0,-1)$的线性相关性;

        \item 判断$\mathbf{R}^3$中向量$(1,-3,1),(-1,2,-2),(1,1,3)$的线性相关性;

        \item \label{item:3:线性相关性:3}
              判断$\mathbf{R}[x]_3$中$p_1(x)=1+x,\ p_2(x)=1-x,\ p_3(x)=x+x^2$的线性相关性;

        \item 判断连续函数全体构成的线性空间中$1,\ \sin^2x,\ \cos^2x$的线性相关性;

        \item \label{item:3:线性相关性:5}
              判断连续函数全体构成的线性空间中$1,\ 2^x,\ 2^{-x}$的线性相关性.
    \end{enumerate}
\end{example}

\begin{solution}
    \begin{enumerate}
        \item 根据定义,应求解方程
              \[\lambda_1(1,1,0) + \lambda_2(0,1,1) + \lambda_3(1,0,-1) = 0,\]
              即
              \[ \begin{cases}
                      \lambda_1 + \lambda_3 = 0 \\
                      \lambda_1 + \lambda_2 = 0 \\
                      \lambda_2 - \lambda_3 = 0
                  \end{cases} \]
              解得基础解系$k(1,-1,-1)^\mathrm{T}$,所以存在非零解,向量组线性相关.

        \item 求解方程
              \[\lambda_1(1,-3,1) + \lambda_2(-1,2,-2) + \lambda_3(1,1,3) = 0,\]
              即
              \[ \begin{cases}
                      \lambda_1 - \lambda_2 + \lambda_3 = 0    \\
                      -3\lambda_1 + 2\lambda_2 + \lambda_3 = 0 \\
                      \lambda_1 - 2\lambda_2 + 3\lambda_3 = 0
                  \end{cases} \]
              解得 $\lambda_1 = \lambda_2 = \lambda_3 = 0$,此向量组线性无关.

        \item 求解方程
              \begin{align*}
                  \lambda_1p_1(x) + \lambda_2p_2(x) + \lambda_3p_3(x)                           & = 0 \\
                  (\lambda_1 + \lambda_2) + (\lambda_1 - \lambda_2 + \lambda_3)x + \lambda_3x^2 & = 0
              \end{align*}
              所以需要求解方程组
              \[ \begin{cases}
                      \lambda_1 + \lambda_2 = 0             \\
                      \lambda_1 - \lambda_2 + \lambda_3 = 0 \\
                      \lambda_3 = 0
                  \end{cases} \]
              解得 $\lambda_1 = \lambda_2 = \lambda_3 = 0$,此向量组线性无关.

        \item 易知 $-1 + \sin^2x + \cos^2x = 0$,对应系数为 $-1, 1, 1$,不全为零,所以此向量组线性相关.

        \item 求解方程
              \[\lambda_1 + \lambda_2·2^{-x} + \lambda_3·2^x = 0\]
              很明显会发现仅凭此方程是难以求解的,方程数目不足. 注意到此方程应该对于任意的 $x$ 均成立,所以取 $x = 0, x = 1, x = -1$,得到方程组
              \[ \begin{cases}
                      \lambda_1 + \lambda_2 + \lambda_3 = 0              \\[1ex]
                      \lambda_1 + \dfrac{1}{2}\lambda_2 + 2\lambda_3 = 0 \\[1ex]
                      \lambda_1 + 2\lambda_2 + \dfrac{1}{2}\lambda_3 = 0
                  \end{cases} \]
              解得 $\lambda_1 = \lambda_2 = \lambda_3 = 0$,此向量组线性无关.
    \end{enumerate}
\end{solution}
注意上述 \ref*{item:3:线性相关性:3} 到 \ref*{item:3:线性相关性:5} 题为不能代入特殊的$x$值来说明,例如 \ref*{item:3:线性相关性:3} 令$x=0$得到线性相关的做法是错误的,因为\ref*{item:3:线性相关性:3} 中线性空间就是多项式构成的线性空间,其中的元素就是多项式,不能代入值. 注意 \ref*{item:3:线性相关性:5} 是特殊题型,需要构造更多的方程来求解这一问题.

\subsection{线性相关性的定理}

实际上,除了定义之外,线性相关性还有大量的等价描述. 我们将在本节介绍常见的等价描述,它们是理解线性空间结构等后续内容的基础,因此希望读者对以下结论及其证明十分熟练并且要有深刻的理解. 我们的主线思路是从不同方面理解线性相关性:
\begin{enumerate}
    \item 从线性组合看(定义)

          向量组线性相关$\iff$它们有系数不全为0的线性组合等于零向量;

          向量组线性无关$\iff$它们只有系数全为0的线性组合才会等于零向量.

    \item 从线性表示看(教材定理2.3)
          \begin{theorem}
              线性空间$V(\mathbf{F})$中的向量组$\alpha_1,\alpha_2,\ldots,\alpha_m\enspace(m \geqslant 2)$线性相关的充分必要条件是$\alpha_1,\alpha_2,\ldots,\alpha_m$中有一个向量可由其余向量在域$\mathbf{F}$上线性表示.
          \end{theorem}
          这一定理等价描述为,向量组线性无关的充分必要条件是其中的向量无法互相表示. 这是显然的,因为向量组能互相表示利用定义可以轻松写出非零系数的线性表示. 总结一下即为:

          向量组线性相关$\iff$其中至少有一个向量可以由其余向量线性表示;

          向量组线性无关$\iff$其中每一个向量都不能由其余向量线性表示.

    \item 从齐次线性方程组看(教材P66例3,实际上这一点与定义十分类似)

          列向量组$\alpha_1,\alpha_2,\ldots,\alpha_m$线性相关$\iff$齐次线性方程组$x_1\alpha_1+x_2\alpha_2+\cdots+x_m\alpha_m=0$有非零解;

          列向量组$\alpha_1,\alpha_2,\ldots,\alpha_m$线性无关$\iff$齐次线性方程组$x_1\alpha_1+x_2\alpha_2+\cdots+x_m\alpha_m=0$只有零解.

    \item 从向量组与它的部分组的关系看(教材P67例6)

          如果向量组的一个部分组线性相关,那么整个向量组也线性相关;

          如果向量组线性无关,那么它的任何一个部分组也线性无关.

    \item 从向量组线性表示一个向量的方式看(教材定理2.4)
          \begin{theorem}\label{thm:3:线性无关等价表示唯一}
              若向量组$\alpha_1,\alpha_2,\ldots,\alpha_m$线性无关,而向量组$\beta,\alpha_1,\alpha_2,\ldots,\alpha_m$线性相关,则$\beta$可由$\alpha_1,\alpha_2,\ldots,\alpha_m$线性表示,且表示法唯一.
          \end{theorem}
          这一定理证明十分经典,特别是唯一性的证明需要掌握,因此此处我们给出证明:

          \begin{proof}
              由于向量组$\beta,\alpha_1,\alpha_2,\ldots,\alpha_m$线性相关,故存在不全为0的$\lambda_0,\lambda_1,\ldots,\lambda_m$使得
              \begin{equation}\label{eq:3:线性无关等价定理}
                  \lambda_0\beta+\lambda_1\alpha_1+\lambda_2\alpha_2+\cdots+\lambda_m\alpha_m=0,
              \end{equation}
              其中$\lambda_0$必不为0,因为如果将$\lambda_0=0$代入\autoref{eq:3:线性无关等价定理},则由于向量组$\alpha_1,\alpha_2,\ldots,\alpha_m$线性无关,必有$\lambda_1=\lambda_2=\cdots=\lambda_m=0$,与$\lambda_0,\lambda_1,\ldots,\lambda_m$不全为0的假设矛盾.

              因此我们有
              \[\beta=-\frac{\lambda_1}{\lambda_0}\alpha_1-\frac{\lambda_2}{\lambda_0}\alpha_2-\cdots-\frac{\lambda_m}{\lambda_0}\alpha_m.\]
              由此我们知道$\beta$可由$\alpha_1,\alpha_2,\ldots,\alpha_m$线性表示. 接下来我们证明表示方式的唯一性. 假设有两种表示方法:
              \begin{gather*}
                  \beta=\mu_1\alpha_1+\mu_2\alpha_2+\cdots+\mu_m\alpha_m, \\
                  \beta=\nu_1\alpha_1+\nu_2\alpha_2+\cdots+\nu_m\alpha_m.
              \end{gather*}
              两式相减可得
              \[0=(\mu_1-\nu_1)\alpha_1+(\mu_2-\nu_2)\alpha_2+\cdots+(\mu_m-\nu_m)\alpha_m.\]
              由于$\alpha_1,\alpha_2,\ldots,\alpha_m$线性无关,因此$\mu_i-\nu_i=0\enspace(i=1,2,\ldots,m)$,即$\mu_i=\nu_i\enspace(i=1,2,\ldots,m)$,因此表示方式唯一.
          \end{proof}

          事实上关于这一定理我们有一个直接的推论
          \begin{corollary}
              若向量组外另一向量可由这一组向量线性表示,则
              \begin{enumerate}
                  \item \label{item:3:线性无关等价表示唯一:1}
                        向量组线性无关$\iff$表示方式唯一;

                  \item \label{item:3:线性无关等价表示唯一:2}
                        向量组线性相关$\iff$表示方式有无穷多种.
              \end{enumerate}
          \end{corollary}
          推论的证明非常简单,此处考虑到读者可能处于初学阶段,给出证明范例:

          \begin{proof}
              我们设向量组为$\alpha_1,\alpha_2,\ldots,\alpha_m$,向量组外的向量为$\beta$. 对于 \ref*{item:3:线性无关等价表示唯一:1},向量组线性无关$\implies$表示方式唯一就是\autoref{thm:3:线性无关等价表示唯一} 的直接结论,因此我们只需考虑表示方式唯一$\implies$向量组线性无关. 利用反证法,假设向量组线性相关,则存在不全为0的$\lambda_1,\lambda_2,\ldots,\lambda_m$使得
              \begin{equation}\label{eq:3:线性无关等价推论1}
                  0=\lambda_1\alpha_1+\lambda_2\alpha_2+\cdots+\lambda_m\alpha_m.
              \end{equation}
              由于$\beta$可由$\alpha_1,\alpha_2,\ldots,\alpha_m$线性表示,因此存在$\mu_1,\mu_2,\ldots,\mu_m$使得
              \begin{equation}\label{eq:3:线性无关等价推论2}
                  \beta=\mu_1\alpha_1+\mu_2\alpha_2+\cdots+\mu_m\alpha_m.
              \end{equation}
              事实上,我们只需将\autoref{eq:3:线性无关等价推论1} 两边乘以任意的$k\in\mathbf{F}$($\mathbf{F}$为向量组所在线性空间定义的数域),然后加到\autoref{eq:3:线性无关等价推论2} 的两边即可得到
              \[\beta=(\mu_1+k\lambda_1)\alpha_1+(\mu_2+k\lambda_2)\alpha_2+\cdots+(\mu_m+k\lambda_m)\alpha_m.\]
              因此表示方式不唯一(且有无穷多种),与假设矛盾,因此向量组线性无关. 事实上这一证明也将 \ref*{item:3:线性无关等价表示唯一:2} 中向量组线性无关$\implies$表示方式有无穷多种证明给出,\ref*{item:3:线性无关等价表示唯一:2} 的另一边同样用反证法可以回到 \ref*{item:3:线性无关等价表示唯一:1} 的证明,由此推论得证.
          \end{proof}
\end{enumerate}

\section{基与维数}

\subsection{引入:向量组的秩与极大线性无关组}

在上一节中我们介绍了很基本的线性无关的等价表述,现在我们回到我们的主线,即我们希望解决有限维线性空间至少需要多少个向量张成的问题,接下来的讨论将逐步逼近问题的答案.
\begin{lemma}\label{lem:3:线性相关性引理}
    设$\alpha_1,\alpha_2,\ldots,\alpha_m$线性相关,则有$j\in\{1,2,\ldots,m\}$使得:
    \begin{enumerate}
        \item \label{item:3:线性相关性引理:1}
              $\alpha_j \in \spa(\alpha_1,\alpha_2,\ldots,\alpha_{j-1})$;

        \item \label{item:3:线性相关性引理:2}
              从$\alpha_1,\alpha_2,\ldots,\alpha_m$中删去向量$\alpha_j$,剩余向量张成空间仍等于$\spa(\alpha_1,\alpha_2,\ldots,\alpha_m)$.
    \end{enumerate}
\end{lemma}
可能大家看见 \ref*{item:3:线性相关性引理:1} 的记号可能又有些许陌生了,但只需简单回顾线性扩张的定义,我们知道证明\ref*{item:3:线性相关性引理:1} 就是证明$\alpha_j$可以被$\alpha_1,\alpha_2,\ldots,\alpha_{j-1}$线性表示. 这一结论初看和\autoref{thm:3:线性无关等价表示唯一} 很类似,但细看发现不太一样:我们要求必须有一个向量可以由排列在它前面的向量线性表示,而非被其余所有向量线性表示. 因此这一结论并不平凡,证明的过程中也有一个技巧,我们给出证明供读者参考学习:

\begin{proof}
    由于$\alpha_1,\alpha_2,\ldots,\alpha_m$线性相关,因此存在不全为0的$\lambda_1,\lambda_2,\ldots,\lambda_m$使得
    \[\lambda_1\alpha_1+\lambda_2\alpha_2+\cdots+\lambda_m\alpha_m=0.\]

    设$j$是$1,2,\ldots,m$中使得$\lambda_j\neq 0$的最大者,则有
    \begin{equation}\label{eq:3:线性相关性引理}
        \alpha_j=-\frac{\lambda_1}{\lambda_j}\alpha_1-\frac{\lambda_2}{\lambda_j}\alpha_2-\cdots-\frac{\lambda_{j-1}}{\lambda_j}\alpha_{j-1}.
    \end{equation}
    因此$\alpha_j$可由$\alpha_1,\alpha_2,\ldots,\alpha_{j-1}$线性表示,即$\alpha_j\in\spa(\alpha_1,\alpha_2,\ldots,\alpha_{j-1})$,故 \ref*{item:3:线性相关性引理:1} 得证.

    接下来我们证明 \ref*{item:3:线性相关性引理:2}. 首先$\spa(\alpha_1,\ldots,\alpha_{j-1},\alpha_{j+1},\ldots,\alpha_m)\subseteq\spa(\alpha_1,\alpha_2,\ldots,\alpha_m)$是显然的,因为任意被$\alpha_1,\ldots,\alpha_{j-1},\alpha_{j+1},\ldots,\alpha_m$线性表示的向量实际上也是被
    \[\alpha_1,\alpha_2,\ldots,\alpha_m\]
    线性表示了,只是$\alpha_j$前的系数恒为0.

    然后证明另一边包含关系,即
    \[\spa(\alpha_1,\alpha_2,\ldots,\alpha_m)\subseteq\spa(\alpha_1,\ldots,\alpha_{j-1},\alpha_{j+1},\ldots,\alpha_m).\]
    任取$\beta\in\spa(\alpha_1,\ldots,\alpha_m)$,则存在$\mu_1,\mu_2,\ldots,\mu_m$使得
    \[\beta=\mu_1\alpha_1+\mu_2\alpha_2+\cdots+\mu_m\alpha_m.\]
    将$\alpha_j$用\autoref{eq:3:线性相关性引理} 表示,代入上式可得任意$\spa(\alpha_1,\ldots,\alpha_m)$中的向量都可以由\[\alpha_1,\ldots,\alpha_{j-1},\alpha_{j+1},\ldots,\alpha_m\]线性表示,因此$\beta\in\spa(\alpha_1,\alpha_2,\ldots,\alpha_{j-1},\alpha_{j+1},\ldots,\alpha_m)$,故引理得证.
\end{proof}

事实上 \ref*{item:3:线性相关性引理:1} 中证明最核心的步骤就是取$j$是$1,2,\ldots,m$中使得$\lambda_j\neq 0$的最大者,这一最大者是一定存在的,因为首先存在$\lambda_i\neq 0$,其次$\lambda_i\neq 0$的个数是有限的,因此一定存在最大者. 这一证明的技巧十分重要,通俗的记忆方法为``从右往左检查,找到第一个不为0的系数(即最大的不为0的系数)''. 我们给出一个推论,推论的证明思想就是如此,我们放在习题中供读者练习:
\begin{corollary}
    $\alpha_1,\alpha_2,\ldots,\alpha_m$线性相关(其中$\alpha_1\neq 0$)的充要条件是存在一个向量$\alpha_i\enspace(1<i\neq m)$可由$\alpha_1,\alpha_2,\ldots,\alpha_{i-1}$线性表示,且表示法唯一.
\end{corollary}
事实上这一推论也可以作为线性无关的等价表述之一.

接下来我们继续我们的主线思路,事实上\autoref{lem:3:线性相关性引理} 的 \ref*{item:3:线性相关性引理:2} 给我们了一个很重要的启示,即对于线性相关的向量组,我们丢弃其中某些(可以被其他向量线性表示)的向量后,张成的空间是不变的. 因此我们可以重复丢弃这样的向量,并仍然保持张成空间不变. 一个自然的问题是,这样丢弃的操作直到什么时候停止呢?

事实上答案也是非常自然的,即我们最后一次从向量组中丢弃向量(并保证张成的空间不变)后,剩余的向量组恰好线性无关时即可停止丢弃. 原因非常简单,因为如果这最后一次不丢弃,则根据\autoref{lem:3:线性相关性引理} 我们一定还能选出一个向量,使得丢弃这一向量后仍能保持张成空间不变. 但一旦丢弃向量后向量组线性无关,这时一定不能继续丢弃,例如这时剩余的线性无关向量组为$\beta_1,\ldots,\beta_m$,这时丢弃其中任意一个$\beta_i,\enspace i\in\{1,2,\ldots,m\})$,则原向量组张成的空间中,至少$\beta_i$无法被剩余向量组线性表示(否则$\beta_i$可以被$\beta_1,\ldots,\beta_{i-1},\beta_{i+1},\ldots,\beta_m$线性表示,则$\beta_1,\ldots,\beta_m$必线性相关),因此我们一定不能继续丢弃.

在上述过程中我们可以引入两个重要的概念,即向量组的秩和极大线性无关组:
\begin{definition}
    设向量组$S=\{\alpha_1,\alpha_2,\ldots,\alpha_m\}$张成的线性空间为$V$,若存在$S$的一个线性无关向量组$B=\{\alpha_{k1},\alpha_{k2},\ldots,\alpha_{kr}\}$,使得$V=\spa(B)$,则称$B$为$S$的一个\term{极大线性无关组}\index{jidaxianxing@极大线性无关组 (maximal linearly independent system)},并称极大线性无关组的长度$r=r(S)$为$S$的\term{秩}\index{zhi@秩 (rank)}.
\end{definition}
定义中``极大''一词我们只需简单思考前述过程即可明白其含义,因为我们要求丢弃后的向量组一旦线性无关就要停止继续丢弃向量,因此这一剩余向量组的长度一定是所有线性无关向量组中最大的.

要注意的是,极大线性无关组在本讲义、教材甚至其它教材(如丘维声老师的《高等代数》)中的定义都有所不同,实际上不同的版本只是为了顺应不同讲解思路而提出的,本质上并无区别,相信读者在完全理解本节内容后能认识到这一点.

由此我们关于有限维线性空间至少需要多少个向量张成的问题有了初步的解答,即如果我们已知这一线性空间是可以由某一向量组张成的,那么这一向量组的秩(即极大线性无关组的长度)就是张成空间需要的最少向量个数. 可能初看这一段话,其中出现的``极大''和``最小''容易导致思维的混乱,但我们可以用一句话清晰地总结:极大线性无关组的长度就是张成空间需要的最少向量个数(如果仍然混乱,我们可以回忆丢弃向量的过程:我们不断丢弃向量得到``最小''的仍然满足张成空间不变的向量组,而这一向量组必须是所有线性无关向量组中最长的,因为向量组丢到线性无关后不能再丢了).

\subsection{向量组的性质}

事实上,我们会有一个自然的疑问,即极大线性无关组的长度是否唯一?我们在丢弃向量的时候,如果向量的排序不同,我们丢弃的次序也可能不同,因此我们最终得到的极大线性无关组是有可能不同的. 但长度不同表明向量组的秩不唯一,这样向量组的秩就失去了很多研究价值——数学喜欢唯一确定的,例如数学分析中表达式的极限不唯一我们会称其极限不存在;又例如定积分的值如果可以是不唯一的,那么我们一定会重新思考积分的定义,否则面积、体积甚至物理中的很多问题都会产生意义不明的多解.

因此我们需要尝试证明极大线性无关组的长度是唯一的,我们从下面这一非常重要的定理开始:
\begin{theorem}\label{thm:3:线性表示}
    设$V(\mathbf{F})$中向量组$ \beta_1,\beta_2,\ldots,\beta_s $的每个向量可由另一向量组$\alpha_1,\alpha_2,\ldots,\alpha_r$线性表示. 若$s>r$,则$ \beta_1,\beta_2,\ldots,\beta_s $线性相关.
\end{theorem}
这一定理的等价(逆否)命题为,$ \beta_1,\beta_2,\ldots,\beta_s $线性无关则必有$s\leqslant r$.

这一定理可通俗概括为:多的向量组可以被少的向量组线性表示,多的一定线性相关. 反过来说,线性无关的向量只能被等长或更长的向量组线性表示. 定理的证明思想上非常简单,但写起来可能有些许复杂,我们给出证明:

\begin{proof}
    设 $\beta_j = \displaystyle\sum_{i = 1}^r \lambda_{ij} \alpha_i,\enspace \lambda_{ij} \in \mathbf{F},\enspace j = 1, 2, \ldots, s$. 由线性相关的定义,再设
    \[x_1\beta_1 + x_2\beta_2 + \cdots + x_s\beta_s = 0,\]
    即
    \[\sum_{j = 1}^s x_j\beta_j = \sum_{j = 1}^s x_j\left(\sum_{i = 1}^r \lambda_{ij} \alpha_i\right) = \sum_{i = 1}^r \left(\sum_{j = 1}^s \lambda_{ij}x_j\right)\alpha_i = 0\]
    事实上我们现在只需证明存在一组不全为零的 $x_1, x_2, \ldots, x_s$ 使得上式成立即可,因为这就是线性相关的定义. 因此我们只需要找出这么一组不全为零的数即可,怎么寻找呢?我们发现若 $\alpha_i$ 前的系数均取 $0$,则此方程必然成立,我们看看这种情况下能不能找到一组解,事实上此时有
    \[\sum_{j = 1}^s \lambda_{ij}x_j = 0,\enspace i = 1, \ldots, r.\]
    此为关于 $x_1, x_2, \ldots, x_s$ 的齐次线性方程组,其方程个数 $r$ 小于未知数数量 $s$,因此此方程组必然有非零解,于是我们就找到了一组不全为零的 $x_1, x_2, \ldots, x_s$ 使式子成立,故 $ \beta_1,\beta_2,\ldots,\beta_s $ 线性相关.
\end{proof}

事实上,\autoref{thm:3:线性表示} 因其重要性又被称为源泉定理,因为我们可以基于此得到大量的推论,下面我们将给出几个简单的作为代表,习题中会出现更为复杂的应用:
\begin{example}\label{ex:3:线性表示推论}
    证明以下\autoref*{thm:3:线性表示} 的推论:
    \begin{enumerate}[label=(\arabic*)]
        \item 若向量组$B_1$可以被向量组$B_2$线性表示,则有$r(B_1)\leqslant r(B_2)$;

        \item \label{item:3:线性表示推论:2}
              设$B_1$和$B_2$是两个线性无关向量组,若$B_1$可以被$B_2$线性表示,$B_2$也可以被$B_1$线性表示,则$B_1$和$B_2$长度相等.
    \end{enumerate}
\end{example}

\begin{proof}
    \begin{enumerate}
        \item $B_1$ 可被其极大线性无关组 $A_1$ 表示,$B_2$ 可被其极大线性无关组 $A_2$ 表示,所以原条件等价于 $A_1$ 可以被 $A_2$ 线性表示. 而由极大线性无关组的定义,$A_1, A_2$ 中的向量个数分别是 $r(B_1), r(B_2)$,根据\autoref{thm:3:线性表示},有 $r(B_1)\leqslant r(B_2)$.

        \item 因为 $B_1, B_2$ 可以互相表示,所以 $r(B_1) \leqslant r(B_2),r(B_2) \leqslant r(B_1)$,所以 $r(B_1) = r(B_2)$. 又极大线性无关组的秩就是其向量个数,所以 $B_1, B_2$ 长度相等.
    \end{enumerate}
\end{proof}

事实上,\autoref{ex:3:线性表示推论} \ref*{item:3:线性表示推论:2} 中两个向量组$B_1$和$B_2$可以互相表示也可以称$B_1$和$B_2$等价. 这里的等价和\autoref{def:1:等价关系} 中描述的等价关系一致,即向量组等价同样满足自反性、对称性和传递性,即
\begin{enumerate}
    \item 自反性:任意向量组 $B$ 本身总是与自己等价,即向量组本身可以由本身表示;

    \item 对称性:设向量组 $B_1$ 等价于向量组 $B_2$,则向量组 $B_2$ 等价于向量组 $B_1$,因为它们可以相互表示;

    \item 传递性:设向量组 $B_1$ 等价于向量组 $B_2$,向量组 $B_2$ 等价于向量组 $B_3$,则向量组 $B_1$ 等价于向量组 $B_3$. 因为 $B_1$ 和 $B_2$ 可以相互表示,$B_2$ 和 $B_3$ 可以相互表示就有 $B_1$ 和 $B_3$ 可以相互表示.
\end{enumerate}
三个条件的成立是显然的,我们不再赘述,接下来我们基于等价向量组的定义给出\autoref{thm:3:线性表示} 的进一步结论,直至证明向量组的秩唯一:
\begin{corollary}
    关于等价的向量组,我们有如下结论:
    \begin{enumerate}
        \item 向量组与其极大线性无关组等价;

        \item 向量组的任意两个极大线性无关组等价;

        \item 向量组的任意两个极大线性无关组长度相等,即向量组的秩唯一.
    \end{enumerate}
\end{corollary}

\begin{proof}
    \begin{enumerate}
        \item 依据极大线性无关组的定义,并且注意到极大线性无关组是原向量组的子集即可;

        \item 设向量组 $B$ 的任意两个极大线性无关组为 $A_1, A_2$,由定义得 $B$ 可被 $A_1$ 表示,也可被 $A_2$ 表示,而 $A_1 \subseteq B, A_2 \subseteq B$,所以 $A_1, A_2$ 可以相互表示;

        \item 由上知向量组的任意两个极大线性无关组是等价的,结合\autoref{ex:3:线性表示推论} \ref*{item:3:线性表示推论:2} 即可得到二者长度相等,由向量组的秩的定义可知其唯一.
    \end{enumerate}
\end{proof}

由此我们证明了向量组的秩是唯一的,因此这一定义对我们将来的研究非常友好.

\subsection{基与维数}

在前几小节中,我们讨论了这一问题:给定向量组$B$,我们能否选出一个长度最小的向量组$B_1$使其张成的空间与$B$能张成的空间相同. 接下来我们讨论更一般化的情形,即我们不给定向量组$B$,直接讨论能张成一个线性空间的线性无关向量组.
\begin{definition}
    若线性空间$V(\mathbf{F})$的有限子集$B=\{\alpha_1,\alpha_2,\ldots,\alpha_n\}$线性无关,且$\spa(B) = V$,则称$B$为$V$的一组基,并称$n$为$V$的维数,记作$\dim V = n$.
\end{definition}

关于基与维数的定义,我们有以下几点需要强调:
\begin{enumerate}
    \item \label{item:3:基与维数:1}
          我们有一个自然的问题:有限维线性空间是否一定有基,若是,则上述定义的基和维数对所有有限维线性空间都是存在的. 事实上结论是显然的. 根据定义,有限维线性空间$V$一定能被其某一有限子集$S$张成,我们根据求取极大线性无关组的算法取出$S$的极大线性无关组$B$,则$B$一定是$V$的基.

    \item 由 \ref*{item:3:基与维数:1} 我们发现,基的存在依赖于极大线性无关组的存在,二者只是在定义上有差别:极大线性无关组是一个向量组的最短等价向量组,而基是张成线性空间的最短向量组. 但二者本质统一,实际上极大线性无关组就是它能张成的线性空间的一组基,其长度(向量组的秩)也就是线性空间的维数.

    \item 有限维线性空间的基不一定唯一,但它们的长度必定唯一(即维数唯一). 这一推导和向量组的秩唯一完全一致. 我们可以假设有限维线性空间$V$有两组基$B_1$和$B_2$,根据基的定义(即它们可以张成$V$,也就是可以表示出$V$中的所有向量). 因此$B_1$中的每一个向量都可以由$B_2$线性表示,反之亦然,因此$B_1$和$B_2$等价,由此我们可以得到$B_1$和$B_2$的长度相等,即因此有限维线性空间维数唯一.

    \item 我们还需要提及一个概念:自然基. 例如三维空间的自然基为$\vec{e}_1=(1,0,0),\vec{e}_2=(0,1,0),\vec{e}_3=(0,0,1)$. $n$维空间也有类似的推广(即$n$个只有一位为 1 其余全为 0 的向量. 此后若没有特殊说明,$\vec{e}_i$就表示$\mathbf{R}^n$第$i$位为1,其余位置为0的自然基). 对于多项式我们则将$1,x,x^2,\ldots$称为自然基,矩阵、函数等构成的线性空间也有相关的常用的基.

    \item \label{item:3:基与维数:5}
          基与维数的意义可以由这个性质反映出来:对一 $n$ 维线性空间 $V$ 而言,其中的任意 $n + 1$ 个向量必然线性相关,而其中的任意 $n - 1$ 个向量必然无法张成空间 $V$,这也是\autoref{thm:3:线性表示} 的直接推论.
\end{enumerate}

事实上,定义出基和维数之后我们对线性空间的研究方式就更明朗了:我们从开始的令人眼花缭乱的 8 条运算性质,利用这些线性运算的特点导出线性扩张与子空间的关联,然后经过线性相关性的讨论最终得到线性空间的本质结构实际上就是可以由基经过一系列线性运算扩张而来,因此我们对线性空间的研究很多时候只需要研究其基和维数即可,由此我们的抽象上升一层,即我们不需要观察线性空间中无限个向量,事实上只需要研究有限个向量的性质即可对整个线性空间有较为全面的了解. 实际上这一思想与之后我们得到矩阵等讨论是密切相关的,因此在我们整个向着对线性方程组解的结构的讨论的路径中也称得上是一块关键的里程碑.

我们经常会遇到验证线性空间的基的问题(求解基的题目最后往往也需要验证你写出的向量组确实是基),我们主要有如下两个角度:
\begin{enumerate}
    \item 根据定义,我们只需验证基的两个条件:线性无关和张成空间. 线性无关利用定义即可,张成空间则需要验证任意向量都可以由基线性表示.

    \item 若我们能确认线性空间$V$的维数$\dim V$,那么我们只需找到$\dim V$个线性无关的向量即可,因为它们必然是$V$的基. 这一结论的证明是容易的,在下面的例题中我们给出一个更一般的结论的证明供读者参考.
\end{enumerate}

\begin{example}
    在$n$维线性空间$V$中,$n$个向量$\alpha_1,\ldots,\alpha_n$线性无关的充要条件是它们可以线性表示出$V$中的任意向量.
\end{example}

\begin{proof}
    充分性:因为 $\alpha_1,\ldots,\alpha_n$ 可以线性表示出 $V$ 中的任意向量,所以 $V$ 的一组基 $\beta_1, \ldots, \beta_n$ 也能由 $\alpha_1,\ldots,\alpha_n$ 表示. 而由基的性质,$\alpha_1,\ldots,\alpha_n$ 又能被 $\beta_1, \ldots, \beta_n$ 表示,所以这两个向量组等价,$\alpha_1,\ldots,\alpha_n$ 的秩就是 $n$,所以 $\alpha_1,\ldots,\alpha_n$ 线性无关.

    必要性:由 \ref*{item:3:基与维数:5} 可知,$\forall \beta \in V, \alpha_1, \ldots, \alpha_n, \beta$ 必线性相关,又 $\alpha_1,\ldots,\alpha_n$,由\autoref{thm:3:线性无关等价表示唯一} 可知,$\beta$ 可以被 $\alpha_1,\ldots,\alpha_n$ 唯一表示,因此 $V$ 中的任意向量都可以被 $\alpha_1,\ldots,\alpha_n$ 线性表示.
\end{proof}

除此之外,我们也在此给出一些求解或验证线性空间的基和维数的基本例题,在习题以及后续章节中会有更多的例子.
\begin{example}\label{ex:3:不同数域的维数}
    证明:线性空间$\mathbf{C}(\mathbf{C})$维数为1,不同于线性空间$\mathbf{C}(\mathbf{R})$维数为2.
\end{example}

\begin{proof}
    对于线性空间$\mathbf{C}(\mathbf{C})$中的任一向量 $a + b\i$,其总可以被向量 $\alpha = 1$表示,数乘系数为 $\lambda = a + b\i$,所以$\mathbf{C}(\mathbf{C})$维数为1;对于线性空间$\mathbf{C}(\mathbf{R})$,向量组 $\alpha = 1, \beta = \i$线性无关,且任一向量 $u = a + b\i = a·\alpha + b·\beta$,可被 $\alpha, \beta$表示,所以$\mathbf{C}(\mathbf{R})$维数为2.
\end{proof}

\begin{example}
    证明:$1,(x-5)^2,(x-5)^3$是$\mathbf{R}[x]_4$的子空间$U$的一组基,其中$U$定义为
    \[U=\{p\in\mathbf{R}[x]_4 \mid p'(5)=0\}.\]
\end{example}

\begin{proof}
    易知$1,(x-5)^2,(x-5)^3$线性无关,且这三个向量都属于子空间$U$,下证$U$的维数是 3.

    设 $p = a + bx + cx^2 + dx^3 \in U$. 因为 $p'(5) = 0$,即 $b + 10c +75d = 0$,所以将$b$代入后有 $p = a + c·(-10x + x^2) + d·(-75x + x^3)$,因此$U$中任意向量可被$1, -10x + x^2, -75x + x^3$表示,所以$U$的维数是 3,进而由基的性质可知$1,(x-5)^2,(x-5)^3$是$U$的一组基.
\end{proof}

我们在后续讨论中经常会涉及子空间和原空间之间的关联,特别是它们的基之间的关联,下面这一定理能很好地满足我们的需求:
\begin{theorem}
    如果$W$是$n$维线性空间$V$的一个子空间,则$W$的基可以扩充为$V$的基.
\end{theorem}
这一定理的应用非常广泛,事实上笔者认为这一定理结论重要性高于证明,因此不在此给出证明,对证明感兴趣的读者可以参看教材70页.

实际上还有关于向量组的秩、基与维数有关的很多结论,事实上都可以由前述的定理推导而来,很多结论事实上都非常自然,我们将习题中展示. 考虑到本讲概念、定理内容多而杂. 我们在本讲最后也会给出一个思维导图,读者可以参考.

\subsection{极大线性无关组的求法}

我们在前述讲解中实际上已经给出一个求解极大线性无关组的方法,即不断丢弃线性相关的向量,最后一次从向量组中丢弃向量(并保证张成的空间不变)后,剩余的向量组恰好线性无关时即可停止丢弃. 但这一方法适用于证明极大线性无关组一定存在,如果考试中要求取极大线性无关组我们应当考虑教材71页给出的``通用而简便''的方法. 事实上教材中给出的方法以及解释已经非常细致,我们只总结其关键步骤,读者可以参考教材中进行细致的学习.
\begin{lemma}[极大线性无关组的求法]

    我们将题目给定的向量按列排成矩阵,然后将矩阵作初等变换化成阶梯矩阵,找到主元所在的列,提取出原矩阵对应列的向量即可.
\end{lemma}

注意极大线性无关组是不唯一的,但上面给出了一个程式化的方法. 实际上如果能一眼看出结果的也不必如此麻烦(当然题目直接要求极大线性无关组还是应当写具体过程的).
\begin{example}\label{ex:3:求解极大线性无关组}
    求向量组
    \[\{\alpha_1=(1,-1,2,4),\alpha_2=(0,3,1,2),\alpha_3=(3,0,7,14),\alpha_4=(1,-1,2,0),\alpha_5=(2,1,5,6)\}\]
    的极大线性无关组和秩.
\end{example}

\begin{solution}
    将向量排列成矩阵得
    \[ \begin{pmatrix}
            1  & 0 & 3  & 1  & 2 \\
            -1 & 3 & 0  & -1 & 1 \\
            2  & 1 & 7  & 2  & 5 \\
            4  & 2 & 14 & 0  & 6
        \end{pmatrix}
        \xLongrightarrow{\text{初等行变换}}
        \begin{pmatrix}
            1 & 0 & 3 & 1 & 2 \\
            0 & 1 & 1 & 0 & 1 \\
            0 & 0 & 0 & 1 & 1 \\
            0 & 0 & 0 & 0 & 0
        \end{pmatrix} \]
    由此可知向量组秩为 3,并分别提取出含有对应不重复主元的 3 个向量. 如 $\alpha_1, \alpha_2, \alpha_4$,即为极大线性无关组.
\end{solution}

学会求解极大线性无关组后,我们还能解决一个重要的问题,就是如何扩张一个线性无关向量组成为线性空间的一组基. 之前我们只说明了这样的扩张是存在的,但具体如何取到并没有给出. 虽然在未来实际应用中我们大部分时候可能只需要扩充一两个向量就行,很多时候我们随手取或者依靠之后的行列式等工具就很好解决. 但实践中我们发现很多同学在教材没给出固定算法的情况下完全无法接受``随手取''这样的描述,因此在此笔者还是给出一种虽然暴力但一定有效的算法.

设线性空间$V$维数为$n$,我们已有的线性无关向量组为$B=\{\alpha_1,\alpha_2,\ldots,\alpha_s\},\enspace s<n$. 我们的目标是将这个向量组扩充为$V$的一组基$B'=\{\alpha_1,\alpha_2,\ldots,\alpha_s,\alpha_{s+1},\ldots,\alpha_n\}$,我们的算法如下:
\begin{enumerate}
    \item 首先,如果$V$不是$\mathbf{F}^n$空间,我们取$B$在$V$的任意一组基下的坐标(如果有自然基最好取自然基方便计算);

    \item 任取$V$的一组基$B_0=\{\beta_1,\beta_2,\ldots,\beta_n\}$,这组基和前面取的是否一致无所谓(看了后面的例子就明白了). 我们得到了一个新的向量组$B_1=\{\alpha_1,\alpha_2,\ldots,\alpha_s,\beta_1,\beta_2,\ldots,\beta_n\}$;

    \item 求$B_1$的极大线性无关组即可,特别注意最后选向量的时候不能把$\alpha_1,\alpha_2,\ldots,\alpha_s$扔掉了,只能扔后面的向量,因为我们求的是从$\alpha_1,\alpha_2,\ldots,\alpha_s$扩充来的一组基;

    \item 最后将我们上面得到的坐标结合第一步取的$V$的基得到由$B$扩充而来的一组基.
\end{enumerate}

\begin{example}
    设$V=\mathbf{R}[x]_4$,我们已有向量组$B=\{1+x,x^3+x^2+3x\}$,请将其扩充为$V$的一组基.
\end{example}

\begin{solution}
    按讲义中的方法,取定 $\mathbf{R}[x]_4$ 的自然基,给出$B$中向量的坐标
    \[\alpha_1 = (1, 1, 0, 0), \alpha_2 = (0, 3, 1, 1).\]
    再取一组基
    \[\beta_1 = (1, 0, 0, 0), \beta_2 = (0, 1, 0, 0), \beta_3 = (0, 0, 1, 0), \beta_4 = (0, 0, 0, 1).\]
    将这 6 个向量排列成矩阵,求解极大线性无关组.
    \[ \begin{pmatrix}
            1 & 0 & 1 & 0 & 0 & 0 \\
            1 & 3 & 0 & 1 & 0 & 0 \\
            0 & 1 & 0 & 0 & 1 & 0 \\
            0 & 2 & 0 & 0 & 0 & 1
        \end{pmatrix}
        \xLongrightarrow{\text{初等行变换}}
        \begin{pmatrix}
            1 & 0 & 1  & 0 & 0 & 0  \\
            0 & 1 & 0  & 0 & 0 & 1  \\
            0 & 0 & -1 & 1 & 0 & -3 \\
            0 & 0 & 0  & 0 & 1 & -1
        \end{pmatrix} \]
    则可取 $\alpha_1, \alpha_2, \beta_1, \beta_3$作为$V$的一组基,即$\{1 + x, 3x + x^2 + x^3, 1, x^2\}$.
\end{solution}

\section{向量的坐标}

坐标的概念实际上我们已经熟悉,例如高中所学的平面向量的坐标表示就是向量在二维平面的基$(1,0),(0,1)$下的坐标表示. 我们现在将这个概念拓展到更一般的线性空间:
\begin{definition}
    设$B=\{\beta_1,\beta_2,\ldots,\beta_n\}$是$n$维线性空间$V(\mathbf{F})$的一组基,如果$V$中元素$\alpha$表示为$\alpha=a_1\beta_1+a_2\beta_2+\cdots+a_n\beta_n$,则其系数组$a_1,a_2,\ldots,a_n$称为$\alpha$在基$B$下的坐标,记为$\alpha_B=(a_1,a_2,\ldots,a_n)$.
\end{definition}

\begin{example}
    分别求$p(x)=a_0+a_1x+a_2x^2$在基$B_1=\{1,x,x^2\}$和$B_2=\{1,x-1,(x-1)^2\}$下的坐标.
\end{example}

\begin{solution}
    通过待定系数法解方程即可.

    在 $B_1$ 下:$(a_0, a_1, a_2)$. 在 $B_2$ 下:$(a_0 + a_1 + a_2, a_1 + 2a_2, a_2)$.
\end{solution}
关于向量的定义我们有以下几点需要强调:
\begin{enumerate}
    \item 若向量$\alpha$在基$\beta_1,\beta_2,\ldots,\beta_n$下的坐标为$\alpha_B=(a_1,a_2,\ldots,a_n)$,则我们也可以写为
          \[\alpha=(\beta_1, \beta_2, \ldots, \beta_n)\begin{pmatrix} a_1 \\ a_2 \\ \vdots \\ a_n \end{pmatrix}\]
          实际上我们在学习矩阵乘法后就会意识到这一记号是很自然的,因为这样的表示就等价于
          \[\alpha=a_1\beta_1+a_2\beta_2+\cdots+a_n\beta_n,\]
          但当前没有学习矩阵乘法,因此我们只能将其视为一种记号.

    \item 坐标与向量是一一对应的:一个坐标可以确定唯一的向量,一个向量在基下表示的系数也必然唯一(因为基是线性无关的);

    \item 坐标保持元素间的线性运算关系不变:$(\alpha+\beta)_B=\alpha_B+\beta_B$和$(\lambda\alpha)_B=\lambda\alpha_B$成立,例如$\mathbf{R}[x]_3$中的向量$\alpha=x^2+2x+1$和$\beta=2x^2+3x+1$,则$\alpha+\beta=3x^2+5x+2$,对应于向量运算,我们有$(\alpha+\beta)_B=(3,5,2)=(1,2,0)+(2,3,2)=\alpha_B+\beta_B$.

    \item 由以上两点我们可以知道:我们对各种各样的$n$维线性空间的研究都可以首先通过坐标转化为$\mathbf{F}^n$中的元素进行研究,例如
          \begin{example}\label{ex:3:转化为坐标}
              求$\mathbf{R}[x]_4$中向量组$\{p_1=x^3-x^2+2x+4,p_2=3x^2+x+2,p_3=3x^3+7x+14,p_4=x^3-x^2+2x,p_5=2x^3+x^2+5x+6\}$的极大线性无关组.
          \end{example}
          \begin{solution}
              我们首先将所有多项式先转化为坐标,然后就会发现和\autoref{ex:3:求解极大线性无关组} 完全一致,最后将坐标转回多项式即可.
          \end{solution}

          事实上,将任意的线性空间转化为$\mathbf{F}^n$研究的思想是非常重要的,因为这可以带来进一步的抽象,即我们甚至可以遮蔽线性空间基的特点,只关注其维数进行研究,这与此后线性空间的同构以及矩阵表示都有密不可分的联系. 事实上,我们一直都在使用这一基本思想,我们每次设线性空间有一组基$\alpha_1,\ldots,\alpha_n$时,事实上我们只关注其维数$n$而遮蔽了基的特点:它可以是向量,可以是多项式,可以是矩阵、函数等等,但这些都不重要,我们都可以将这些元素视为几何空间$\mathbf{F}^n$中的向量,获得更直观的理解,从而可以忽视一些使我们理解困难的细节.

    \item 容易验证$\mathbf{R}^n$中的向量在自然基下的坐标实际上就是向量本身,例如$(x,y,z)=xe_1+ye_2+ze_3$,故在$\mathbf{R}^3$自然基下的坐标仍然为$(x,y,z)$,需要牢记,有时可以加速解题.
\end{enumerate}

\vspace{2ex}
\centerline{\heiti \Large 内容总结}

本节内容相对而言概念和定理非常多,涉及的题型也很多,因此我们在这里给出一个思维导图,供读者捋顺思路(读者也可以将其他看到过的,例如习题中的命题进一步加入思维导图).
\begin{figure}[htbp]
    \centering
    \includegraphics[scale=0.6]{figs/3-1.png}
\end{figure}

事实上,与其他内容风格不一样的是,本讲中很大一部分的定理我们都给出了证明,一方面是为了提升阅读体验,防止在初学时就被多个``显然''等词汇困惑,另一方面也是希望读者能够从这些比较规范的证明中得到一些证明的技巧.

也许读到这里很多读者都会有些迷惑与焦急——为什么我们仿佛在学习很多看起来十分抽象而且似乎没什么实际应用的知识呢?或许我们需要在这里给读者一个``定心丸''. 事实上,我们在上一讲中定义的线性空间运算法则就是从一般向量加法数乘运算法则抽象而来的最为抽象和基本的内容,我们仅仅建立在这一基础上,伴随着线性表示、线性扩张、线性相关等概念的提出,导出了(有限维)线性空间都具有一种统一的本质结构描述——基和维数,由此我们从抽象的运算规则走到了比较具体的结构. 在此基础上,我们基本上将单个线性空间的研究完成,之后我们将会讨论线性空间之间的关系——一方面可以定义线性空间之间的运算,我们将在下一讲详细介绍,另一方面可以建立两个线性空间之间的某种映射,在关于这种映射的讨论中我们会发现线性空间的本质结构是维数,甚至基之间的差异都可以完全被遮蔽(只需通过本讲介绍的坐标即可),然后我们对线性空间的认识便可以从某种比较抽象的结构走向大家熟悉的一定长度的向量,接下来便可以定义更为具象的矩阵. 这一路上我们实际上是从最为抽象的内容逐步定义概念,说明定理,走向具象的内容. 不同于一般线性代数从行列式、矩阵开始,这样的思路一定能让读者对线性代数有更为深刻的认识.

\vspace{2ex}
\centerline{\heiti \Large 习题}

\vspace{2ex}
{\kaishu 给我五个系数,我将画出一头大象;给我六个系数,大象将会摇动尾巴。}
\begin{flushright}
    \kaishu
    ——柯西
\end{flushright}

\centerline{\heiti A组}
\begin{enumerate}
    \item 请先完成教材P87--88第二章习题第10题的判断题;

    \item 证明:如果向量组线性相关,把每个向量去掉$m$个位置一致的分量,得到的缩短组仍线性相关;如果向量组线性无关,把每个向量添加$m$个位置一致的分量,得到的缩短组仍线性无关;

    \item $a$取何值时,$\beta_1=(1,3,6,2)^\mathrm{T},\beta_2=(2,1,2,-1)^\mathrm{T},\beta_3=(1,-1,a,-2)^\mathrm{T}$线性无关?

    \item 设$\alpha_1,\alpha_2,\ldots,\alpha_n\in\mathbf{F}^n$. 证明:$\alpha_1,\alpha_2,\ldots,\alpha_n$线性无关的充要条件是$\mathbf{F}^n$中任一向量都可以由它们线性表示.

    \item 设$S_1=\{\alpha_1,\ldots,\alpha_s\},S_2=\{\beta_1,\ldots,\beta_t\}$是向量空间$V$的两个线性无关的子集,证明:$\alpha_1,\ldots,\alpha_s,\beta_1,\ldots,\beta_t$线性无关$\iff \spa(S_1)\cap \spa(S_2)=\{0\}$.

    \item 已知$\alpha_1=(1,2,4,3),\alpha_2=(1,-1,-6,6),\alpha_3=(-2,-1,2,-9),\alpha_4=(1,1,-2,7),\beta=(4,2,4,a)$.
          \begin{enumerate}
              \item 求子空间$\spa(\alpha_1,\alpha_2,\alpha_3,\alpha_4)$的维数和一组基;

              \item 求$a$的值使得$\beta\in W$,并求$\beta$在 (1) 所选基下的坐标.
          \end{enumerate}

    \item 证明:$B=\{1,x-a,(x-a)^2\}\enspace(a\neq 0)$是$\mathbf{R}[x]_3$的一组基,并求$\mathbf{R}[x]_3$的自然基$\{1,x,x^2\}$中每个向量关于基$B$的坐标.

    \item 已知向量组$A=\{\alpha_1,\alpha_2,\alpha_3\},\enspace B=\{\alpha_1,\alpha_2,\alpha_3,\alpha_4\},\enspace C=\{\alpha_1,\alpha_2,\alpha_3,\alpha_5\}$的秩分别为$r(A)=r(B)=3,\enspace r(C)=4$. 证明:$\{\alpha_1,\alpha_2,\alpha_3,\alpha_5-\alpha_4\}$的秩为4.

    \item 设向量组$\alpha_1,\alpha_2,\ldots,\alpha_s$的秩为$r$. 在其中任取$m$个向量$\alpha_{i1},\alpha_{i2},\ldots,\alpha_{im}$,证明:向量组$\alpha_{i1},\alpha_{i2},\ldots,\alpha_{im}$的秩$\geqslant r+m-s$.

    \item 已知$\alpha_1,\alpha_2,\ldots,\alpha_n$线性无关,而$\alpha_1,\alpha_2,\ldots,\alpha_n,\beta,\gamma$线性相关. 证明:要么$\beta,\gamma$可以由$\alpha_1,\alpha_2,\ldots,\alpha_n$线性表示,要么$\alpha_1,\alpha_2,\ldots,\alpha_n,\beta$与$\alpha_1,\alpha_2,\ldots,\alpha_n,\gamma$等价.
\end{enumerate}

\centerline{\heiti B组}
\begin{enumerate}
    \item 已知$\alpha_1\neq 0$,则$\alpha_1,\alpha_2,\ldots,\alpha_n$线性相关的充要条件是存在$i\enspace(2 \leqslant i \leqslant n)$使得$\alpha_i$可由$\alpha_1,\alpha_2,\ldots,\alpha_{i-1}$线性表示,且表示法唯一.

    \item 证明以下两个结论:
          \begin{enumerate}
              \item 设$U$和$W$都是$V$的非零子空间,如果$U\subseteq W$,那么$\dim U \leqslant \dim W$;

              \item 设$U$和$W$都是$V$的非零子空间,$U\subseteq W$,且$\dim U = \dim W$,则$U = W$.
          \end{enumerate}

    \item 设向量组$\alpha_1,\alpha_2,\ldots,\alpha_n$线性无关. 证明:在向量组$\beta,\alpha_1,\alpha_2,\ldots,\alpha_n$中至多有一个向量$\alpha_i\enspace(1 \leqslant i \leqslant r)$可被其前面的$i$个向量$\beta,\alpha_1,\alpha_2,\ldots,\alpha_{i-1}$线性表示.

    \item 证明:$1,e^{\lambda_1\cdot x},e^{\lambda_2\cdot x}$($\lambda_1\neq\lambda_2$且均不为0)线性无关.

    \item 设线性空间$V(\mathbf{F})$中,向量$\beta$是$\alpha_1,\ldots,\alpha_r$的线性组合,但不是$\alpha_1,\ldots,\alpha_{r-1}$的线性组合. 证明:$\spa(\alpha_1,\ldots,\alpha_{r-1},\alpha_r)=\spa(\alpha_1,\ldots,\alpha_{r-1},\beta)$.

    \item 设$\mathbf{R}_+$是所有正实数组成的集合,加法和数乘定义如下:
          \[ \forall a,b \in \mathbf{R}_+,\enspace k\in \mathbf{R}\colon\enspace a\oplus b = ab,\enspace k\odot a = a^k \]
          则 $\mathbf{R}_+$关于这一加法和数乘构成一个实线性空间. 求$\mathbf{R}_+$的一组基.
\end{enumerate}

\centerline{\heiti C组}
\begin{enumerate}
    \item 已知$m$个向量$\alpha_1,\alpha_2,\ldots,\alpha_m$线性相关,但其中任意$m-1$个都线性无关,证明:
          \begin{enumerate}
              \item 若$k_1\alpha_1+\cdots+k_m\alpha_m=0$,则$k_1,\ldots,k_m$全为0或全不为0;

              \item 若以下等式成立
                    \begin{align*}
                        k_1\alpha_1+\cdots+k_m\alpha_m & =0 \\
                        l_1\alpha_1+\cdots+l_m\alpha_m & =0
                    \end{align*}
                    其中$l_1\neq 0$,证明:$\dfrac{k_1}{l_1}=\cdots=\dfrac{k_m}{l_m}$.
          \end{enumerate}

    \item (替换定理)设$\alpha_1,\alpha_2,\ldots,\alpha_r$线性无关,且可以被$\beta_1,\beta_2,\ldots,\beta_n$线性表示,则可以将$\beta_1,\beta_2,\ldots,\beta_n$中的$r$个向量替换成$\alpha_1,\alpha_2,\ldots,\alpha_r$后得到与$\beta_1,\beta_2,\ldots,\beta_n$等价的新向量组(注:可以使用数学归纳法证明).

    \item 设线性空间$V=\mathbf{F}^n$. 证明:
          \begin{enumerate}
              \item 存在$V$的子空间$W$,使得$W$的任一非零向量的分量均不为0;

              \item 若$V$的子空间$W$的任一非零向量的分量均不为0,则$\dim W=1$;

              \item 若$V$的子空间$W$的任一非零向量的零分量个数均不超过$r$,则$\dim W \leqslant r+1$.
          \end{enumerate}

    \item 延续上一讲对于 $\mathbf{F_4}(\mathbf{Z}_2)$ 的讨论,尝试求 $\mathbf{F_4}$ 在 $\mathbf{Z}_2$ 上的一组基及其维数,以及其中每个元素的坐标表示.
\end{enumerate}

\input{./专题/4 线性空间的运算.tex}
\input{./专题/5 线性映射.tex}
\input{./专题/6 线性映射基本定理.tex}
\section*{7 线性映射矩阵表示(I)}
\addcontentsline{toc}{section}{7 线性映射矩阵表示(I)}

\vspace{2ex}

\centerline{\heiti A组}
\begin{enumerate}
    \item 教材 P154/9,此处略.
    \item 教材 P154/10,此处略.
\end{enumerate}

\centerline{\heiti B组}
\begin{enumerate}
    \item $T(\beta_1,\beta_2,\cdots,\beta_n)=(\beta_1,\beta_2,\cdots,\beta_n)\begin{pmatrix}0 & 0 & 0 & \cdots & 0 & a_1 \\ 1 & 0 & 0 & \cdots & 0 & a_2 \\ 0 & 1 & 0 & \cdots & 0 & a_3 \\ \vdots & \vdots & \vdots & & \vdots & \vdots \\ 0 & 0 & 0 & \cdots & 1 & a_n\end{pmatrix}$
    所以 $A=\begin{pmatrix}0 & 0 & 0 & \cdots & 0 & a_1 \\ 1 & 0 & 0 & \cdots & 0 & a_2 \\ 0 & 1 & 0 & \cdots & 0 & a_3 \\ \vdots & \vdots & \vdots & & \vdots & \vdots \\ 0 & 0 & 0 & \cdots & 1 & a_n\end{pmatrix}$ 是 $T$ 关于基 $B$ 的表示矩阵.

    $T$ 是同构 $\Leftrightarrow$ $T$ 是双射 $\Leftrightarrow$ $r(T) = n$,所以 $A$ 满秩即 $a_1\neq 0$ 时 $T$ 是同构映射.
    \item \begin{enumerate}
        \item 略.
        \item 设 $(\sigma(f_1),\sigma(f_2),\sigma(f_3))=(f_1,f_2,f_3)A$,可用待定系数法解得 $A=\begin{pmatrix}-1 & -2 & -2 \\ 3 & 2 & 3 \\ -1 & -1 & -1\end{pmatrix}$.
        \item 用待定系数求出 $f=-2f_1+3f_2$,故 $\sigma(f)=-2\sigma(f_1)+3\sigma(f_2)=-4+3x+2x^2$.
    \end{enumerate}
    \item \begin{enumerate}
        \item 略.
        \item $\lambda \neq -1$ 时,$A,B$ 均可逆,故 $\varphi^{-1}(X)=A^{-1}XB^{-1}$.
        \item 取 $V$ 的一组基 $\begin{pmatrix}1 & 0 \\ 0 & 0\end{pmatrix},\begin{pmatrix}0 & 1 \\ 0 & 0\end{pmatrix},\begin{pmatrix}0 & 0 \\ 1 & 0\end{pmatrix},\begin{pmatrix}0 & 0 \\ 0 & 1\end{pmatrix}$,有
        \[\mathrm{Im}\sigma = \mathrm{span}(\varphi(\alpha_1),\varphi(\alpha_2),\varphi(\alpha_3),\varphi(\alpha_4))=\mathrm{span}(\begin{pmatrix}1 & 2 \\ -1 & -2\end{pmatrix},\begin{pmatrix}-1 & -1 \\ 1 & 1\end{pmatrix})\]
        \[\mathrm{Ker}\sigma = \mathrm{span}(\begin{pmatrix}2 & -3 \\ 0 & 1\end{pmatrix},\begin{pmatrix}1 & 0 \\ 1 & 0\end{pmatrix})\]
        (省略步骤,答案不唯一)
        \item 取 $\begin{pmatrix}1 & 2 \\ -1 & -2\end{pmatrix},\begin{pmatrix}-1 & -1 \\ 1 & 1\end{pmatrix},\begin{pmatrix}0 & 0 \\ 1 & 0\end{pmatrix},\begin{pmatrix}0 & 0 \\ 0 & 1\end{pmatrix}$ 即可.
        
        此时矩阵为 $\begin{pmatrix}2 & -2 & -1 & 0 \\ 4 & -2 & 0 & -1 \\ 0 & 0 & 0 & 0 \\ 0 & 0 & 0 & 0\end{pmatrix}$(答案不唯一).
    \end{enumerate}
    \item 求基的过程与求 $V=\{X\in \mathbf{R}^4\ |\ x_1+x_2+x_3=0\}$ 类似,求 $V$ 的基只需求解方程组 $x_1+x_2+x_3=0$ 即可,得到基础解系 $\begin{pmatrix}-1 \\ 0 \\ 1 \\ 0\end{pmatrix},\begin{pmatrix}-1 \\ 1 \\ 0 \\ 0\end{pmatrix},\begin{pmatrix}0 \\ 0 \\ 0 \\ 1\end{pmatrix}$.
    
    换回本题,有基为 $A_1=\begin{pmatrix}-1 & 0 \\ 1 & 0\end{pmatrix},A_2=\begin{pmatrix}-1 & 1 \\ 0 & 0\end{pmatrix},A_3=\begin{pmatrix}0 & 0 \\ 0 & 1\end{pmatrix}$.
    而 $\sigma(A_1)=A_1+A_2,\sigma(A_2)=A_1+A_2,\sigma(A_3)=2A_3$,可得 $\sigma(A_1+A_2)=2(A_1+A_2),\sigma(A_1-A_2)=0,\sigma(A_3)=2A_3$.

    所以取基 $\{A_1-A_2,A_1+A_2,A_3\}$ 有
    \[(\sigma(A_1-A_2),\sigma(A_1+A_2),\sigma(A_3))=(A_1-A_2,A_1+A_2,A_3)\begin{pmatrix}0 & 0 & 0 \\ 0 & 2 & 0 \\ 0 & 0 & 2\end{pmatrix}\]
    为对角矩阵.
    \item \begin{enumerate}
        \item 求核空间:设 $f(x)=ax^3+bx^2+cx+d$,由 $f(-1)=f(0)=f(1)=0$ 解得 $f(x)=a(x^3-x)$. 故 $N(T)=\mathrm{span}(x^3-x)$.
        
        求像空间:取常用基 $\{1,x,x^2,x^3\}$,我们要求 $T(1),T(x),T(x^2),T(x^3)$ 的极大线性无关组,我们发现 $T(x)=T(x^3)$,故先舍弃 $T(x^3)$,然后令 $k_1T(1)+k_2T(x)+k_3T(x^2)=0$,可解得 $k_1=k_2=k_3=0$,故 $T(1),T(x),T(x^2)$ 线性无关,故 $R(T)=\mathrm{span}(\begin{pmatrix}1 & 1 \\ 1 & 1\end{pmatrix},\begin{pmatrix}0 & 1 \\ -1 & 0\end{pmatrix},\begin{pmatrix}0 & 1 \\ 1 & 0\end{pmatrix})$.
        \item 由上一问有 $\mathrm{dim}N(T)=1,\mathrm{dim}R(T)=3$,又有 $\mathrm{dim}\mathbf{R}[x]_4=4$. 则维数公式成立.
    \end{enumerate}
    \item
        \begin{enumerate}
            \item
                对非齐次线性方程组$AX=\xi_1$,\\
                $\bar{A}=\begin{pmatrix}
                    1 & -1 & -1 & -1 \\
                    -1 & 1 & 1 & 1 \\
                    0 & -4 & -2 & -2
                \end{pmatrix}
                \rightarrow\begin{pmatrix}
                    1 & -1 & -1 & -1 \\
                    0 & 1 & \frac 12 & \frac 12 \\
                    0 & 0 & 0 & 0
                \end{pmatrix}
                \rightarrow\begin{pmatrix}
                    1 & 0 & -\frac 12 & -\frac 12 \\
                    0 & 1 & \frac 12 & \frac 12 \\
                    0 & 0 & 0 & 0
                \end{pmatrix}$,则\\
                $\xi_2=C_1\begin{pmatrix}
                    \frac 12 \\
                    -\frac 12 \\
                    1
                \end{pmatrix} + \begin{pmatrix}
                    -\frac 12 \\
                    \frac 12 \\
                    0
                \end{pmatrix}=\frac 12\begin{pmatrix}
                    C_1 - 1 \\
                    -C_1 + 1 \\
                    2C_1
                \end{pmatrix}$(其中$C_1$为任意常数).\\
                $A^2=\begin{pmatrix}
                    2 & 2 & 0 \\
                    -2 & -2 & 0 \\
                    4 & 4 & 0
                \end{pmatrix}$,对齐次线性方程组$A^2X=\xi_1$,\\
                $\bar{B}=\begin{pmatrix}
                    A^2 & \xi_1
                \end{pmatrix}=\begin{pmatrix}
                    2 & 2 & 0 & 1 \\
                    -2 & -2 & 0 & 1 \\
                    4 & 4 & 0 & -2
                \end{pmatrix}
                \rightarrow\begin{pmatrix}
                    1 & 1 & 0 & -\frac 12 \\
                    0 & 0 & 0 & 0 \\
                    0 & 0 & 0 & 0
                \end{pmatrix}$,\\
                则$A^2X=\xi_1$的通解\\
                $\xi_3=C_2\begin{pmatrix}
                    -1 \\
                    1 \\
                    0
                \end{pmatrix}+C_3\begin{pmatrix}
                    0 \\
                    0 \\
                    1
                \end{pmatrix}+\begin{pmatrix}
                    -\frac 12 \\
                    0 \\
                    0
                \end{pmatrix}=\begin{pmatrix}
                    -C_2 - \frac 12 \\
                    C_2 \\
                    C_3
                \end{pmatrix}$(其中$C_2, C_3$为任意常数).
            \item
                因为$|\xi_1,\xi_2,\xi_3|=\frac 12\begin{vmatrix}
                    -1 & C_1 - 1 & -C_2 - \frac 12 \\
                    1 & -C_1 + 1 & C_2 \\
                    -2 & 2C_1 & C_3
                \end{vmatrix}=-\frac 12\neq 0$,\\
                所以$\xi_1,\xi_2,\xi_3$线性无关.
        \end{enumerate}
    \item \begin{enumerate}
        \item 对新的一组基,使用过渡矩阵进行表达如下:
        \[(\beta_{1}, \beta_{2}, \beta_{3})=(\alpha_{1}, \alpha_{2}, \alpha_{3})\begin{pmatrix}
        2 & 1 & -1 \\
        1 & 1 & 1 \\
        3 & 2 & 1    
        \end{pmatrix}=(\alpha_{1}, \alpha_{2}, \alpha_{3}) C\]
        其中 $C$ 是可逆矩阵,且
        \[(\alpha_{1},\ \alpha_{2},\ \alpha_{3})=(\beta_{1},\ \beta_{2},\ \beta_{3}) C^{-1}\]
        将上式代入已知条件得
        \[\sigma\left(\left(\beta_{1},\ \beta_{2},\ \beta_{3}\right) C^{-1}\right)=\left(\left(\beta_{1},\ \beta_{2},\ \beta_{3}\right) C^{-1}\right) A\]
        容易验证(只需利用线性变换和矩阵的等价性然后利用矩阵乘法结合律即可)上式左端等于 $(\sigma(\beta_{1}, \beta_{2}, \beta_{3})) C^{-1}$,所以
        \[(\sigma(\beta_{1},\ \beta_{2},\ \beta_{3})) C^{-1}=(\beta_{1},\ \beta_{2},\ \beta_{3})(C^{-1} A)\]
        从而得 $\sigma(\beta_{1},\ \beta_{2},\ \beta_{3})=(\beta_{1},\ \beta_{2},\ \beta_{3})(C^{-1} A C)$,故 $\sigma$ 关于基 $\{\beta_{1},\ \beta_{2},\ \beta_{3}\}$ 下对应的矩阵 $B=C^{-1} A C=\begin{pmatrix}2 & 0 & 1 \\ 0 & 2 & 1 \\ 3 & 1 & -1\end{pmatrix}$.
        \item $\sigma$ 的值域是 $A$ 列向量组的极大线性无关组,由于 $A $ 的第 $1$ 列可以由第 $2$ 列和第 $3$ 列线性表示,从而 $\sigma(V)=L(2 \alpha_{1}+\alpha_{2},\ -\alpha_{1}+\alpha_{3})$.$\operatorname{Ker} \sigma$ 是线性方程组 $AX=0$ 的解空间,从而 $\mathrm{Ker} \sigma=\mathrm{span}(\alpha_{1}-2 \alpha_{2}-3 \alpha_{3})$.
        \item 由于 $\alpha_{1}$ 不能由 $2 \alpha_{1}+\alpha_{2}$ 和 $-\alpha_{1}+\alpha_{3}$ 线性表示,可以把 $\sigma(V)$ 的基扩充为 $V$ 的基 $\{\alpha_{1},\ 2 \alpha_{1}+\alpha_{2},\ -\alpha_{1}+\alpha_{3}\}$,$\sigma$ 在这个基下对应的矩阵是 $\begin{pmatrix}0 & 0 & 0 \\ 2 & 5 & -2 \\ 3 & 6 & -2\end{pmatrix}$.
        \item 由于 $\alpha_{1},\ \alpha_{2}$ 不能由 $\alpha_{1}-2 \alpha_{2}-3 \alpha_{3}$ 线性表示,可以把 $\mathrm{Ker} \sigma$ 的基扩充为 $V$ 的基 $\{\alpha_{1},\ \alpha_{2},\ \alpha_{1}-2 \alpha_{2}-3 \alpha_{3}\}$,$\sigma$ 在这个基下对应的矩阵是 $\begin{pmatrix}2 & 2 & 0 \\ 0 & 1 & 0 \\ 1 & 0 & 0\end{pmatrix}$.
    \end{enumerate}
    \item \begin{enumerate}
        \item 因为 $A^k=\begin{pmatrix}\lambda_1^k & 0 \\ 0 & \lambda_2^k\end{pmatrix}$,所以 $f(A)=a_mA^m+a_{m-1}A^{m-1}+\cdots+a_1A+a_0E = \begin{pmatrix}f(\lambda_1) & 0 \\ 0 & f(\lambda_2)\end{pmatrix}$.
        \item $A=PBP^{-1}$,则 $A^2=(PBP^{-1})(PBP^{-1})=PB^2P^{-1}$. 由归纳法得 $A^k=PB^kP^{-1}$,于是
        \[f(A)=a_mPB^mP^{-1}+a_{m-1}PB^{m-1}P^{-1}+\cdots+a_0=P\begin{pmatrix}f(\lambda_1) & 0 \\ 0 & f(\lambda_2)\end{pmatrix}P^{-1}=Pf(B)P^{-1}\]
    \end{enumerate}
\end{enumerate}

\centerline{\heiti C组}
\begin{enumerate}
    \item
\end{enumerate}

\clearpage

\section*{8 线性映射矩阵表示(II)}
\addcontentsline{toc}{section}{8 线性映射矩阵表示(II)}

\vspace{2ex}

\centerline{\heiti A组}
\begin{enumerate}
    \item 
\end{enumerate}

\centerline{\heiti B组}
\begin{enumerate}
    \item 为使得“每行元素之和”的条件有用,我们用 $\alpha=\begin{pmatrix}1 \\ 1 \\ \vdots \\ 1\end{pmatrix}$ 去乘以 $A$. 则 $A\alpha=\begin{pmatrix}k \\ k \\ \vdots \\ k\end{pmatrix}=k\alpha$.
    因为 $A$ 可逆所以 $k\neq 0$,同时由上面的式子有 $\alpha=kA^{-1}\alpha$,得 $A^{-1}\alpha=\dfrac{1}{k}\alpha(k\neq 0)$.
    故 $A^{-1}$ 每行和为 $\dfrac{1}{k}$ 成立.
    \item \begin{enumerate}
        \item 由题意 $r(A)=r(B)$. $A$ 可由初等变换为 $\begin{pmatrix}1 & 2 & a \\ 1 & 3 & 0 \\ 0 & 0 & 0\end{pmatrix}$,$B$ 可由初等变换为 $\begin{pmatrix}1 & a & 2 \\ 0 & 1 & 1 \\ 0 & 0 & 2-a\end{pmatrix}$.
        由于秩相同故 $2-a=0,a=2$.
        \item $(A,B)=\begin{pmatrix}1 & 2 & a & 1 & a & 2 \\ 1 & 3 & 0 & 0 & 1 & 1 \\ 2 & 7 & -a & -1 & 1 & 1\end{pmatrix}\rightarrow\begin{pmatrix}1 & 0 & 6 & 3 & 4 & 4 \\ 0 & 1 & -2 & -1 & -1 & -1 \\ 0 & 0 & 0 & 0 & 0 & 0\end{pmatrix}$
        
        $X = \begin{pmatrix}-6k_1+3 \\ 2k_1-1 \\ k_1\end{pmatrix},Y = \begin{pmatrix}-6k_2+4 \\ 2k_2-1 \\ k_2\end{pmatrix},Z = \begin{pmatrix}-6k_3+4 \\ 2k_3-1 \\ k_3\end{pmatrix}$
        
        $\begin{pmatrix}-6k_1+3 & -6k_2+4 & -6k_3+4 \\ 2k_1-1 & 2k_2-1 & 2k_3-1 \\ k_1 & k_2 & k_3\end{pmatrix}\rightarrow \begin{pmatrix}1 & 1 & 1 \\ 0 & 1 & 1 \\ 0 & 0 & k_3-k_2\end{pmatrix}$

        因为 $P$ 可逆,所以 $k_2\neq k_3$,故 $P=\begin{pmatrix}-6k_1+3 & -6k_2+4 & -6k_3+4 \\ 2k_1-1 & 2k_2-1 & 2k_3-1 \\ k_1 & k_2 & k_3\end{pmatrix},k_2\neq k_3, k_1,k_2,k_3 \in \mathbf{R}$
    \end{enumerate}
\end{enumerate}

\centerline{\heiti C组}
\begin{enumerate}
    \item 首先设
    \[A=\begin{pmatrix}a_1 & a_2 \\ a_3 & a_4\end{pmatrix},B=\begin{pmatrix}b_1 & b_2 \\ b_3 & b_4\end{pmatrix},C=\begin{pmatrix}c_1 & c_2 \\ c_3 & c_4\end{pmatrix}.\]
    由于 $A,B,C$ 在 $M_2(\mathbf{C})$ 中线性无关,所以将 $A,B,C$ 的元素排为一列,可知矩阵
    \[\begin{pmatrix}a_1 & b_1 & c_1 \\ a_2 & b_2 & c_2 \\ a_3 & b_3 & c_3 \\ a_4 & b_4 & c_4\end{pmatrix}\]
    的秩为 $3$,这里不妨设前三个行向量线性无关,即有
    \[D=\begin{pmatrix}a_1 & b_1 & c_1 \\ a_2 & b_2 & c_2 \\ a_3 & b_3 & c_3\end{pmatrix}\]
    为可逆矩阵.

    另外,注意到对任意的 $x_1,x_2,x_3$,有
    \[x_1A+x_2B+x_3C=\begin{pmatrix}a_1x_1+b_1x_2+c_1x_3 & a_2x_1+b_2x_2+c_2x_3 \\ a_3x_1+b_3x_2+c_3x_3 & a_4x_1+b_4x_2+c_4x_3\end{pmatrix}.\]
    现在考虑方程组
    \[\begin{cases}a_1x_1+b_1x_2+c_1x_3 = 0 \\ a_2x_1+b_2x_2+c_2x_3 = 1 \\ a_3x_1+b_3x_2+c_3x_3 = 1\end{cases}\]
    其系数矩阵为 $D$,这是一个可逆矩阵,所以上述方程存在唯一解,不妨记为 $(x_1',x_2',x_3')$,此时就有 $\lvert x_1'A+x_2'B+x_3'C \rvert = \begin{vmatrix}0 & 1 \\ 1 & \ast\end{vmatrix} = -1$.
    
    所以 $x_1'A+x_2'B+x_3'C$ 为可逆矩阵.
\end{enumerate}

\clearpage
 % 射影空间
\begingroup
\SetLUChapterNumberingStyle{1}
\def\theHchapter{\arabic{chapter}ε}

\chapter{有限域上的矩阵}

\ResetChapterNumberingStyle{8}
\endgroup

\section*{9 线性映射矩阵表示(III)}
\addcontentsline{toc}{section}{9 线性映射矩阵表示(III)}

\vspace{2ex}

\centerline{\heiti A组}
\begin{enumerate}
    \item $\alpha = (x_1,x_2,x_3)^{\mathrm{T}},\beta = (y_1,y_2,y_3)^{\mathrm{T}},\alpha^{\mathrm{T}}\beta=x_1y_1+x_2y_2+x_3y_3$,实际上是 $\alpha\beta^{\mathrm{T}}$ 对角线元素之和,故 $\alpha^{\mathrm{T}}\beta=-5$.
\end{enumerate}

\centerline{\heiti B组}
\begin{enumerate}
    \item $A=(a_{ij})_{n\times n}$,$A^\mathrm{T}A$ 对角线元素为 $\sum\limits_{i=1}^na_{i1}^2,\sum\limits_{i=1}^na_{i2}^2,\cdots,\sum\limits_{i=1}^na_{in}^2$ 均为 $0$,
    故 $A$ 中所有元素均为 $0$,可得 $A=O$.
    另一方法:由于 $r(A^\mathrm{T}A) = r(A)$,又 $r(A^\mathrm{T}A)=0$,所以 $r(A)=0$,从而 $A=O$.
    \item \begin{enumerate}
        \item 根据定义易证.
        \item 令 $E_{ij}$ 表示第 $i$ 行第 $j$ 列位置是 $1$,其余位置都是 $0$ 的 $n$ 阶方阵,则 $V$ 的一组基是
        \[E_{11},\cdots,E_{nn},E_{12}+E_{21},\cdots,E_{n-1,n}+E_{n,n-1}\]
        故 $V$ 的维数是 $1+2+\cdots+n=\dfrac{n(n+1)}{2}$.
        \end{enumerate}
    \item 观察对称性有
    \[AA^{\mathrm{T}}=(a^2+b^2+c^2+d^2)E\]
    从而有 $A^{-1}=\dfrac{1}{a^2+b^2+c^2+d^2}A^{\mathrm{T}}$.
    \item 设 $E_{ij}$ 表示第 $i$ 行第 $j$ 列位置是 $1$,其余位置都是 $0$ 的 $n$ 阶方阵.
        \begin{enumerate}
            \item $k=0$ 时,$W$ 为主对角线全为 $0$ 的上三角矩阵全体,则 $B_1=\{E_{ij}\ |\ i>j\}$ 为 $W$ 的一组基,且 $\mathrm{dim}W=\dfrac{n(n-1)}{2}$.
            \item $k=1$ 时,$W$ 为对称矩阵全体,$\forall A\in W,A = \sum\limits_{i=1}^n\sum\limits_{j=1}^na_{ij}E_{ij}=\sum\limits_{i<j}a_{ij}(E_{ij}+E_{ji})+\sum\limits_{i=1}^na_{ii}E_{ii}$,
            而 $B_2=\{E_{ij}+E_{ji},E_{kk}\ |\ 1\leq i,j,k\leq n,i<j\}$ 线性无关,故 $B_2$ 是 $W$ 的一组基,$\mathrm{dim}W=\dfrac{n(n+1)}{2}$.
            \item $k=2$ 时,可得 $a_{ii}=2a_{ii},a_{ii}=0$,故 $\forall A\in W,A = \sum\limits_{i=1}^n\sum\limits_{j=1}^na_{ij}E_{ij}=\sum\limits_{i<j}a_{ij}(E_{ij}+2E_{ji})$,
            而 $B_3=\{E_{ij}+2E_{ji}\ |\ i<j\}$ 线性无关,故 $B_3$ 是 $W$ 的一组基,$\mathrm{dim}W=\dfrac{n(n-1)}{2}$.
        \end{enumerate}
\end{enumerate}

\centerline{\heiti C组}
\begin{enumerate}
    \item 
\end{enumerate}

\clearpage

\begingroup
\SetLUChapterNumberingStyle{2}
\def\theHchapter{\arabic{chapter}ε}

\chapter{多重线性映射与张量的计算}

\ResetChapterNumberingStyle{9}
\endgroup

\section*{10 矩阵运算进阶(I)}
\addcontentsline{toc}{section}{10 矩阵运算进阶(I)}

\vspace{2ex}

\centerline{\heiti A组}
\begin{enumerate}
    \item 由题意有 $P_2AP_1 = E$,从而有 $A=P_2^{-1}P_1^{-1}=P_2P_1^{-1}$.
    \item 由题意知 $B = E_{ij}A$,所以 $BA^{-1}=E_{ij}$ 从而 $B$ 可逆,同时可得 $AB^{-1}=A(A^{-1}E_{ij}^{-1})=E_{ij}$.
    \item $Q = (\alpha_1+\alpha_2,\alpha_2,\alpha_3)=(\alpha_1,\alpha_2,\alpha_3)\begin{pmatrix}1 & 0 & 0 \\ 1 & 1 & 0 \\ 0 & 0 & 1\end{pmatrix}=P\begin{pmatrix}1 & 0 & 0 \\ 1 & 1 & 0 \\ 0 & 0 & 1\end{pmatrix}$.
    
    记 $E_{21}=\begin{pmatrix}1 & 0 & 0 \\ 1 & 1 & 0 \\ 0 & 0 & 1\end{pmatrix}$,有 $Q^{-1}AQ=E_{21}^{-1}P^{-1}APE_{21}=\begin{pmatrix}1 & 0 & 0 \\ 0 & 1 & 0 \\ 0 & 0 & 2\end{pmatrix}$.
    \item 见本章例 10.5,此处不赘述.
\end{enumerate}

\centerline{\heiti B组}
\begin{enumerate}
    \item 此处仅给出答案,具体过程略.
    \begin{enumerate}
        \item $\begin{pmatrix}a_{21} & a_{22} & a_{23} \\ a_{11} & a_{12} & a_{13} \\ a_{31} & a_{32} & a_{33}\end{pmatrix}$.
        \item $\begin{pmatrix}-a_{11} & -a_{12} & -a_{13} \\ a_{21} & a_{22} & a_{23} \\ a_{31} & a_{32} & a_{33}\end{pmatrix}$.
        \item $\begin{pmatrix}-a_{13} & -a_{12} & a_{11} \\ a_{23} & a_{22} & -a_{21} \\ a_{33} & a_{32} & -a_{31}\end{pmatrix}$.
        \item $\begin{pmatrix}-a_{11}-a_{12} & -a_{12}-a_{13} & -a_{13}-a_{11} \\ a_{21}+a_{22} & a_{22}+a_{23} & a_{23}+a_{21} \\ a_{31}+a_{32} & a_{32}+a_{33} & a_{33}+a_{31}\end{pmatrix}$.
    \end{enumerate}
    \item \begin{enumerate}
        \item 略.
        \item 设 $A=(a_{ij})_{3\times 2}$,$e_{ij}$ 为 $\mathbf{R}^{3\times 2}$ 的自然基。因为 $PAQ = \begin{pmatrix}a_{12}+a_{22} & 0 \\ a_{22} & 0 \\ 0 & 0\end{pmatrix}$,
        所以 $\sigma(e_{12}) = e_{11},\sigma(e_{22}) = e_{11}+e_{21},\sigma(e_{11})=\sigma(e_{21})=\sigma(e_{31})=\sigma(e_{32})=0$.
        
        于是 $\mathrm{Ker}\sigma = \mathrm{span}(e_{11},e_{21},e_{31},e_{32}),\mathrm{Im}\sigma = \mathrm{span}(e_{11},e_{11}+e_{21})$. 
        \item 令 $B_1=\{e_{12},e_{22},e_{11},e_{21},e_{31},e_{32}\},B_2=\{e_{11},e_{11}+e_{21},e_{12},e_{22},e_{31},e_{32}\}$,则均为 $\mathbf{R^{3\times 2}}$ 的基,且 $\sigma(\epsilon)=(\eta)\begin{pmatrix}E_2 & 0 \\ 0 & 0\end{pmatrix}$.
    \end{enumerate}
    \item 见教材 P147/例 5
\end{enumerate}

\centerline{\heiti C组}
\begin{enumerate}
    \item 使用数学归纳法。当 $n=1$ 时,$A=\begin{pmatrix}a\end{pmatrix}(a\neq 0)$,$B$ 取任意一阶矩阵均成立;
    假设 $n-1$ 阶成立,$A = \begin{pmatrix}A_1 & \alpha \\ \beta & a_{nn}\end{pmatrix}$,其中 $A_1$ 为 $n-1$ 阶矩阵且存在 $n-1$ 阶下三角矩阵 $B_1$ 使得 $B_1A_1$ 为上三角矩阵,则有 
    \[\begin{pmatrix}B_1 & O \\ O & 1\end{pmatrix}\begin{pmatrix}A_1 & \alpha \\ O & a_{nn}-\beta A^{-1}\alpha\end{pmatrix} = \begin{pmatrix}B_1A_1 & B_1\alpha \\ O & a_{nn}-\beta A_1^{-1}\alpha\end{pmatrix}\]为上三角矩阵.
    而 \[\begin{pmatrix}E_{n-1} & O \\ -\beta A_1^{-1} & 1\end{pmatrix}\begin{pmatrix}A_1 & \alpha \\ \beta & a_{nn}\end{pmatrix}=\begin{pmatrix}A_1 & \alpha \\ O & a_{nn}-\beta A_1^{-1} \alpha\end{pmatrix}\]
    故 $B=\begin{pmatrix}B_1 & O \\ O & 1\end{pmatrix}\begin{pmatrix}E_{n-1} & O \\ -\beta A_1^{-1} & 1\end{pmatrix}$ 符合条件($B_1$ 为下三角矩阵,故 $B$ 也是).
    \item 任取 $\alpha\in W$,有 $A_{12}\alpha=0$.
    \[\begin{pmatrix}O_{k\times 1} \\ \alpha\end{pmatrix}=A^{-1}A\begin{pmatrix}O_{k\times 1} \\ \alpha\end{pmatrix}=A^{-1}\begin{pmatrix}A_{11} & A_{12} \\ A_{21} & A_{22}\end{pmatrix}\begin{pmatrix}O_{k\times 1} \\ \alpha\end{pmatrix}=A^{-1}\begin{pmatrix}O_{l\times 1} \\ A_{22}\alpha\end{pmatrix}\]
    \[=\begin{pmatrix}B_{11} & B_{12} \\ B_{21} & B_{22}\end{pmatrix}\begin{pmatrix}O_{l\times 1} \\ A_{22}\alpha\end{pmatrix}=\begin{pmatrix}B_{12}A_{22}\alpha \\ B_{22}A_{22}\alpha\end{pmatrix}\]
    故 $B_{12}A_{22}\alpha=0$,故我们可以推测以下定义:$\sigma\in L(W,U),\sigma(\alpha)=A_{22}\alpha$.
    证明单射、满射即可.

    单射:$\sigma(\alpha)=\sigma(\beta)\Rightarrow A_{22}(\alpha-\beta)=0$. 又有 $A_{12}\alpha=A_{12}\beta=0\Rightarrow A_{12}(\alpha-\beta)=0$,可得 $\begin{pmatrix}A_{12} \\ A_{22}\end{pmatrix}(\alpha-\beta) = 0$. 由于 $\begin{pmatrix}A_{12} \\ A_{22}\end{pmatrix}$ 列满秩,故有 $\alpha=\beta$. 成立.

    满射:$\forall \gamma\in U,B_{12}\gamma = 0.$ 类似我们一开始的步骤,$\begin{pmatrix}O_{l\times 1} \\ \gamma\end{pmatrix}=AA^{-1}\begin{pmatrix}O_{l\times 1} \\ \gamma\end{pmatrix}= \cdots = \begin{pmatrix}A_{12}B_{22}\gamma \\ A_{22}B_{22}\gamma\end{pmatrix}$,故 $\exists B_{22}\gamma \in W,A_{22}B_{22}\gamma=\gamma\in U$. 故满射成立,证毕.
\end{enumerate}

\clearpage

\section*{11 矩阵的秩}
\addcontentsline{toc}{section}{11 矩阵的秩}

\vspace{2ex}

\centerline{\heiti A组}
\begin{enumerate}
    \item 取 $\mathbf{R^4}$ 标准基 $\varepsilon_1,\varepsilon_2,\varepsilon_3,\varepsilon_4$.
    那么 $(\alpha_1,\alpha_2,\alpha_3,\alpha_4)=(\varepsilon_1,\varepsilon_2,\varepsilon_3,\varepsilon_4)A,(\beta_1,\beta_2,\beta_3,\beta_4)=(\varepsilon_1,\varepsilon_2,\varepsilon_3,\varepsilon_4)B.$
    其中 \[A=\begin{pmatrix}1 & 1 & 1 & 1 \\ 1 & 1 & -1 & -1 \\ 1 & -1 & 1 & -1 \\ 1 & -1 & -1 & 1\end{pmatrix},B=\begin{pmatrix}1 & 2 & 1 & 0 \\ 1 & 1 & 1 & 1 \\ 0 & 3 & 0 & -1 \\ 1 & 1 & 0 & -1\end{pmatrix}.\] 
    由此可知 \[(\beta_1,\beta_2,\beta_3,\beta_4)=(\varepsilon_1,\varepsilon_2,\varepsilon_3,\varepsilon_4)B=(\alpha_1,\alpha_2,\alpha_3,\alpha_4)A^{-1}B.\]
    过渡矩阵 \[A^{-1}B=\dfrac{1}{4}\begin{pmatrix}3 & 7 & 2 & -1 \\ 1 & -1 & 2 & 3 \\ -1 & 3 & 0 & -1 \\ 1 & -1 & 0 & -1\end{pmatrix}\]
    另外,容易求得 $\xi$ 在 $\alpha_1,\alpha_2,\alpha_3,\alpha_4$ 下的坐标为 $\begin{pmatrix}0 \\ \frac{1}{2} \\ \frac{1}{2} \\ 0\end{pmatrix}$
    \item 证明:考虑矩阵 $A$ 的行向量组的极大线性无关组,若添加的一行可由其极大线性无关组线性表示,则秩不变. 否则秩增加 $1$. 
    \item 证明:设 $A$ 的行向量组为 $\{\alpha_1,\alpha_2,\cdots,\alpha_s\}$,$r(A)=r$; $B$ 的行向量组为 $\{\alpha_1,\alpha_2,\cdots,\alpha_m\},r(B)=k$.
    不妨设:$B$ 的行向量组的极大线性无关组为 $\{\alpha_1,\alpha_2,\cdots,\alpha_k,\alpha_{i_1},\cdots,\alpha_{i_{r-k}}\}$,其中 $\{\alpha_{i_1},\cdots,\alpha_{i_{r-k}}\}$(共 $r-k$ 个向量)是包含在 $\{\alpha_{m+1},\cdots,\alpha_s\}$(共 $s-m$ 个向量)之中的. 显然有
    \[r-k \leq s-m,\]
    即
    \[r(B)=k\ge r+m-s=r(A)+m-s.\]
\end{enumerate}

\centerline{\heiti B组}
\begin{enumerate}
    \item 已知 $r(A+B) \leq r(A)+r(B)$,把 $B$ 写成 $-B$ 则有 $r(A-B) \leq r(A)+r(-B)=r(A)+r(B)$. 不等式右半部分得证.
    
    另外,$r(A)=r(A-B+B) \leq r(A-B)+r(B)$,从而 $r(A-B) \ge r(A)-r(B)$,当然,加个绝对值也是没有问题的:$r(A-B) \ge \lvert r(A)-r(B) \rvert$. 同理,有 $r(A+B) \ge \lvert r(A)+r(B) \rvert$. 证毕.
    \item $V$ 的基 $B_1$ 到 $B_2$ 的过渡矩阵 $P$ 具有下述形式:
    \[P=\begin{pmatrix}\mathrm{Im} & B_1 \\ 0 & B_2\end{pmatrix}\]
    其中 $B_1,B_2$ 分别是域 $\mathbf{F}$ 上 $m\times (n-m),(n-m)\times (n-m)$ 矩阵,
    \[\beta_j=b_{j1}\delta_1+\cdots+b_{jm}\delta_m+b_{j,m+1}\delta_{m+1}+\cdots+b_{jn}a_n\]
    其中 $j=m+1,\cdots,n$. 于是
    \[\beta_j+W=b_{j,m+1}(\alpha_{m+1}+W)+\cdots+b_{jn}(\alpha_n+W)\]
    因此商空间 $V/W$ 的基 $\alpha_{m+1}+W,\cdots,\alpha_n+W$ 到 $\beta_{m+1}+W,\cdots,\beta_n+W$ 的过渡矩阵是 $B_2$.
    \item 设 $\beta_1=\alpha_1+\alpha_2,\cdots,\beta_{n-1}=\alpha_{n-1}+\alpha_n,\beta_n=\alpha_n+\alpha_1$. 由于
    \[(\beta_1,\beta_2,\cdots,\beta_n) = (\alpha_1,\alpha_2,\cdots,\alpha_n)A,\]
    其中
    \[A=\begin{pmatrix}1 & & & & 1 \\ 1 & 1 & & & \\ & 1 & \ddots & & \\ & & \ddots & 1 & \\ & & & 1 & 1\end{pmatrix}.\]
    则有 $\lvert A \rvert = 1 + (-1)^{n+1}=2\neq 0$,所以 $A$ 可逆,可得 $\{\beta_1,\beta_2,\cdots,\beta_n\}$ 和 $\{\alpha_1,\alpha_2,\cdots,\alpha_n\}$ 等价. 这就说明了 $\alpha_1,\alpha_2,\cdots,\alpha_n$ 线性无关的充要条件是 $\beta_1,\beta_2,\cdots,\beta_n$ 线性无关.
    \item 记 $B_1=\{e_{11},e_{12},e_{21},e_{22}\},B_2=\{g_1,g_2,g_3,g_4\}$.\begin{enumerate}
        \item 设 $k_1g_1+k_2g_2+k_3g_3+k_4g_4=O$,可得 $k_1=k_2=k_3=k_4=0$,所以 $g_1,g_2,g_3,g_4$ 线性无关,从而是 $M_2(\mathbf{R})$ 的一组基.
        \item 由 $M_2(\mathbf{R}) \cong \mathbf{R^4}$,所以 $\{e_{11},e_{12},e_{21},e_{22}\}$ 可以表示为 $\mathbf{R^4}$ 中的自然基 $\{e_1,e_2,e_3,e_4\}$,而 $\{g_1,g_2,g_3,g_4\}$ 可表示为 $\{(1,0,0,0)^{\mathbf{T}},(1,1,0,0)^{\mathbf{T}},(1,1,1,0)^{\mathbf{T}},(1,1,1,1)^{\mathbf{T}}\}$.
        
        于是,由 \[\begin{pmatrix}g_1 & g_2 & g_3 & g_4\end{pmatrix}=\begin{pmatrix}e_{11} & e_{12} & e_{21} & e_{22}\end{pmatrix}C\]
        得 \[\begin{pmatrix}e_{11} & e_{12} & e_{21} & e_{22}\end{pmatrix}=\begin{pmatrix}g_1 & g_2 & g_3 & g_4\end{pmatrix}C^{-1}\]
        所以基 $B_2$ 变为 $B_1$ 的变换矩阵为 $C^{-1}=\begin{pmatrix}1 & -1 & 0 & 0 \\ 0 & 1 & -1 & 0 \\ 0 & 0 & 1 & -1 \\ 0 & 0 & 0 & 1\end{pmatrix}$.
        \item 考虑从 $A^2=A$ 中选取较为简单的矩阵,例如由
        \[\begin{pmatrix}a & b \\ 0 & 0\end{pmatrix}^2=\begin{pmatrix}a^2 & ab \\ 0 & 0\end{pmatrix}=\begin{pmatrix}a & b \\ 0 & 0\end{pmatrix}\]
        取 $a=1,b=0$ 或 $1$,得 $A_1=\begin{pmatrix}1 & 0 \\ 0 & 0\end{pmatrix},A_2=\begin{pmatrix}1 & 1 \\ 0 & 0\end{pmatrix}$

        类似地,可取 $A_3=\begin{pmatrix}0 & 0 \\ 0 & 1\end{pmatrix},A_4=\begin{pmatrix}0 & 0 \\ 1 & 1\end{pmatrix}$.

        这就取出了一组满足 $A^2=A$ 的线性无关的 $\{A_1,A_2,A_3,A_4\}$,是 $M_2(\mathbf{R})$ 的一组基 $B_3$.
        \item 先记 $B_2$ 变为 $B_3$ 的变换矩阵为 $D$,即 \[\begin{pmatrix}A_1 & A_2 & A_3 & A_4\end{pmatrix}=\begin{pmatrix}g_1 & g_2 & g_3 & g_4\end{pmatrix}D\]\
        按题 $(2)$ 中所述,此时有 \[\begin{pmatrix}1 & 1 & 0 & 0 \\ 0 & 1 & 0 & 0 \\ 0 & 0 & 0 & 1 \\ 0 & 0 & 1 & 1\end{pmatrix}=\begin{pmatrix}1 & 1 & 1 & 1 \\ 0 & 1 & 1 & 1 \\ 0 & 0 & 1 & 1 \\ 0 & 0 & 0 & 1\end{pmatrix}D\]
        由于上式右端已知矩阵的逆矩阵为上面的 $C^{-1}$,所以在上式两边左乘 $C^{-1}$,可得 \[D = \begin{pmatrix}1 & 0 & 0 & 0 \\ 0 & 1 & 0 & -1 \\ 0 & 0 & -1 & 0 \\ 0 & 0 & 1 & 1\end{pmatrix}\]
        由于矩阵 $A$ 关于 $B_2$ 的坐标为 $(1,1,1,1)^{\mathbf{T}}$,所以 $A$ 关于 $B_3$ 的坐标为
        \[Y=D^{-1}X=\begin{pmatrix}1 \\ 3 \\ -1 \\ 2\end{pmatrix}.\]
    \end{enumerate}
    \item \begin{enumerate}
        \item 初等变换即可.
        \item 同上.
        \item 矩阵 $A$ 秩为 $r$ 可写作 $A=P\begin{pmatrix}E_r & 0 \\ 0 & 0\end{pmatrix}Q = P(E_{11}+E_{22}+\cdots+E_{rr})Q$($E_r$ 是 $r\times r$ 的单位矩阵,$E_{ii}$ 是 $n\times n$ 的只有第 $i$ 行 $i$ 列的这个元素为 $1$,其他元素均为 $0$ 的矩阵). 
        每个 $PE_{ii}Q$ 都是秩为 $1$ 的矩阵,故得证.
        \item 记 $r(A)=r$,把 $A$ 写成 $P\begin{pmatrix}E_r & 0 \\ 0 & 0\end{pmatrix}Q$ 的形式. 构造 $B=Q^{-1}\begin{pmatrix}E_r & 0 \\ 0 & 0\end{pmatrix}P^{-1}$ 可以发现其满足条件,故得证.
    \end{enumerate}
    \item $r(BC)\leq r(B) \leq 1$,得证. 
    
    反之,若 $A$ 是秩为 $1$ 的 $3\times 3$ 矩阵,则存在可逆矩阵 $P,Q$ 使得 $A=P^{-1}E_{11}Q^{-1}$,其中 $E_{11}=\begin{pmatrix}1 & 0 & 0 \\ 0 & 0 & 0 \\ 0 & 0 & 0\end{pmatrix}=\begin{pmatrix}1 \\ 0 \\ 0\end{pmatrix}\begin{pmatrix}1 & 0 & 0\end{pmatrix}$.
    则取 $B=P^{-1}\begin{pmatrix}1 \\ 0 \\ 0\end{pmatrix},C=\begin{pmatrix}1 & 0 & 0\end{pmatrix}Q^{-1}$,有 $A=BC$,证毕.
    \item \begin{enumerate}
        \item $r(\alpha \alpha^{\mathbf{T}})\leq r(\alpha) = 1,r(\beta \beta^{\mathbf{T}})\leq r(\beta) = 1$. 由 $r(A+B) \leq r(A)+r(B)$ 得 $r(A)=r(\alpha \alpha^{\mathbf{T}}+\beta \beta^{\mathbf{T}}) \leq r(\alpha)+r(\beta)=2.$
        \item 若 $\alpha,\beta$ 均为 $\mathbf{0}$ 向量,显然. 否则假设 $\alpha$ 不为 $0$,则由于两向量线性相关,必有确定的 $k$ 使得 $\beta = k\alpha$,把 $\beta$ 用 $\alpha$ 表示之后易证.
    \end{enumerate}
    \item $r(A)=r$ 则 $AX=0$ 的解空间维数 $\mathrm{dim}N(A) = n-r$. 由 $r(A)+r(B)=k$ 得 $r(B)=k-r \leq n-r=\mathrm{dim}N(A)$. 要求 $AB=O$,说明 $B$ 的列向量均为 $AX=0$ 的解,那么只需要选择合适的列向量组拼接成 $B$ 即可(这一定能做到,因为 $B$ 维数不会超过解空间维数).
    \item 由于 $A$ 是 $m\times n$ 矩阵,$r(A)=m$,可知对于矩阵 $A$ 做初等列变换,可使其前 $m$ 列变为单位矩阵,后 $n-m$ 列变为全 $0$ 列.
    因此,存在 $n$ 阶可逆矩阵 $P$ 使得
    \[AP=\begin{pmatrix}E_m & O_{m\times (n-m)}\end{pmatrix}\]
    于是\[AP(AP)^{\mathrm{T}} = \begin{pmatrix}E & O\end{pmatrix} \begin{pmatrix}E \\ O\end{pmatrix}=E_m\]
    所以存在 $B=(PP^{\mathrm{T}}A^{\mathrm{T}})$ 为 $n\times m$ 矩阵,使 $AB=E$.
    \item 利用 $A,B$ 的相抵标准形. 存在 $n$ 阶可逆矩阵 $P_1,Q_1,P_2,Q_2$ 使得
    \[P_1AQ_1=\begin{pmatrix}E_{r_A} & O \\ O & O\end{pmatrix},P_2BQ_2=\begin{pmatrix}O & O \\ O & E_{r_B}\end{pmatrix}\]
    于是 \[AQ_1=P_1^{-1}\begin{pmatrix}E_{r_A} & O \\ O & O\end{pmatrix},P_2B=\begin{pmatrix}O & O \\ O & E_{r_B}\end{pmatrix}Q_2^{-1}\]
    所以 \[AQ_1P_2B=P_1^{-1}\begin{pmatrix}E_{r_A} & O \\ O & O\end{pmatrix}\begin{pmatrix}O & O \\ O & E_{r_B}\end{pmatrix}Q_2^{-1}=O\]
    取 $M=Q_1P_2$ 即可.
    \item \begin{enumerate}
        \item 易证,此处略去.
        \item 注意到 $B$ 的列向量均为 $AX=0$ 的解,设 $AX=0$ 的基础解系为 $\alpha_1,\cdots,\alpha_t(t=n-r)$,则易知
        \[B_{11}=(\alpha_1,0,\cdots,0),B_{12}=(0,\alpha_1,\cdots,0),\cdots,B_{1n}=(0,0,\cdots,\alpha_1),\]
        \[B_{21}=(\alpha_2,0,\cdots,0),B_{22}=(0,\alpha_2,\cdots,0),\cdots,B_{2n}=(0,0,\cdots,\alpha_2),\]
        \[\vdots\]
        \[B_{t1}=(\alpha_t,0,\cdots,0),B_{t2}=(0,\alpha_t,\cdots,0),\cdots,B_{tn}=(0,0,\cdots,\alpha_t)\]
        为 $S(A)$ 的一组基,故 $\mathrm{dim}S(A)=n(n-r)$.
    \end{enumerate}
\end{enumerate}

\centerline{\heiti C组}
\begin{enumerate}
    \item 对 $\begin{pmatrix}E_n & A' \\ A & E_s\end{pmatrix}$ 利用打洞原理有
    \[\begin{pmatrix}E_n-A'A & O \\ O & E_s\end{pmatrix} \leftarrow \begin{pmatrix}E_n & A' \\ A & E_s\end{pmatrix} \rightarrow \begin{pmatrix}E_n & O \\ O & E_s-AA'\end{pmatrix}\]
    所以 $r\begin{pmatrix}E_n-A'A & O \\ O & E_s\end{pmatrix}=r\begin{pmatrix}E_n & O \\ O & E_s-AA'\end{pmatrix}$,即 $s+r(E_n-A'A)=n+r(E_s-AA')$,即
    \[r(E_n-A'A)-r(E_s-AA')=n-s.\]
    \item \begin{enumerate}
        \item 由 \[\begin{pmatrix}A & 0 \\ 0 & B\end{pmatrix}\rightarrow \begin{pmatrix}A & AC \\ 0 & B\end{pmatrix}\rightarrow \begin{pmatrix}A & AC+BD \\ 0 & B\end{pmatrix}=\begin{pmatrix}A & E \\ 0 & B\end{pmatrix}\]
        \[\rightarrow \begin{pmatrix}0 & E \\ -AB & B\end{pmatrix}\rightarrow \begin{pmatrix}0 & E \\ AB & 0\end{pmatrix}\]
        可得.
    \item 用分块矩阵的方法,我们知道 
    \[\begin{pmatrix}A & O \\ O & B\end{pmatrix}\rightarrow \begin{pmatrix}A & O \\ A & B\end{pmatrix}\rightarrow \begin{pmatrix}A & A \\ A & A+B\end{pmatrix}\]
    结合 $AB=BA$,我们知道
    \[\begin{pmatrix}A & A \\ A & A+B\end{pmatrix}\begin{pmatrix}A+B & O \\ -A & E\end{pmatrix}=\begin{pmatrix}AB & A \\ O & A+B\end{pmatrix}\]
    于是
    \[r(A)+r(B)=r\begin{pmatrix}A & O \\ O & B\end{pmatrix}=r\begin{pmatrix}A & A \\ A & A+B\end{pmatrix}\ge \begin{pmatrix}AB & A \\ O & A+B\end{pmatrix}\ge r(AB)+r(A+B)\] 
    \end{enumerate}
    \item 略有超纲,使用贝祖定理,
    \[\exists u(x),v(x),u(x)f_1(x)+v(x)f_2(x)=1\]
    \[r\begin{pmatrix}f_1(A) & O \\ O & f_2(A)\end{pmatrix}=r\begin{pmatrix}f_1(A) & f_1(A)u(A)+f_2(A)v(A) \\ O & f_2(A)\end{pmatrix}=r\begin{pmatrix}f_1(A) & E \\ O & f_2(A)\end{pmatrix}\] 
    \[=r\begin{pmatrix}f_1(A) & E \\ -f_2(A)f_1(A) & O\end{pmatrix}=r\begin{pmatrix}O & E \\ f(A) & O\end{pmatrix}\]
    \item 由于 $A$ 是列满秩矩阵,$B$ 是行满秩矩阵,知存在可逆矩阵 $P_{3\times 3},Q_{2\times 2}$ 使得
    \[A=P\begin{pmatrix}E_2 \\ O\end{pmatrix},B=\begin{pmatrix}E_2 & O\end{pmatrix}Q\]
    于是 \[BA=\begin{pmatrix}E_2 & O\end{pmatrix}QP\begin{pmatrix}E_2 \\ O\end{pmatrix}\]
    由 $(AB)^2=9AB$ 有 \[P\begin{pmatrix}E_2 \\ O\end{pmatrix}\begin{pmatrix}E_2 & O\end{pmatrix}QP\begin{pmatrix}E_2 \\ O\end{pmatrix}\begin{pmatrix}E_2 & O\end{pmatrix}Q=9P\begin{pmatrix}E_2 \\ O\end{pmatrix}\begin{pmatrix}E_2 & O\end{pmatrix}Q\]
    即 \[\begin{pmatrix}E_2 \\ O\end{pmatrix}BA\begin{pmatrix}E_2 & O\end{pmatrix}=9\begin{pmatrix}E_2 \\ O\end{pmatrix}\begin{pmatrix}E_2 & O\end{pmatrix}\]
    也就是 \[\begin{pmatrix}BA & O \\ O & O\end{pmatrix}=\begin{pmatrix}9E_2 & 0 \\ 0 & 0\end{pmatrix}\]
    所以 $BA=9E_2$.
    \item 本题求核空间困难,但只需要求维数,我们考虑求像空间之后求出像空间维数,然后用维数公式求解.
    
    取 $F^{n\times p}$ 的自然基 $\{e_{11},e_{12},\cdots,e_{np}\}$($e_{ij}$ 表示仅有第 $i$ 行第 $j$ 列的元素为 $1$,其他均为 $0$ 的矩阵)

    则 $\mathrm{Im}\ \sigma=\mathrm{span}(\sigma(e_{11}),\cdots,\sigma(e_{np}))$.

    取 $A$ 的列向量,写成 $A=\begin{pmatrix}\alpha_1,\alpha_2,\cdots,\alpha_n\end{pmatrix}$,则 $\sigma(e_{ij})$ 可排列如下:
    \[(\alpha_1,0,\cdots,0),(0,\alpha_1,\cdots,0),\cdots,(0,0,\cdots,\alpha_1)\]
    \[(\alpha_2,0,\cdots,0),(0,\alpha_2,\cdots,0),\cdots,(0,0,\cdots,\alpha_2)\]
    \[\cdots\]
    \[(\alpha_n,0,\cdots,0),(0,\alpha_n,\cdots,0),\cdots,(0,0,\cdots,\alpha_n)\]
    由于 $r(A)=r$,故 $\alpha_1,\alpha_2,\cdots,\alpha_n$ 的极大线性无关组有 $r$ 个向量,不妨设为 $\alpha_1,\alpha_2,\cdots,\alpha_r$. 则下列向量:
    \[(\alpha_{r+1},0,\cdots,0),(0,\alpha_{r+1},\cdots,0),\cdots,(0,0,\cdots,\alpha_{r+1})\]
    \[\cdots\]
    \[(\alpha_n,0,\cdots,0),(0,\alpha_n,\cdots,0),\cdots,(0,0,\cdots,\alpha_n)\]
    均可以被其他向量线性表出. 观察除了上述向量的剩下的向量,可以发现这 $r\times p$ 个向量线性无关,从而 $\mathrm{dim}(\mathrm{Im}\ \sigma) = r\times p$.

    故由维数公式,得 $\mathrm{dim}(\mathrm{Ker}\ \sigma) = \mathrm{dim}F^{n\times p}-\mathrm{dim}(\mathrm{Im} \ \sigma) = (n-r)p$.
\end{enumerate}

\clearpage

\input{./专题/12 矩阵运算进阶(II).tex}
\chapter{行列式}

接下来我们将开始介绍大部分线性代数或高等代数教材中开头就会介绍的内容——行列式. 在本讲义的思路中,我们更多将行列式视为一个帮助我们研究的工具,无论是当前的主线——线性方程组解的理论还是之后我们要介绍的矩阵标准形的内容. 因此我们会将这一章只作类似``工具介绍''的作用,而非其他教材那样从行列式出发引出相关概念,因为我们研究的核心和出发点是之前的抽象空间和映射.

事实上,我们将在本讲义之后再介绍一次行列式,那时我们将会介绍行列式更丰富的应用,并表明是否引入行列式可能对于线性代数完整理论的构建而言重要程度是有限的. 但是我们完全无法否认行列式的历史地位,从17世纪起行列式就是用于求解线性方程组的重要的工具,历经数百年也逐渐发展出了许多重要的理论和应用,因此我们仍然需要完整的章节来讲述行列式的相关内容.

\section{行列式的定义}

很多教材采用``逆序数''定义行列式,但是本教材未提及,而且也缺乏直观,因此我们不在本讲展开描述. 我们会在史海拾遗中结合历史给出相关的定义,当然感兴趣的同学可以参考丘维声《高等代数》等教材. 本教材使用公理化定义(使用一些规则描述)并讲解了递归式定义(按行(列)展开).

\subsection{公理化定义}

\begin{definition}[行列式] \label{def:13:公理化定义} \index{hanglieshi@行列式 (determinant)}
    数域$\mathbf{F}$上的一个$n$阶\term{行列式}是取值于$\mathbf{F}$的$n$个$n$维向量$\alpha_1,\alpha_2,\ldots,\alpha_n \in \mathbf{F}^n$的一个函数,且$\forall \alpha_i,\beta_i \in \mathbf{F}^n$和$\forall \lambda \in \mathbf{F}$,满足下列规则:
    \begin{enumerate}
        \item \label{item:13:齐性}
              (齐性) $D(\alpha_1,\ldots,\lambda\alpha_i,\ldots,\alpha_n)=\lambda D(\alpha_1,\ldots,\alpha_i,\ldots,\alpha_n)$;

        \item \label{item:13:加性}
              (加性,与 \ref*{item:13:齐性} 合称线性性) \\
              $D(\alpha_1,\ldots,\alpha_i+\beta_i,\ldots,\alpha_n)=D(\alpha_1,\ldots,\alpha_i,\ldots,\alpha_n)+D(\alpha_1,\ldots,\beta_i,\ldots,\alpha_n)$;

        \item \label{item:13:反对称性}
              (反对称性) $D(\alpha_1,\ldots,\alpha_i,\ldots,\alpha_j,\ldots,\alpha_n)=-D(\alpha_1,\ldots,\alpha_j,\ldots,\alpha_i,\ldots,\alpha_n)$;

        \item \label{item:13:规范性}
              (规范性) $D(e_1,e_2,\ldots,e_n)=1$.
    \end{enumerate}
\end{definition}
在公理化定义中,我们将行列式定义为一个满足特定的运算性质的从列向量组合到数的函数. 事实上,公理化定义从是逆序数定义可以推导出的行列式的运算性质,教材采用这种定义避开了繁琐的说明.

除此之外,我们不难看出公理化定义可以形象地理解为对$n$维空间中体积的定义,对几何意义感兴趣的同学可以参考 \href{https://b23.tv/BV1ys411472E}{3b1b《线性代数的本质》系列视频}相关内容.
\begin{example} \label{ex:13:公理化定义}
    使用\autoref{def:13:公理化定义} 验证下述命题的正确性:
    \begin{enumerate}
        \item 若行列式有一列为零向量,则行列式的值等于0.

        \item 若行列式有两列元素相同,则行列式的值等于0.

        \item 若行列式有两列元素成比例,则行列式的值等于0.

        \item 对行列式做倍加列变换,行列式的值不变.

        \item 若$\alpha_1,\alpha_2,\ldots,\alpha_n$线性相关,则$D(\alpha_1,\alpha_2,\ldots,\alpha_n)=0$.
    \end{enumerate}
\end{example}

\begin{proof}
    \begin{enumerate}
        \item 由于行列式满足\autoref{def:13:公理化定义} 的\ref*{item:13:齐性},设行列式第$i$列为零向量,因此
              \begin{align*}
                  D(\alpha_1,\ldots,0,\ldots,\alpha_n) & =D(\alpha_1,\ldots,0\cdot\alpha_i,\ldots,\alpha_n)  \\
                                                       & =0\cdot D(\alpha_1,\ldots,\alpha_i,\ldots,\alpha_n) \\
                                                       & =0
              \end{align*}

        \item 由于行列式满足\autoref{def:13:公理化定义} 的\ref*{item:13:反对称性},设行列式第$i$列和第$j$列元素相同,因此
              \begin{align*}
                  D(\alpha_1,\ldots,\alpha_i,\ldots,\alpha_j,\ldots,\alpha_n) & =D(\alpha_1,\ldots,\alpha_j,\ldots,\alpha_i,\ldots,\alpha_n)  \\
                                                                              & =-D(\alpha_1,\ldots,\alpha_i,\ldots,\alpha_j,\ldots,\alpha_n)
              \end{align*}
              从而$D(\alpha_1,\ldots,\alpha_i,\ldots,\alpha_j,\ldots,\alpha_n)=0$.

        \item 由于行列式满足\autoref{def:13:公理化定义} 的\ref*{item:13:齐性},设行列式第$i$列和第$j$列元素成比例,$\alpha_i=k\alpha_j$,因此
              \begin{align*}
                  D(\alpha_1,\ldots,\alpha_i,\ldots,\alpha_j,\ldots,\alpha_n)
                   & =D(\alpha_1,\ldots,k\alpha_j,\ldots,\alpha_j,\ldots,\alpha_n) \\
                   & =kD(\alpha_1,\ldots,\alpha_j,\ldots,\alpha_j,\ldots,\alpha_n) \\
                   & =0
              \end{align*}
              其中最后一个等号用到了本例的第二条结论.

        \item 事实上,根据\autoref{def:13:公理化定义} 的\ref*{item:13:加性} 以及本例第3条结论,我们可以得到
              \begin{align*}
                  D(\alpha_1,\ldots,\alpha_i+k\alpha_j,\ldots,\alpha_j,\ldots,\alpha_n)
                   & =D(\alpha_1,\ldots,\alpha_i,\ldots,\alpha_j,\ldots,\alpha_n)   \\&+D(\alpha_1,\ldots,k\alpha_j,\ldots,\alpha_j,\ldots,\alpha_n) \\
                   & =D(\alpha_1,\ldots,\alpha_i,\ldots,\alpha_j,\ldots,\alpha_n)+0 \\
                   & =D(\alpha_1,\ldots,\alpha_i,\ldots,\alpha_j,\ldots,\alpha_n).
              \end{align*}

        \item 设$\alpha_1,\alpha_2,\ldots,\alpha_n$线性相关,因此存在不全为0的数$k_1,k_2,\ldots,k_n$使得$k_1\alpha_1+k_2\alpha_2+\cdots+k_n\alpha_n=0$,不妨设$k_1 \neq 0$,因此
              \[\alpha_1=-\frac{k_2}{k_1}\alpha_2-\frac{k_3}{k_1}\alpha_3-\cdots-\frac{k_n}{k_1}\alpha_n,\]
              因此
              \begin{align*}
                  D(\alpha_1,\alpha_2,\ldots,\alpha_n) & =D(-\frac{k_2}{k_1}\alpha_2-\frac{k_3}{k_1}\alpha_3-\cdots-\frac{k_n}{k_1}\alpha_n,\alpha_2,\ldots,\alpha_n)     \\
                                                       & =-\frac{k_2}{k_1}D(\alpha_2,\alpha_2,\ldots,\alpha_n)-\cdots-\frac{k_n}{k_1}D(\alpha_n,\alpha_2,\ldots,\alpha_n) \\
                                                       & =0.
              \end{align*}
    \end{enumerate}
\end{proof}

\begin{example} \label{ex:13:公理化定义2}
    设向量$\alpha_1,\alpha_2,\beta_1,\beta_2$为三维列向量,又$A=(\alpha_1,\alpha_2,\beta_1),B=(\alpha_1,\alpha_2,\beta_2)$,且$|A|=3$,$|B|=2$,求$|2A+3B|$.
\end{example}

\begin{solution}
    $2A+3B=(2\alpha_1+3\alpha_1,2\alpha_2+3\alpha_2,2\beta_1+3\beta_2)=(5\alpha_1,5\alpha_2,2\beta_1+3\beta_2)$,因此
    \begin{align*}
        |2A+3B| & =|5\alpha_1,5\alpha_2,2\beta_1+3\beta_2|                       \\
                & =25|\alpha_1,\alpha_2,2\beta_1+3\beta_2|                       \\
                & =25(2|\alpha_1,\alpha_2,\beta_1|+3|\alpha_1,\alpha_2,\beta_2|) \\
                & =25(2|A|+3|B|)=300.                                            \\
    \end{align*}
\end{solution}

\subsection{递归式定义}

首先我们需要引入余子式和代数余子式的概念:
\begin{definition} \label{def:13:余子式}
    在$n$阶行列式$D=|a_{ij}|_{n \times n}$中,去掉元素$a_{ij}$所在的第$i$行和第$j$列的所有元素而得到的$n-1$阶行列式称为元素$a_{ij}$的\term{余子式}\index{yuzishi@余子式 (minor)},记作$M_{ij}$,并把数$A_{ij}=(-1)^{i+j}M_{ij}$称为元素$a_{ij}$的\term{代数余子式}\index{yuzishi!daishu@代数余子式 (cofactor)}.
\end{definition}
注意,虽然余子式和代数余子式在名称中含有式,但实际上他们是一个值. 实际上行列式也称为``式'',但这些``式''只是形状上有个形式,实际上只是一个值.
\begin{example} \label{ex:13:余子式}
    根据\autoref{def:13:余子式} 计算行列式$\begin{vmatrix}
            2  & 1 & 3  \\
            -1 & 0 & 2  \\
            1  & 5 & -2
        \end{vmatrix}$每个元素的余子式和代数余子式.
\end{example}

\begin{solution}
    我们只举一个例子,第二行第一列元素$-1$的余子式和代数余子式. 根据定义,它的余子式是去掉第二行和第一列所有元素剩余的二阶行列式
    \[\begin{vmatrix}
            1 & 3  \\
            5 & -2
        \end{vmatrix}=-17,\]
    因此它的代数余子式是$A_{21}=(-1)^{2+1}(-17)=17$. 读者可以自行计算其他元素的余子式和代数余子式.
\end{solution}

接下来我们便可以给出递归式定义:
\begin{definition} \label{def:13:递归式定义}
    设$D=|a_{ij}|_{n \times n}$,则
    \begin{align}
        \label{eq:13:递归式定义1}
        D=\sum_{k=1}^{n}a_{kj}A_{kj}=a_{1j}A_{1j}+a_{2j}A_{2j}+\cdots+a_{nj}A_{nj} & \qquad j=1,2,\ldots,n \\
        \label{eq:13:递归式定义2}
        D=\sum_{k=1}^{n}a_{ik}A_{ik}=a_{i1}A_{i1}+a_{i2}A_{i2}+\cdots+a_{in}A_{in} & \qquad i=1,2,\ldots,n
    \end{align}
\end{definition}
其中$A_{ij}$即为\autoref{def:13:余子式} 给出的代数余子式,\autoref{eq:13:递归式定义1} 称为$D$对第$j$列的展开式,\autoref{eq:13:递归式定义2} 称为$D$对第$i$行的展开式. 事实上,这一定义被称为递归式定义的原因是显然的(如果在程序设计课程中已经学习过递归的概念),它使用$n-1$阶行列式定义$n$阶行列式,因此我们对任意$n$阶行列式都可以递归展开到1阶,从而得到最终行列式计算结果.

除此之外,我们需要强调的是,这里的递归式定义能称之为定义,必须要使得其与之前的公理化定义不冲突. 事实上二者等价的证明都是技术性的,教材175页定理5.1说明了我们如何从公理化定义推出递归式定义,反过来我们只需要对公理化定义中每个性质利用公理化定义逐个展开验算即可,我们放在习题中供感兴趣的读者自行验证. 因为都是技术性的问题,这里不展开叙述,事实上也不是我们核心的内容.
\begin{example} \label{ex:13:递归式定义}
    利用\autoref{def:13:递归式定义} 计算\autoref{ex:13:余子式} 中的行列式,可以行列展开均使用并在上述公式中选取不同$i$和$j$以熟悉\autoref*{def:13:递归式定义},并注意体会递归式定义的含义.
\end{example}

\begin{solution}
    我们选取$i=2$进行按行展开,由\autoref{def:13:余子式} 可知$A_{21}=17,A_{22}=-7,A_{23}=-9$,因此
    \begin{align*}
        D & =\sum_{k=1}^{3}a_{2k}A_{2k}              \\
          & =a_{21}A_{21}+a_{22}A_{22}+a_{23}A_{23}  \\
          & =(-1) \cdot 17+0 \cdot (-7)+2 \cdot (-9) \\
          & =-35.
    \end{align*}
    同理,我们选取$j=3$进行按列展开,由\autoref{def:13:余子式} 可知$A_{13}=-5,A_{23}=-9,A_{33}=1$,因此
    \begin{align*}
        D & =\sum_{k=1}^{3}a_{k3}A_{k3}             \\
          & =a_{13}A_{13}+a_{23}A_{23}+a_{33}A_{33} \\
          & =3 \cdot (-5)+2 \cdot (-9)+(-2) \cdot 1 \\
          & =-35.
    \end{align*}
    读者可以自行计算按其他行列展开的结果.
\end{solution}

递归式定义有一个重要的结论如下:
\begin{theorem} \label{thm:13:递归性质}
    $n$阶行列式$D=|a_{ij}|_{n \times n}$的某一行(列)元素与另一行(列)相应元素的代数余子式的乘积之和等于0,即
    \begin{align}
        \label{eq:13:递归式定义3}
        \sum_{k=1}^{n}a_{kj}A_{ki}=a_{1j}A_{1i}+a_{2j}A_{2i}+\cdots+a_{nj}A_{ni}=0 & \qquad j \neq i \\
        \label{eq:13:递归式定义4}
        \sum_{k=1}^{n}a_{jk}A_{ik}=a_{j1}A_{i1}+a_{j2}A_{i2}+\cdots+a_{jn}A_{in}=0 & \qquad j \neq i
    \end{align}
\end{theorem}

我们简要说一下定理的证明. 虽然这一定理看着下标满天飞,似乎很难证明,但如果我们首先将第$j$列元素替换为第$i$列元素,然后根据\autoref{def:13:递归式定义} 按第$j$列展开求行列式,这一结果一定是0,因为此时矩阵第$i$和$j$两列完全相同. 同时我们发现,我们上面展开写出的式子就是\autoref{eq:13:递归式定义3}(注意此时$a_{ki}=a_{kj}$),由此得证.

到目前为止,读者可能对\crefrange*{eq:13:递归式定义1}{eq:13:递归式定义4} 式繁杂的下标感到陌生,因此安排了\crefrange*{ex:13:公理化定义2}{ex:13:递归式定义} 希望大家熟悉这些公式.
\begin{example} \label{ex:13:递归式定义2}
    对\autoref{ex:13:递归式定义} 中的矩阵验证\autoref{thm:13:递归性质} 的正确性.
\end{example}

\begin{solution}
    例如我们选取第一行元素和第二行的代数余子式,由\autoref{def:13:余子式} 可知$A_{21}=17,A_{22}=-7,A_{23}=-9$,因此
    \begin{align*}
        D & =\sum_{k=1}^{3}a_{1k}A_{2k}             \\
          & =a_{11}A_{21}+a_{12}A_{22}+a_{13}A_{23} \\
          & =2 \cdot 17+1 \cdot (-7)+3 \cdot (-9)   \\
          & =0.
    \end{align*}
\end{solution}

这一节中行列式是按照一行(列)展开的,若按多行(列)展开则需要相应的 Laplace 定理,我们将在下一讲行列式计算进阶中介绍.

\subsection{行列式的常用性质}

设$A,B \in \mathbf{F}^{n \times n}$,$k \in \mathbf{F}$,则
\begin{enumerate}
    \item 一般情况下,$|A \pm B| \neq |A|\pm|B|$;

    \item $|kA|=k^n|A|$;
          \begin{proof}
              由\autoref{def:13:公理化定义} 的\ref*{item:13:齐性},设$A=(\alpha_1,\alpha_2,\ldots,\alpha_n)$,则
              \begin{align*}
                  |kA| & =|k\alpha_1,k\alpha_2,\ldots,k\alpha_n| \\
                       & =k^n|\alpha_1,\alpha_2,\ldots,\alpha_n| \\
                       & =k^n|A|.
              \end{align*}
          \end{proof}

    \item \label{item:13:行列式性质:4}
          初等矩阵行列式(注意初等矩阵不分行列,左乘右乘区分初等行列变换):\\
          $|E_{ij}|=-1,\enspace |E_i(c)|=c,\enspace |E_{ij}(k)|=1$;

    \item 利用 \ref*{item:13:行列式性质:4} 中的结论可以得到$|AB|=|A||B|,\enspace|A^k|=|A|^k$;

    \item 利用 \ref*{item:13:行列式性质:4} 中的结论可以得到$A$可逆$\iff |A| \neq 0$;

    \item 利用 \ref*{item:13:行列式性质:4} 中的结论可以得到$|A^\mathrm{T}|=|A|$;

    \item 利用 \ref*{item:13:行列式性质:4} 中的结论可以得到上、下三角矩阵行列式均为主对角线元素的乘积;

    \item 利用 \ref*{item:13:行列式性质:4} 中的结论(求出初等矩阵逆矩阵行列式)可以得到若$A$可逆,则$|A^{-1}|=|A|^{-1}$.
          \begin{proof}
              由$|AB|=|A||B|$,设$B=A^{-1}$,则$|E|=|AA^{-1}|=|A||A^{-1}|$,因此$|A||A^{-1}|=1$,从而$|A^{-1}|=|A|^{-1}$.
          \end{proof}
\end{enumerate}

以上性质都可以基于定义或上述其他性质得到,其中3-6条可参考教材5.3节,7可参考教材172页例3. 这些性质结论希望读者熟悉,实际上推导过程重要性不大,虽然也并不复杂. 下面介绍的性质需要用到``打洞法''(分块矩阵初等变换)来证明:

\begin{enumerate}
    \item $\begin{vmatrix}
                  A & O \\ O & B
              \end{vmatrix} = \begin{vmatrix}
                  A & O \\ C & B
              \end{vmatrix} = \begin{vmatrix}
                  A & D \\ O & B
              \end{vmatrix} = |A||B|,\enspace\begin{vmatrix}
                  O & A \\ B & C
              \end{vmatrix} = (-1)^{kr}|A||B|$;
          \begin{proof}
              证明从略,感兴趣的读者可以参考教材179页例2,实际上我们很多时候只需要基于这些结论证明进一步的性质.
          \end{proof}

    \item 当$A$可逆时,有$\begin{vmatrix}
                  A & B \\ C & D
              \end{vmatrix} = |A||D-CA^{-1}B|$,当$D$可逆时,有$\begin{vmatrix}
                  A & B \\ C & D
              \end{vmatrix} = |D||A-BD^{-1}C|$,当$B$可逆时,有$\begin{vmatrix}
                  A & B \\ C & D
              \end{vmatrix} = (-1)^{mn}|B||C-DB^{-1}A|$,当$C$可逆时,有$\begin{vmatrix}
                  A & B \\ C & D
              \end{vmatrix} = (-1)^{mn}|C||B-AC^{-1}D|$;
          \begin{proof}
              由于
              \[\begin{pmatrix}
                      E & O \\ -CA^{-1} & E
                  \end{pmatrix}\begin{pmatrix}
                      A & B \\ C & D
                  \end{pmatrix}= \begin{pmatrix}
                      A & B \\ O & D-CA^{-1}B
                  \end{pmatrix},\]
              两边取行列式,并注意到
              \[\begin{vmatrix}
                      E & O \\ -CA^{-1} & E
                  \end{vmatrix}=1,\]
              因此
              \[\begin{vmatrix}
                      A & B \\ C & D
                  \end{vmatrix}=\begin{vmatrix}
                      A & B \\ O & D-CA^{-1}B
                  \end{vmatrix}=|A||D-CA^{-1}B|.\]

              读者会发现,我们在上面的证明中多次使用$\begin{vmatrix}
                      A & O \\ C & B
                  \end{vmatrix} = \begin{vmatrix}
                      A & D \\ O & B
                  \end{vmatrix} = |A||B|$这一性质,因此这一性质是相当重要的,需要读者熟悉.

              本条的其他结论推导类似于上方,在此不再赘述,感兴趣的读者可以自行推导,实际上结论并不是很重要,重要的是在于领悟使用行列式分块计算性质和打洞法的基本方法.
          \end{proof}

    \item 设$A,B$分别是$n \times m$和$m \times n$矩阵,则$|E_n \pm AB|=|E_m \pm BA|$,且 \\
          $|\lambda E_n \pm AB|=\lambda^{n-m}|\lambda E_m \pm BA|,\enspace n \geqslant m$.
          \begin{proof}
              由前述第二条性质直接可得
              \[|E_n \pm AB|=\begin{vmatrix}
                      E_n & A \\ \mp B & E_m
                  \end{vmatrix}=|E_m \pm BA|,\]
              也有
              \begin{align*}
                  |\lambda E_n \pm AB|
                   & =\begin{vmatrix}
                          \lambda E_n & A \\ \mp B & E_m
                      \end{vmatrix}=\lambda^n
                  \begin{vmatrix}
                      E_n & \lambda^{-1}A \\ \mp B & E_m
                  \end{vmatrix}=\lambda^{n-m}\begin{vmatrix}
                                                 E_n & A \\ \mp B & E_m
                                             \end{vmatrix} \\
                   & = \lambda^{n-m}|\lambda E_m \pm BA|.
              \end{align*}
              其中第一行第二个等号来源于前$n$行每行提出一个$\lambda$,第一行第三个等号来源于后$m$列每列乘以$\lambda$.
          \end{proof}

          事实上,这里的结果在特征值与特征向量的部分中我们会给出更深入的解释.
\end{enumerate}

还有一部分由这些性质可以推导的其他性质将出现在C组习题中供参考. 这部分主要是技巧性内容,可以选择性完成.

\section{行列式的基本运算}

本节内容按照往年经验不是考试重点,但是我们要保证教材中涉及的的方法都掌握. 本节我们简要说明教材中提及的基本行列式计算方法,在下一讲行列式计算进阶中我们将用一整讲详细展开行列式的计算技巧.

首先我们用一个简单的三阶行列式的例子回顾行列式的多种基本计算方法. 这里选取三阶行列式主要原因也是三阶行列式在未来实际解题中最为常见,这里希望读者比较选择最适合自己的方法在未来更便捷地使用:
\begin{example}
    计算行列式$D=\begin{vmatrix}
            1 & 2 & 3 \\
            2 & 3 & 1 \\
            3 & 1 & 2
        \end{vmatrix}$.
\end{example}

\begin{solution}
    \begin{enumerate}
        \item (公理化定义与性质)参考教材171页例2的方法,此处展开较为复杂不再赘述,也不推荐使用这一方法.

        \item (公式法,实际上就是逆序数定义)我们知道三阶行列式的计算公式为(直接展开也很容易验证)$D=\begin{vmatrix}
                      a_{11} & a_{12} & a_{13} \\
                      a_{21} & a_{22} & a_{23} \\
                      a_{31} & a_{32} & a_{33}
                  \end{vmatrix}=a_{11}a_{22}a_{33}+a_{12}a_{23}a_{31}+a_{13}a_{21}a_{32}-a_{13}a_{22}a_{31}-a_{12}a_{21}a_{33}-a_{11}a_{23}a_{32}$,因此$D=1 \cdot 3 \cdot 2+2 \cdot 1 \cdot 3+3 \cdot 2 \cdot 1-3 \cdot 3 \cdot 3-2 \cdot 2 \cdot 2-1 \cdot 1 \cdot 1=-18$.

        \item (化为上三角形式)参考教材172页例4,具体过程不在此展开.

        \item (递归式定义展开)我们对第一行展开,由\autoref{def:13:递归式定义} 可知
              \begin{align*}
                  D & =1 \cdot
                  \begin{vmatrix}
                      3 & 1 \\
                      1 & 2
                  \end{vmatrix}-2
                  \cdot \begin{vmatrix}
                            2 & 1 \\
                            3 & 2
                        \end{vmatrix}+3
                  \cdot \begin{vmatrix}
                            2 & 3 \\
                            3 & 1
                        \end{vmatrix}                           \\
                    & =1 \cdot (6-1)-2 \cdot (4-3)+3 \cdot (2-9) \\
                    & =-18.
              \end{align*}
    \end{enumerate}
\end{solution}

接下来我们需要介绍一个非常重要的行列式,我们称之为Vandermonde行列式:
\begin{example}
    证明:$n$阶Vandermonde行列式
    \[V_n=\begin{vmatrix}
            1         & 1         & \cdots & 1         \\
            x_1       & x_2       & \cdots & x_n       \\
            \vdots    & \vdots    & \ddots & \vdots    \\
            x_1^{n-1} & x_2^{n-1} & \cdots & x_n^{n-1}
        \end{vmatrix}=\prod_{1 \leqslant i < j \leqslant n}(x_j-x_i)\]
\end{example}

教材177--178页例2给出了对上式的详细解释以及证明,这里我们不再赘述. 我们需要强调的是Vandermonde行列式的重要性,事实上,Vandermonde行列式有着广泛的应用,在之后不少的习题中我们将使用它. 在此我们证明\autoref{thm:4:覆盖定理} 的有限维情形作为一个例子:
\begin{example}\label{ex:13:行列式证明覆盖定理}
    设$V_1,V_2,\ldots,V_s$是有限维线性空间$V$的$s$个非平凡子空间,证明:$V$中至少存在一个向量不属于$V_1,V_2,\ldots,V_s$中的任何一个,即$V_1 \cup V_2 \cup \cdots \cup V_s\subsetneq V.$
\end{example}

\begin{proof}
    设$\dim V=n$,设$\alpha_1,\alpha_2,\ldots,\alpha_n$为$V$的一组基,构造向量组$\{\beta_k\}$中每个元素满足
    \[\beta_k=\alpha_1+k\alpha_2+\cdots+k^{n-1}\alpha_n,\enspace k=1,2,3,\ldots\]
    任取上述向量组中的$n$个向量$\beta_{k_1},\beta_{k_2},\ldots,\beta_{k_n}$,其中$k_1<k_2<\cdots<k_n$,则有
    \[(\beta_{k_1},\beta_{k_2},\ldots,\beta_{k_n})=(\alpha_1,\alpha_2,\ldots,\alpha_n)C\]
    其中
    \[C=\begin{pmatrix}
            1         & 1         & \cdots & 1         \\
            k_1       & k_2       & \cdots & k_n       \\
            \vdots    & \vdots    & \ddots & \vdots    \\
            k_1^{n-1} & k_2^{n-1} & \cdots & k_n^{n-1}
        \end{pmatrix}\]
    则$|C|$是一个 Vandermonde 行列式. 由 Vandermonde 行列式的性质可知$|C| \neq 0$,因此$C$可逆. 又由于$\alpha_1,\alpha_2,\ldots,\alpha_n$是$V$的一组基,因此$\beta_{k_1},\beta_{k_2},\ldots,\beta_{k_n}$线性无关,从而向量组$\{\beta_k\}$中任意$n$个向量均构成$V$的一组基.

    由于$V_1,V_2,\ldots,V_s$是$V$的非平凡子空间,因此每个子空间最多包含$\{\beta_k\}$中$n-1$个向量,进而$V_1\cup V_2\cup\cdots\cup V_s$只包含$\{\beta_k\}$中有限个向量,所以必然存在一个向量$\beta_j$使得$\beta_j \notin V_1\cup V_2\cup\cdots\cup V_s$.
\end{proof}

除此之外教材上还有一些基于递推的方法的例子,我们将会在下一讲中展开介绍这一方法以及其它技巧性更强的方法.

\section{伴随矩阵}

伴随矩阵是一个重要的概念,它给出了逆矩阵与原矩阵的关联,并且其性质都比较经典,很适合于练习.
\begin{definition}[伴随矩阵] \index{bansuijuzhen@伴随矩阵 (adjugate matrix)}
    称矩阵$A^*=\begin{pmatrix}
            A_{11} & A_{21} & \cdots & A_{n1} \\
            A_{12} & A_{22} & \cdots & A_{n2} \\
            \vdots & \vdots & \ddots & \vdots \\
            A_{1n} & A_{2n} & \cdots & A_{nn}
        \end{pmatrix}$为$A$的\term{伴随矩阵},其中$A_{ij}$是元素$a_{ij}$的代数余子式.
\end{definition}
我们要特别注意伴随矩阵代数余子式的下标与通常矩阵下标不一致,与转置下标一致. 伴随矩阵具有以下几个重要性质,我们将给出大部分性质的证明,部分性质我们放在朝花夕拾中证明:
\begin{example} \label{ex:13:伴随矩阵}
    证明下列关于$n$阶矩阵$A$的伴随矩阵$A^*$的性质:
    \begin{enumerate}
        \item \label{item:13:伴随矩阵:1}
              $AA^*=A^*A=|A|E$,若$A$可逆,则有$A^{-1}=|A|^{-1}A^*,\enspace A^*=|A|A^{-1},\enspace (A^*)^{-1}=(A^{-1})^*=|A|^{-1}A$.

        \item $|A^*|=|A|^{n-1}$,无论$A$是否可逆.

        \item \label{item:13:伴随矩阵:3}
              $(AB)^*=B^*A^*,\enspace (A^\mathrm{T})^*=(A^*)^\mathrm{T},\enspace (kA)^*=k^{n-1}A^*$,要求掌握$A$和$B$可逆时的证明,若不可逆则需要使用第二节习题C组中对角占优的推论证明.

        \item $A$可逆时,$(A^*)^*=|A|^{n-2}A,\enspace |(A^*)^*|=|A|^{(n-1)^2}$(本题结论可以推广到更多重的伴随矩阵).

        \item 对正整数$k$,$(A^k)^*=(A^*)^k$.

        \item $r(A^*)=\begin{cases}
                      n & r(A)=n     \\
                      1 & r(A)=n-1   \\
                      0 & r(A) < n-1
                  \end{cases}$.
    \end{enumerate}
\end{example}

\begin{proof}
    \begin{enumerate}
        \item 由\autoref{eq:13:递归式定义3} 和\autoref{eq:13:递归式定义4},$AA^*$的第$i$行第$j$列元素为
              \[\sum_{k=1}^{n}a_{ik}A_{kj}=\begin{cases}
                      |A| & i=j      \\
                      0   & i \neq j
                  \end{cases}\]
              因此$AA^*=|A|E$,同理可证$A^*A=|A|E$.

              若$A$可逆,则$|A| \neq 0$,从而由$AA^*=|A|E$可知$A^{-1}=|A|^{-1}A^*$,$A^*=|A|A^{-1}$,$(A^*)^{-1}=|A|^{-1}A$.

              而我们知道$(A^{-1})^*A^{-1}=|A^{-1}|E=|A|^{-1}E$,因此$(A^{-1})^*=|A|^{-1}A$.

        \item 由$AA^*=|A|E$,$|AA^*|=|A||A^*|=|A|^n$,因此$|A^*|=|A|^{n-1}$.

        \item 只证明$A$和$B$可逆的情况,由$A^*=|A|A^{-1}$可知,$(AB)^*=|AB|(AB)^{-1}=|A||B|B^{-1}A^{-1}=B^*A^*$.

              由$(A^\mathrm{T})^*=|A^\mathrm{T}|(A^\mathrm{T})^{-1}=|A|(A^{-1})^\mathrm{T}=(|A|A^{-1})^\mathrm{T}=(A^*)^\mathrm{T}$.

              由$(kA)^*=|kA|(kA)^{-1}=k^n|A|\cdot k^{-1}A^{-1}=k^{n-1}A^*$.

        \item 由$(A^*)^*=|A^*|A^{*-1}$,$|A^*|=|A|^{n-1}$,$(A^*)^{-1}=|A|^{-1}A$,可知$(A^*)^*=|A|^{n-2}A$. 由$|(A^*)^*|=||A|^{n-2}A|^{n-1}=|A|^{n(n-2)+1}=|A|^{(n-1)^2}$.

        \item 由$(A^k)^*=|A^k|(A^k)^{-1}=|A|^k(A^{-1})^k=(|A|A^{-1})^k=(A^*)^k$.

        \item 证明见\autoref{ex:15:伴随矩阵的秩}.
    \end{enumerate}
\end{proof}

在计算行列式时若出现伴随矩阵,我们经常使用\autoref{ex:13:伴随矩阵} 中的 \ref*{item:13:伴随矩阵:1},\ref*{item:13:伴随矩阵:3} 进行计算.

使用伴随矩阵求逆矩阵是一种矩阵求逆的方式,我们通过一个简单的例子复习:
\begin{example}
    判断矩阵$\begin{pmatrix}
            1 & 2 & 3 \\ 2 & 1 & 2 \\ 1 & 3 & 3
        \end{pmatrix}$是否可逆. 若可逆,利用伴随矩阵求其逆矩阵.
\end{example}

\begin{solution}
    见教材182页例1.
\end{solution}

\section{Cramer法则}

从历史角度来开,引入行列式是用于求解线性方程组的. 瑞士数学家克莱姆(Cramer)于1750年在他的《线性代数分析导言》中发表了这一方法. 事实上莱布尼兹〔1693〕,以及麦克劳林〔1748〕亦研究了这一法则,但他们的记法不如克莱姆清晰. 接下来我们介绍这一充满历史底蕴的定理:
\begin{theorem}[Cramer法则] \label{thm:13:Cramer} \index{Cramer@Cramer 法则 (Cramer's rule)}
    对线性方程组
    \begin{gather}
        \label{eq:13:线性方程组1}
        \begin{cases} \begin{aligned}
                a_{11}x_1+a_{12}x_2+\cdots+a_{1n}x_n & = 0             \\
                a_{21}x_1+a_{22}x_2+\cdots+a_{2n}x_n & = 0             \\
                                                     & \vdotswithin{=} \\
                a_{n1}x_1+a_{n2}x_2+\cdots+a_{nn}x_n & = 0
            \end{aligned} \end{cases}
        \\
        \label{eq:13:线性方程组2}
        \begin{cases} \begin{aligned}
                a_{11}x_1+a_{12}x_2+\cdots+a_{1n}x_n & = b_1           \\
                a_{21}x_1+a_{22}x_2+\cdots+a_{2n}x_n & = b_2           \\
                                                     & \vdotswithin{=} \\
                a_{n1}x_1+a_{n2}x_2+\cdots+a_{nn}x_n & = b_n
            \end{aligned} \end{cases}
    \end{gather}

    令$D=\begin{vmatrix}
            a_{11} & a_{12} & \cdots & a_{1n} \\
            a_{21} & a_{22} & \cdots & a_{2n} \\
            \vdots & \vdots & \ddots & \vdots \\
            a_{n1} & a_{n2} & \cdots & a_{nn}
        \end{vmatrix}$,称为系数行列式.

    令$D_1=\begin{vmatrix}
            b_1    & a_{12} & \cdots & a_{1n} \\
            b_2    & a_{22} & \cdots & a_{2n} \\
            \vdots & \vdots & \ddots & \vdots \\
            b_n    & a_{n2} & \cdots & a_{nn}
        \end{vmatrix},\ldots,D_n=\begin{vmatrix}
            a_{11} & a_{12} & \cdots & b_1    \\
            a_{21} & a_{22} & \cdots & b_2    \\
            \vdots & \vdots & \ddots & \vdots \\
            a_{n1} & a_{n2} & \cdots & b_n
        \end{vmatrix}$.

    \begin{enumerate}
        \item 方程组 \ref{eq:13:线性方程组1} 只有零解$\iff D \neq 0$,有非零解(无穷多解)$\iff D=0$,即$r(A)<n$;

        \item 方程组 \ref{eq:13:线性方程组2} 有唯一解$\iff D \neq 0$,此时$x_i=\dfrac{D_i}{D}\enspace(i=1,2,\ldots,n)$,当$D=0$时,方程组 \ref{eq:13:线性方程组2} 要么无解,要么有无穷多解.
    \end{enumerate}
\end{theorem}

\begin{proof}
    我们不区分齐次与非齐次方程组进行证明,实际上齐次的结论只是下面证明的特例. 事实上,对于任意线性方程组$AX=b$,其中$b$可以是零向量,若$D=|A|\neq 0$,则$A$可逆,因此方程组有解
    \[X=A^{-1}b=\frac{1}{|A|}A^*b=\frac{1}{D}A^*b,\]
    即
    \[\begin{pmatrix}
            x_1 \\ x_2 \\ \vdots \\ x_n
        \end{pmatrix}= \dfrac{1}{D}\begin{pmatrix}
            A_{11} & A_{21} & \cdots & A_{n1} \\
            A_{12} & A_{22} & \cdots & A_{n2} \\
            \vdots & \vdots & \ddots & \vdots \\
            A_{1n} & A_{2n} & \cdots & A_{nn}
        \end{pmatrix}\begin{pmatrix}
            b_1 \\ b_2 \\ \vdots \\ b_n
        \end{pmatrix},\]
    于是$x_i=\dfrac{1}{D}(b_1A_{1i}+b_2A_{2i}+\cdots+b_nA_{ni})=\dfrac{D_i}{D}\enspace(i=1,2,\ldots,n)$(根据\autoref{def:13:递归式定义}).

    事实上此时解唯一,因为可逆矩阵$A$应当是可逆线性映射$\sigma$关于某组基的表示矩阵. 对于可逆映射而言,首先必须是单射,因此$\sigma(a)=b$只能有唯一解,因此$AX=b$只能有唯一解.

    反之,若方程组$AX=b$只有唯一解,说明方阵$A$对应的线性映射$\sigma$是单射,并且因为$A$是方阵,因此$\sigma$出发空间与到达空间维数相同,由\autoref{thm:6:双射等价条件} 可知$\sigma$是双射,因此$A$可逆,从而$|A|\neq 0$. 综上可以证明方程组$AX=b$有唯一解$\iff D=|A|\neq 0$.

    剩下的无解、无穷解的结论就很显然了. 若$AX=b$无解或有无穷解,反证法,若$D\neq 0$,则$A$可逆,从而$AX=b$有唯一解,矛盾,故$D=0$. 反之,若$D=0$,则$A$不可逆,反证法,若$AX=b$有唯一解,$A$可逆,矛盾,故$AX=b$无解或有无穷解(完全就是前面$AX=b$有唯一解$\iff D=|A|\neq 0$的推论).
\end{proof}

我们可以用 Cramer 法则求解线性方程组,但要注意只有方程个数与未知数个数相等时才能使用,并且需要系数行列式不为0.
\begin{example}
    求解方程组$\begin{cases}
            x_1+x_2+x_3=1          \\
            a_1x_1+a_2x_2+a_3x_3=0 \\
            a_1^2x_1+a_2^2x_2+a_3^2x_3=0
        \end{cases}$,其中$a_1,a_2,a_3$两两不等.
\end{example}

\begin{solution}
    $a_1,a_2,a_3$两两不等时,我们有
    \[D=\begin{vmatrix}
            1     & 1     & 1     \\
            a_1   & a_2   & a_3   \\
            a_1^2 & a_2^2 & a_3^2
        \end{vmatrix}=(a_2-a_1)(a_3-a_1)(a_3-a_2)\neq 0,\]
    根据Cramer法则,方程组有唯一解,且
    \[D_1=\begin{vmatrix}
            1 & 1     & 1     \\
            0 & a_2   & a_3   \\
            0 & a_2^2 & a_3^2
        \end{vmatrix}=(a_3-a_2)a_2a_3,\]
    \[D_2=\begin{vmatrix}
            1     & 1 & 1     \\
            a_1   & 0 & a_3   \\
            a_1^2 & 0 & a_3^2
        \end{vmatrix}=(a_1-a_3)a_1a_3,\]
    \[D_3=\begin{vmatrix}
            1     & 1     & 1 \\
            a_1   & a_2   & 0 \\
            a_1^2 & a_2^2 & 0
        \end{vmatrix}=(a_2-a_1)a_1a_2,\]
    因此
    \[x_1=\dfrac{D_1}{D}=\dfrac{a_2a_3}{(a_2-a_1)(a_3-a_1)},\]\[x_2=\dfrac{D_2}{D}=\dfrac{a_1a_3}{(a_1-a_2)(a_3-a_2)},\]\[x_3=\dfrac{D_3}{D}=\dfrac{a_1a_2}{(a_1-a_3)(a_2-a_3)}.\]
\end{solution}

事实上,Cramer法则在定理内容中已经给我们提供了关于线性方程组解的理论的重要结论——它实现了我们第一讲中提到的方程组解的情况的更一般化的讨论,即不需要化为简化阶梯矩阵就可以利用更为一般化的结论判断方程组解的情况,我们将在朝花夕拾中进行更完整的讨论.

\section{行列式的秩}

\subsection{行列式的秩}

首先我们需要给出矩阵的子式、主子式的定义,然后给出相关的顺序主子式的定义.
\begin{definition}
    矩阵$A=(a_{ij})_{n \times n}$的任意$k$行($i_1<i_2<\cdots<i_k$行)和任意$k$列($j_1<j_2<\cdots<j_k$列)的交点上的$k^2$个元素排成的行列式
    \[\begin{vmatrix}
            a_{i_1j_1} & a_{i_1j_2} & \cdots & a_{i_1j_k} \\
            a_{i_2j_1} & a_{i_2j_2} & \cdots & a_{i_2j_k} \\
            \vdots     & \vdots     & \ddots & \vdots     \\
            a_{i_kj_1} & a_{i_kj_2} & \cdots & a_{i_kj_k}
        \end{vmatrix}\]
    称为矩阵$A$的一个$k$阶子式,若子式等于0则称$k$阶零子式,否则称非零子式.

    当$A$为方阵且$i_t=j_t\enspace(t=1,2,\ldots,k)$(即选取相同行列)时,称为$A$的$k$阶\term{主子式}\index{zhuzishi@主子式 (principal minor)}. 若$i_t=j_t=t\enspace(t=1,2,\ldots,k)$,称为$A$的$k$阶\term{顺序主子式}\index{zhuzishi!shunxu@顺序主子式 (leading principal minor)}(取前$k$行$k$列的左上角主子式).
\end{definition}

\begin{example}
    写出矩阵$\begin{pmatrix}
            1 & 5 & -2 \\ 2 & 3 & 4 \\ -1 & -3 & 0
        \end{pmatrix}$的所有一阶、二阶子式、主子式和顺序主子式.
\end{example}

\begin{solution}
    \begin{enumerate}
        \item 一阶子式:根据子式定义,任意1行1列交点组成的1个元素就是一阶子式,即所有的元素本身都是一阶子式;

        \item 二阶子式:根据子式定义,任意2行2列交点组成的4个元素排成的行列式就是二阶子式,即
              \[\begin{vmatrix}
                      1 & 5 \\
                      2 & 3
                  \end{vmatrix},\begin{vmatrix}
                      1 & -2 \\
                      2 & 4
                  \end{vmatrix},\begin{vmatrix}
                      5 & -2 \\
                      3 & 4
                  \end{vmatrix},\begin{vmatrix}
                      1  & 5  \\
                      -1 & -3
                  \end{vmatrix},\begin{vmatrix}
                      1  & -2 \\
                      -1 & 0
                  \end{vmatrix},\]\[\begin{vmatrix}
                      5  & -2 \\
                      -3 & 0
                  \end{vmatrix},\begin{vmatrix}
                      2  & 3  \\
                      -1 & -3
                  \end{vmatrix},\begin{vmatrix}
                      2  & 4 \\
                      -1 & 0
                  \end{vmatrix},\begin{vmatrix}
                      3  & 4 \\
                      -3 & 0
                  \end{vmatrix};\]

        \item 主子式:根据主子式定义,要求取的行列号相同,故一阶主子式就是1行1列、2行2列、3行3列的元素,二阶主子式就是选取1、2/1、3/2、3行与列构成的子式,即
              \[\begin{vmatrix}
                      1 & 5 \\ 2 & 3
                  \end{vmatrix},\begin{vmatrix}
                      1 & -2 \\ -1 & 0
                  \end{vmatrix},\begin{vmatrix}
                      3 & 4 \\ -3 & 0
                  \end{vmatrix};\]
              三阶主子式就是矩阵本身对应的行列式,不再赘述.

        \item 顺序主子式根据定义,一阶就是第一行第一列的元素,二阶就是前两行前两节元素构成的子式,三阶就是本身的行列式.
    \end{enumerate}
\end{solution}

接下来我们给出行列式的秩的定义.
\begin{definition}
    矩阵$A$的非零子式的最高阶数$r$称为$A$的行列式秩.
\end{definition}
即矩阵$A$的行列式秩为$r$的含义为$A$至少有一个$r$阶子式不为0,但所有$r+1$阶子式均为0. 事实上,我们可以通过如下定理统一矩阵的秩和行列式秩:
\begin{theorem}\label{thm:13:行列式秩等于行列式秩}
    矩阵$A$的秩$r(A)=r \iff A$的行列式的秩为$r$.
\end{theorem}
我们可以得到上一个专题中矩阵的秩的等价定义. 这一定理的证明见教材183页,事实上是很简单的. 我们可以这么理解,最高非零子式的阶数实际上就是矩阵行、列向量极大线性无关组的长度(更多行、列向量就会使得子式等于0,此时必不满秩),那么这一定理就很显然了.

\begin{definition}
    矩阵$A$的非零子式的最高阶数$r$称为矩阵$A$的秩,记为$r(A)$.
\end{definition}

需要注意的是,前面定义的子式、行列式秩等都是对矩阵定义的,原因是行列式虽名为``式''但实际上只是一个数,只有矩阵有形可以定义上述概念.
\begin{example}
    利用定义求矩阵$\begin{pmatrix}
            1 & 1 & -1 & 3 \\ 1 & 2 & 1 & 1 \\ 2 & 3 & 0 & 4
        \end{pmatrix}$的行列式秩.
\end{example}

\begin{solution}
    记该矩阵为$A$,由于$A$为3行4列矩阵,因此$r(A)\leqslant 3$. 又我们可以发现其三阶子式
    \[\begin{vmatrix}
            1 & -1 & 3 \\ 1 & 1 & 1 \\ 2 & 0 & 4
        \end{vmatrix}=-4\neq 0,\]
    故$r(A)\geqslant 3$,因此$r(A)=3$.
\end{solution}

\subsection{关于秩的总结}

本学期我们一共学习了四个秩的概念:向量组的秩,线性映射的秩,矩阵的秩和行列式的秩. 事实上,我们在很多地方都讨论了它们的统一性:
\begin{enumerate}
    \item 在矩阵的秩的定义以及三秩的统一中体现了向量组的秩(行秩、列秩的定义基于向量组的秩)和线性映射的秩(矩阵的秩的定义基于线性映射的秩)与矩阵的秩的统一;

    \item 在\autoref{thm:13:行列式秩等于行列式秩} 中统一了矩阵的秩和行列式的秩.
\end{enumerate}
虽然线性映射的秩、矩阵的秩、行列式的秩的定义各不相同,但本质都在于向量组的秩(回顾线性映射的秩的定义,矩阵行秩、列秩的定义,乃至\autoref*{thm:13:行列式秩等于行列式秩} 的证明). 这给我们的启示是上述提到的概念都可以互相转化考虑. 例如考虑可逆时,我们可以考虑行、列向量是否线性无关/矩阵对应的线性映射是否可逆/行列式是否为0等. 虽然说起来很简单,但是实际做题的时候很多同学还是容易思维局限,因此我们需要将这些概念的统一性放在重要的位置.
\begin{example}
    求多项式$f(x)=\begin{vmatrix}
            1 & a_1   & a_2     & a_3     \\
            1 & a_1+x & a_2     & a_3     \\
            1 & a_1   & a_2+x+1 & a_3     \\
            1 & a_1   & a_2     & a_3+x+2
        \end{vmatrix}$的所有零点.
\end{example}

\begin{solution}
    事实上,本题可以直接首先展开求出四阶行列式的值然后解方程$f(x)=0$即可,但我们这里使用更为本质的方法. $f(x)=0$实际上就是行列式等于零,即此时$x$使得行列式中两列(或两行)线性相关了. 事实上,我们很容易发现$x=0$时,第一列与第二列成比例,故此时$f(x)=0$成立. 同理,$x=-1$和$x=-2$也是$f(x)=0$的解.

    事实上,我们知道这一行列式展开后是一个次数最高为3的多项式,因此$0,-1,-2$就是$f(x)$的所有零点.
\end{solution}

\vspace{2ex}
\centerline{\heiti \Large 内容总结}

在这一讲中我们引入了一个重要的工具——行列式. 我们不同于一般教材的逆序数定义(我们将会在史海拾遗中从历史角度介绍这一定义),首先给出了公理化的定义,并且发现行列式事实上就是在描述$n$维空间中物体的体积. 接下来我们也介绍了递归式定义(即按一行一列展开),并讨论了行列式的一些性质和基本运算(进阶问题我们将在下一讲讨论),介绍了常用的范德蒙行列式. 接下来我们也介绍了伴随矩阵及其大量性质,在性质的证明中希望读者体会这类证明的一般想法. 我们也介绍了Cramer法则,它是最开始研究线性方程组理论的一个核心结果,因此在讨论线性方程组一般理论的朝花夕拾一讲中我们还会再遇见它的身影. 最后我们讨论了行列式的秩,这也是我们最后一个``秩''的定义,我们讨论了向量组、线性映射、矩阵、行列式的秩的统一性,这也是我们这一学期学习的秩的概念的一个总结,也希望读者能在练习中更深刻体会它们的关联.

事实上,我们很难说服读者行列式以什么样的方式引入是最为合适的,或许在史海拾遗的历史讲述中我们可能才能窥见行列式诞生的奥秘,那是最为自然的描述,但需要过多的准备以至于可能令人厌烦. 但至少公理化定义是非常简单的,并且有一定的几何背景,由此也直接可以得出行列式大量的优良性质,例如矩阵可逆等价于行列式等于零——这一性质在将来关于线性方程组、特征多项式等的讨论中是核心的.

\vspace{2ex}
\centerline{\heiti \Large 习题}

\vspace{2ex}
{\kaishu 新的数学方法和概念,常常比解决数学问题本身更重要。}
\begin{flushright}
    \kaishu
    ——华罗庚
\end{flushright}

\centerline{\heiti A组}
\begin{enumerate}
    \item 递归式定义推导公理化定义.

    \item 设$\alpha_1,\alpha_2,\alpha_3$为三维列向量,令$A=(\alpha_1,\alpha_2,\alpha_3)$,且$|A|=2$,求$|\alpha_1+\alpha_2+\alpha_3,\alpha_1+3\alpha_2+9\alpha_3,\alpha_1+4\alpha_2+16\alpha_3|$.

    \item 求证以下命题:
          \begin{enumerate}
              \item 奇数阶反对称矩阵不可逆;

              \item 若$A$是$n$阶可逆对称矩阵,$B$是$n$阶反对称矩阵,则当$n$为奇数时,齐次线性方程组$(AB)X=O$有非零解.
          \end{enumerate}

    \item 设$A$、$B$分别为$m$、$n$阶可逆矩阵,且$|A|=a$,$|B|=b$,求$\begin{pmatrix}
                  A & O \\ O & B
              \end{pmatrix}^*$和$\begin{pmatrix}
                  O & A \\ B & O
              \end{pmatrix}^*$.

    \item 证明:
          \begin{enumerate}
              \item 若$A$为幂等矩阵,则$A^*$也为幂等矩阵;$A$为幂零矩阵,则$A^*$也为幂零矩阵;

              \item 若$A$为对称矩阵,则$A^*$也为对称矩阵;$A$为反对称矩阵,则$A^*$为偶数阶时也为反对称矩阵,奇数阶时为对称矩阵.
          \end{enumerate}

    \item 证明:上(下)三角矩阵的伴随矩阵是上(下)三角矩阵(对角矩阵为特例).

    \item 设$A$为$n$阶方阵,证明:若$|A|=0$,则$A$中任意两行(列)对应元素的代数余子式成比例.

    \item 设向量$\alpha_1,\alpha_2,\alpha_3$线性无关,讨论向量$\alpha_1-\alpha_2-2\alpha_3,\ 2\alpha_1+\alpha_2-\alpha_3,\ 3\alpha_1+\alpha_2+2\alpha_3$的线性相关性.

    \item 设$W=\spa(\alpha_1,\alpha_2)$是$\mathbf{R}^4$的一个子空间,其中$\alpha_1=(1,2,1,-1)^\mathrm{T}$,$\alpha_2=(1,4,-1,-1)^\mathrm{T}$,试将$\alpha_1,\alpha_2$扩充为$\mathbf{R}^4$的基.
\end{enumerate}

\centerline{\heiti B组}
\begin{enumerate}
    \item 设$D=\begin{vmatrix}
                  3 & 0 & 4 & 1 \\ 2 & 3 & 1 & 4 \\ 0 & -7 & 8 & 3 \\ 5 & 3 & -2 & 2
              \end{vmatrix}$,求
          \begin{enumerate}
              \item $A_{21}+A_{22}+A_{23}+A_{24}$;

              \item $A_{31}+A_{33}$;

              \item $M_{41}+M_{42}+M_{43}+M_{44}$.
          \end{enumerate}

    \item 求参数 $a,b$  的值,使得$\begin{vmatrix}1 & 1 & 1 \\ x & y & z \\u & v & w\end{vmatrix}=1,
              \begin{vmatrix}1 & 2 & -5 \\ x & y & z \\u & v & w\end{vmatrix}=2,
              \begin{vmatrix}2 & 3 & b \\ x & y & z \\u & v & w\end{vmatrix}=a$都成立,并求$\begin{vmatrix}x & y & z \\ 1 & -1 & 5 \\u & v & w\end{vmatrix}$.

    \item 设$A,B$为三阶矩阵,且$|A|=3,|B|=2$,且$|A^{-1}+B|=2$,求$|A+B^{-1}|$.

    \item 设$A$为$n$阶正交矩阵,即$AA^\mathrm{T}=A^\mathrm{T}A=E$,且$|A|<0$,证明:$|E+A|=0$.

    \item 已知齐次线性方程组
          \[\begin{cases} \begin{aligned}
                      a_{11}x_1+a_{12}x_2+\cdots+a_{1n}x_n          & = 0             \\
                      a_{21}x_1+a_{22}x_2+\cdots+a_{2n}x_n          & = 0             \\
                                                                    & \vdotswithin{=} \\
                      a_{n-1,1}x_1+a_{n-1,2}x_2+\cdots+a_{n-1,n}x_n & = 0
                  \end{aligned} \end{cases}\]
          设$M_j\enspace(j=1,2,\ldots,n)$表示$A=(a_{ij})_{n-1 \times n}$划掉第$j$列所得的$n-1$阶子式,证明:
          \begin{enumerate}
              \item $(M_1,-M_2,\ldots,(-1)^{n-1}M_n)$是方程组的一个解;

              \item 若$r(A)=n-1$,则方程组的解全是$(M_1,-M_2,\ldots,(-1)^{n-1}M_n)$的倍数.
          \end{enumerate}

    \item 设$A,B$均为$n$阶矩阵,且$|A|=2,|B|=1$,求$|2A^*B^*-A^{-1}B^{-1}|$.

    \item 若$n$阶非零矩阵$A$满足$A^\mathrm{T}=A^*$,证明:
          \begin{enumerate}
              \item $|A|>0$;

              \item $|A|=1$(补充:若$A$第一行元素相等,求第一行元素的值);

              \item $A$为正交矩阵,即$AA^\mathrm{T}=A^\mathrm{T}A$;

              \item $n>2$且为奇数时,$|E-A|=0$.
          \end{enumerate}

    \item 已知$A$是一个秩为$n-1$的$n\enspace(n \geqslant 2)$阶方阵,且已知某个元素$a_{ij}$的代数余子式$A_{ij} \neq 0$,求方程组$AX=0$的基础解系.

    \item 设$D=|a_{ij}|_{n \times n}$,$A_{ij}$是$a_{ij}$的代数余子式. 求证:
          \[\begin{vmatrix}
                  A_{11}    & A_{12}    & \cdots & A_{1,n-1}   \\
                  A_{21}    & A_{22}    & \cdots & A_{2,n-1}   \\
                  \vdots    & \vdots    & \ddots & \vdots      \\
                  A_{n-1,1} & A_{n-1,2} & \cdots & A_{n-1,n-1}
              \end{vmatrix}=a_{nn}D^{n-2}.\]

    \item 设$a_1,a_2,\ldots,a_n$为互不相等的实数,$b_1,b_2,\ldots,b_n$为任意给定的实数. 证明:存在唯一的$n-1$次多项式,满足$f(a_i)=b_i,\enspace i=1,2,\ldots,n$.

    \item 证明:$n$维向量组$\alpha_1,\alpha_2,\ldots,\alpha_n$线性无关的充要条件是
          \[\begin{vmatrix}
                  \alpha_1^\mathrm{T}\alpha_1 & \alpha_1^\mathrm{T}\alpha_2 & \cdots & \alpha_1^\mathrm{T}\alpha_n \\
                  \alpha_2^\mathrm{T}\alpha_1 & \alpha_2^\mathrm{T}\alpha_2 & \cdots & \alpha_2^\mathrm{T}\alpha_n \\
                  \vdots                      & \vdots                      & \ddots & \vdots                      \\
                  \alpha_n^\mathrm{T}\alpha_1 & \alpha_n^\mathrm{T}\alpha_2 & \cdots & \alpha_n^\mathrm{T}\alpha_n
              \end{vmatrix}\neq 0.\]

    \item 设$a_1,\ldots,a_n$为$n$个$n$维向量,证明:向量组$a_1,\ldots,a_n$线性无关的充要条件是任一个$n$维向量都可以由其线性表示(不使用线性空间维数的方式完成).

    \item 设$s \times n\enspace(s\leqslant n)$矩阵为
          \[\begin{pmatrix}
                  1      & a      & a^2    & \cdots & a^{n-1}    \\
                  1      & a^2    & a^4    & \cdots & a^{2(n-1)} \\
                  \vdots & \vdots & \vdots & \ddots & \vdots     \\
                  1      & a^s    & a^{2s} & \cdots & a^{s(n-1)}
              \end{pmatrix}\]
          且$a^r\neq 1\enspace(0<r<n)$,求$A$的秩和它的列向量组的一个极大线性无关组.

    \item 设$A,B,C,D \in \mathbf{F}^{n \times n}$,定义变换 $ T : \mathbf{F}^{n \times n} \to \mathbf{F}^{n \times n}$ 为
          \[ T(X) = AXB+CX+XD \]
          证明:
          \begin{enumerate}
              \item $T$为$\mathbf{F}^{n \times n}$上的线性变换;

              \item 当$C=D=0$时,$T$可逆的充要条件是$|AB| \neq 0$.
          \end{enumerate}

    \item 设$A$为$n$阶矩阵,且$r(A) < n$,又$A_{11} \neq 0$,证明:存在常数$k$,使得$(A^*)^2=kA^*$.

    \item 设$V$是一个$n$维实线性空间,证明:存在$V$中的一个由可列无穷多个向量组成的向量组$\{\alpha_i \mid i\in\mathbf{Z}_+\}$,使得其中任意$n$个向量组成的向量组都是$V$的一组基.
\end{enumerate}

\centerline{\heiti C组}
\begin{enumerate}
    \item 设$A,B,C,D$均为$n$阶方阵,$\lvert A \rvert \neq 0$且$AC=CA$. 证明:
          \[\begin{vmatrix}
                  A & B \\ C & D
              \end{vmatrix} = |AD-CB|.\]

    \item 设$A$为$n$阶可逆矩阵,$\alpha,\beta$为$n$维列向量,证明:
          \[|A+\alpha\beta^{\mathrm{T}}|=|A|(1+\beta^\mathrm{T}A^{-1}\alpha).\]

    \item 设$A,B$均为$n$阶方阵,证明:
          \[\begin{vmatrix}
                  A & B \\ B & A
              \end{vmatrix} = |A+B||A-B|.\]

    \item 设$A,B,C,D$均为$n$阶方阵,且$r\begin{pmatrix}
                  A & B \\ C & D
              \end{pmatrix}=n$,证明:
          \[\begin{vmatrix}
                  |A| & |B| \\ |C| & |D|
              \end{vmatrix} = 0.\]

    \item (对角占优)设$A=(a_{ij})_{n \times n}$是一个$n$级矩阵,证明:
          \begin{enumerate}
              \item 若$A$为复矩阵,且$|a_{ii}|>\displaystyle\sum_{j \neq i}|a_{ij}|$,那么$|A|\neq 0$;

              \item 若$A$为实矩阵,且$a_{ii}>\displaystyle\sum_{j \neq i}|a_{ij}|$,那么$|A|>0$;

              \item (推论)存在充分大的实数$M$,使得$t>M$时,$tE+A$可逆.
          \end{enumerate}

    \item 求$\begin{pmatrix}
                  A & C \\ O & B
              \end{pmatrix}^*$,并求当$A$可逆时的$\begin{pmatrix}
                  A & B \\ C & D
              \end{pmatrix}^*$.

    \item 下面三个小问探讨伴随矩阵的反问题,即对任意给定的$n$阶方阵$B$,是否存在$n$阶方阵$A$使得$A^*=B$.
          \begin{enumerate}
              \item 证明:若$n=2$,则存在唯一的2阶方阵$A$使得$A^*=B$;

              \item 证明:若$n > 2$,则存在$n$阶方阵$A$使得$A^*=B$的充要条件为$r(B) \in \{0,1,n\}$,并且
                    \begin{enumerate}
                        \item $r(B)=n$时,$A=\sqrt[n-1]{|B|}B^{-1}$;

                        \item $r(B)=1$时,$A=Q^{-1}\begin{pmatrix}
                                      0 & O \\ O & X_{n-1}
                                  \end{pmatrix}P^{-1}$,且$|X_{n-1}|=|PQ|$,$B=P\begin{pmatrix}
                                      1 & O \\ O & O
                                  \end{pmatrix}Q$;
                    \end{enumerate}

              \item 设$A=\begin{pmatrix}
                            1 & 1 & 1 \\ 1 & 1 & 1 \\ 1 & 1 & 1
                        \end{pmatrix}$,求矩阵$B$使得$B^*=A$.
          \end{enumerate}
\end{enumerate}

\input{./专题/14 行列式计算进阶.tex}
\begingroup
\SetLUChapterNumberingStyle{3}
\def\theHchapter{\arabic{chapter}ε}

\chapter{矩阵空间}

\ResetChapterNumberingStyle{14}
\endgroup

\section*{15 朝花夕拾}
\addcontentsline{toc}{section}{15 朝花夕拾}

\vspace{2ex}

\centerline{\heiti A组}
\begin{enumerate}
    \item
        \begin{enumerate}
            \item 参考教材定理$6.1$.
            \item 提示. 考虑$A=\begin{pmatrix}
                a_{11} & a_{12} & \cdots & a_{1n} \\
                a_{21} & a_{22} & \cdots & a_{2n} \\
                \vdots & \vdots & \ddots & \vdots \\
                a_{m1} & a_{m2} & \cdots & a_{mn}
            \end{pmatrix}.$
            令$\alpha_1=\begin{pmatrix}
                a_{11} \\
                a_{21} \\
                \vdots \\
                a_{m1}
            \end{pmatrix},\cdots,\alpha_n=\begin{pmatrix}
                a_{1n} \\
                a_{2n} \\
                \vdots \\
                a_{mn}
            \end{pmatrix}.$
            $AX=0\implies x_1\alpha_1+\cdots+x_n\alpha_n=0$\\
            只有零解表明$\alpha_1,\cdots,alpha_n$线性无关,故$A$列满秩,$r(A)=n$.\\
            有非零解(无穷解)则相反.
            \item 提示同上.
            \item 参考教材定理$6.2$.
            \item $A(k_1X_1+\cdots+k_sX_s)=k_1AX_1+\cdots+k_sAX_s=0+\cdots+0=0.$
            \item $A(k_1X_1+\cdots+k_sX_s+\eta_0)=k_1AX_1+\cdots+k_sAX_s+A\eta_0=b$
            \item $A(\eta_1-\eta_2)=A\eta_1-A\eta_2=b-b=0$
            \item
                令$\bar{X}=k_1X_1+\cdots+k_sX_s$,有
                \begin{align*}
                    A\bar{X}=b&\iff A(k_1X_1+\cdots+k_sX_s)=b\\
                    &\iff k_1AX_1+\cdots+k_sAX_s=b\\
                    &\iff (k_1+\cdots+k_s)b=b\\
                    &\iff k_1+\cdots+k_2=1\quad (b\neq 0).
                \end{align*}
            \item 类似上题.
            \item
                错误.\\
                必要性正确. 令$A$为$m\times n$的矩阵,则$AX=b$有唯一解表明$r(A)=n$. 而$r(A)=n\implies AX=0$只有零解成立.\\
                充分性错误. 注意到$AX=0$只有零解$\implies r(A)=n$. 但$r(A)=m$不代表$AX=b$有唯一解,因为还有无解的可能.\\
                (齐次线性方程组一定有解,但非齐次线性方程组不一定有解,一定要注意这个差别)\\
                简单的反例如$A=\begin{pmatrix}
                    1 & 0 \\
                    0 & 1 \\
                    0 & 0
                \end{pmatrix},b=\begin{pmatrix}
                    0 \\
                    0 \\
                    1
                \end{pmatrix}$
            \item
                正确. 令$B=(\beta_1,\cdots,\beta_s)$. $AB=0\implies A(\beta_1,\cdots,\beta_s)=0$,即$(A\beta_1, \cdots, A\beta_s)=0$,故$B$的列向量为方程组$AX=0$的解.
                (注:$A(\beta_1,\cdots,\beta_s)=(A\beta_1, \cdots, A\beta_s)$利用分块矩阵性质即可)
            \item 利用上一题的结论易知正确.
            \item 设$A$的解空间为$N(A)$,$B$的解空间为$N(B)$,则由题意有$N(A)\subseteq N(B)$,故$\dim{N(A)}\leqslant \dim{N(B)}$. 而$r(A)+\dim{N(A)}=r(B)+\dim{N(B)}$(参考第1题),故$r(A)\geqslant r(B)$正确.
            \item
                错误.\\
                必要性正确. $AX=0$与$BX=0$为同解方程组可得$N(A)=N(B)\implies\dim{N(A)}=\dim{N(B)}$. 同上一题,可得$r(A)=r(B)$.\\
                充分性错误. 反例有$A=\begin{pmatrix}
                    1 & 0 & 0 \\
                    0 & 1 & 0
                \end{pmatrix}, B=\begin{pmatrix}
                    1 & 0 & 0 \\
                    0 & 0 & 1
                \end{pmatrix}$.\\
                实际上,只需考虑解空间均为$\mathbf{R}^n$的子空间,$\mathbf{R}^n=\spa(\alpha_1,\cdots,\alpha_n)$. 令$N(A)=\spa(\alpha_i), N(B)=\spa(\alpha+j),i\neq j$即为反例.
        \end{enumerate}
    \item
        $r(A^*)=
        \begin{cases}
            1, & r(A)=3 \\
            0, & r(A)<3
        \end{cases}$,而已知$A_{21}\neq 0$,故$r(A^*)=1$,故$r(A)=3$,故$AX = 0$解空间维数为$4-r(A)=1$. 而$|A|=0$,由行列式展开可知
        \begin{equation*}
            a_{i1}A_{21}+a_{i2}A_{22}+a_{i3}A_{23}+a_{i4}A_{24}=0 \ (i=1,2,3,4)
        \end{equation*}
        故解为$k(A_{21},A_{22},A_{23},A_{24})^\mathrm{T}\ (k\in \mathbf{R})$.
    \item
        由题意得$r(A)=3<4$且$\alpha_1-4\alpha_3=0$,或$\alpha_1=4\alpha_3$.
        由$r(A)=3$得$r(A^*)=1$,$A^*X=0$的基础解系含$3$个线性无关的解向量.
        由$A*A=|A|E=O$得$\alpha_1,\alpha_2,\alpha_3,\alpha_4$为$A^*X=0$的解,从而$\alpha_1,\alpha_2,\alpha_4$或$\alpha_2,\alpha_3,\alpha_4$为$A^*X=0$的基础解系.
    \item $\forall \alpha \in \mathbf{R}^n, \alpha^\mathrm{T}A\beta =0\implies A\beta =0.$
    \item
        $r(A)=3\implies \dim{N(A)}=4-r(A)=1$.\\
        $\beta = x_1\alpha_1+x_2\alpha_2+x_3\alpha_3+x_4\alpha_4\implies$特解为$(1,-1,0,3)^\mathrm{T}$\\
        $\alpha_1-2\alpha_2+\alpha_3=0\implies$导出组基础解系为$(1,-2,1,0)^\mathrm{T}$.\\
        故通解为$k(1,-2,1,0)^\mathrm{T}+(1,-1,0,3)^\mathrm{T} \ (k\in\mathbf{R})$
    \item
        设方程组$Ax=b$,由$r(A)=3$知其导出组$Ax=0$的基础解系只含一个解向量.\\
        而$A\eta_1=b,A(\eta_2+\eta_3)=2b$,故$2\eta_1-(\eta_2+\eta_3)=(3,4,5,6)^\mathrm{T}$为$Ax=0$的基础解系,从而所求通解为
        \begin{equation*}
            x = \eta_1+c(3,4,5,6)^\mathrm{T} = (2,3,4,5)^\mathrm{T} + c(3,4,5,6)^\mathrm{T},
        \end{equation*}
        其中$c$为任意常数.
    \item
        \begin{enumerate}
            \item 取$\beta_1-\beta_2,\beta_1-\beta_3$(取法不唯一),有$A(\beta_1-\beta_2)=A(\beta_1-\beta_3)=0$且$\beta_1-\beta_2,\beta_1-\beta_3$线性无关,否则$\beta_1,\beta_2,\beta_3$线性相关.
            \item 取$2\beta_1+k_1(\beta_1+\beta_2)+k_2(\beta_2+\beta_3)$可行(取法不唯一).
        \end{enumerate}
    \item
        对于任意的$b_{s\times 1}$,$AX=b$都有解,说明$A$的列向量可以线性表出出任意$s$维向量$b$,从而$r(A)\geqslant s$,而$r(A)\leqslant s$,所以$r(A)=s$.\\
        反之,如果$r(A)=s$,则$A$有$s$个线性无关的列向量,它们就是$s$维空间$V$的一组基,对于$V$中任意向量$b$,都可以由$A$的列向量线性表出,从而$AX=b$有解.\\
        当然,也可以直接用秩不等式:$s=r(A)\leqslant r(A,b)\leqslant s$得到$r(A,b)=s$,所以$AX=b$有解.
    \item
        $AB=0\implies r(A)+r(B)\leqslant n,r(B)=n\implies r(A)\leqslant 0\implies A=O.$
    \item
        必要性. 有条件可知$V_1\cap V_2=\{0\}$,$\begin{pmatrix}A\\B\end{pmatrix}\in\mathbf{F}^{n\times n}$,而$\begin{pmatrix}A\\B\end{pmatrix}x=0$只有零解,故$\begin{pmatrix}A\\B\end{pmatrix}$可逆,从而$r(A)=m,r(B)=n-m$,于是
        \begin{align*}
            \dim V_1&=n-r(A)=n-m, \\
            \dim V_2&=n-r(B)=n-(n-m)=m.
        \end{align*}
        又$V_1+V_2$是$\mathbf{F}^n$的子空间,且$\dim(V_1+V_2)=\dim V_1+\dim V_2-\dim(V_1\cap V_2)=\dim\mathbf{F}^n$,故$\mathbf{F}^n=V_1\oplus V_2$.\\
        充分性. 若$\begin{pmatrix}A\\B\end{pmatrix}x=0$有非零解$x_1$,则$x_1\in V_1\cap\V_2.$这与$\mathbf{F}^n=V_1\oplus V_2$矛盾.
    \item
        \begin{enumerate}
            \item
                由条件知,方程组系数矩阵的为$2$,系数矩阵
                \begin{equation*}
                    A=\begin{pmatrix}
                        0 & 1 & a & b \\
                        -1 & 0 & c & d \\
                        a & c & 0 & -e \\
                        b & d & e & 0
                    \end{pmatrix}\rightarrow
                    \begin{pmatrix}
                        0 & 1 & a & b \\
                        -1 & 0 & c & d \\
                        0 & 0 & 0 & ad-e-bc \\
                        0 & 0 & -(ad-e-bc) & 0
                    \end{pmatrix},
                \end{equation*}
                故$ad-e-bc=0$.
            \item 易求得基础解系为$(c,-a,1,0)^\mathrm{T},(d,-b,0,1)^\mathrm{T}$.
        \end{enumerate}
\end{enumerate}

\centerline{\heiti B组}
\begin{enumerate}
    \item 见教材P210第10题.
    \item
        由$r(A)=m$可知,$A$的$m$个行向量($A^\mathrm{T}$的$m$个列向量)线性无关,它们是方程组$CX=0$的一个基础解系. 由$CA^\mathrm{T}=O$和$n-r(C)=m$,得$r(C)=n-m$. 因此,由$CA^\mathrm{T}B^\mathrm{T}=C(BA^\mathrm{T})=O$,可知$(BA)^\mathrm{T}$的$m$个行向量线性无关,它们也是$CX=0$的一个基础解系.
    \item
        必要性:设$AX=b$有无穷多解,则$r(A)<n$,从而$|A|=0$,于是$A^*b=A^*AX=|A|X=0.$\\
        充分性:设$A^*b=0$,即方程组$A^*b=0$有非零解,则$r(A^*)<n$,又$A_{11}\neq 0$,所以$r(A)=n-1$.\\
        令$A=(\alpha_1,\alpha_2,\cdots,\alpha_n)$,因为$A^*A=|A|E=O$,所以$\alpha_1,\alpha_2,\cdots,\alpha_n$为方程组$A^*X=0$的解,又因为$A_{11}\neq 0$,所以$\alpha_1,\alpha_2,\cdots,\alpha_n$线性无关.\\
        由$r(A^*)=1$,得方程组$A^*X=0$的基础解系含有$n-1$个线性无关的解向量,所以$\alpha_2,\alpha_3,\cdots,\alpha_n$为方程组$A^*X=0$的一个基础解系.\\
        因为$A^*b=0$,所以$b$为方程组$A^*X=0$的一个解,从而$b$可由$\alpha_2,\alpha_3,\cdots,\alpha_n$线性表示,$b$也可由$\alpha_1,\alpha_2,\cdots,\alpha_n$线性表示,于是$r(A) = r\begin{pmatrix}A & b\end{pmatrix}=n-1<n$,故方程组$AX=b$有无穷多解.
    \item
        由各列的元素之和等于$0$,得$|A|=0$.
        利用教材第$6$章习题$7$的结论:\\
        如果$r(A)<n-1$,则$r(A^*)=0$,$A^*=O$,所以,$A_{ij}=0\ (i,j=1,2,\cdots,n)$;\\
        如果$r(A)=n-1$,则$r(A^*)=1$,且$AA^*=O$,于是$A^*$的每一列$\left[A_{i1},A_{i2},\cdots A_{in}\right]^\mathrm{T} \ (i=1,2,\cdots,n)$都是$AX=0$的解.\\
        由于$AX=0$的解空间的维数为\begin{equation*} n-r(A)=n-(n-1)=1 \end{equation*}
        所以,$AX=0$的任意两个解成比例. 又元素全部为$1$的$n$元向量$e=(1,1,\cdots,1)^\mathrm{T}$满足方程组$AX=0$,因此$A^*$任意一列都与$e$成比例,即
        \begin{equation*}
            A_{i1}=A_{i2}=\cdots=A_{in} \ (i=1,2,\cdots,n).
        \end{equation*}
        所以,$A^*$的每一列元素(即$|A|$的每一行元素的代数余子式)都相等.\\
        同理,$(A^\mathrm{T})^*$的每一列元素(即$|A^\mathrm{T}|$的每一行元素,也是$|A|$的每一列元素的代数余子式)都相等,即
        \begin{equation*}
            A_{1j}=A_{2j}=\cdots=A_{nj} \ (j=1,2,\cdots,n).
        \end{equation*}
    \item 见2019-2020学年线性代数I(H)期末第五题.
    \item 令原方程组$AX=0$,解系组成矩阵$B$. 则新方程组为$B^\mathrm{T}X=0$. 而$B^\mathrm{T}A^\mathrm{T}=(AB)^\mathrm{T}=0$,推测$A^\mathrm{T}$的列向量为基础解系. 而由维数公式,
    \begin{equation*}
        r(A)=\dim V-\dim\ker(A) = 2n-n=n.
    \end{equation*}
    故$A^\mathrm{T}$列空间维数为$n$,即$r_c(A^\mathrm{T})=n$. 而由题意$r(B)=n$(否则不是基础解系),故
    \begin{equation*}
        \dim{\ker(B)}=\dim{V^{\prime}}-r(B)=2n-n=n=r_c(A^\mathrm{T}).
    \end{equation*}
    猜想成立.
    \item
        \begin{enumerate}
            \item 略.
            \item
                \begin{align*}
                    \dim(V_1+V_2) &=\dim V_1+\dim V_2-\dim(V_1\cap V_2)\\
                                  &=n(n-r)+n(n-s)-n(n-k).
                \end{align*}
        \end{enumerate}
    \item
        \begin{enumerate}
            \item 容易计算$|aE=bA|=(a+nb)a^{n-1}$.
            \item
                由$1<r(aE+bA)<n$知$|aE+bA|=0$. 故$a\neq 0$,且$a+nb=0$,此时$aE+bA$左上角的$n-1$阶子式
                \begin{equation*}
                    \begin{vmatrix}
                        a+b & b & \cdots & b \\
                        b & a+b & \cdots & b \\
                        \vdots & \vdots & \ddots & \vdots \\
                        b & b & \cdots & a+b
                    \end{vmatrix}=[a+(n-1)b]a^{n-2}=\frac{a^{n-1}}{n}\neq 0,
                \end{equation*}
                故$\dim{W}=n-r(aE+bA)=n-(n-1)=1$.
        \end{enumerate}
    \item
        设系数矩阵为$A$,第二个矩阵为$B$. 由于$(A,b)$为$B$子矩阵,故$r(A)\leqslant r(A,b)\leqslant r(B)$,而$r(A)=r(B)$,故$r(A)=r(A,b)$成立.
    \item 见教材P213第6题.
    \item 假. 取$A=\begin{pmatrix}
        1-i & 1+i \\
        1+i & -1+i
    \end{pmatrix}$,有$A^\mathrm{T}A=0$.
    \item 见教材P214第8题.
    \item
        由$XA=0$与$XAA^{\mathrm{T}}=0$同解. 又由条件知$(C-B)AA^\mathrm{T}=0$,故$(C-B)A=0$. 即$CA=BA$.
    \item 由$r(A)+r(E-A)=n \iff A^2=A$, 易证.
    \item
        \textbf{方法一.}用分块矩阵的方法,我们知道
        \begin{equation*}
            \begin{pmatrix}
                A & O \\
                O & B
            \end{pmatrix}
            \rightarrow
            \begin{pmatrix}
                A & O \\
                A & B
            \end{pmatrix}
            \rightarrow
            \begin{pmatrix}
                A & A \\
                A & A+B
            \end{pmatrix}.
        \end{equation*}
        结合$AB=BA$,我们知道
        \begin{equation*}
            \begin{pmatrix}
                A & A \\
                A & A+B
            \end{pmatrix}
            \underbrace{
                \begin{pmatrix}
                    A+B & O \\
                    -A & E
                \end{pmatrix}
            }_{\text{非广义初等变换,难以想到}}
            =
            \begin{pmatrix}
                AB & A \\
                O & A+B
            \end{pmatrix}.
        \end{equation*}
        于是
        \begin{equation*}
            r(A)+r(B)=r
            \begin{pmatrix}
                A & O \\
                O & B
            \end{pmatrix}=
            \begin{pmatrix}
                A & A \\
                A & A+B
            \end{pmatrix}\geqslant r
            \begin{pmatrix}
                AB & A \\
                O & A+B
            \end{pmatrix}\geqslant
            r(AB)+r(A+B).
        \end{equation*}
        \textbf{方法二.}设方程组$AX=0$与$BX=0$的解空间分别是$V_1, V_2$,方程组$ABX=BAX=0$与$(A+B)X=0$的解空间分别为$W_1, W_2$,则$V_1\subseteq W_1, V_2\subseteq W_1$,从而$V_1+V_2\subseteq W_1$,同时$V_1\cap V_2\subseteq W_1$,同时$V_1\cap V_2\subseteq W_2$,利用维数公式就有
        \begin{equation*}
            \dim V_1+\dim V_2=\dim(V_1+V_2)+\dim(V_1\cap V_2)\leqslant \dim W_1+\dim W_2.
        \end{equation*}
        即
        \begin{equation*}
            (n-r(A))+(n-r(B))\leqslant (n-r(AB))+(n-r(A+B)).
        \end{equation*}
        化简便知$r(A)+r(B)\geqslant r(AB)+r(A+B)$.
    \item
        \begin{enumerate}
            \item
                必要性:$ABX=0$与$BX=0$同解可知它们基础解系所含向量个数相同,即
                \begin{equation*}
                    n-r(AB)=n-r(B)\implies r(AB)=r(B).
                \end{equation*}
                充分性:由必要性,当$r(AB)=r(B)$时,$ABX=0$与$BX=0$的基础解系所含向量个数相同,而$BX=0$的解都是$ABX=0$的解,所以$ABX=0$与$BX=0$同解.
            \item
                $r(AB)=r(B)$说明$ABX=0$与$BX=0$同解,用$CX$代替$X$就得到$ABCX=0$与$BCX=0$同解,从而有$r(ABC)=r(BC)$.\\
                \textbf{推论.}设$A$是一个方阵,且存在正整数$k$使得$r(A^{k+1})=r(A^k)$,递推就有
                \begin{equation*}
                    r(A^k)=r(A^{k+1})=r(A^{k+2})=\cdots.
                \end{equation*}
            \item
                当$A$可逆时,结论是显然的.
                当$A$不可逆时,有$r(A)\leqslant n-1$,现在考虑$n+1$个矩阵$A,A^2,\cdots,A^{n+1}$,有
                \begin{equation*}
                    n-1\geqslant r(A)\geqslant r(A^2)\geqslant\cdots\geqslant r(A^n)\geqslant r(A^{n+1})\geqslant 0.
                \end{equation*}
                这$n+1$个矩阵的秩只能从$0,1,\cdots,n-1$这$n$个数中取,所以必有两个矩阵的秩相同,即存在$m(1\leqslant m\leqslant n)$使得$r(A^m)=r(A^{m+1})$,由上面的推论可得:
                \begin{equation*}
                    r(A^m)=r(A^{m+1})=\cdots=r(A^{n})=r(A^{n+1})=\cdots.
                \end{equation*}
                特别的,对于任意的正整数$k$有$r(A^n)=r(A^{n+k})$.
        \end{enumerate}
    \item
        增广矩阵
        $A=\begin{pmatrix}
            1 & 1 & b & -1 & 1 \\
            2 & 3 & 1 & 1 & -2 \\
            0 & 1 & a & 3 & -4 \\
            -3 & -3 & -3b & b & a+2
        \end{pmatrix}.$由解空间维数为$3$可知,$r(A)=2$.由于
        $\beta_1=\begin{pmatrix}
            1 \\
            2 \\
            0 \\
            3
        \end{pmatrix}
        \beta_2=\begin{pmatrix}
            1 \\
            3 \\
            1 \\
            -3
        \end{pmatrix}$线性无关,则后三列必能被其表示.对于
        $\alpha_2=\begin{pmatrix}
            -1 \\
            1 \\
            3 \\
            b
        \end{pmatrix}=k_1\beta_1+k_2\beta_2\implies k_1=-4,k_2=3\implies b=3$\\
        $\alpha_3=\begin{pmatrix}
            1 \\
            -2 \\
            -4 \\
            a+2
        \end{pmatrix}=k_1\beta_1+k_2\beta_2\implies k_1=5,k_2=1\implies a=-25.$\\
        而$\alpha_1=\begin{pmatrix}
            b \\
            1 \\
            a \\
            -3b
        \end{pmatrix}=\begin{pmatrix}
            3 \\
            1 \\
            -5 \\
            -9
        \end{pmatrix}=k_1\beta_1+k_2\beta_2\implies k_1=8,k_2=-5$成立.故$a=-5,b=3$.解空间的基略.\\
        若解空间为$2$维,则$r(A)=3$,令
        $A=\begin{pmatrix}
            \alpha_1 \\
            \alpha_2 \\
            \alpha_3 \\
            \alpha_4
        \end{pmatrix}$,由于$\alpha_1,\alpha_2$必线性无关,故$\alpha_3,\alpha_4$中有且仅有一个可被$\alpha_1,\alpha_2$表示.若为$\alpha_3$,$\alpha_2-2\alpha_1=(0,1,1-2b,3,-4)=(0,1,4,3,-4)$,则$a=1-2b$.故$a+2b=1$,且$\alpha_4$不能被表示,故$\alpha_4\neq -3\alpha_1\implies b\neq 3, a\neq -5$.
        故$\begin{cases}
            a+2b=1 \\
            b\neq 3, a\neq -5
        \end{cases}$即可.
    \item
        $W_1\cap W_2:\begin{pmatrix}
            A \\ B
        \end{pmatrix}X=0.$\qquad
        $W_1+ W_2:\begin{cases}
            AX = 0\\
            BX = 0
        \end{cases}.$
    \item
        \textbf{方法一.}令$x_4=t$,则方程组
        $\begin{cases}
            x_1+x_4=-1 \\
            x_2-2x_4=d \\
            x_3+x_4=e
        \end{cases}$的一般解为
        $\begin{cases}
            x_1=-1-t \\
            x_2=d+2t \\
            x_3=e-t \\
            x_4=t.
        \end{cases}$ \\
        代入方程组
        $\begin{cases}
            x_1+x_2+ax_3+x_4=1 \\
            -x_1+x_2-x_3+bx_4=2 \\
            2x_1+x_2+x_3+x_4=c
        \end{cases}$,可得
        $\begin{cases}
            (2-a)t=2-d-ae \\
            (b+4)t=1-d+e \\
            0=c-d-e+2.
        \end{cases}$
        由$t$的任意性,可得$a=2,b=-4$.从而
        $\begin{cases}
            0=2-d-2e \\
            0=1-d+e \\
            0=c-d-e+2.
        \end{cases}$
        解得$d=\frac 43,e=\frac 13,c=-\frac 13$.\\
        \textbf{方法二.}由于
        $$\begin{pmatrix}
            A & b \\
            B & d
        \end{pmatrix}=
        \begin{pmatrix}
            1 & 1 & a & 1 & 1 \\
            -1 & 1 & -1 & b & 2 \\
            2 & 1 & 1 & 1 & c \\
            1 & 0 & 0 & 1 & -1 \\
            0 & 1 & 0 & -2 & d \\
            0 & 0 & 1 & 1 & e
        \end{pmatrix}\rightarrow
        \begin{pmatrix}
            1 & 0 & 0 & 0 & 0 \\
            0 & 0 & 0 & b-a+6 & 3-2d+(1-a)e \\
            0 & 0 & 0 & 2a-4 & c-2+d(2a-1)e \\
            0 & 0 & 0 & a-2 & d-2+ae \\
            0 & 1 & 0 & 0 & 0 \\
            0 & 0 & 1 & 0 & 0
        \end{pmatrix}.$$
        易知$r(B)=3$,由
        $r\begin{pmatrix}
            A & b \\
            B & d
        \end{pmatrix}=r(B)$可得
        $\begin{cases}
            a-2=0 \\
            2a-4=0 \\
            b-a+6=0 \\
            3-2d+(1-a)e=0 \\
            c-2+d(2a-1)e=0 \\
            d-2+ae=0
        \end{cases}$,解得
        $\begin{cases}
            a=2 \\
            b=-4 \\
            c=-\frac 13 \\
            d=\frac 43 \\
            e=\frac 13
        \end{cases}.$
    \item
        注意如果$X$是两个方程组的公共解,这等价于存在$t_1,t_2,k_1,k_2$使得
        \begin{equation*}
            X=\gamma+t_1\eta_1+t_2\eta_2=\delta+k_1\xi_1+k_2\xi_2.
        \end{equation*}
        从而对应有
        \begin{equation*}
            t_1\eta_1+t_2\eta_2-k_1\xi_1-k_2\xi_2=\gamma-\delta.
        \end{equation*}
        将$t_1,t_2,k_1,k_2$设为未知量,对上述方程组的增广矩阵进行初等行变换,可得
        \begin{equation*}
            (\eta_1,\eta_2,-\xi_1,-\xi_2,\delta-\gamma)=
            \begin{pmatrix}
                -6 & -5 & -8 & -10 & -16 \\
                5 & 4 & 1 & 2 & 6 \\
                1 & 0 & -1 & 0 & 0 \\
                0 & 1 & 0 & -1 & 0
            \end{pmatrix}\rightarrow
            \begin{pmatrix}
                1 & 0 & -1 & 0 & 0 \\
                0 & 1 & 0 & -1 & 0 \\
                0 & 0 & 1 & 1 & 1 \\
                0 & 0 & 0 & 1 & 2.
            \end{pmatrix}
        \end{equation*}
        故$(t_1,t_2,k_1,k_2)^\prime$有唯一解$(-1,2,-1,2)^\prime$.因此公共解为
        \begin{equation*}
            X=\gamma-\eta_1+2\eta_2=\delta-\xi_1+2\xi_2=(1,0,-1,2)^\prime.
        \end{equation*}
    \item
        方程组(1)、(2)有公共解,即方程组$\begin{cases}
            x_1+2x_2+ax_3=0,\\
            x_1+4x_2+a^2x_3=0,\\
            x_1+2x_2+x_3=a-1
        \end{cases}$有解,对其增广矩阵进行初等行变换,
        \begin{align*}
            \bar{A}=&\begin{pmatrix}
                1 & 1 & 1 & 0 \\
                1 & 2 & a & 0 \\
                1 & 4 & a^2 & 0 \\
                1 & 2 & 1 & a-1
            \end{pmatrix}\rightarrow\begin{pmatrix}
                1 & 1 & 1 & 0 \\
                0 & 1 & a-1 & 0 \\
                0 & 3 & a^2-1 & 0 \\
                0 & 1 & 0 & a-1
            \end{pmatrix}\rightarrow\\&\begin{pmatrix}
                1 & 1 & 1 & 0 \\
                0 & 1 & a-1 & 0\\
                0 & 0 & a^2-3a+2 & 0\\
                0 & 0 & 1-a & a-1
            \end{pmatrix}\rightarrow\begin{pmatrix}
                1 & 1 & 1 & 0\\
                0 & 1 & a-1 & 0 \\
                0 & 0 & 1-a & a-1\\
                0 & 0 & 0 & (a-1)(a-2)
            \end{pmatrix}.
        \end{align*}
        当$a\neq 1$且$a\neq 2$时,方程组(1)、(2)没有公共解.\\
        当$a=1$时,$\bar{A}\rightarrow\begin{pmatrix}
            1 & 1 & 1 & 0 \\
            0 & 1 & 0 & 0 \\
            0 & 0 & 0 & 0 \\
            0 & 0 & 0 & 0
        \end{pmatrix}\rightarrow\begin{pmatrix}
            1 & 0 & 1 & 0 \\
            0 & 1 & 0 & 0 \\
            0 & 0 & 0 & 0 \\
            0 & 0 & 0 & 0
        \end{pmatrix}$,因为$r(A)=r(\bar{A})=2$,所以方程组(1)、(2)有公共解,公共解为$X=C\begin{pmatrix}
            -1 \\
            0 \\
            1
        \end{pmatrix}$(其中$C$为任意常数).\\
        当$a=1$时,$\bar{A}\rightarrow\begin{pmatrix}
            1 & 1 & 1 & 0 \\
            0 & 1 & 1 & 0 \\
            0 & 0 & -1 & 1 \\
            0 & 0 & 0 & 0
        \end{pmatrix}\rightarrow\begin{pmatrix}
            1 & 0 & 0 & 0 \\
            0 & 1 & 0 & 1 \\
            0 & 0 & 1 & -1 \\
            0 & 0 & 0 & 0
        \end{pmatrix}$,方程组(1)、(2)有唯一的公共解为$X=\begin{pmatrix}
            0 \\
            1 \\
            -1
        \end{pmatrix}.$
\end{enumerate}

\centerline{\heiti C组}
\begin{enumerate}
    \item
        令$f(x)=a_nx^n+\cdots+a_1x+a_0$为一个$n$次多项式,设$\lambda_1,\cdots,\lambda_{n+1}$为$f(x)$的$n+1$个不同的根,则有齐次线性方程组
        $\left\{\begin{array}{cl}
            \lambda_1^n a_n+\cdots+\lambda_1a_1+a_0&=0, \\
            \lambda_2^n a_n+\cdots+\lambda_2a_1+a_0&=0, \\
            \vdots& \\
            \lambda_{n+1}^n a_n+\cdots+\lambda_{n+1}a_1+a_0&=0.
        \end{array}\right.$
        其系数矩阵为
            $A=\begin{pmatrix}
                \lambda_1^n & \lambda_1^{n-1} & \cdots & 1 \\
                \lambda_2^n & \lambda_2^{n-1} & \cdots & 1 \\
                \vdots & \vdots & \ddots & \vdots \\
                \lambda_{n+1}^n & \lambda_{n+1}^{n-1} & \cdots & 1
            \end{pmatrix}$.
        显然$|A|\neq 0$,从而$a_n=a_{n-1}=\cdots=a_1=a_0=0$,这与$f(x)$是$n$次多项式矛盾.
    \item 见教材P214第7题.
    \item
        \begin{enumerate}
            \item
                由于$r(A)=n$,所以$A$可逆,由打洞原理可知$|B|=|A|(0-\beta^\prime A^{-1}\beta)=-|A|\beta^\prime A^{-1}\beta$,从而$B$可逆的充要条件为$\beta^\prime A^{-1}\beta\neq 0$.
            \item
                必要性. 由于
                \begin{equation*}
                    r=r(A)\leqslant r(A,\beta)\leqslant r\begin{pmatrix}
                        A & \beta \\
                        \beta^\prime & 0
                    \end{pmatrix}=r(B)=r.
                \end{equation*}
                再结合$A^\prime=A$,可知
                \begin{equation*}
                    r\begin{pmatrix}
                        A & \beta \\
                        \beta^\prime & 0
                    \end{pmatrix}
                    =r(A,\beta)=r\begin{pmatrix}
                        A\\
                        \beta^\prime
                    \end{pmatrix}.
                \end{equation*}
                于是由定理$15.1$可知方程组$\begin{pmatrix}
                    A\\
                    \beta^\prime
                \end{pmatrix}X=\begin{pmatrix}
                    \beta\\
                    0
                \end{pmatrix}$有解,即$\begin{cases}
                    AX=\beta, \\
                    \beta^\prime X=0.
                \end{cases}$有解.\\
                充分性. 由于$\begin{cases}
                    AX=\beta, \\
                    \beta^\prime X=0.
                \end{cases}$有解,从而$AX=\beta$也有解,即有$r(A)=r(A,\beta)$. 另外$\begin{cases}
                    AX=\beta, \\
                    \beta^\prime X=0.
                \end{cases}$有解也说明$\begin{pmatrix}
                    A\\
                    \beta^\prime
                \end{pmatrix}X=\begin{pmatrix}
                    \beta\\
                    0
                \end{pmatrix}$有解,于是结合定理$15.1$可知$r\begin{pmatrix}
                    A & \beta \\
                    \beta^\prime & 0
                \end{pmatrix}=r\begin{pmatrix}
                    A\\
                    \beta^\prime
                \end{pmatrix}.$而显然$r\begin{pmatrix}
                    A\\
                    \beta^\prime
                \end{pmatrix}=r(A,\beta)=r(A)$,于是$r(B)=r\begin{pmatrix}
                    A & \beta \\
                    \beta^\prime & 0
                \end{pmatrix}=r.$
            \item
                必要性. 若$B$可逆,则$B$的行向量组线性无关,从而$r(A,\beta)=n$,又由于$r(A)=n-1$,所以$r(A,\beta)=r(A)+1$,从而由定理$15.1$知$AX=\beta$无解.\\
                充分性. 由于$r(A)=n-1$,若$AX=\beta$无解,则由定理$15.1$可知$r(A,\beta)=r(A)+1=n$,于是
                \begin{equation*}
                    r(B)=r\begin{pmatrix}
                        A & \beta \\
                        \beta^\prime & 0
                    \end{pmatrix}\geqslant r(A,\beta)=n.
                \end{equation*}
                若$r(B)=n$,则$r(B)=r\begin{pmatrix}
                    A & \beta \\
                    \beta^\prime & 0
                \end{pmatrix}=r(A,\beta)$,取转置结合$A^\prime=A$有$r\begin{pmatrix}
                    A & \beta \\
                    \beta^\prime & 0
                \end{pmatrix}=r\begin{pmatrix}
                    A\\
                    \beta^\prime
                \end{pmatrix}$,再次结合定理$15.1$可知方程组$\begin{cases}
                    AX=\beta, \\
                    \beta^\prime X=0.
                \end{cases}$
                有解,特别地,$AX=\beta$也有解,这就与已知产生了矛盾.所以$r(B)=n+1$,即$B$可逆.
        \end{enumerate}
    \item
        \begin{enumerate}
            \item
                当$n$为偶数时,将增广矩阵$\bar{A}$的第$i(i=1,2,\cdots,n-1)$行乘以$(-1)^i$加到最后一行,得
                \begin{equation*}
                    \bar{A}\rightarrow\begin{pmatrix}
                        1 & 1 & \cdots & 0 & 0 & b_1 \\
                        0 & 1 & \cdots & 0 & 0 & b_2 \\
                        0 & 0 & \cdots & 0 & 0 & b_3 \\
                        \vdots & \vdots &  & \vdots & \vdots & \vdots \\
                        0 & 0 & \cdots & 1 & 1 & b_{n-1} \\
                        0 & 0 & \cdots & 0 & 0 & \sum\limits_{i=1}^{n}(-1)^ib_i
                    \end{pmatrix}.
                \end{equation*}
                故当$\sum\limits_{i=1}^{n}(-1)^ib_i=0$时,方程组有无穷多解,一般解为
                $$\begin{cases}
                    x_1&=\sum\limits_{i=1}^{n-1}(-1)^{i-1}b_i+(-1)^1x_n, \\
                    x_2&=\sum\limits_{i=2}^{n-1}(-1)^{i-2}b_i+(-1)^2x_n, \\
                    &\vdots \\
                    x_{n-2}&=b_{n-2}-b_{n-1}+(-1)^{n-2}x_n, \\
                    x_{n-1}&=b_{n-1}+(-1)^{n-1}x_n, \\
                \end{cases}$$
                其中$x_n$为自由未知量.\\
            \item
                当$n$为奇数时,有
                \begin{equation*}
                    \bar{A}\rightarrow\begin{pmatrix}
                        1 & 1 & \cdots & 0 & 0 & b_1 \\
                        0 & 1 & \cdots & 0 & 0 & b_2 \\
                        0 & 0 & \cdots & 0 & 0 & b_3 \\
                        \vdots & \vdots &  & \vdots & \vdots & \vdots \\
                        0 & 0 & \cdots & 1 & 1 & b_{n-1} \\
                        0 & 0 & \cdots & 0 & 2 & b_n+\sum\limits_{i=1}^{n-1}(-1)^ib_i
                    \end{pmatrix}.
                \end{equation*}
                此时无论$b_1,b_2,\cdots,b_n(n\geqslant 2)$取何值,方程组都有唯一解为
                $$\begin{cases}
                    x_1&=\sum\limits_{i=1}^{n-1}(-1)^{i-1}b_i+(-1)^{n-1}x_n,\\
                    x_2&=\sum\limits_{i=2}^{n-1}(-1)^{i-2}b_i+(-1)^{n-2}x_n,\\
                    &vdots\\
                    x_{n-2}&=b_{n-2}-b_{n-1}+(-1)^2x_n,\\
                    x_{n-1}&=b_{n-1}+(-1)^1x_n,\\
                    x_n&=\frac 12b_n+\sum\limits_{i=1}^{n-1}(-1)^ib_i.
                \end{cases}$$
        \end{enumerate}
    \item
        \begin{enumerate}
            \item
                必要性. 设$A$的行向量为$\alpha_1,\alpha_2,\cdots,\alpha_s$,$B$的行向量为$\beta_1,\beta_2,\cdots,\beta_m$,由$AX=0$与$BX=0$同解得$AX=0$与$\begin{pmatrix}
                    A\\
                    B
                \end{pmatrix}=0$同解,从而系数矩阵的秩相同,即行向量$\alpha_1,\alpha_2,\cdots,\alpha_s$与$\alpha_1,\alpha_2,\cdots,\alpha_s,\beta_1,\beta_2,\cdots,\beta_m$秩相同,%“由命题3.2.5知道”?
                因此$\alpha_1,\alpha_2,\cdots,\alpha_s$与$\alpha_1,\alpha_2,\cdots,\alpha_s,\beta_1,\beta_2,\cdots,\beta_m$等价,即有$\beta_1,\beta_2,\cdots,\beta_m$可由$\alpha_1,\alpha_2,\cdots,\alpha_s$线性表出. 同理,$\alpha_1,\alpha_2,\cdots,\alpha_s$可由$\beta_1,\beta_2,\cdots,\beta_m$线性表出,即$A$与$B$的行向量等价.\\
                充分性可由必要性的证明得到. 但是由于$A,B$列向量的维数都可能不同,所以不存在等价关系.
            \item
                提示. $V_1\subseteq V_2$等价于$AX=0$与$\begin{pmatrix}
                    A \\
                    B
                \end{pmatrix}X=0$同解.
            \item
                必要性. 由题意可知$AX=a$与$\begin{pmatrix}
                    A \\
                    B
                \end{pmatrix}X=\begin{pmatrix}
                    a \\
                    b
                \end{pmatrix}$同解,从而它们的增广矩阵秩相同,即$r(A,a)=r\begin{pmatrix}
                    A & a \\
                    B & b
                \end{pmatrix}$,可知$(A,a)$与$\begin{pmatrix}
                    A & a \\
                    B & b
                \end{pmatrix}$的行向量组等价,从而$(B,b)$的每一个行向量都可以由$(A,a)$的行向量线性表出.\\
                充分性可由必要性得到.
            \item 证明留给读者.
        \end{enumerate}
\end{enumerate}

\clearpage

\begingroup
\SetLUChapterNumberingStyle{4}
\def\theHchapter{\arabic{chapter}ε}

\chapter{线性同余方程与纽结}

\ResetChapterNumberingStyle{15}
\endgroup

\chapter{史海拾遗} \label{chap:史海拾遗}

通过前面十余讲的讨论,我们已经将线性代数的一个核心问题——有关线性方程组的解的一般理论完成,其意已尽. ``历史是一面镜子,它照亮现实,也照亮未来. '' 我们很有必要抓住这一时机来完整讨论有关于线性代数的历史,循着历代数学家的脚步重新审视所学的内容,再次感受其中逻辑的自然与顺畅. 很多时候,知道一件事情为什么、怎么来的更为可贵. 另一方面,我们也将在史海中搜寻整个数学大厦中与线性代数紧密相连的部分,开始我们后一阶段更多``未竟之美''的讨论.

提示:本讲中可能出现大量未学习的内容,事实上很多是后续会学习的,也有部分是非常前沿的介绍. 读者可以留个印象,日后再回过头来看,一定会感觉无比亲切.

\section{起点:初等代数}

\subsection{初等代数简介}

代数的英语为 algebra ,源于阿拉伯语单字``al-jabr''(本义为``重聚''),出自《代数学》(阿拉伯语:al-Kitāb al-muḫtaṣar fī ḥisāb al-ğabr wa-l-muqābala)这本书的书名上,意指移项和合并同类项之计算的摘要,其为波斯回教数学家花拉子米于820年所著.

事实上,初见代数一词我们脑海中便会浮现出小学、初中阶段老师反复强调的``用字母表示数''的思想,``一元一次方程''、``合并同类项''、``因式分解''等熟悉的词汇也会出现在眼前. 事实也是如此,初等代数的由来正是用字母表示数后,得到了一系列方程和多项式的有趣问题.

亚历山大港的丢番图(Dióphantos ho Alexandreús,公元200--284),是罗马时代的数学家. 大部分有关丢番图生平的信息来源于5世纪时希腊人梅特罗多勒斯(Metrodorus)在其文集中收录的一篇具有数学谜题性质的《丢番图墓志铭》:
\begin{quote}
    \kaishu
    坟中安葬着丢番图.

    多么令人惊讶,它忠实地记录了所经历的道路.

    上帝给予的童年占六分之一,

    又过十二分之一,两颊长胡,

    再过七分之一,点燃起结婚的蜡烛.

    五年之后天赐贵子,

    可怜迟到的宁馨儿,享年仅及其父之半,便进入冰冷的墓.

    悲伤只有用数论的研究去弥补,

    又过四年,他也走完了人生的旅途.
\end{quote}
据此列一元一次方程可知,丢番图享寿84岁,于33岁时成婚,38岁时生子,80岁时丧子. 丢番图作著的丛书《算术》(Arithmetica)处理求解代数方程组的问题,其中有不少已经遗失,但他的研究在数论中占有重要地位,如丢番图方程、丢番图几何、丢番图逼近等都是数学里的重要领域. 我们简要展开丢番图方程的讨论:
\begin{definition}
    形如
    \[a_1x_1^{b_1}+a_2x_2^{b_2}+\cdots+a_nx_n^{b_n}=c,\enspace a_i,b_i,c\in\mathbf{Z}\enspace(i=1,2,\ldots,n)\]
    的方程称为丢番图方程(或不定方程).
\end{definition}

简而言之,丢番图方程就是未知数只能使用整数的整数系数多项式等式,虽然定义看起来十分简单,学过初等数论的同学应当有些熟悉(事实上在之后的讨论中我们可以看到初等代数与初等数论之间是紧密联系的),但可以说这一方程在数学史上留下了浓墨重彩的一笔. 后来当法国数学家费马研究《算术》一书时,对其中某个方程颇感兴趣并认为其无解,说他对此``已找到一个绝妙的证明'',但却没有记录下来,直到三个世纪后才出现完整的证明. 我们这里简要介绍这一定理,即费马大定理:
\begin{theorem}
    丢番图方程$x^n+y^n-z^n=0$在$n>2$时无正整数解.
\end{theorem}

自费马提出猜想的三百余年以来,无数数学家为证明费马大定理而费尽心血,直至1995年,英国数学家安德鲁·怀尔斯(Andrew Wiles)最终给出了证明. 这一证明使用了代数数论、代数几何等大量现代数学工具. 如果读者有兴趣自行搜索费马大定理证明的历史,我们可以看到对这一看似简单定理的证明的不懈追求推动了现代数学的发展,可谓是意义重大. 或许在数学史中这些中间过程不过是短短的一行文字描述,但这就是人类探索真理的历程的缩影.

或许阅读这一讲义的很多同学都来源于计算机相关专业,我们这里还可以简要介绍丢番图方程与理论计算机的关联. 1900年,希尔伯特提出丢番图问题的可解答性为他在巴黎的国际数学家大会演说中所提出的23个重要数学问题的第十题. 这个问题是对于任意多个未知数的整系数不定方程,要求给出一个可行的算法,使得借助于它通过有限次运算,可以判定该方程有无整数解.

第十问题的解决是众人集体的智慧结晶. 其中美国数学家马丁·戴维斯(Martin Davis)、希拉里·普特南(Hilary Putnam)和朱莉娅·罗宾逊(Julia Robinson)做出了突出的贡献. 而最终的结果,是由俄国数学家尤里·马季亚谢维奇(Yuri Matiyasevich)于1970年所完成的:不可能存在一个算法能够判定任何丢番图方程是否有解. 这一问题涉及到``可计算性''的问题,相关的讨论读者在学习计算理论后将有进一步的了解. 想必读者听闻过罗素悖论或理发师悖论,或者是图灵停机问题,这些都与可计算性密不可分. 实际上,``可计算性''与人类逻辑与知识的边界密切相关——这显然是个异常宏大的主题,留待后人不断探索.

\subsection{西方初等代数发展史简述}

我们回到对于初等代数历史的讨论——刚刚我们显然有些跳脱了,但这些讨论对于了解数学之美也是必要的. 在古代西方,还有几个重要的时间节点值得提及:
\begin{enumerate}
    \item 公元前1800年左右,旧巴比伦斯特拉斯堡泥板书中记述其寻找著二次椭圆方程的解法;

    \item 公元前1600年左右,普林顿322号泥板书中记述了以巴比伦楔形文字写成的勾股数列表;

    \item 公元前800年左右,印度数学家包德哈亚那在其著作《包德哈尔那绳法经》中以代数方法找到了勾股数,给出了一些二次方程的几何解法,且找出了两组丢番图方程组的正整数解;

    \item 公元前300年左右,在《几何原本》的第二卷里,欧几里德给出了有正实数根之二次方程的解法,使用尺规作图的几何方法;

    \item 公元前100年左右,写于古印度的巴赫沙里手稿中使用了以字母和其他符号写成的代数标记法,且包含有三次与四次方程,多达五个未知量的线性方程之代数解,二次方程的一般代数公式,以及不定二次方程与方程组的解法.
\end{enumerate}

此为丢番图之前的初等代数发展重要节点,蕴含着古人朴素的智慧. 丢番图之后,499年,印度数学家阿耶波多在其所著之阿耶波多书里以和现代相同的方法求得了线性方程的自然数解,描述不定线性方程的一般整数解,给出不定线性方程组的整数解,而描述了微分方程;
628年,印度数学家婆罗摩笈多在其所著之梵天斯普塔释哈塔中,介绍了用来解不定二次方程的宇宙方法,且给出了解线性方程和二次方程的规则. 他发现二次方程有两个根,包括负数和无理数根.

此后便迎来一个更为重要的时间节点. 820年,代数(algebra)一词出现,其描述于波斯数学家花拉子米所著之完成和平衡计算法概要中对于线性方程与二次方程系统性的求解方法. 花拉子米常被认为是``代数之父'',其大多数的成果简化后会被收录在书籍之中,且成为现在代数所用的许多方法之一. 990年左右,波斯阿尔卡拉吉在其所著之al-Fakhri中更进一步地以扩展花拉子米的方法论来发展代数,加入了未知数的整数次方及整数开方. 他将代数的几何运算以现代的算术运算代替,定义了单项式并给出了任两个单项式相乘的规则.

此后,初等代数的发展逐步向着现代符号体系与研究方法发展,逐渐演化为了两个方向的问题的讨论:
\begin{enumerate}
    \item 未知数更多的一次方程组的解;

    \item 未知数次数更高的高次方程的解.
\end{enumerate}

前者与我们的主角:线性代数相关,而后者则引发了另一个学科——抽象代数的开端. 初等代数学逐步解决了2、3、4次方程求解问题,这些方程的解都可用系数的四则运算与根式运算给出,即可用根式解这些方程,此时初等代数也因此而达到顶峰. 但当时的数学家们继续探索
5次与5次以上方程的解也试图用根式解出这些方程,经过200余年却无重要进展,直到19世纪抽象代数的发展才有了转机,后续我们也将介绍这其中的故事.

\subsection{中国初等代数发展史简述}

在本节的最后,我们将视角转向东方,总结古代中国人在初等代数学中作出的贡献. 相信读者都十分熟悉这一问题:
\begin{quote}
    \kaishu
    今有雉兔同笼,上有三十五头,下有九十四足,问雉兔各几何?
\end{quote}

这是《孙子算经》(不晚于473年)中提出的著名的鸡兔同笼问题. 在《孙子算经》中还提出了读者在初等数论中就已十分熟悉的``中国剩余定理'',直至现代的密码学研究也无法离开这一重要定理.

实际上,早在《孙子算经》出现前500年左右(公元前100年左右),中国古代数学名著《九章算术》中便处理了代数方程的问题. 其中的``方程章''是世界上最早的系统研究代数方程的专门论著. 它在世界数学历史上最早创立了多元一次方程组的筹式表示方法,以及它的多种求解方法. 《九章算术》把这些线性方程组的解法称为``方程术'',其实质相当于现今的高斯消元法(早于高斯约1900年).

除去线性方程组的贡献,在高次方程方面,中国古代也有相当丰富的成果. 625年左右,中国数学家王孝通在《缉古算经》中找出了三次方程的数值解;1247年,南宋数学家秦九韶在《数书九章》中用秦九韶算法解一元高次方程. 1248年,金朝数学家李冶的《测圆海镜》利用天元术将大量几何问题化为一元多项式方程,是一部几何代数化的代表作. 1300年左右,中国数学家朱世杰处理了多项式代数,发明四元术解答了多达四个未知数的多项式方程组,发明非线性多元方程的消元法,将相关多项式进行乘法、加法和减法运算,逐步消元,将多元非线性方程组化为单个未知数的高次多项式方程;并以数值解出了288个四次、五次、六次、七次、八次、九次、十次、十一次、十二次和十四次多项式方程.

\section{演化:线性代数的产生与发展}

如前所述,初等代数经过数个世纪的发展逐渐演化为了两个大的方向:未知数更多的一次方程组和未知数次数更高的高次方程. 在这两个方向上的发展,使得代数学发展到高等代数的阶段,上面两个方向简而言之就是现在大家熟悉的线性方程组理论(线性代数)和多项式理论(以致后来的抽象代数). 本节我们主要讨论前者,后者我们将在下一节中讨论.

\subsection{行列式与Cramer法则的引入}

在这一部分,我们首先将重点介绍线性方程组理论的开山鼻祖——莱布尼茨. 莱布尼茨(Gottfried Wilhelm Leibniz,1646--1716),德国自然科学家、数学家、物理学家、历史学家和哲学家,和牛顿同为微积分的创建人. 他博览群书,涉猎百科,对丰富人类的科学知识宝库做出了不可磨灭的贡献,行列式与线性方程组理论是他留给人类的财富中很小但很重要的一部分.

莱布尼茨的第一个大的贡献便是引入了新符号. 莱布尼兹首先创立了采用两个记号的双标码记法,他在方程中使用系数
10,11,12;20,21,22;30,31,32,因为两个数字各有所指,所以相当于现代数学中方程系数符号的下标,即相当于$a_{10},a_{11},\ldots$的下标. 莱布尼兹在1693年给洛必达的一封信中给出了一个方程组:
\[\begin{cases}
        10+11x+12y=0 \\
        20+21x+22y=0 \\
        30+31x+32y=0
    \end{cases}\]

事实上,这一方程组有两个未知量和三个方程,当常数项不全为0时,这是一个非齐次线性方程组. 然后莱布尼茨首先对第一行和第二行消去变量$y$,有
\[10\cdot 22+11\cdot 22x-12\cdot 20-12\cdot 21x=0,\]
然后对第一行和第三行消去变量$y$,有
\[10\cdot 32+11\cdot 32x-12\cdot 30-12\cdot 31x=0,\]
对上述两式消去$x$,有
\[10\cdot 21\cdot 32+11\cdot 22\cdot 30+12\cdot 20\cdot 31-12\cdot 21\cdot 30-11\cdot 20\cdot 32-10\cdot 22\cdot 31=0,\]
事实上,这一式与现在我们所熟知的行列式形式
\[\begin{vmatrix}
        10 & 11 & 12 \\
        20 & 21 & 22 \\
        30 & 31 & 32
    \end{vmatrix}=0\]
完全一致. 回顾线性方程组有解的条件,即$(10,20,30)^\mathrm{T}$可以由$(11,21,31)^\mathrm{T},(12,22,32)^\mathrm{T}$线性表示,因此上面行列式等于0是方程组有解的必要条件,即莱布尼茨通过消元法解出了现在线性方程组有解的一个必要条件.

进一步地,莱布尼茨用记号$\overline{1\cdot 2\cdot 3\cdot 4}$表示现在的四阶行列式. 莱布尼兹说式$\overline{1\cdot 2\cdot 3\cdot 4}$由$4!=24$项组成,这些项可以由其中某一项指数的所有置换而得到. 这里为了叙述的完整性,我们先给出置换的相关定义:
\begin{definition}[置换]
    一个集合$S$的\term{置换}\index{zhihuan@置换 (permutation)}是一个从$S$到$S$的双射$p:S\to S$.
\end{definition}

\begin{example}
    设$S=\{1,2,3\}$,定义$p(1)=2,p(2)=3,p(3)=1$是$S$的一个置换,因为它是从$S$到$S$的双射. 我们通常将其记为$p_1=\begin{pmatrix}
            1 & 2 & 3 \\
            2 & 3 & 1
        \end{pmatrix}$,上面一行是按顺序排列的$S$的元素,下一行是按置换后的顺序排列的$S$的元素.

    同理$p_2=\begin{pmatrix}
            1 & 2 & 3 \\
            1 & 3 & 2
        \end{pmatrix}$,$p_3=\begin{pmatrix}
            1 & 2 & 3 \\
            1 & 2 & 3
        \end{pmatrix}$都是$S$的置换.
\end{example}

由此我们发现,对于有限集合$S$而言,其上的置换就是集合中元素多次对换位置,这里所谓的对换就是将两个元素交换. 当一个置换可以表示成连续偶数个对换时,称其为偶置换,否则称其为奇置换. 例如,$p_1$是偶置换,因为我们要先交换1,2得到2,1,3然后交换1,3得到结果. 同理,$p_2$是奇置换,$p_3$是偶置换.

莱布尼茨在文章中表示,由11,22,33,44的第二位数通过偶置换得到的那些项有相同的符号(取正号),其余取相反的符号(负号). 本质上,莱布尼兹知道现代一个行列式的一个组合定义,区别仅在于根据奇偶置换所确定的符号规则被逆序数代替,并且柯西也将这一低阶行列式的情形扩展为了一般的$n$阶情形,我们将在介绍柯西时给出相关的结论,届时我们也将进一步完善上面对于奇偶置换的讨论.

除此之外,基于这一``行列式''的定义此莱布尼茨也给出了最原始的Cramer法则,因此可以称为这一方向理论的鼻祖. 然而,莱布尼茨的很多工作都是后来(1850年左右)才被人们发现的,所以他的方法对后来其他数学家提出的规则几乎没有影响. 事实上,同时代的日本数学家关孝和在其著作《解伏元题法》中也提出了行列式的概念与算法. 而在Cramer法则上,麦克劳林和Cramer的工作更早被人们认识到.

麦克劳林(Maclaurin,1698--1746)是18世纪英国最具有影响的数学家之一. 他自幼聪慧勤奋,11岁便步入大学校门,17岁就以有关引力研究的论文获硕士学位,19岁受聘为阿伯丁马里沙尔学院数学教授,21岁当选为英国皇家学会会员. 麦克劳林最为读者熟知的贡献想必是麦克劳林级数展开式,实际上他还有几何学等方面其他贡献. 线性代数方面,在他1748年的遗著《代数论著》(\emph{A Treatise of Algebra})中,麦克劳林最先开创了用行列式的方法来求解含2个、3个和4个未知量的联立线性方程组. 遗憾的是,麦克劳林没能进一步给出一个明确的法则来确定符号. 虽然,书中的记法不太好,符号变化的规则又比较模糊,但它确实就是我们今天所使用的Cramer法则.

事实上,现在我们所熟知的Cramer法则是由瑞士数学家加布里埃尔·克莱姆(Gabriel Cramer,1704--1752)在1750年的著作《线性代数分析导言》(\emph{Introduction à l'analyse des lignes courbes algébriques})中给出的. 为了确定经过5个点的一般二次曲线的系数,他引入了这一著名的法则,并且因其符号上更为简洁明了的优越性而被人们所接受. 事实上,克莱姆最著名的工作是在1750年发表关于代数曲线方面的权威之作. 他最早证明一个第$n$度的曲线是由$n(n + 3)/2$个点来决定的.

\subsection{线性方程组与行列式的进一步研究}

在前人工作的基础上,关于线性方程组以及行列式的理论有了更快的发展. 裴蜀(E. Bézout,1730--1783),法国数学家. 曾在海军学校和皇家炮兵学校任教,主要从事代数方程理论的研究并取得一系列的成果. 1764年,裴蜀发表论文提出了行列式中项的构成规则和符号的形成规则. 他给出了行列式的一个循环构造规律,同时用不同于莱布尼兹、克莱姆的方法,给出了项的构成规则和符号确定规则. 他所作的成就对后来行列式理论的奠基和发展起着非常重要的作用. 同时,裴蜀在该文中证明了含$n$个未知量的$n$个齐次线性方程组有非零解的条件是其``结式''(系数行列式)等于零,跳出了前人对于求解方程组计算问题的讨论,转向对一般理论的讨论.

在行列式的发展史上,第一个对行列式理论做出连贯的逻辑的阐述,即把行列式理论与线性方程组求解相分离的人,是法国数学家范德蒙(A-T. Vandermonde,1735--1796)——他不仅把行列式应用于解线性方程组,而且对行列式理论本身进行了开创性研究. 范德蒙自幼在父亲的指导下学习音乐,但对数学有浓厚的兴趣,后来终于成为法兰西科学院院士. 他给出了用二阶子式和它们的余子式来展开行列式的法则,这跳出了前人从线性方程组角度研究行列式的范畴,因此就对行列式本身这一点来说,他是这门理论的奠基人. 当然,范德蒙还有一个读者十分熟知的工作,便是计算了范德蒙行列式,这一行列式对于后续的研究有非常重要的地位.

除此之外,范德蒙的工作也得到了进一步的推广. 1772年,拉普拉斯在一篇论文中证明了范德蒙提出的一些规则,并推广了他的展开行列式的方法,便有了大家熟知的按多行(多列)展开的拉普拉斯定理. 1779年,裴蜀(正是前面所介绍的,实际上这里介绍的数学家很多都有工作的交织)发表了一篇《代数方程的一般理论》的文章,这篇论文给出了解决非齐次线性方程组的方法,这个方法是他在克莱姆、范德蒙和拉普拉斯行列式理论基础上的总结. 除此之外,裴蜀在论文中还有其他很多关于行列式理论的发现:他改进了拉普拉斯展开式的另一个改进形式;得出了行列式的两行或两列相同则结果为零的结论;并结合线性方程的消元法得出了著名的``裴蜀定理''等.

接下来对行列式理论做了可谓``大一统''工作的是著名数学家柯西——是的,又是他,一个和欧拉、高斯一样无处不在的数学家. 1812年,柯西率先使用了双下标的方式表示方程组系数(即$a_{11}$这样的有两个数字组成的下标),有趣的是柯西当年还没有使用双竖线的方式表示行列式,而是采用$S(\pm a_{11}a_{22}\cdots a_{nn})$的形式. 现在为人熟知的双竖线的表示形式是后文将要介绍的矩阵论创始人凯莱在1841年率先使用的. 事实上,柯西给出了与现代行列式定义几乎完全一致的版本. 为了展开叙述柯西的理论,我们在此进一步讨论有关于置换的概念. 我们来看一个简单的置换
\[p=\begin{pmatrix}
        1 & 2 & 3 & 4 & 5 \\
        2 & 1 & 4 & 5 & 3
    \end{pmatrix},\]
我们可以将得到上述置换的过程分为两步. 第一步我们对1和2进行置换,得到
\[p_1=\begin{pmatrix}
        1 & 2 & 3 & 4 & 5 \\
        2 & 1 & 3 & 4 & 5
    \end{pmatrix},\]
然后我们对3,4,5进行置换,但保持上一步中已经改变的1和2不变,这一过程可以写为
\[p_2=\begin{pmatrix}
        1 & 2 & 3 & 4 & 5 \\
        1 & 2 & 4 & 5 & 3
    \end{pmatrix}.\]
就可以得到$p$. 事实上这接续的两步和映射的复合运算含义完全一致. 回忆$h=g\circ f$,$h(x)=g(f(x))$是先对$x$作用$f$然后作用$g$,这里实际上也是对$1,2,3,4,5$先做了$p_1$的置换然后做了$p_2$的置换,即$p=p_1\circ p_2$,写成乘法形式即为
\[\begin{pmatrix}
        1 & 2 & 3 & 4 & 5 \\
        2 & 1 & 4 & 5 & 3
    \end{pmatrix}=\begin{pmatrix}
        1 & 2 & 3 & 4 & 5 \\
        1 & 2 & 4 & 5 & 3
    \end{pmatrix}\begin{pmatrix}
        1 & 2 & 3 & 4 & 5 \\
        2 & 1 & 3 & 4 & 5
    \end{pmatrix}.\]
实际上,上面的置换$p_1$相当于将1变为了2,然后2又变回了1,这构成了一个长度为2的循环.$p_2$相当于将3变为了4,4变为了5,然后5又变回了3,因此构成了一个长度为3的循环. 因此上式也可以改写为
\[p=(3,4,5)(1,2)\]
其中$(3,4,5)$就表示从3到4,4到5,然后5到3的循环(事实上$(4,5,3),(5,3,4)$同理表示同一个循环),$(1,2)$也是同理. 事实上,因为$3,4,5$和$1,2$之间没有元素是重复的,因此可以称它们是不相交的. 事实上我们有如下定理,我们不加证明地直接给出:
\begin{theorem}
    $S$上的任意置换$p$都可以表示为长度$\geqslant 2$且不相交的循环的乘积,且这一分解式在不考虑循环顺序(即上面所说的$(3,4,5)$和$(4,5,3)$实则表示一个循环)下是唯一确定的.
\end{theorem}

事实上,我们在前面讲的对换(即两个元素交换顺序)就是长度为2的循环. 关于对换,我们也有一个重要的结论:
\begin{theorem}\label{thm:16:对换乘积}
    $S$上的任意置换$p$都可以表示为对换的乘积.
\end{theorem}

实际上这一结论是很直观的,$1,2,\ldots,n$的任意置换实际上都可以通过两两交换顺序得到. 最愚蠢的方式就是反过来思考如何从任意置换反推到$1,2,\ldots,n$,然后全部顺序调换即可. 反推的方式就是首先找到1的位置,然后一直向左对换到最左端,然后开始找2,对换到第二个位置,以此类推,因此这一结论是显然正确的. 但接下来的证明将会从另一个角度给出更为丰富的结果:

\begin{proof}

\end{proof}

事实上在介绍莱布尼茨的工作时我们就介绍了莱布尼茨利用奇数或偶数次对换作为依据确定行列式定义中每一个排列的乘积前的符号,这里我们给出严谨的关于符号的说明:
\begin{theorem}[置换的符号] \label{thm:16:置换的符号}
    设$p$是$S$上的一个置换,将$p$分解为对换的乘积:
    \[p=p_1\cdots p_k,\]
    则称
    \[\tau(p)=(-1)^k\]
    为$p$的\term{符号}(亦称符号差或奇偶性),它由置换$p$唯一确定且不依赖于对换分解的方法. 此外任取$q,r$也为$S$上的置换,则有
    \[\tau(qr)=\tau(q)\tau(r).\]
\end{theorem}

\begin{proof}

\end{proof}

基于这一定理,我们可以有如下合理的定义:
\begin{definition}
    若$\tau(p)=1$,则称$p$为$S$上的偶置换;若$\tau(p)=-1$,则称$p$为$S$上的奇置换.
\end{definition}

事实上我们在之前也定义了奇置换和偶置换,即从顺序排列的数列经过奇数还是偶数次对换可以得到最终的置换出发进行定义. 两个定义实际上是统一的,统一性由下面这一定理保证:
\begin{theorem}\label{thm:16:置换符号计算公式}
    设$S$上的置换$p$分解为长为$l_1,l_2,\ldots,l_m$的互不相交的循环的乘积,则
    \[\tau(p)=(-1)^{\sum\limits_{k=1}^m(l_k-1)}.\]
\end{theorem}

因为对换是长度为2的循环,事实上每一个$l_i-1$都等于1. 如果上述定理成立,那么奇数次的对换将会使得置换符号为$-1$,因为$-1$的奇数次方,偶置换则会得到符号为$-1$的偶数次方,即为1,因此两个定义统一.

\begin{proof}
    事实上,根据\autoref{thm:16:置换的符号},我们有
    \[\tau(p)=\tau(p_1)\cdots\tau(p_m),\]
    根据\autoref{thm:16:对换乘积} 的证明我们知道,$p_i$可以被写为$l_i-1$个对换的乘积,因此我们有$\tau(p_i)=(-1)^{l_k-1}$,因此有
    \[\tau(p)=(-1)^{\sum\limits_{k=1}^m(l_k-1)}.\]
\end{proof}

接下来我们就要看如何将上述置换的分解与符号的理论用于计算行列式. 柯西从$n$个数$a_1,\ldots,a_n$出发,作乘积$a_1\cdots a_n$,然后类似于范德蒙行列式,作所有不同元之间的差的积$\displaystyle\prod_{1\leqslant i\leqslant j\leqslant n}(a_j-a_i)$,得到乘积
\begin{equation}\label{eq:16:柯西行列式}
    a_1\cdots a_n\prod_{1\leqslant i\leqslant j\leqslant n}(a_j-a_i)
\end{equation}
柯西对这个乘积中各项所含的幂改写成第二个下标,例如把$a_2^3$改写为$a_{23}$,把这样改写后得到的表达式定义为一个行列式,记作$S(\pm a_{11}a_{22}\cdots a_{nn})$.

我们以$n=3$为例展示上面的过程. 根据柯西描述的算法,乘积为
\begin{align*}
               & a_1a_2a_3(a_2-a_1)(a_3-a_1)(a_3-a_2)                                                                               \\
    =          & a_1a_2^2a_3^3-a_1^2a_2a_3^3+a_1^3a_2a_3^2-a_1a_2^3a_3^2+a_1^2a_2^3a_3-a_1^3a_2^2a_3                                \\
    \triangleq & a_{11}a_{22}a_{33}-a_{12}a_{21}a_{33}+a_{13}a_{21}a_{32}-a_{11}a_{23}a_{32}+a_{12}a_{23}a_{31}-a_{13}a_{22}a_{31}.
\end{align*}

事实上这与今天我们熟悉的三阶行列式计算公式完全一致,事实上$n$阶都是一致的. 柯西天才地用一个很简短的抽象公式将前人找到的规律描述了出来,同时也发明了双下标的表示,将行列式可以写成$n\times n$的矩形方阵形式,并且沿用至今.

事实上,由\autoref{eq:16:柯西行列式} 展开得到的式子中的项不难看出都可以写成如下形式:
\[\prod a_{1p(1)}\ldots a_{np(n)}\]
其中$p$是集合$S=\{1,2,\ldots,n\}$上的一个置换,因为乘积前面的$a_1\cdots a_n$会保证每一个数都出现,而后面的乘积由排列组合的知识可知只有$a_1,\ldots,a_n$的次数分别为$0,1,\ldots,n-1$的一个置换才会留下来且前面的系数为1或$-1$. 接下来便有一个问题,即前面的系数究竟是1还是$-1$. 柯西使用的方法与前面介绍的几乎完全一致!他就是通过计算$(-1)^{\sum\limits_{k=1}^m(l_k-1)}$这一方式判断的,其中$l_k$是上述置换$p$进行循环分解后各项的长度. 基于此,我们有行列式定义如下:
\begin{definition}
    $n$阶行列式
    \[\begin{vmatrix}
            a_{11} & a_{12} & \cdots & a_{1n} \\
            a_{21} & a_{22} & \cdots & a_{2n} \\
            \vdots & \vdots & \ddots & \vdots \\
            a_{n1} & a_{n2} & \cdots & a_{nn}
        \end{vmatrix}=\sum_{p\in S_n}\tau(p)\prod_{i=1}^na_{ip(i)},\]
    其中$S_n$是集合$S=\{1,2,\ldots,n\}$上置换的全体,$\tau(p)$是置换$p$的符号.
\end{definition}

在很多教科书上,这一定义也被称为逆序数定义,这是因为置换的符号实际上也可以视为所谓``逆序对''个数的体现. 那么何为逆序对呢?其实一对数就是$(a_i,a_j)$,其中$i<j$,当$a_i>a_j$时,我们称这一对数为逆序对. 例如,对于置换
\[p=\begin{pmatrix}
        1 & 2 & 3 \\
        3 & 1 & 2
    \end{pmatrix},\]
$(3,1)$和$(3,2)$都是逆序对,因为3在1前面但比1大,3在2前面但比2大. 而数对$(1,2)$则是顺序对,因为1在2前面且比2小,这与原先$1,2,3$的排列顺序是统一的.

基于逆序对的定义我们可以有奇置换和偶置换的另一个定义:
\begin{definition}
    我们称逆序数为奇数的置换为奇置换,逆序数为偶数的置换为偶置换.
\end{definition}

显然我们也必须要求这一定义和前面的定义是一致的,事实上我们有如下定理:
\begin{theorem}
    任意置换$p$的逆序数的奇偶性与其可以被分解为对换乘积的个数的奇偶性相同.
\end{theorem}

\begin{proof}

\end{proof}

由这一定理以及\autoref{thm:16:置换符号计算公式} 我们知道逆序数为奇数时,置换符号为$-1$,逆序数为偶数时,置换符号为1,因此这一定义与前面的定义是一致的.

由此我们可以看出,行列式的逆序数定义实际上最开始来源于莱布尼茨、克莱姆等人最朴素的从低阶出发的探索,它们找到了一些规律,这些规律由天才数学家柯西进行抽象总结,得到具有普适性的方法,变成了上面沿用至今的严格定义. 而后人又结合置换、逆序数等理论进行重新叙述,再嵌套一层抽象,最终上面完整的叙述逻辑. 在这里我们看到一个很抽象甚至初看没有什么道理的定义是如何自然演化而来的,实际上是数学理论螺旋式进步以及多个理论(可能对应很多条数学发展支线)在教育家们的手中进行合理排列后所呈现的状态.

更进一步地,1815年,柯西发表了一篇关于行列式理论的基础性文章. 在这篇文章中它不仅用这个名字代替了几个旧的术语,也在文章中给出了系统的一般行列式乘法定理,证明了新组的行列式是原来两个组的行列式的乘积. 在这篇论文中,柯西第一次论述了包括一个给定的矩阵的伴随矩阵的思想,以及通过展开任何行或者列来计算行列式的步骤,完善了范德蒙和拉普拉斯的工作,给出了严谨的证明. 在柯西的行列式的工作中,还涉及到对称矩阵以及相似变换等问题. 在柯西1826年的《微积分在几何中的应用教程》中,讨论了后续学习中将要介绍的一些二次型理论,以及实对称矩阵特征值均为实数(后续会讲解)等重要结论. 除此之外,柯西在相似行列式的研究中,证明了大家熟知的相似变换有相同的特征值的结论. 由此可见,柯西这一数学天才对于后世的影响是无比深远的,从记号层面的革新,到行列式展开、行列式乘法等理论的大一统,以及现在大家熟知的结论的证明,都能看出柯西贡献的突出与伟大.

继柯西之后在行列式理论方面最高产的人就是德国数学家雅可比(C. Jacobi,1804--1851),他引进了函数行列式,即``Jacobi 行列式''(读者学习多元微积分时会十分熟悉这一名词),指出函数行列式在多重积分的变量替换中的作用,给出了函数行列式的导数公式. 雅可比的著名论文《论行列式的形成和性质》标志着行列式系统理论的建成. 事实上,行列式在数学分析、几何学、线性方程组理论、二次型理论等多方面的应用,促使行列式理论自身在19世纪也得到了很大发展. 整个19世纪都有行列式的新结果. 除了一般行列式的大量定理之外,还有许多有关特殊行列式的其他定理都相继得到.

最后笔者要在此说明一点. 我们在讲义中介绍了行列式最常见的三种定义方式. 事实上,按照历史的发展脉络,的确是现在看来最不直观的逆序数定义的思想首先出现的. 《大学数学:代数与几何》中给出的公理化定义有很强的几何直观性,也与列向量组的线性相关性等有很直接的关联,但实际上几何直观源于后来拉格朗日发现行列式和以其列向量构成的四面体的体积之间的关系,是远在莱布尼茨的思想出现之后才讨论的. 而关于行列式展开的定义根据上面的讨论也知道,是范德蒙、拉普拉斯和柯西接力提出并给出严谨证明才得到的.

事实上,笔者在编写讲义的时候就发现,无论从哪个定义出发定义行列式都是显得``毫无道理''的,因为完全缺乏引入,这不像之前研究线性空间那般自然(因为我们发现了方程组行向量间的线性相关性影响了解的唯一性,并且线性空间也有高中学习的平面向量的直观作为基础),行列式的定义直接丢出来会显得非常笨拙而且没有道理,但我们研究其性质会发现它和可逆、矩阵的秩甚至以后的特征值理论有很强的关联,行列式仿佛成为了一个无头但有尾的理论,这可能也是为什么《线性代数应该这样学》完全抛弃了行列式来讲述线性代数——因为这非常难引入且不是必要的. 但笔者还是希望保留行列式这一具有重要历史地位的理论,并且它对于之后的很多研究都有重要意义,所以笔者选择在史海拾遗中从历史的角度提供一种``直观''——它来源于数学家最开始对一些问题的研究. 因此我们详细地描述了莱布尼茨如何从消元法解线性方程组得到类似于现代行列式的定义的过程. 尽管这是低维的情况,但接下来在讲述柯西的工作时,$n$维的情况在逻辑上就像是自然的推广了(我们相信在历史中也是如此,前人对于线性方程组解的形式(如Cramer法则)和行列式的研究都构成了柯西研究的出发点,在此基础上柯西做了更进一步的抽象得到了现在的行列式定义,后人又结合了置换等概念将这一理论进一步形式化),这样我们也勉强算是梳理出了一个有引入的能更让初学者接受的行列式理论:至少我们从线性方程组的解的讨论起步——于是根据朝花夕拾的知识我们知道行列式一定和线性相关、矩阵的秩等概念有很强的关联,得到一些简单的结论,然后不断抽象直至今天呈现在读者面前的令人摸不着头脑的理论,虽然非常冗长,但相信能让读者接受这一概念的引入也是比较自然的.

数学不是魔术,不是从无到有的魔法,其发展历程必然是螺旋式上升的过程,很多不直观的概念可能只是因为历经数百年很多数学家不断改进而使人很难看出当年原本自然的想法源于何处. 希望读者读完本讲后再看教材时,能体会到每一个概念、每一个定理背后的历史厚重感,它们都是历代数学家在前人肩膀上不断总结、创新而来的,这些想法或是沉稳的推进,也可能是天才的智慧. 也许在未来很多年以后的教材上,有一个全新的概念或者结论,它可能只是短短的一行描述,但那也许凝结着现在正阅读着这段文字的你未来很多年研究的心血——这也许就是一种价值的实现.

\subsection{矩阵理论的发展}

随着线性方程组和行列式理论的建立和发展,在行列式基础之上的矩阵理论发展非常迅速. ``矩阵''这个词是由西尔维斯特首先使用的,他是为了将数字的矩形阵列区别于行列式而发明了这个术语. 而实际上,矩阵这个课题在诞生之前就已经发展的很好了. 从行列式的大量工作中明显的表现出来,不管行列式的值是否与问题有关,方阵本身都可以研究和使用,矩阵的许多基本性质也是在行列式的发展中建立起来的. 在逻辑上,矩阵的概念应先于行列式的概念,然而在历史上次序正好相反.

虽然矩阵一词是西尔维斯特率先发明的,但英国数学家凯莱(A. Cayley,1821--1895)一般被公认为是矩阵论的创立者,因为在西尔维斯特创用矩阵术语以前,凯莱对于矩阵的有关概念及其性质就有所研究. 1843年,凯莱即己研究三阶以上的高阶矩阵的行列式理论(\emph{On the theory of
determinants}),L. Gegenbauer、M-Lecat、L. H. Rice等在这个领域又进行了扩展. 1846年,凯莱定义了转置矩阵以及对称矩阵,与现代的定义完全一致. 在1855--1858年间,凯莱在矩阵方面做了许多开创性的工作. 1855年,凯莱注意到在线性方程组中使用矩阵是非常芳便的,因而引进矩阵以简化记号,这就有了现在我们使用的阶梯矩阵等术语以及$AX=\vec{b}$的记号.

1858年,凯莱发表了重要文章《矩阵论的研究报告》(\emph{A memoir on the theory of matrices}). 在该研究报告中,凯莱系统地阐述了矩阵的理论体系,如矩阵概念的引入、相关概念和运算的定义,使得矩阵从零散的知识发展为系统完善的理论体系. 凯莱定义了矩阵加法和数乘运算,并且从变换的复合引入了矩阵乘法的运算法则,也给出了一些特殊矩阵例如零矩阵、单位矩阵等,同时也说明了两个矩阵相乘不符合交换律,但也着重强调了矩阵乘法是可结合的. 除此之外,凯莱也引入了逆矩阵的概念. 凯莱在文章中采用单个的符号表示矩阵,证明了矩阵$A$可逆时,方程$AX=\vec{b}$的解可以写为$X=A^{-1}\vec{b}$,并且也给出了矩阵可逆时
\[A^{-1}=\frac{1}{|A|}A^*.\]
凯莱还利用一般的代数运算和矩阵运算的相似性得出了矩阵的一些结论. 例如当行列式为零时矩阵不可逆,零矩阵不可逆,两个非零矩阵乘积可以为零矩阵等结论. 除此之外,凯莱在文章中采用单个的符号表示矩阵,推出了方阵的特征多项式的形式,并说明了特征多项式的根就是特征值的重要结论. 除此之外,凯莱也证明了我们后面要详细介绍的``哈密顿-凯莱''定理的一部分,这被称为``矩阵理论中最著名的理论之一''.

凯莱第一个把矩阵作为独立的概念提出来,并作为独立的理论加以研究. 可以说,《矩阵论的研究报告》的公开发表,标志着矩阵理论作为一个独立数学分支的诞生. 但我们之前也提到,矩阵一词是西尔维斯特在研究方程的个数与未知量的个数不相同的线性方程组时最先使用的. 因此我们也很有必要接着介绍凯莱的挚友——西尔维斯特在矩阵理论方面的成果.

詹姆斯·约瑟夫·西尔维斯特(James Joseph Sylvester,1814--1897),1829年进入设在利物浦的皇家学会的学校学习,他学习努力,成绩突出,曾因解决了美国抽彩承包人提出的一个排列问题而得到500美元的数学奖金.1846年西尔维斯特进入内殿(Inner Temple)法学协会,并于1850年取得律师资格. 在这期间他和同时进入林肯法律学会的凯莱建立起了深厚的友谊. 他们在从事法律业务的间隙,经常在一起交流数学研究的成果. 西尔维斯特一生致力于纯数学的研究,他和凯莱、哈密顿等人一起开创了自牛顿以来英国纯粹数学的繁荣局面. 西尔维斯特的成就主要在代数方面,在代数方程论、数论等诸领域都有重要的贡献. 西尔维斯特一生创造过许多数学名词,流传至今的如矩阵、判别式等都是他首先使用的.

1850年,西尔维斯特在研究方程的个数与未知量的个数不相同的线性方程组时,由于无法使用行列式(因为行列式要求行列数相同),所以引入了矩阵一词来表由$m$行$n$列元素组成的矩形阵列,西尔维斯特也引入了对角矩阵、数量矩阵等概念.1879年弗罗伯纽斯给出矩阵的秩的概念后,1884年,西尔维斯特给出了零性的概念和零性律:他把矩阵的阶数与秩的差叫做矩阵的零性,并说明了两个(而且可以推广为任意有限数目)矩阵乘积的零性不能比任意因子的零性小,也不能比组成这一乘积的因子的零性之和大. 西尔维斯特的这一零性律现在应当叙述为:
\[r(A)+r(B)-n\leqslant r(AB)\leqslant\min\{r(A),r(B)\},\]
这是矩阵理论中关于矩阵乘积的秩的一个重要定理. 除此之外,西尔维斯特和凯莱也就矩阵方程
\[A_1XB_1+\cdots+A_kXB_k=C\]
和
\[AX-XB=O,\]
这里篇幅有限我们不再展开叙述. 事实上,在之后的二次型理论中,我们还会学习到所谓的西尔维斯特惯性定理,这也是非常核心的理论.

在矩阵论发展史上,弗罗贝尼乌斯(G. Frobenius, 1849--1917)的贡献是不可磨灭的. 1870年左右,群论成为数学研究的主流之一,弗罗贝尼乌斯在柏林时就受到库默尔和克罗内克的影响,对抽象群理论产生兴趣并从事这方面的研究,发表了多篇有价值的论文. 1892年,他重返柏林大学任数学教授. 1893年当选为柏林普鲁士科学院院士. 弗罗贝尼乌斯的主要数学贡献在群论方面,在行列式、矩阵、双线性型以及代数结构方面也有出色的工作. 矩阵论方面,1878年,弗罗贝尼乌斯引进了西尔维斯特$\lambda$矩阵的行列式因子、不变因子和初等因子等概念(在一般的高等代数教材中都会由此引入讨论若当标准形),给出了正交矩阵、相似矩阵、合同矩阵等概念(与现在的定义是完全一致的),指出了各种不同类型的矩阵的关系. 1894年,他又对1878年的不变因子和初等因子理论做了更深入的工作,进一步整理了维尔斯特拉斯不变因子和初等因子的理论. 1879年,弗罗贝尼乌斯引进了矩阵的秩的概念: 矩阵的秩就是矩阵中非零子式的最大阶数. 他也引进了行列式秩的定义:如果一个行列式的所有$r+1$阶子式为0,但至少有一个$r$阶子式不为零,那么就称$r$为行列式的秩.

在上述三位数学家的工作下,矩阵论中的一个核心问题:矩阵约化与分解不断地有了新的突破. 我们将在后续章节用大量的篇幅介绍这一主题——因此这里也许可以算是承上启下的一节. 1854年,约当研究了矩阵化为标准型的问题. 1892年,梅茨勒(H. Metzler)引进了矩阵的超越函数概念并将其写成矩阵的幂级数的形式. 傅立叶、西尔和庞加莱的著作中还讨论了无限阶矩阵问题,这主要是适用方程发展的需要而开始的. 事实上,矩阵由最初作为一种工具经过两个多世纪的发展,现在已成为独立的一门数学分支——矩阵论. 而矩阵论又可分为矩阵方程论、矩阵分解论和广义逆矩阵论等矩阵的现代理论. 矩阵及其理论现已广泛地应用于现代科技的各个领域,相信很多工科读者在将来的学习中将会大量运用这一方面的结果.

\subsection{线性代数的应用:解析几何的发展}

线性代数的发展与解析几何的发展有着密切的联系,应该说二者间在数学发展史上来看是互相促进的关系. 一方面,从希腊时代到1600年几何统治着数学,代数居于附庸的地位. 而解析几何为确立代数在数学界的地位铺平了道路. 1600年以后,代数才从几何统治的桎梏下解放出来,成为一门独立的基础数学科目,占据了它在数学中应有的地位. 另一方面,我们接下来也将会展开很多用线性代数知识解决几何问题的实例.

\subsection{线性空间与线性映射的角度}

前面的讨论我们一直讨论的是线性方程组与行列式关联的历史,本讲义中最重视的线性空间和线性映射理论被搁置了. 这一节我们将重点放在这一部分,供

物理学的发展带动了向量理论及向量空间的发展,而向量理论和向量空间的发展也打开了新的数学前景. 当今数学意义上的向量及向量空间的概念内容丰富,形式多样.

向量空间第一个具体定义的是由皮亚诺(G. Peano,1858--1932)在1888年的《几何计算》中给出,但是影响并不广泛. 直到1918年外尔(Hermann Weyl,1885--1955)的工作,使得人们重新认识到了皮亚诺公理化定义的重要性. 大约在1920年左右,分析学家巴拿赫、哈恩和维纳等对皮亚诺的向量空间做了进一步的研究,并引起了广泛的影响,随之而产生了如赋范向量空间、希尔伯特空间等等.

\section{推广:线性代数之后的线性和代数}

\subsection{泛函分析}

\subsection{抽象代数}

本小节我们接着前面初等代数发展为高等代数的两个方向之二,继续讨论有关于高次方程与多项式理论的历史.

\section{进阶:(线性)代数的进一步发展}

\subsection{抽象代数}

\subsection{泛函分析}

\subsection{抽象代数}

\section{进阶:本世纪的线性代数}

在这一节中,我们就几个专题探讨线性代数中的一些概念在上世纪末到本世纪的过程中的发展,主要探讨矩阵论而非线性空间理论的部分. 这是因为矩阵论的发展相对而言比较接地气,大部分从结果上看非常简捷明快. 本节提及的很多结论,在证明中都有众多数学分支如草蛇灰线穿行其中. 但是,一个初步的展现未必需要完整的表述,省略一些证明也不妨碍读者领略现代矩阵论研究的风采. 至于从线性代数真正意义上``生发开去''的东西,我们将留在下一节讨论.

\subsection{正定性:从数到矩阵,以及本世纪的矩阵论}

二次型在这份讲义当中尚未被置于中心地位. 往早期讲,它的研究缘起于费马大定理,而往晚近讲,二次型的代数理论研究几近一个完全的分支. 我们在此抛开Witt理论之类的东西不谈,暂且只就谈正定性及其与矩阵论的关系,对于二次空间(quadratic space)的研究,及二次型与二次互反律(quadratic reciprocity law)的关系等等主题,读者自可参考志村五郎(Goro Shimura)的书《二次型的算术》(\textit{Arithmetic of Quadratic Forms}).

首先考虑这个问题:如何判断一个多项式是非负的?从初中开始,我们就知道有配方法:如果能把一个多项式拆成一堆非负项的和,那么它当然就是非负的,这个方法在二次型正定性的讨论中也被多次用到. 很自然的,下面这个问题就被提出了:

\begin{quote}
    \kaishu
    是不是任意一个非负多项式都能被拆成一系列多项式的平方和?
\end{quote}

答案是否定的. 一个典型的反例在 Motzkin 1967 年出版的一本关于代数-几何不等式的书中给出:
\[ P(x, y, z) = x^6 + y^4z^2 + y^2z^4 -3x^2y^2z^2 \]

希尔伯特第十七问题就是这个问题的一个推广:
\begin{quote}
    \kaishu
    是不是任意一个非负多项式都能被拆成一系列有理函数的平方和?
\end{quote}

这样的推广终归是成立了,它的证明由 E. Artin 在 1927 年给出,C. N. Delzell 在 1984 年甚至给出了一个算法以构造这样的平方和. 但是,故事到这里还没有结束,把``数''推广到``矩阵'',有趣的事情发生了:

\begin{theorem}[Helton, 2002]\label{thm:16:helton2002}
    对称的矩阵多项式是半正定的,当且仅当它能被拆成一组矩阵多项式的平方和.
\end{theorem}

当然,我们需要澄清一些概念:

\begin{definition}
    一个 $n$ 元的矩阵单项式是一个形如 $am_1m_2 \cdots m_k$ 的连乘积,其中 $a \in \R$,$m_i$ 为矩阵变元 $x_j \enspace(j \leqslant n)$ 或其转置. 一个 $n$ 元的矩阵多项式是有限个 $n$ 元矩阵单项式的和.
\end{definition}

\begin{definition}
    一个 $n$ 元的矩阵多项式 $Q$ 的值域 $\im Q$ 被定义为

    \[ \bigcup_{m = 1}^\infty\left\{Q(A_1, A_2, \ldots, A_n) \mid A_i \in \mathbf{M}_m(\R)\right\} \]

    即在代入任意 $n$ 个 $m$ 阶方阵之后所能得到的结果.
\end{definition}

\begin{definition}
    称一个矩阵多项式 $Q$ 是对称的,如果所有 $\im Q$ 中的矩阵都是对称的;称它是半正定的,如果所有 $\im Q$ 中的矩阵都是半正定的.
\end{definition}

\begin{definition}
    所谓矩阵多项式的平方和,指的是如下形式的有限和:
    \[ P(x_1, x_2, \ldots, x_n) = \sum_{i = 1}^k h_i(x_1, x_2, \ldots, x_n)h_i^\mathrm{T}(x_1, x_2, \ldots, x_n) \]
    其中 $h_i$ 均为 $n$ 元矩阵多项式. 显然,这个平方和一定是半正定的.
\end{definition}

\autoref{thm:16:helton2002} 的证明已经超出了这份讲义所能探讨的范畴. 其原论文足有二十页,证明过程可谓相当详尽,有兴趣的读者可自行查阅.

下面我们再给出一个将数推广到矩阵的相关问题,它在某种意义上也与正定性有关,而且也是本世纪的成果. 最小二乘法在本讲义前面很早的地方就有介绍,同样将其推广到矩阵,就要求我们推广距离的概念:

\begin{definition}
    考虑 $n$ 阶正定矩阵 $A, B$,定义它们之间的迹度量距离(trace metric distance)为:
    \[ \delta(A, B) = \sqrt{\sum_{i = 1}^n \log^2\lambda_i(A^{-1}B)} \]
    其中 $\lambda_i(M)$ 表示矩阵 $M$ 的第 $i$ 个特征值.
\end{definition}

\begin{definition}
    $k$ 个矩阵 $A_1, A_2, \ldots, A_k$ 的 Karcher 均值(Karcher mean)指的是正定矩阵 $X$,使得 $X$ 到所有 $A_i$ 的迹度量距离的平方和最小,将其记作 $\sigma(A_1, A_2, \ldots, A_k)$
\end{definition}

这个概念由 Karcher 在 1973 年先引入到任意度量空间中,这里呈现的是 Moakher 在 2005 年将其引入到矩阵中的版本. 在 2006 年,Bhatia 和 Holbrook 合作的论文呈现了它的一系列性质,并给出了一个猜想. 这个猜想是对下面这个简单事实的推广:

\begin{lemma}
    考虑 $x_1, x_2, \ldots, x_n, y_1, y_2, \ldots, y_n \in \R$,且 $\forall i \leqslant n,\enspace x_i \leqslant y_i$. 记 $x, y$ 分别为使得
    \[ \sum_{i = 1}^n (x - x_i)^2, \quad \sum_{i = 1}^n (y - y_i)^2 \]
    取得极小值的 $x$ 和 $y$,则 $x \leqslant y$.
\end{lemma}

很显然,对吧?他们猜测,Karcher 均值也有这种美好的单调性,不过,首先要在正定矩阵上引进序关系:

\begin{definition}
    如果 $B - A$ 正定,则 $A \leqslant B$.
\end{definition}

\begin{theorem}[Bhatia-Holbrook, 2006, Lawson-Lim, 2011]
    考虑 $A_1, A_2, \ldots, A_n, B_1, B_2, \ldots, B_n$ 为 $m$ 阶正定矩阵,如果 $\forall i \leqslant n,\enspace A_i \leqslant B_i$,则
    \[ \sigma(A_1, A_2, \ldots, A_n) \leqslant \sigma(B_1, B_2, \ldots, B_n) \]
\end{theorem}

等等,为什么是定理?正如我们已经暗示的,这个结果由 Lawson 和 Lim 在 2011 年成功证明了,其证明用到了 Loewner-Heinz 全局非正曲率度量空间(Loewner-Heinz globally nonpositively curved metric spaces)的一些性质,事实上这就是正定矩阵集合全体的复杂性带来的.

提及特征值,我们不妨再看看另外一个不那么幸运的结果,它是一个被证伪的猜想.

\begin{definition}
    称一个方阵 $A$ 的谱半径(spectral radius)为其特征值的绝对值的最大者,记为 $\rho(A)$.
\end{definition}

这个定义是非常有用的,如果我们注意到以下两个定理:

\begin{theorem}
    对于可对角化的矩阵 $U$,以下式子成立:
    \[ \lVert Uv \rVert \leqslant \rho(U) \lVert v \rVert \]
\end{theorem}

这是谱定理的一个直接推论. 注意,如果 $U$ 不是可对角化的,那么这个式子不成立,读者可以自行尝试构造反例,应当不难.

\begin{theorem}
    \[ \rho(A_1A_2\cdots A_k) \leqslant \rho(A_1)\rho(A_2)\cdots \rho(A_k) \]
\end{theorem}

这是 Gelfand 公式的一个直接推论. 也就是说,如果我们限制了所有矩阵 $A_i$ 都是可对角化的(例如在许多物理过程中),可以通过谱半径的乘积来估计乘积的谱半径的上界,这对于很多连续进行线性变换的系统的估计都是非常关键的. 而且,下面这个定义也一样重要:

\begin{definition}[广义谱半径]
    设 $\Sigma$ 为有限个 $m$ 阶方阵的集合,称其广义谱半径(generalized spectral radius)为
    \[ \rho(\Sigma) = \varlimsup_{k \to \infty} \rho_k(\Sigma) \]
    其中
    \[ \rho_k(\Sigma) = \max\{\rho(A_1A_2 \cdots A_k) \mid A_i \in \Sigma,\enspace i \leqslant k\} \]
    $\varlimsup$ 意指上极限,即最大的聚点.
\end{definition}

这个定义的物理意义可以类似地理解:当我们有一大堆可能的线性变换的时候,进行趋近于无穷次的随机选取变换,最终结果的范数最大值很可能落在哪里. 当然,很容易看出来,对于任意的 $k$,都有 $\rho_k(\Sigma) \leqslant \rho(\Sigma)$. 这个倒霉猜想由 Lagarias 和 Wang 在 1995 年提出,被称为有限性猜想:

\begin{quote}
    \kaishu
    对于任意 $\Sigma$,存在 $k$ 使得 $\rho_k(\Sigma) = \rho(\Sigma)$.
\end{quote}

也就是说,广义谱范数可以在有限步内抵达. 但不幸的是,这个猜想在七年后被 Bousch 和 Mairesse 证伪了,证伪的方法事实上是直接构造了一个 $\Sigma$ 的例子. 但是,他们的构造在此如要呈现将花费过多的笔墨,因为他们完全是从迭代函数系统(iterative function system)出发完成构造的,要用到关于 Lyapunov 指数的一些知识. 当然,对于读完这份讲义的读者,这些内容并不会太困难,不妨找来他们的论文一探究竟.

顺便一提,证明对于绝对值最小的特征值来说类似的猜想不成立特别简单,构造的例子如下,读者不妨自行思考一下为什么.
\[
    \begin{bmatrix}
        2 & 0 \\ 0 & 1/2
    \end{bmatrix}, \quad \begin{bmatrix}
        1/3 & 0 \\ 0 & 3
    \end{bmatrix}
\]

\subsection{线性方程组的解:快一点,再快一点!}

这一节讨论的主要是如何求解一个线性方程组
\[ A\vec{x} = \vec{b} \]

当然,读者会说,高斯消元法嘛,第一讲就已经讲过了,如果 $A$ 可逆那就是 $\vec{x} = A^{-1}\vec{b}$ 呗. 也不算错,那如果 $A$ 干脆不是一个方阵呢?确切的解自然不存在,最小二乘解也由 Penrose-Moore 逆给出:

\begin{definition}
    考虑 $m \times n$ 矩阵 $A$,它的 Penrose-Moore 逆为一个 $n \times m$ 矩阵 $A^\dagger$ 满足以下条件:

    \begin{enumerate}
        \item $A A^\dagger A = A$

        \item $A^\dagger A A^\dagger = A^\dagger$

        \item $A^\dagger A$ 和 $AA^\dagger$ 都是对称(或者厄米)的
    \end{enumerate}

    这样的矩阵存在且唯一.
\end{definition}

当然,不难写出近似的精度的定义:

\begin{definition}
    称 $\widetilde{\vec{x}}$ 是 $A\vec{x} = \vec{b}$ 的一个精度为 $\varepsilon$ 的近似解,如果:
    \[ \lVert \widetilde{\vec{x}} - A^\dagger \vec{b} \rVert_A \leqslant \varepsilon \lVert A^\dagger \vec{b} \rVert_A \]
    其中
    \[ \lVert \vec{x} \rVert_A = \sqrt{\vec{x}^\mathrm{T}A\vec{x}} \]
    称为由 $A$ 诱导的范数.
\end{definition}

这是对于一般的矩阵的结果. 通常情况下,我们只要处理实对称矩阵,甚至还能伴随一个更强的条件:

\begin{definition} \index{ruozhuduijiao@弱主对角占优 (weakly diagonally dominant)}
    称矩阵 $(a_{ij})$ 是\term{弱主对角占优}的,如果
    \[ a_{ii} \geqslant \sum_{j \neq i} |a_{ij}| \]
    对于任意 $i$ 成立.
\end{definition}

实对称的主对角占优矩阵对应的线性方程组求解当然可以使用直接求逆的方法,但是由于在处理图论问题以及求解一些椭圆型偏微分方程时,它往往以非常大的规模出现,所以哪怕是非线性时间的算法,也显得比较缓慢了,而到达近线性的突破到现在还不到十年:

\begin{theorem}[Spielman-Teng, 2014]
    存在一个在任意精度 $\varepsilon$ 下求解实对称的主对角占优矩阵 $A$ 对应的方程组 $Ax = b$ 的随机算法,其时间复杂度为
    \[ O(m \log^c n \log(1 / \varepsilon)) \]
    其中 $m$ 为 $A$ 中的非零元个数,$c$ 为一常数.
\end{theorem}

我们无意在此详述这个算法如何运作. 事实上,稍有图论基础的读者应不难看懂原论文. 它的核心在于,使用好的图采样来稀疏化,进而优化求解.

\subsection{随机矩阵:吸引了陶哲轩的未解之谜}

既然提到随机算法,那不妨也介绍一下随机矩阵的相关研究. 这就不像前面几个领域在本世纪内的研究较为松散,反而是如火如荼,哪怕是将近十年的主要结果罗列一番也破费精力. 由于笔者对概率论不甚熟悉,也不曾深究这些结论的证明,在此,我们仅罗列一些结果,有些较新的结果的准确性可能需要明眼的读者自行甄别.

首先,何谓随机矩阵?我们这里只探讨最简单的模型:Bernoulli 矩阵.

\begin{definition}
    称一个矩阵为 Bernoulli 矩阵,如果它的每一个元素都是独立同分布、成功概率 $p = 1/2$ 的 Bernoulli 随机变量.
\end{definition}

第一个问题是,这玩意有多大概率是奇异的?简单观察一下,不难猜想,奇异性的来源也就是行或者列相同,因此,我们给出以下猜想:

\begin{theorem}[Tikhomirov, 2020] \label{thm:16:tik2020}
    一个 $n$ 阶 Bernoulli 矩阵奇异的概率为:
    \[ P_n = \left( \frac{1}{2} + o(1) \right)^n \]
\end{theorem}

这个结果的证明几经波折. 原始的猜想早在上世纪就被正式阐述过,早在 1967 年,Koml\'os 就已经表明,$\displaystyle\lim_{n \to \infty} P_n = 0$. 直到 1995 年,Kahn 等人才给出第一个指数级别的估计 $P_n = O(0.999^n)$. 在 2007 年,陶哲轩和 Van Vu 给出了 $P_n = O(0.958^n)$ 的估计并在第二年获得了如下突破:

\begin{theorem}[Tao-Vu, 2007]
    一个 $n$ 阶 Bernoulli 矩阵奇异的概率为:
    \[ P_n = \left( \frac{3}{4} + o(1) \right)^n \]
\end{theorem}

下一个突破性进展在 2010 年,Bourgain 等人进一步压下了上界:

\begin{theorem}[Bourgain, 2010]
    一个 $n$ 阶 Bernoulli 矩阵奇异的概率为:
    \[ P_n = \left( \frac{1}{\sqrt{2}} + o(1) \right)^n \]
\end{theorem}

最终形成了我们看到的\autoref{thm:16:tik2020}. 事实上,Tikhomirov 证明的结果比这个还要广一点:

\begin{theorem}[Tikhomirov, 2020]
    一个 $n$ 阶,成功概率为 $p \in (0, 1/2]$ 的 Bernoulli 矩阵奇异的概率为:
    \[ P_n = \left( 1 - p + o(1) \right)^n \]
\end{theorem}

Jain 等人在同年也给出了另一个关于任意有限支集的分布代替独立同分布的结果.

另外的问题还有:
\begin{enumerate}
    \item 对于高斯分布的矩阵,结果如何?

    \item 最大特征值,也就是谱半径的分布如何?(Tracy-Widom 规则及其推广)

    \item 所有特征值的分布如何?(Circular law conjecture,由陶哲轩在 2010 年证明)

    \item 范数和逆的范数的分布如何?(Spielman-Teng 猜想,2002 年被形式化提出,Littlewood-Offord 问题的反问题,陶哲轩和 Van Vu 在 2009 年给出了一个初步结果)

    \item 其它非独立同分布的随机模型如何?(关于随机带矩阵(random band matrix),Khorunzhiy 猜想在 2010 年被 Sodin 证明)
\end{enumerate}

\subsection{机器学习!}

当然咯,提到本世纪的线性代数,也不能不提及机器学习的一些发展. 我们仅就几个特定问题,讨论线性代数方法在机器学习中的应用. 这里提及的很多结论会比前面的结果更加轻松,亦可供对此感兴趣的读者仔细参详,甚至自行写出证明.

我们要谈的第一个问题叫做压缩感知(compressed sensing). 给定一个稀疏信号和有限次的测量,测量的次数很可能远少于要复原的变量的个数,这个时候,我们需要多少次测量才能以某个精度复原出原始信号?最初的结果来自于 Donoho 在 2006 年的一篇论文,在 Google 学术上,这篇论文被引次数在笔者写下这一小节时已经达到了惊人的 33115 次,很可能是本世纪引用次数最高的一篇数学论文. 我们下面来介绍一下这篇论文的主要结果,首先需要给出几个定义:

\begin{definition}
    一个向量 $x$ 的 $l_p$-范数定义为:
    \[ \lVert x \rVert_p = \left( \sum_{i}|x_i|^p \right)^{\frac{1}{p}} \]
    其中 $x_i$ 为其分量.
\end{definition}

\begin{definition}
    称一个向量是 $l_p$-稀疏的,如果它满足:
    \[ \lVert x \rVert_p < R \]
    其中 $R$ 为给定正常数,$0 < p \leqslant 1$.
\end{definition}

\section{未来:从线性代数出发能望到多远}

\begin{quote}

    \kaishu
    对于模或者其它具备某种线性操作的数学结构, 例如稍后要介绍的 Abel 范畴中的对象, 复形及其上同调的研究是通称为同调代数的学科的经典内容; 这是本真意义的``线性代数''的一个真子集, 也是本书核心.

    \begin{flushright}
        \kaishu
        ——李文威《代数学方法》卷二:线性代数
    \end{flushright}

\end{quote}

到底什么是线性代数?想必大部分读者都看过那张线性代数九宫格. 当然,手抓饼也可以是线性代数. 但是在此,为了使得我们的讨论不致过于冗长,准李文威老师的思路,我们以线性性作为基准,以此衡量何谓线性代数的发展. 一切事关线性性的探讨都可以被称为线性代数,而我们在这里仅取一瓢之水,希望读者窥一斑可见全豹.

\subsection{模作为线性空间的推广}

当然,我们说线性性,很显然的问题就是,什么样的东西是一个足以保线性性的东西?一个域上的线性空间,那当然可以. 它自身有 Abel 群结构,对于域的加法和数乘作用相容——等等,为什么一定要是一个域呢?域中的除法,以及域的交换性,这些东西看起来对线性性并没有多少意义. 如果你注意到了这一点,那么恭喜你,你已经能够写出模的定义了:

\begin{definition}
    置 $\langle R : +, \cdot \rangle$ 为一环,$\langle M : + \rangle$ 为一 Abel 群. 称 $M$ 为 $R$ 上的左模,如果我们能给它赋予一个映射 $R \times M \to M$ ,以左乘记,满足以下条件:

    \begin{itemize}
        \item $r(m_1 + m_2) = rm_1 + rm_2,\enspace \forall m_1, m_2 \in M,\enspace r \in R$

        \item $(r_1 + r_2)m = r_1m + r_2m,\enspace \forall m \in M,\enspace r_1, r_2 \in R$

        \item $(r_1r_2)m = r_1(r_2m),\enspace \forall m \in M,\enspace r_1, r_2 \in R$

        \item $1_Rm = m,\enspace \forall m \in M$
    \end{itemize}

    对称地,我们也可以定义右模. 事实上,右模无外乎反环 $R^{\mathrm{op}}$ 上的左模.
\end{definition}

对于模理论,深究起来亦破费一番功夫. 这里,我们只提示一个在笔者看来非常有用的性质:

\begin{theorem}[Freyd-Mitchel, 1964]
    任意小的 Abel 范畴都可以满、忠实且正合地嵌入到某个环 $R$ 的左模范畴中去.
\end{theorem}

深入介绍这个定理会让这一节变得相当复杂,或许需要单开一本书来讲. 但是,单从字面意义上讲,读者不难看出,这个意思就是,对 Abel 范畴的研究可以在某个环 $R$ 的左模范畴中完成,而事实上另有一个嵌入定理将另一个更广泛的结构嵌入到 Abel 范畴中:

\begin{theorem}[Quillen-Gabriel, 1972]
    任意一个小的 Quillen 正合范畴都可以被保正合地嵌入到 Abel 范畴中.
\end{theorem}

也就是说,$R$ 的左模范畴可以成为研究一类更广泛的结构的样板,这是其在范畴论中不可或缺的意义之一. 至于模自身的理论,也不可谓不庞大,而且在代数几何等领域的现代发展中发挥着重要的作用. 不过,为了让读者少听点天书,我们还是进入下一个主题吧. 如果读者有兴趣,自可参阅各种现代代数学发展相关的资料.

\subsection{张量积:一个过渡性章节}

张量积的构造其实无外乎线性映射的推广:

\begin{definition}
    置 $V, W, Z$ 为线性空间,称 $V \times W \to Z$ 是双线性的(bilinear),如果它关于 $V$ 和关于 $W$ 都是线性的.
\end{definition}

\begin{definition} \label{thm:16:tensorprod}
    $V$ 和 $W$ 的张量积 $V \otimes W$ 无外乎一种具备泛性质的双线性映射 $\varphi$,即满足对于任意双线性映射 $h$,存在唯一线性映射 $\widetilde h$ 使得下图交换:

    \begin{center}
        \begin{tikzcd}
            V \times W \ar[r, "\varphi"] \ar[rd, "h"] & V \otimes W \ar[d, dashed, "\widetilde h"] \\
            & Z
        \end{tikzcd}
    \end{center}

    也就是说,$\widetilde h \circ \varphi = h$.
\end{definition}

当然,为了确立这个定义的正确性,我们要保证这个空间 $V \otimes W$ 的唯一性. 事实上,它就相当于积空间商掉一些部分:

\begin{align*}
    R = \spa ( & (v_1 + v_2, w) - (v_1, w) - (v_2, w), \\
               & (v, w_1 + w_2) - (v, w_1) - (v, w_2), \\
               & (sv, w) - s(v, w),                    \\
               & (v, sw) - s(v, w))
\end{align*}

于是:
\[ V \otimes W \cong (V \times W) / R \]
不难验证这样给出的 $V \otimes W$ 满足\autoref{thm:16:tensorprod}.

\subsection{从代数到单子:往程序设计范式前进}

\subsection{线性化:群表示的艺术}

我们已经提及,线性性是一个非常好的性质. 对于一般的群,这种性质都是不可能存在的. 但是,如何给一个非线性的结构赋予线性结构呢?我们先看一个事实.

\begin{lemma}
    $\mathcal{L}(V)$ 关于复合操作构成一个群.
\end{lemma}

这个结果的验证应该颇为简单. 随后,我们自然想到,如果研究一个群到这个群的群同态,即那些保群乘法结构的映射,那么我们不久相当于把对群的研究线性化了吗?这样,我们就有了以下定义:

\begin{definition}
    从群 $G$ 到线性空间 $V$ 上的表示(representation)意指一个从 $G$ 到 $\mathcal{L}(V)$ 的群同态.
\end{definition}

通常地,我们会研究实表示和复表示,也就是把 $V$ 取为 $\R^n(\R)$ 和 $\C^n(\C)$. 我们将表示空间的维数称为表示的维数. 对表示论的研究很大程度上是对特征标的研究:

\begin{definition}
    考虑有限维表示 $\rho: G \to \mathcal{L}(V)$,$V$ 是域 $\mathbf{F}$ 上的向量空间,表示 $\rho$ 的特征标被定义为:
    \begin{align*}
        \chi_\rho \colon & G \to F                 \\
                         & g  \mapsto  \tr \rho(g)
    \end{align*}
\end{definition}

当然,对表示论的介绍也足以撑起一本巨著,在此,我们仅提示它在对群论研究中的一些意义,罗列一些用这种方式能够轻松证明,而用通常方法无法证明的结果:
\begin{itemize}
    \item Burnside 定理:所有 $p^aq^b$ 阶的群都是可解群. 这个结果就笔者所知尚未有不使用表示论做出的证明,事实上,表示论给出的证明相当简短.

    \item Feit-Thompson 定理:所有奇数阶群都是可解群. 这个结果的证明非常厚,笔者没看过,但是被其厚度震撼过.

    \item 有限单群分类定理,或称宏伟定理(the Grand theorem). 这是上世纪到本世纪群论最重要的结果之一,其主要部分证明就是应用了表示论引入的特征标理论.

    \item Ore 猜想:有限非 Abel 单群中的任意元素都是换位子. 这个猜想提出于 1951 年,在 2010 年,Liebeck 等人应用李型单群的 Deligne-Lusztig 不可约特征标理论将其完全攻克.
\end{itemize}

此外,表示论在许多物理学领域中发挥着关键性的作用. 一个典型的例子就是现代粒子物理学,其主要工具之一就是李群的表示论. 在量子物理中,哈密顿量的对称群的表示也能够引出多重态(multiplet)的概念,其中不可约的表示部分表征了系统的可能能级. 如果对于这方面的研究感兴趣,笔者(作为一个不怎么懂物理的人)推荐参考 GTM267,B. C. Hall 所著的《写给数学家的量子理论》(\textit{Quantum Theory for Mathematicians})第 17 章的内容.

这里我们另外指明很漂亮的一点,群的表示可以对应地用来建构多面体. 给定三维空间中的一个基向量 $v$,对其应用某有限群中各个元素的矩阵表示,我们可以很自然地得出一个多面体——通过这种方式,我们可以自然得到五种正多面体,它们是多面体点群的三维表示生成的.

如果读者对多面体足够熟悉,那么很自然的一个发现是,我们也可以取三个向量和原点出发构筑一个四面体,然后对这个四面体进行对称操作,即应用矩阵表示得到最终的结果. 事实上,在正多面体的情形下,我们得到的就是一个面的顶点、棱心、面心和原点构成的四面体. 因此,如果将其进行推广,取任意一个它的变形,都可以得到一个具备对应对称性的多面体. 这种方式往往也是计算机中表示一个具备某种对称性的多面体的方法,它的好处是可以利用群表示的方式节省所需的存储空间.

\subsection{拓扑向量空间:从布尔巴基学派的遗产走出}

\subsection{仿射簇,以及代数几何的问题}

\section*{附:本讲义未竟专题概览}

\section*{参考资料}

\begin{enumerate}
    \item \href{https://zh.wikipedia.org/wiki/%E4%BB%A3%E6%95%B0}{维基百科:代数}

    \item \href{https://zhuanlan.zhihu.com/p/574858845}{知乎:代数发展史}

    \item
\end{enumerate}
\vspace{2ex}
\centerline{\heiti \Large 内容总结}

\vspace{2ex}
\centerline{\heiti \Large 习题}

\vspace{2ex}
{\kaishu 如果我们想要预见数学的将来,适当的途径是研究这门科学的历史和现状.}
\begin{flushright}
    \kaishu
    ——庞加莱
\end{flushright}

\centerline{\heiti A组}
\begin{enumerate}
    \item
\end{enumerate}

\centerline{\heiti B组}
\begin{enumerate}
    \item
\end{enumerate}

\centerline{\heiti C组}
\begin{enumerate}
    \item
\end{enumerate}

\input{./专题/17 多项式.tex}
\begingroup
\SetLUChapterNumberingStyle{5}
\def\theHchapter{\arabic{chapter}ε}

\chapter{纽结的多项式不变量}

\ResetChapterNumberingStyle{17}
\endgroup

\input{./专题/18 不变子空间.tex}
\begingroup
\SetLUChapterNumberingStyle{6}
\def\theHchapter{\arabic{chapter}ε}

\chapter{控制法及其在矩阵理论中的应用}

\ResetChapterNumberingStyle{18}
\endgroup

\input{./专题/19 相似标准形(I).tex}
\input{./专题/20 相似标准形(II).tex}
\input{./专题/21 多项式的进一步讨论.tex}
\input{./专题/22 若当标准形.tex}
\begingroup
\SetLUChapterNumberingStyle{7}
\def\theHchapter{\arabic{chapter}ε}

\chapter{对称多项式和Young图}

\ResetChapterNumberingStyle{22}
\endgroup

\input{./专题/23 内积空间.tex}
\input{./专题/24 内积空间上的算子(I).tex}
\input{./专题/25 内积空间上的算子(II).tex}
\begingroup
\SetLUChapterNumberingStyle{8}
\def\theHchapter{\arabic{chapter}ε}

\chapter{希尔伯特空间引论}

\ResetChapterNumberingStyle{25}
\endgroup

\chapter{线性代数与解析几何基础}

解析几何很大程度上是线性代数发展的初衷,在研究点线面以及几何体时,将集体的几何问题抽象化为代数问题使其方便解决与计算,即是解析几何的主要思想. 本节我们将会从线性代数的角度探究解析几何的一些基本概念与方法. 此在线性代数课程的考察中也会有少部分的解析几何内容,但内容较浅,主要考察点、直线、平面等之间的关系.

\section{欧几里得空间}

在前面的学习中我们已经较为全面地学习了内积空间的相关知识,而在解析几何中,我们在更多情况下会研究\term{欧几里得空间}\index{oujilidekongjian@欧几里得空间 (Euclidean space)}下的问题.
\begin{definition}[欧几里得空间]
    欧几里得空间(欧氏空间)是一个有限维实内积空间.
\end{definition}
同学们可能对欧氏空间的几何直观更为熟悉. 当欧氏空间的维数为 2 或 3 时,我们可以用熟悉的平面直角坐标系与空间直角坐标系来描述欧氏空间中的向量,并用点积作为向量的内积.

\section{欧氏空间上的运算}

我们也已经基本掌握了模、内积、夹角等在内积空间中的基本概念,在此我们引入一些在先前的学习中接触较少的概念.
\begin{definition}[点积] \index{dianji@点积 (dot product)}
    \term{点积}是在三维欧氏空间中对两个向量的运算,用$\vec{a}\cdot\vec{b}$表示. 两向量点积得到的数值等于两向量模长的乘积与两向量夹角的余弦的乘积.
\end{definition}
特别的,三维欧氏空间中的向量点积$(a_1,a_2,a_3)\cdot(b_1,b_2,b_3)$可以表示为\[a_1b_1+a_2b_2+a_3b_3\]
由点积的计算,我们可以很方便地得到两向量夹角的余弦,即\[\cos\theta=\frac{\vec{a}\cdot\vec{b}}{|\vec{a}||\vec{b}|}\]
\begin{definition}[叉乘] \index{chacheng@叉乘 (cross product)}
    \term{叉乘}是在三维欧氏空间中对两个向量的运算,用$\vec{a}\times\vec{b}$表示. 两向量叉乘得到的向量垂直于两向量,方向遵循右手定则,其模长为两向量的模的乘积与两向量夹角的正弦的乘积.
\end{definition}
由定义可知,叉乘仅在三维欧氏空间中有定义,且叉乘的结果是一个向量,而不是一个数. 关于叉乘向量的计算有另一种更常用的用行列式表示的计算方法,即
\[(a_1,a_2,a_3)\times(b_1,b_2,b_3)=\begin{vmatrix}
        \vec{i} & \vec{j} & \vec{k} \\
        a_1     & a_2     & a_3     \\
        b_1     & b_2     & b_3
    \end{vmatrix}\]
其中$\vec{i},\vec{j},\vec{k}$为三维欧氏空间的自然基.

在解析几何中,叉乘的一个重要应用是求解与两向量垂直的向量.
\begin{definition}[混合积] \index{hunheji@混合积 (mixed product)} \index{biaoliangsancongji@标量三重积 (scalar triple product)}
    \term{混合积}(或称\term{标量三重积},不同于\term{矢量三重积})是三维欧氏空间中对三个向量的运算,用$[\vec{a},\vec{b},\vec{c}]$表示,等价于$(\vec{a}\times\vec{b})\cdot\vec{c}$.
\end{definition}
混合积的几何意义是以$\vec{a},\vec{b},\vec{c}$为邻边的平行六面体的体积,可以用行列式表示为
\[[(a_1,a_2,a_3),(b_1,b_2,b_3),(c_1,c_2,c_3)]=\begin{vmatrix}
        a_1 & a_2 & a_3 \\
        b_1 & b_2 & b_3 \\
        c_1 & c_2 & c_3
    \end{vmatrix}\]
同时读者也不难验证 $ (\vec{a}\times\vec{b})\cdot\vec{c} = \vec{a}\cdot(\vec{b}\times\vec{c}) $. 其应用之一是可以用来判断三个向量是否共面.

\section{点、直线、平面的表示}

一个点在欧氏空间中可以用一个向量来表示. 在三维欧氏空间中,我们可以用三个实数来表示一个点的坐标.

\subsection{平面的方程}

平面是欧氏空间中的一个基本几何对象,我们有多种代数方法来表示平面.

平面的一般方程是平面的一种最基本的表示方法,即$Ax+By+Cz+D=0$. 平面的一般方程十分简洁,但是我们很难由此方程得到平面的几何性质,因此我们还需要考虑其他的表示方法. 例如,一个平面由平面上一点与平面上两个不共线的向量来表示. 假设已知平面上一点$P(x_0,y_0,z_0)$和平面上两个不共线的向量$\vec{u}=(a,b,c)$和$\vec{v}=(d,e,f)$,则平面上的任意一点$Q(x,y,z)$都满足$\overrightarrow{PQ}$与$\vec{u}$和$\vec{v}$线性相关,即
\[\overrightarrow{PQ}=k_1\vec{u}+k_2\vec{v}\]
化为坐标形式即为
\[\begin{cases}
        x=x_0+k_1a+k_2d \\
        y=y_0+k_1b+k_2e \\
        z=z_0+k_1c+k_2f
    \end{cases}\]
这就是平面的参数方程,其中$k_1,k_2$是参数.

此外,平面还可以由平面上一点和平面的法向量来表示. 假设已知平面上一点$P(x_0,y_0,z_0)$和平面的法向量$\vec{n}=(A,B,C)$,则平面上的任意一点$Q(x,y,z)$都满足向量$\overrightarrow{PQ}$与$\vec{n}$垂直,即点积为$0$. 由此可得其方程为\[A(x-x_0)+B(y-y_0)+C(z-z_0)=0\]这种表示方法称为\term{点法式}.

我们发现这跟平面的一般方程十分相似,实际上,我们可以直接通过平面的一般方程得到平面的法向量.

在得到一张由其他方式表示的平面时,我们往往也会将其转化为一般式或点法式,以便于我们计算其与其他几何对象的关系. 例如,得到一个由平面上一点与平面上两不共线的向量表示的平面,则可以通过求两向量的叉积得到平面的法向量,从而得到平面的点法式.

\begin{example}
    若已知一个平面上有三点$A(1,2,0),\enspace B(0,1,-1),\enspace C(1,1,1)$,求该平面的一般方程.
\end{example}

\subsection{直线的方程}

直线在欧氏空间中也是一个基本对象,同样有多种代数方法可以表示直线.

首先直线可以用某两张平面的交表示. 假设有两相交平面的方程,联立可得直线方程
\[\begin{cases}
        A_1x+B_1y+C_1z+D_1=0 \\
        A_2x+B_2y+C_2z+D_2=0
    \end{cases}\]
即为直线的一般方程. 这种联立方程的表示方法最为基本,但是不够简洁,大多情况下也不够直观. 所以更多情况下我们希望在表示中可以直观体现直线的一些特征. 因此,可以用直线上的一个点和直线的方向(即方向向量)来确定一条直线.

假设已知直线上的一点$A_0(x_0,y_0,z_0)$和直线的方向向量$\vec{l}=(a,b,c)$,则直线上的任意一点$A(x,y,z)$都满足$\overrightarrow{AA_0}$与$\vec{l}$平行,用具体的方程则表示为
\[\frac{x-x_0}{a}=\frac{y-y_0}{b}=\frac{z-z_0}{c}\]
其中$a,b,c$不为零. 这种表示方法称为\term{点向式}.

如果我们对上述式子进行替换,令\[t=\frac{x-x_0}{a}=\frac{y-y_0}{b}=\frac{z-z_0}{c}\]
则可得
\[\begin{cases}
        x=x_0+at \\
        y=y_0+bt \\
        z=z_0+ct
    \end{cases}\]
这样就得到了直线的参数方程,其中$t$为参数.

当然还有以两点确定一条直线的表示方法,我们可以轻松地算出直线的方向向量,然后用点向式或参数方程来表示. 最后可以得出方程
\[\frac{x-x_1}{x_2-x_1}=\frac{y-y_1}{y_2-y_1}=\frac{z-z_1}{z_2-z_1}\]

那么如何实现从一般方程到点向式或参数方程的转换呢?最简单的方法是求解线性方程组再用两点表示或者参数表示,但是这样的方法比较麻烦,事实上我们可以利用法向量进行转换. 假设两平面的一般方程为$A_1x+B_1y+C_1z+D_1=0$与$A_2x+B_2y+C_2z+D_2=0$,则可以得到两平面的法向量分别为$\vec{n}_1=(A_1,B_1,C_1),\enspace\vec{n}_2=(A_2,B_2,C_2)$,因为该直线在两张平面内,所以直线与两个法向量都垂直,所以$\vec{n}_1\times\vec{n}_2$即为直线的方向向量. 再求出一般方程的一个解(即直线上一点)即可得到直线的点向式与参数方程.

\section{平面与直线间的位置关系}

对于三维欧氏空间中的几何对象,我们主要需要研究平行、相交与重合等关系. 我们可以通过平面与直线的方程来判断.

\subsection{线与线的位置关系}

线与线之间的位置关系判断主要依靠它们的方向向量. 如果两条直线的方向向量平行,则两条直线平行或重合,此时再判断两直线是否存在公共点,若联立方程有解,说明两直线重合,否则两条直线平行. 如果两条直线的方向向量不平行,则还需要判断两条直线是否共面,若共面则说明两条直线相交,否则两条直线异面. 此时以两直线方程联立方程组,若有解则说明存在交点,否则说明两条直线异面.

\begin{example}
    已知直线$L_1=\begin{cases}
            x+y+z-1=0 \\
            x-2y+2=0
        \end{cases},\enspace L_2=\begin{cases}
            x=2t  \\
            y=t+a \\
            z=bt+1
        \end{cases}$,试确定$a,b$的值使得$L_1,L_2$是:
    \begin{enumerate}
        \item 平行直线;

        \item 异面直线.
    \end{enumerate}
\end{example}

\subsection{线与面的位置关系}

线与面的位置关系首先需要判断线的方向向量与平面的法向量的关系. 如果方向向量与法向量平行,则说明线与面垂直. 如果两者垂直,则说明该直线与平面平行或者在平面内,只需再判断直线上的点是否在平面内即可.

此外还有一些对于平面不同表示形式的方法. 例如,假设已知直线的方向向量与平面上两个不平行的向量,则可以对这三个向量做混合积,如果混合积为零,则说明三个向量共面,即直线与平面平行或者在平面内.

\subsection{面与面的位置关系}

面与面的位置关系主要依靠两个平面的法向量来判断. 如果两个平面的法向量平行,则说明两个平面平行或重合,再判断两平面是否存在公共点. 若两法向量垂直,则两平面也垂直.

\vspace{2ex}
\centerline{\heiti \Large 内容总结}

这里关于解析几何的部分浅尝辄止,只是简单地介绍了一些基本的概念与方法,希望能够帮助大家对解析几何有一个简单的初步认识. 在线性代数课程中可能的相关考察基本也仅限于点、线、面之间的关系,方程的联立、求解等等,或许大家在未来其他课程的学习中可以学到更多相关的知识.

\vspace{2ex}
\centerline{\heiti \Large 习题}

\vspace{2ex}
{\kaishu 我决心放弃那个仅仅是抽象的几何. 这就是说,不再去考虑那些仅仅是用来练思想的问题. 我这样做,是为了研究另一种几何,即目的在于解释自然现象的几何.}
\begin{flushright}
    \kaishu
    ——笛卡尔
\end{flushright}

\centerline{\heiti A组}
\begin{enumerate}
    \item
\end{enumerate}

\centerline{\heiti B组}
\begin{enumerate}
    \item
\end{enumerate}

\centerline{\heiti C组}
\begin{enumerate}
    \item
\end{enumerate}

\begingroup
\SetLUChapterNumberingStyle{9}
\def\theHchapter{\arabic{chapter}ε}

\chapter{射影几何的代数方法}

\ResetChapterNumberingStyle{26}
\endgroup

\chapter{二次型}

\section{二次型的定义}

\begin{definition}[二次型] \index{ercixing@二次型 (quadratic form)}
    $n$个元$x_1,x_2,\ldots,x_n$的二次齐次多项式
    \begin{align*}
        f(x_1,x_2,\ldots,x_n) & = \sum_{i=1}^{n}a_{ii}x_i^2+\sum\limits_{1\leqslant i<j\leqslant n}2a_{ij}x_ix_j    \\
                              & = a_{11}x_1^2+a_{22}x_2^2+\cdots+a_{nn}x_n^2                                        \\
                              & \quad +2a_{12}x_1x_2+\cdots+2a_{1n}x_1x_n+2a_{23}x_2x_3+\cdots+2a_{n-1,n}x_{n-1}x_n
    \end{align*}
    称为数域$\mathbf{F}$上的$n$元二次型(简称\term{二次型}).
\end{definition}
本学期研究的主要是实二次型. 若令$a_{ij}=a_{ji}\enspace(1\leqslant i<j\leqslant n)$,则二次型可表示为
\[f(x_1,x_2,\ldots,x_n)=\sum_{i=1}^{n}\sum_{j=1}^{n}a_{ij}x_ix_j=X^\mathrm{T}AX\]
其中$X=(x_1,x_2,\ldots,x_n)^\mathrm{T}\in\mathbf{R}^n$,$A=(a_{ij})_{n\times n}$为实对称矩阵,并称对称矩阵$A$为二次型$f(x_1,x_2,\ldots,x_n)$的矩阵.

注意,二次型实际上是一个$\mathbf{R}^n\to\mathbf{R}$的函数,所以本质上代入$x_1,\ldots,x_n$后就是一个实数,写成矩阵形式我们也可以发现矩阵相乘结果为$1\times 1$矩阵,即一个实数,因此不必把二次型想得过于复杂.

同时需要注意,二次型对应矩阵一定是对称矩阵. 实际上一个形如$f(x_1,x_2,\ldots,x_n)=\displaystyle\sum_{i=1}^{n}\displaystyle\sum_{j=1}^{n}a_{ij}x_ix_j$的函数可以对应的矩阵是很多的,但我们要求$a_{ij}=a_{ji}$才能得到二次型对应的矩阵.
\begin{example}
    已知二次型
    \[f(X)=(x_1,x_2,x_3,x_4)\begin{pmatrix}
            1 & 2 & 3 & -4 \\ 3 & 2 & 1 & 4 \\ -4 & 3 & -7 & 2 \\ 0 & -6 & 8 & 4
        \end{pmatrix}\begin{pmatrix}
            x_1 \\ x_2 \\ x_3 \\ x_4
        \end{pmatrix}\]
    写出二次型$f(X)$的矩阵.
\end{example}

\begin{example}
    回答以下问题:
    \begin{enumerate}
        \item 已知$A$是一个$n$阶矩阵,则$A$为反对称矩阵的充要条件是对任意$n$元列向量$X$都有$X^\mathrm{T}AX=0$;

        \item 若二次型$f(x_1,x_2,\ldots,x_n)=X^\mathrm{T}AX$对任意$n$元列向量$X$都有$f(x_1,x_2,\ldots,x_n)=0$,证明:$A=O$;

        \item 设二次型$f(x_1,x_2,\ldots,x_n)=X^\mathrm{T}AX,\enspace g(x_1,x_2,\ldots,x_n)=X^\mathrm{T}BX$.\\
              证明:若$f(x_1,x_2,\ldots,x_n)=g(x_1,x_2,\ldots,x_n)$,则$A=B$.
    \end{enumerate}
\end{example}

\section{矩阵相合的定义与性质}

\begin{definition}
    我们称$n$阶矩阵$A$相合于$B$(记作$A\simeq B$),如果存在可逆矩阵$C$使得$B=C^\mathrm{T}AC$.
\end{definition}
矩阵相合(合同)有如下基本性质:
\begin{enumerate}
    \item 合同是等价关系;合同不同于相似,是与数域有关的;合同要求$C$必须可逆,因此是一种特殊的相抵;

    \item $A\simeq B$一般不能得到$A^m\simeq B^m$(但是$A,B$为实对称矩阵时可以),但如果可逆,我们有$A^{-1}\simeq B^{-1}$,同时如果$A_1\simeq A_2,B_1\simeq B_2$,则有$\begin{pmatrix}
                  A_1 & O \\ O & B_1
              \end{pmatrix}\simeq\begin{pmatrix}
                  A_2 & O \\ O & B_2
              \end{pmatrix}$;

    \item $A\simeq B$表明$A$可以每次做相同的初等行列变换得到$B$,反之亦然. 这实际上就是初等变换法求相合标准形的基本原理,详见教材260页小字部分,感兴趣同学可以了解,一般不会要求使用这一方法.
\end{enumerate}

\begin{example}
    设$A\simeq B$,$C\simeq D$,且它们都是$n$阶实对称矩阵,问:$A+C\simeq B+D$ 是否成立.
\end{example}

\begin{example}
    判断:矩阵相似是否一定合同?矩阵合同是否一定相似?对于实对称矩阵上述论断又是否正确呢?正确请说明理由,不正确请举出反例.
\end{example}
实际上,教材中引入合同与二次型使用了双线性函数这一概念,实际上与双线性函数的度量矩阵有关,感兴趣的同学可以了解,但这部分属于小字,考试一般不做考查要求.

% TODO 小字介绍

% 小字介绍

\section{二次型标准形的定义与求解}

实际上二次型可以视为一个空间曲线/曲面方程,我们希望这些方程化为标准形式,有助于我们讨论一些问题. 由于实二次型对应矩阵为实对称矩阵,实对称矩阵一定可以相似对角化,故有下面的定理:
\begin{theorem}
    任意二次型$f(X)=X^\mathrm{T}AX$总可以通过可逆的线性变换$X=PY$(其中$P$可逆)化为标准形,即$f(X)=X^\mathrm{T}AX\xlongequal{X=PY}Y^\mathrm{T}(P^\mathrm{T}AP)Y=d_1y_1^2+d_2y_2^2+\cdots+d_ny_n^2$.
\end{theorem}
一般而言,我们有三种方法求解二次型标准形,分别为正交变换法,配方法和初等变换法. 正交变换法由于涉及正交因此不作要求,初等变换法之前已经提及并且较为复杂,不推荐优先使用. 因此我们接下来主要使用配方法.

注意,求二次型标准形不应使用之前求相似标准形的一般方法,因为只有正交矩阵才能保证$P^{-1}=P^\mathrm{T}$,一般矩阵无法保证. 当然实际上求得的对角矩阵都是由特征值按重数排列而成的,只是矩阵$P$不合要求,应当做 Schmidt 正交化.

配方法的思想非常简单,就是利用配方消除混合乘积项,将二次型表示成几个平方和的形式,最后通过坐标变换$X=CY$(又称仿射变换,其中$C$可逆)化标准形.
\begin{example}
    用配方法把三元二次型
    \[f(x_1,x_2,x_3)=2x_1^2+3x_2^2+x_3^2+4x_1x_2-4x_1x_3-8x_2x_3\]
    化为标准形,并求所用的坐标变换$X=CY$即变换矩阵$C$.
\end{example}
配方法是合理的,因为$X=CY$,其中$C$可逆,则$X^\mathrm{T}AX=Y^\mathrm{T}(C^\mathrm{T}AC)Y$,配方法使得$C^\mathrm{T}AC$为对角矩阵,因此可以得到相合标准形. 但是这种方法不能用来求相似对角化,原因仍然是$C^{-1}=C^\mathrm{T}$需要$C$为正交矩阵,但坐标变换矩阵不一定满足. 所以一定要区分好求解相似、相合标准形使用的方法,不能因为题目经常给的是实对称矩阵而混淆,只有正交变换法是通用的,因为正交矩阵满足$P^{-1}=P^\mathrm{T}$使得相似、相合的定义统一.

注意:有的同学可能知道正交变换法的具体操作流程,如果能保证计算正确且题目不强制配方法时可以使用,但是历年考试经常出现部分题目求解特征值时三次方程解不出的情况,此时一定要立刻醒悟,转向配方法解决问题.

\section{相合规范形 \quad 惯性定理}

事实上,一个二次型通过正交变换标准化得到的对角矩阵对角线上元素为特征值按重数排列的结果,但是使用配方法、初等变换法则不一定,甚至配方方式或者初等变换顺序不同都会产生不同的对角矩阵,因此相合标准形不唯一. 但我们知道,相抵标准形唯一,相似标准形不考虑排列组合因素也是唯一的,因此我们也需要统一相合标准形.

我们不难发现,任一对角矩阵一定相合于$\diag(1,\ldots,1,-1,\ldots,-1,0,\ldots,0)$(我们很容易写出对应的可逆变换矩阵),我们称这一相合标准形为相合规范形,其中$+1$的个数称为矩阵的正惯性指数,$-1$的个数称为矩阵的负惯性指数. 并且由于变换矩阵可逆,根据相抵标准形的结论,我们有原矩阵$A$的秩$r(A)$等于这一对角矩阵的秩,于是也等于正负惯性指数之和. 显然,$A$可逆时,其相合规范形主对角元没有0.

但我们没有说明一个矩阵的相合规范形是否唯一,实际上这就是下面惯性定理的结果:
\begin{theorem}[惯性定理]
    实对称矩阵的相合规范形唯一.
\end{theorem}
这一定理有很多等价表述,例如实对称矩阵正、负惯性指数唯一,或者实对称矩阵相合标准形中对角线上正、负、零的个数唯一. 或者实对称矩阵特征值中正、负、零的个数唯一等. 这一定理的证明方法比较经典,最关键的一步在于代入数值导出矛盾. 代入的方法是在两种表达的正负号分界线前后分别置0,使得两种表达形式一个大于0,一个小于等于0.
\begin{example}
    解答如下问题:
    \begin{enumerate}
        \item 设$n$元二次型$f(x_1,x_2,\ldots,x_n)=l_1^2+\cdots+l_p^2-l_{p+1}^2-\cdots-l_{p+q}^2$,其中$l_i\enspace (i=1,2,\ldots,p+q)$是关于$x_1,x_2,\ldots,x_n$的一次齐次式. 证明:$f(x_1,x_2,\ldots,x_n)$的正惯性指数$\leqslant p$,负惯性指数$\leqslant q$;

        \item 已知$A$为$m$阶实对称矩阵,$C$为$m\times n$实矩阵,证明:$C^\mathrm{T}AC$的正负惯性指数分别小于等于$A$的正负惯性指数.
    \end{enumerate}
\end{example}

\begin{example}
    确定二次型$f(x_1,x_2,\ldots,x_{10})=x_1x_2+x_3x_4+x_5x_6+x_7x_8+x_9x_{10}$的秩以及正、负惯性指数.
\end{example}

惯性定理的``惯性''二字与物理中的惯性有关,实际上透露着某种不变性. 根据惯性定理,我们有如下结论:
\begin{enumerate}
    \item 我们可以按相合关系对全体$n$阶实对称矩阵分类,因为实对称矩阵相合意味着规范形唯一,我们可以按照$+1$、$-1$、0个数的不同划分为$\vphantom{\cfrac{n+1}{2}}\dfrac{(n+1)(n+2)}{2}$个等价类(相抵、相似也是等价关系,可以思考划分等价类的方式与个数);

    \item 实数域上两个实对称矩阵相合的充要条件是它们有相同的正负惯性指数,两个对角矩阵相合的充要条件是对角线上正、负、零个数相同.
\end{enumerate}
注:复数域上两个对称矩阵相合的充要条件是它们的秩相同(可以思考其证明),例如$E_n$和$-E_n$在复数域上相合,但实数域上不相合.
\begin{example}
    设$A=\begin{pmatrix}
            1 & 2 & 0 \\ 2 & 1 & 0 \\ 0 & 0 & 3
        \end{pmatrix},\enspace B=\begin{pmatrix}
            -2 & 0 & 0 \\ 0 & 2 & 1 \\ 0 & 1 & 2
        \end{pmatrix}$,判断$A$与$B$是否相合.
\end{example}

\section{标准形的应用}

我们在本学期讨论了三种标准形,即相抵标准形,相似标准形和相合标准形,实际上它们之间的关系我们已经讨论,即相似一定相抵,相合一定相抵,但相似和相合互相没有包含关系. 本节我们考虑一些基于矩阵分解的问题,利用之前所学的相抵标准形、相似标准形、相合标准形的分解解决一些问题. 本节内容可以选择性掌握.

首先看一个关于幂等矩阵的例题,需要用到相抵标准形、相似标准形的分解:
\begin{example}
    解答以下两个问题:
    \begin{enumerate}
        \item 证明:任意一个方阵都可以分解成一个可逆矩阵和一个幂等矩阵的乘积;

        \item 已知$A$是一个秩为$r$的$n$级非零矩阵,证明:$A$为幂等矩阵的充要条件是存在列满秩的$n\times r$矩阵$B$和行满秩的$r\times n$矩阵$C$使得$A=BC$且$CB=E_r$.
    \end{enumerate}
\end{example}
下面是一个利用相合标准形进行分解的例子:
\begin{example}
    (与正交有关)证明:每个秩为$r$的$n\enspace(r<n)$阶实对称矩阵均可表示为$n-r$个秩为$n-1$的实对称矩阵的乘积.
\end{example}

\vspace{2ex}
\centerline{\heiti \Large 内容总结}

\vspace{2ex}
\centerline{\heiti \Large 习题}

\vspace{2ex}
{\kaishu }
\begin{flushright}
    \kaishu

\end{flushright}

\centerline{\heiti A组}
\begin{enumerate}
    \item
\end{enumerate}

\centerline{\heiti B组}
\begin{enumerate}
    \item
\end{enumerate}

\centerline{\heiti C组}
\begin{enumerate}
    \item
\end{enumerate}

\begingroup
\SetLUChapterNumberingStyle{10}
\def\theHchapter{\arabic{chapter}ε}

\chapter{有限域上的二次型}

\ResetChapterNumberingStyle{27}
\endgroup

\begingroup
\SetLUChapterNumberingStyle{11}
\def\theHchapter{\arabic{chapter}ε}

\chapter{二次型的几何}

\ResetChapterNumberingStyle{27}
\endgroup

\chapter{极分解与奇异值分解}

让我们稍微复习一下关于算子和数的类比. 自伴算子类似于实数,正算子类似于非负数,等距同构类似于模为 1 的复数. 那么我们的野心不止于此,在复数域的时候我们希望使用我们已经熟悉的数去描述所有复数. 同理,在算子上我们希望使用我们已经熟悉的算子去描述其他一般的算子.

% TODO section

我们还是从数开始出发. 每个非零复数 $ z $ 都可以写成如下形式
\[ z = \left(\frac{z}{\lvert z \rvert}\right)\lvert z \rvert = \left(\frac{z}{\lvert z \rvert}\right)\sqrt{\overline{z}z} \]
其中 $ \dfrac{z}{\lvert z \rvert} $ 是一个单位复数. 那么由此我们可以类比得出一个对 $ V $ 上任意算子的漂亮的描述.

\begin{theorem}[极分解定理] \index{jifenjiedingli@极分解定理 (polar decomposition theorem)}
    设 $ T \in \mathcal{L}(V) $,则存在一个等距同构 $ S \in \mathcal{L}(V) $,使得 $ T = S\sqrt{T^*T} $.
\end{theorem}

它之所以叫这个名字就是因为其形式类似于复数极坐标分解. 以下是证明.

\begin{proof}
    首先 $ \forall T \in \mathcal{L}(V) $,$ T^{*}T $ 都是正算子,所以 $ \sqrt{T^*T} $ 的定义是合理的.

    然后 $ \forall v \in V $,有
    \begin{align*}
        \lVert Tv \rVert^2 = \langle Tv, Tv \rangle & = \langle T^*Tv, v \rangle                   \\
                                                    & = \langle \sqrt{T^*T}\sqrt{T^*T}v, v \rangle \\
                                                    & = \langle \sqrt{T^*T}v, \sqrt{T^*T}v \rangle \\
                                                    & = \lVert \sqrt{T^*T}v \rVert^2.
    \end{align*}

    所以 $ \forall v \in V ,\enspace \lVert Tv \rVert = \lVert \sqrt{T^*T}v \rVert $.

    由此我们定义一个线性映射 $ S_1: \im \sqrt{T^*T} \rightarrow \im T $ 为
    \[ S_1(\sqrt{T^*T}v) = Tv. \]

    接下来要做的就是把这个 $ S_1 $ 扩张成一个等距同构 $ S \in \mathcal{L}(V) $使得 $ T = S\sqrt{T^*T} $. 不过在这之前得先验证 $ S_1 $ 是良定义的.

    设 $ v_1, v_2 \in V $ 使得 $ \sqrt{T^*T}v_1 = \sqrt{T^*T}v_2 $,则
    \begin{align*}
        \lVert Tv_1 - Tv_2 \rVert & = \lVert T(v_1 - v_2) \rVert                    \\
                                  & = \lVert \sqrt{T^*T}(v_1 - v_2)\rVert           \\
                                  & = \lVert \sqrt{T^*T}v_1 - \sqrt{T^*T}x_2 \rVert \\
                                  & = 0,
    \end{align*}

    得出 $ Tv_1 = Tv_2 $,所以 $ S_1 $ 是良定义的. 线性性结合范数便可以验证,不多赘述.

    从而 $ \forall u \in \im \sqrt{T^*T} $ 有 $ \lVert S_1u \rVert = \lVert u \rVert $.

    特别地,$ S_1 $ 是单射,由线性映射基本定理可得
    \[ \dim \enspace \im \sqrt{T^*T} = \dim \enspace \im T. \]
    而这也代表着 $ \dim(\im \sqrt{T^*T})^{\perp} = \dim(\im T)^{\perp} $. 那么就可以取 $ (\im \sqrt{T^*T})^{\perp} $ 的一组标准正交基 $ e_1, \ldots , e_m $ 和$ (\im T)^{\perp} $ 的一组标准正交基 $ f_1, \ldots, f_m $. 因为两个空间的维数相同,所以这两组标准正交基的长度相同(记为 $ m $).

    由此定义线性映射 $ S_2: (\im \sqrt{T^*T})^{\perp} \rightarrow (\im T)^{\perp} $ 为
    \[ S_2(a_1e_1 + \cdots + a_me_m) = a_1f_1 + \cdots + a_mf_m. \]
    所以 $ \forall w \in (\im \sqrt{T^*T})^{\perp} $ 均有 $ \lVert S_2w \rVert = \lVert w \rVert $.

    到这里我们想要的等距同构 $ S $ 就已经呼之欲出了. $ \forall v \in V, v = u + w $,其中 $ u \in \im \sqrt{T^*T} $,$ w \in (\im \sqrt{T^*T})^{\perp} $,将 $ S $ 定义为
    \[ Sv = S_1u + S_2w. \]

    $ \forall v \in V $ 均有
    \[ S(\sqrt{T^*T}v) = S_1(\sqrt{T^*T}v) = Tv, \]

    所以 $ T = S\sqrt{T^*T} $. 下面就是确确实实地证明 $ S $ 事实上是一个等距同构.

    $ \forall v \in V $,有
    \[ \lVert Sv \rVert^2 = \lVert S_1u + S_2w \rVert^2     = \lVert S_1u \rVert^2 + \lVert S_2w \rVert^2 = \lVert u \rVert^2 + \lVert w \rVert^2 = \lVert v \rVert^2 \]

    命题得证.
\end{proof}

极分解定理将所有一般的算子表成了一个等距同构和一个正算子的乘积,而这两种算子我们之前的章节都已经给出了完全的描述,所以我们探讨的内积空间上算子的约化分解的最终结论就是极分解定理.

特别地,考虑 $ \mathbf{F} = \mathbf{C} $ 的情况. 设 $ T \in \mathcal{L}(V) $,其有极分解 $ T = S\sqrt{T^*T} $,$ S $ 是等距同构. 则 $ V $ 有一个标准正交基使得$ S $ 关于这个基有对角矩阵,且 $ V $ 还有一组标准正交基使得 $ \sqrt{T^*T} $ 关于这组基有对角矩阵. 但很难有这样一组标准正交基满足 $ S $ 和 $ \sqrt{T^*T} $ 的同时对角化,不过这也给了我们一个思路,如果我们尝试用两组标准正交基对同一个算子描述可能会得到更简单的矩阵描述.

为了与极分解配对,我们引入奇异值的概念.

\begin{definition}[奇异值] \index{qiyizhi@奇异值 (singular value)}
    设 $ T \in \mathcal{L}(V) $,则 $ T $ 的奇异值就是 $ \sqrt{T^*T} $的特征值,每个特征值 $ \lambda $ 重复 $ \dim E(\lambda, \sqrt{T^*T}) $ 次.
\end{definition}

显然算子的奇异值都是非负的,因为他们都是正算子 $ \sqrt{T^*T} $ 的特征值. 而且对 $ \sqrt{T^*T} $ 应用谱定理和可对角化条件可知,每个 $ T \in \mathcal{L}(V) $都有 $ \dim V $ 个奇异值.

那么依托奇异值和两组标准正交基,我们可以对 $ V $ 上的每个算子都给出简洁的矩阵表示.

\begin{theorem}[奇异值分解] \index{qiyizhi!fenjie@奇异值 (singular value decomposition)}
    设 $ T \in \mathcal{L}(V) $ 有奇异值 $ s_1, \ldots , s_n $,则 $ V $ 存在两组标准正交基$ e_1, \ldots, \e_n $ 和 $ f_1, \ldots, f_n $ 使得 $ \forall v \in V $ 均有$ Tv = s_1 \langle v, e_1 \rangle f_1 + \cdots + s_n \langle v, e_n \rangle f_n $.
\end{theorem}

\vspace{2ex}
\centerline{\heiti \Large 内容总结}

\vspace{2ex}
\centerline{\heiti \Large 习题}

\vspace{2ex}
{\kaishu }
\begin{flushright}
    \kaishu

\end{flushright}

\centerline{\heiti A组}
\begin{enumerate}
    \item
\end{enumerate}

\centerline{\heiti B组}
\begin{enumerate}
    \item
\end{enumerate}

\centerline{\heiti C组}
\begin{enumerate}
    \item
\end{enumerate}

\chapter{实空间上的算子}

\vspace{2ex}
\centerline{\heiti \Large 内容总结}

\vspace{2ex}
\centerline{\heiti \Large 习题}

\vspace{2ex}
{\kaishu }
\begin{flushright}
    \kaishu

\end{flushright}

\centerline{\heiti A组}
\begin{enumerate}
    \item
\end{enumerate}

\centerline{\heiti B组}
\begin{enumerate}
    \item 设 $ V $ 是有限维复内积空间,$ S, T \in \mathcal{L}(V) $ 均为正规算子. 证明:若 $ ST = TS $,则
          \begin{enumerate}
              \item $ V $ 上存在一组标准正交基,使得 $ S, T $ 在此基下的矩阵都是对角矩阵.

              \item $ S $ 与 $ T $ 的复合也是正规算子.
          \end{enumerate}
\end{enumerate}

\centerline{\heiti C组}
\begin{enumerate}
    \item
\end{enumerate}

\begingroup
\SetLUChapterNumberingStyle{12}
\def\theHchapter{\arabic{chapter}ε}

\chapter{实数域的诸扩域}

\ResetChapterNumberingStyle{29}
\endgroup

\begingroup
\SetLUChapterNumberingStyle{13}
\def\theHchapter{\arabic{chapter}ε}

\chapter{线性动力系统}

\ResetChapterNumberingStyle{29}
\endgroup

\chapter{线性代数与微积分}

\vspace{2ex}
\centerline{\heiti \Large 内容总结}

\vspace{2ex}
\centerline{\heiti \Large 习题}

\vspace{2ex}
{\kaishu }
\begin{flushright}
    \kaishu

\end{flushright}

\centerline{\heiti A组}
\begin{enumerate}
    \item
\end{enumerate}

\centerline{\heiti B组}
\begin{enumerate}
    \item
\end{enumerate}

\centerline{\heiti C组}
\begin{enumerate}
    \item
\end{enumerate}

\begingroup
\SetLUChapterNumberingStyle{14}
\def\theHchapter{\arabic{chapter}ε}

\chapter{线性代数之外的代数}

\ResetChapterNumberingStyle{30}
\endgroup


\backmatter
{\small
\printindex
\printindex[sym]
}

\end{document}
